\subsection{Test Results}
All checks within test\_imu\_sensor.py passed as expected. Table \ref{tab:results} shows the test results. Figures \ref{fig:noise} and \ref{fig:walk} show the module output for the process noise and walk bounds checks, respectively.

\begin{table}[H]
	\caption{Test results}
	\label{tab:results}
	\centering \fontsize{10}{10}\selectfont
	\begin{tabular}{c | c | c  } % Column formatting, 
		\hline
		\textbf{Test} 						  			   & \textbf{Pass/Fail} 						   			& \textbf{Notes} 									\\ \hline
		Gyro/Accelerometer I/O 		   				& \input{AutoTex/gyroIOPassFail}              & \input{AutoTex/gyroIOFailMsg}			 \\ \hline
		MRP Switching 					   				& \input{"AutoTex/mrp switchPassFail"} 	   & \input{"AutoTex/mrp switchFailMsg"}  \\ \hline
		Static Bias 										 & \input{AutoTex/biasPassFail} 				 & \input{AutoTex/biasFailMsg} 				   \\ \hline
		Process Noise 							         & \input{AutoTex/noisePassFail} 				& \input{AutoTex/noiseFailMsg}				 \\ \hline
		Discretization 						  			   & \input{AutoTex/discretizationPassFail}    & \input{AutoTex/discretizationFailMsg}  \\ \hline
		Saturation 							   				&  \input{AutoTex/saturationPassFail} 		 &  \input{AutoTex/saturationFailMsg}      \\ \hline
		Accelerometer Center of Mass Offset & \input{"AutoTex/COM offsetPassFail"}    & \input{"AutoTex/COM offsetFailMsg"}  \\ \hline
		IMU Misalignment 							   & \input{AutoTex/misalignmentPassFail}     & \input{AutoTex/misalignmentFailMsg}  \\ \hline
		Bias Walk Bounds 							   & \input{"AutoTex/walk boundsPassFail"}   & \input{"AutoTex/walk boundsFailMsg"}\\ \hline
	\end{tabular}
\end{table}

Fig. \ref{fig:omegaWalkBoundPlot} and Fig. \ref{fig:accelWalkBoundPlot} show that the random walk remains within bounds for the Bias Walk Bounds tests. The bounds are shown by solid, horizontal blue and green lines.

\input{AutoTex/omegaWalkBoundPlot} \label{fig:omegaWalkBoundPlot}
\input{AutoTex/accelWalkBoundPlot} \label{fig:accelWalkBoundPlot}

\pagebreak %needed to keep images/paragraphs in the right place. Cannot \usepackage{float} here because it is not used in the AutoTex implementation.