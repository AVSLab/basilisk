% !TEX root = ./Basilisk-CoarseSunSensor-20170803.tex

\section{User Guide}

\subsection{Setting the CSS Unit Direction Vector}
\subsubsection{Direct Method}
The unit normal vector of each CSS sensor is set through the $\cal B$ frame vector representation
\begin{equation}
		{\tt nHat\_B} \equiv \leftexp{B}{\hat{\bm n}}
\end{equation}
It is possible to set these vectors directly.  However, there are some convenience functions that make this process easier.  

\subsubsection{Via a Common Sensor Platform}
Multiple CSS devices are often arranged together on a single CSS platform.  The orientation of the body-fixed platform frame $\frameDefinition{P}$ relative to the body frame $\frameDefinition{B}$ is given through $[PB]$.  In the CSS module, the DCM is specified through the variable {\tt dcm\_PB}.

Assume the DCM  {\tt dcm\_PB} is set directly via Python.  Two angles then define the orientation of the sensor normal axis $\hat{\bm n}$.  The elevation angle is $\phi$ and the azimuth angle is $\theta$.  These are an Euler angle sequence with the order 3-(-2).  The elevation angle is a positive 3-axis rotation, while the azimuth is a minus 2-axis rotation.  The helper function
$$
{\tt setUnitDirectionVectorWithPerturbation}(\theta_{p}, \phi_{p}) 
$$
where $\theta_{p}$ and $\phi_{p}$ are CSS heading perturbation can can be specified.  The Euler angles implemented are then
\begin{align}
		\phi_{\text{actual}} &= \phi_{\text{true}} + \phi_{p} \\
		\theta_{\text{actual}} &= \theta_{\text{true}} + \theta_{p} 
\end{align}
To setup un-perturbed CSS sensor axes simple set these perturbation angles to zero.  

Instead of setting the DCM {\tt dcm\_PB} variable directly, this can also be set via the helper function
$$
	{\tt setBodyToPlatformDCM}(\psi, \theta, \phi)
$$
where $(\psi, \theta, \phi)$ are classical $3-2-1$ Euler angles that map from the body frame $\cal B$ to the platform frame $\cal P$.  




\subsection{CSS Field-of-View}
The CSS sensor field of view is set by specifying the class variable
$$
	{\tt fov}
$$
This angle is the angle from the bore-sight axis to the edge of the field of view, and is expressed in terms of radians.  





\subsection{CSS Output Scale}
The general CSS signal computation is performed in a normalized manner yielding an unperturbed output between 0 and 1.  The CSS module variable
$$
	{\tt scaleFactor}
$$
is the scale factor $\alpha$ which scales the output to the desired range of values, as well as the desired units.  For example, if the maximum sun signal ($\hat{\bm n}$ points directly at sun) should yield 1 mA, then the scale factor is set to this value.    



\subsection{Specifying CSS Sensor Noise}
Three types of CSS signal corruptions can be simulated.  If not specified, all these corruptions are zeroed.  

The Kelly corruption parameter $\kappa$ is set by specifying the variable
$$
	{\tt KellyFactor}
$$

Second, to add a gaussian noise component to the normalized output the variable, the variable
$$
	{\tt SenNoiseStd}
$$
is set to a non-zero value.  This is the standard deviation of normalized gaussian noise.  Note that this noise magnitude is in terms of normalized units as it is added to the 0-1 nominal signal.  

Lastly, to simulate a signal bias, the variable
$$
	{\tt SenBias}
$$
is set to a non-zero value.   This constant bias of the normalized gaussian noise.  


\subsection{Connecting Messages}
Of the three possible input messages to the CSS module, the following message inputs are required for the module to properly operate:
$$
	{\tt SpicePlanetStateSimMsg} \\
	{\tt SCPlusStatesSimMsg}
$$
The first message is used to know the sun heading vector, while the second message provides the spacecraft inertial orientation,

The last message is optional.  If the {\tt EclipseSimMsg} is connected, then the solar eclipse information is taking into account.  The eclipse info provides 0 if the spacecraft is fully in a planet's shadow, 1 if in free space fully exposed to the, and a value between (0,1) if in the penumbra region.  The cosine sensor value $\hat\gamma$ is scaled by this eclipse value.  If the message is not connected, then this value default to 1, thus simulating a spacecraft that is fully exposed to the sun.  



\subsection{Setting Up CSS Modules}
\subsubsection{Individual CSS Units}
It is possible to add Individual CSS units to the Basilisk simulation.   This is done by invoking instances of the {\tt CoarseSunSensor()} class from python, configuring the required states, and then adding each to the BSK evaluation stack.  Each {\tt CoarseSunSensor} class has it's own {\tt UpdateState()} method that get's evaluated each update period.  

This setup is convenient if only 1-2 CSS units have to be modeled, but can be cumbersome if a larger cluster of CSS units must be administered.  When setup this way, each CSS unit will output an individual CSS output message.

\subsubsection{Array or Constellation of CSS Units}
An alternate method to setup a series of CSS units is to use the {\tt CSSConstellation()} class.  This class is able to store a series of CSS {\tt CoarseSunSensor} objects, and engage the update routine on all of them at once.  This way only the {\tt CSSConstellation} module needs to be added to the BSK update stack.  In this method the 
{\tt CSSConstellation} module outputs a single CSS sensor message which contains an array of doubles with the CSS sensor signal.  Here the individual CSS units {\tt CSS1}, {\tt CSS2}, etc. are setup and configured first.  Next, they are added to the {\tt CSSConstellation} through the python command
$$
	{\tt cssConstellation.sensorList = coarse\_sun\_sensor.CSSVector([CSS1, CSS2, ..., CSS8])}
$$

