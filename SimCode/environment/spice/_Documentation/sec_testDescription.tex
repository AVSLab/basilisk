\section{Test Description and Success Criteria}

\subsection{Sub Tests}

This test is located in {\tt SimCode/environment/spice/\_UnitTest/test\_unitSpice.py}. In order to get good coverage of all the outputs of Spice, the test is broken up into several parts: \par

\begin{enumerate}
\item \underline{Time Increment Check} The check goes through the simulation time advancement check. The steps are verified to be consistent. 
\item \underline{GPS Time Check} At a specific UTC time, the simulation calculates GPS time. We therefore recalculate the expected GPS time at that date and compare it with the simulation for accuracy. 
\item \underline{Julian Day Check} Similarly, we independently calculate the Julian date given the epoch of the simulation, and compare it to the simulation's Julian date.
\item \underline{Mars Position Check} The position for Mars computed by Spice is compared to JPL Horizon's ephemeris for the same epoch and propagation time.
\item \underline{Earth Position Check} The position for Earth computed by Spice is compared to JPL Horizon's ephemeris for the same epoch and propagation time.
\item \underline{Sun Position Check} The position for the Sun computed by Spice is compared to JPL Horizon's ephemeris for the same epoch and propagation time.
\end{enumerate} 

\subsection{Test Success Criteria}

In order to thoroughly test the spice ephemeris module, the test was parametrized studying multiple dates. Through all of the tests, the error tolerances drive the success criteria, and are explained in the next section. 

\underline{Dates studied}:

In order to create a complete simulation of the different possible situations, these tests were run on 24 dates. Starting in February 2015, the simulation is run for the 10th and 20th of every other month. The last day tested is therefore, nearly two years later in December of 2016. For each of the days, we needed the truth vectors for the positions of Mars, Earth and the Sun in the J200 reference frame. We present all the parameters along with the test results in the following section.

\underline{Truth Data}:

The truth data was taken from JPL's Horizon's database. It provides highly accurate ephemerides for solar system objects ( 734640 asteroids, 3477 comets, 178 planetary satellites, 8 planets, the Sun, L1, L2, select spacecraft, and system barycenters ). 
