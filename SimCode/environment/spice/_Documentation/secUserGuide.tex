\section{User Guide}

In order to have the spice\_interface module write out it's ephemeris messages, it must be set up from python. The main steps are listed below:

\begin{itemize}
 \item[-]   \texttt{SpiceObject = spice\_interface.SpiceInterface()}: Construct a spice\_interface object
  \item[-]  \texttt{SpiceObject.ModelTag = "SpiceInterfaceData"}: Define a model tag for the object
 \item[-]  \texttt{SpiceObject.SPICEDataPath = splitPath[0] + '/External/EphemerisData/'}: Give the path to the Spice Kernels
  \item[-]  \texttt{SpiceObject.OutputBufferCount = 10000}: Define a buffer count
 \item[-]  \texttt{SpiceObject.PlanetNames = spice\_interface.StringVector(["earth", "mars barycenter", "sun"])}: Give the celestial body names that which to be tracked
\item[-]  \texttt{SpiceObject.UTCCalInit = "2016 October 20, 00:00:00.0 TDB"}: Give a start date to the object
\end{itemize}

The default frame is \texttt{J2000}.
