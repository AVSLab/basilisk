\section{Test Results}

\subsection{Pass/Fail}
The test results are explained below and summarized in Table~\ref{tab:results}.

\begin{enumerate}
\item{Error Bound Enforcement: We do want to violate the error bound a 
   statistically small number of times as most bounds are specified 3-sigma 
   and we'll need to be at the spec to make sure it works.  All signals remained 
   inside their bounds greater than 1-sigma (~30\%) of the time.  }
\item{Error Bound Usage: As stated above, we want to ensure that the random 
   walk process is effectively utilizing the error bound that it has been 
   given and not remaining mired near zero.  All error signals cross up above 
   80\% of their error bound at least once.}
\item{Corner Case Usage: All errors/warnings were stimulated and the simulation 
   still ran without incident. }
\end{enumerate}

\begin{table}[htbp]
    \caption{Test Results}
\label{tab:results}
    \centering \fontsize{10}{10}\selectfont
\begin{tabular}{|c||c|}
\hline
SubTest & Result \\ \hline \hline
Bound Enforcement& \textcolor{ForestGreen}{Passed} \\ \hline
Bound Usage &  \textcolor{ForestGreen}{Passed}\\ \hline
Corner Case &  \textcolor{ForestGreen}{Passed}\\ \hline
\end{tabular}
\end{table}



\subsection{Test Coverage}
The method coverage for all of the methods included in the simple\_nav 
module are tabulated in Tables~\ref{tab:cov_met} and \ref{tab:cov_met2}.

\begin{table}[htbp]
    \caption{Simple Navigation Test Analysis Results}
   \label{tab:cov_met}
        \centering \fontsize{10}{10}\selectfont
   \begin{tabular}{c | r | r | r} % Column formatting, 
      \hline
      Method Name    & Unit Test Coverage (\%) & Runtime Self (\%) & Runtime Children (\%) \\
      \hline
      UpdateState & 100.0 & 0.08 & 15.0 \\
      SelfInit & 100.0 & 0.0 & 0.0 \\
      CrossInit & 100.0 & 0.0 & 0.0 \\
      computeOutput & 100.0 & 0.0 & 0.0 \\
      \hline
   \end{tabular}
\end{table}

\begin{table}[htbp]
    \caption{GaussMarkov Test Analysis Results}
   \label{tab:cov_met2}
        \centering \fontsize{10}{10}\selectfont
   \begin{tabular}{c | r | r | r} % Column formatting, 
      \hline
      Method Name    & Unit Test Coverage (\%) & Runtime Self (\%) & Runtime Children (\%) \\
      \hline
      computeNextState & 100.0 & 0.71 & 12.4 \\
      setRNGSeed & 100.0 & 0.0 & 0.0 \\
      setPropMatrix & 100.0 & 0.0 & 0.0 \\
      getCurrentState & 100.0 & 0.0 & 0.0 \\
      setUpperBounds & 100.0 & 0.0 & 0.0 \\
      setNoiseMatrix & 100.0 & 0.0 & 0.0 \\
      setPropMatrix & 100.0 & 0.0 & 0.0 \\
      \hline
   \end{tabular}
\end{table}
For all of the code this test was designed for, the coverage percentage is 
100\%.  The CPU usage of the model is higher than would be ideal although this 
might just be a symptom of the level of simplicity present in the overall 
simulation.  The majority of the computations are coming from two pieces of the 
GaussMarkov code.  

The first is the random number generator.  The model is using 
one of the simplest random number generators in the standard template library.  
That is still a relatively expensive operation as random numbers are costly and 
we generate a new random number for each state.  The second factor is in the 
state and noise propagation.  Those are being performed with a matrix 
multiplication that is an $n^2$ operation.  We could save some computations 
here in the future if we took away the cross-correlation capability from some 
of the states which would definitely be easy and accurate.  It would just take 
some more code.
