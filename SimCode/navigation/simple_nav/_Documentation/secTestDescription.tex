\section{Test Description and Success Criteria}


\subsection{Test location}

The unit test for the simple\_nav module is located in:\\

\noindent
{\tt SimCode/navigation/simple\_nav/UnitTest/SimpleNavUnitTest.py} \\
\\

\subsection{Subtests}

\noindent This unit test is designed to functionally test the simulation model 
outputs as well as get complete code path coverage.  The test design is broken 
up into three main parts:\\
\begin{enumerate}
\item{\underline{Error Bound Enforcement}: The simulation is run for 2.4 hours and the 
   error bounds for all of the signals are tested. This test length is long enough to see both
   the walk in the signal and the noise, all the while not being so long as to slow down the test.
   The test ensures that the bounds are crossed no more than 30\% of the time.}
\item{\underline{Error Bound Usage}: The error signals are checked for all of the model 
   parameters over the course of the simulation to ensure that the error gets 
   to at least 80\% of its maximum error bound at least once, ensuring that noise is indeed
   properly introduced.}
\item{\underline{Corner Case Check}: The simulation is intentionally given bad inputs to 
   ensure that it alerts the user and does not crash.}
\end{enumerate}

\subsection{Test success criteria}

These tests are considered to pass if during the whole simulation time of 144 minutes,
all the variables need to say within an allowable statistical error. This means that they
must stay within their bounds $30\%$ of the time.

At the same time, we want each of the errors to get to $80\%$ of their respective error bounds
at least once during the run.

These sigma bounds are defined in Table~\ref{tab:sigmas}. These are chosen in regard to
the simulation's parameters and their orders of magnitude, while the bounds are defined in
Table~\ref{tav:bounds}

\begin{table}[htbp]
    \caption{Sigma Values}
\label{tab:sigmas}
    \centering \fontsize{10}{10}\selectfont
\begin{tabular}{|c||c|c|c|c|c|c|}
\hline
Variable & Position & Velocity & Attitude & Rates & $\Delta$ V & Sun Position \\ \hline \hline
Associated $\sigma$ & 5 (m)& 0.035 (m/s)& $\frac{1}{360}$ (deg) & 0.05 (deg/s) & 1 (deg) & 0.1 (deg) \\ \hline 
\end{tabular}
\end{table}

\begin{table}[htbp]
    \caption{Upper bound Values}
\label{tab:bounds}
    \centering \fontsize{10}{10}\selectfont
\begin{tabular}{|c||c|c|c|c|c|c|}
\hline
Variable & Position & Velocity & Attitude & Rates & $\Delta$ V & Sun Position \\ \hline \hline
Associated bounds & 1000 (m)& 1 (m/s)& 0.29 (deg) & 1.15 (deg/s) & 5 (deg) & 3.03 (deg) \\ \hline 
\end{tabular}
\end{table}

These values were set to make the statistics visible in one test. They are also sometimes set as radians which explains why some values are not round.

