\documentclass[]{BasiliskReportMemo}
\usepackage{AVS}


\newcommand{\submiterInstitute}{Autonomous Vehicle Simulation (AVS) Laboratory,\\ University of Colorado}

\newcommand{\ModuleName}{reactionWheel\_ConfigureRWRequests }
\newcommand{\subject}{Testing Reaction Wheel Model}
\newcommand{\status}{Initial document draft}
\newcommand{\preparer}{J. Alcorn}
\newcommand{\summary}{This unit test validates the internal aspects of the Basilisk reaction wheel (RW) module by comparing module output to expacted output. The Basilisk RW module's {\tt configureRWRequests} method is responsible for converting requested torque into actual RW torque. {\tt configureRWRequests} applies saturation, minimum torque, and coloumb friction to the requested torque to produce an actual applied RW torque.}


\begin{document}


\makeCover


%
%	enter the revision documentation here
%	to add more lines, copy the table entry and the \hline, and paste after the current entry.
%
\pagestyle{empty}
{\renewcommand{\arraystretch}{1.1}
\noindent
\begin{longtable}{|p{0.5in}|p{4.5in}|p{1.14in}|}
\hline
{\bfseries Rev}: & {\bfseries Change Description} & {\bfseries By} \\
\hline
Draft & Initial document creation & J. Alcorn \\
\hline

\end{longtable}
}

\newpage
\setcounter{page}{1}
\pagestyle{fancy}

\tableofcontents
~\\ \hrule ~\\


\section{Introduction}
The Basilisk star tracker module star\_tracker.cpp is responsible for producing sensed Euler parameters from true simulation attitude. Given the true spacecraft structure to inertial attitude as a Modified Rodriguez Parameter (MRP) set, the module outputs an Euler Parameter (EP) set and time stamp. A Gauss-Markov process model is used to add noise to the Euler parameter measurement.

\section{{\tt test\_reactionWheelStateEffector\_ConfigureRWRequests.py} Test Description}

This test {\tt test\_reactionWheelStateEffector\_ConfigureRWRequests.py} is located in \newline {\tt SimCode/dynamics/reactionWheels/\_UnitTest}. In order to get good coverage of all the aspects of the module, the test is broken up into several parts: \par

\begin{enumerate}
	\item \underline{Torque Saturation} The check validates conversion of requested torque to a torque below the maximum the RW can physically produce. This test checks both negative and positive values of torque.
	\item \underline{Minimum Torque} The check validates conversion of requested torque to a torque above the minimum the RW can physically produce. This test checks both negative and positive values of torque.
	\item \underline{Coloumb Friction} The check verifies that Coloumb friction is applied in the correct direction and magnitude based on wheel speed.  This test checks both negative and positive values of torque and wheel speed.
\end{enumerate} 

\subsection{Test Parameters}

This section summarizes the test input/output for each of the checks. 
\begin{itemize}
	\item \underline{Error Tolerance}
	Since none of the tests involve integration, the error tolerance is $0$. That is, in order for the tests to pass the module output must exactly match the expected output.
\end{itemize}


\subsection{Test Results}

All checks within {\tt test\_reactionWheelStateEffector\_ConfigureRWRequests.py} passed as expected. Table \ref{tab:results} shows the test results.

\begin{table}[htbp]
	\caption{Test results.}
	\label{tab:results}
	\centering \fontsize{10}{10}\selectfont
	\begin{tabular}{c | c | c  } % Column formatting, 
		\hline
		\textbf{Test} & \textbf{Pass/Fail} & \textbf{Notes} \\ \hline
		Saturation Torque & \textcolor{ForestGreen}{Passed} & \\ \hline
		Minimum Torque & \textcolor{ForestGreen}{Passed} & \\ \hline
		Coloumb Friction & \textcolor{ForestGreen}{Passed} & \\ \hline
	\end{tabular}
\end{table}

\section{{\tt test\_reactionWheelStateEffector\_integrated} Test Description}

\subsection{Test Parameters}

\subsection{Test Results}

\end{document}
