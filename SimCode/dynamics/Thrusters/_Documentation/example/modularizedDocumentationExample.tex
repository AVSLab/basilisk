\documentclass[]{BasiliskReportMemo}

\usepackage{cite}
\usepackage{AVS}
\usepackage{float} %use [H] to keep tables where you put them
\usepackage{array} %easy control of text in tables
\usepackage{graphicx}
\usepackage{hyperref}
\bibliographystyle{plain}


\newcommand{\submiterInstitute}{Autonomous Vehicle Simulation (AVS) Laboratory,\\ University of Colorado}


\newcommand{\ModuleName}{Thrusters}
\newcommand{\subject}{Thruster dynamic effector module}
\newcommand{\status}{Initial Document}
\newcommand{\preparer}{T. Teil}
\newcommand{\summary}{ This test compares the simulation's forces and torques due to thrusters with expected values. This test runs a variety of thrust scenarios, creates truth values, and compares them point by point to the simulation. It therefore analyzes the thrust behavior down to test rate precision.} %include packages, styles, and new commands in this file.

\begin{document}

\makeCover

%
%	enter the revision documentation here
%	to add more lines, copy the table entry and the \hline, and paste after the current entry.
%
\pagestyle{empty}
{\renewcommand{\arraystretch}{2}
\noindent
\begin{longtable}{|p{0.5in}|p{4.5in}|p{1.14in}|}
\hline
{\bfseries Rev}: & {\bfseries Change Description} & {\bfseries By} \\
\hline
1.0 & First draft & S. Carnahan \\
\hline

\end{longtable}
} %when you update the file, update the revision table.

\newpage
\setcounter{page}{1}
\pagestyle{fancy}

\tableofcontents %Autogenerate the table of contents
~\\ \hrule ~\\ %Makes the line under table of contents
	
\section{Model Description}

\subsection{Introduction}
The ephemeris converter module has the purpose of copying the spice sim messages into a flight software interface message. This allow flight software module to access planet ephemeris data.

\subsection{I/O}

The spice message, which is read in by the module, contains the following variables:

\begin{itemize}
\item J2000Current: the time of validity for the planet state
\item PositionVector : the true position of the planet for the time
    \item  VelocityVector[3]: the tue velocity of the planet for the time
    \item  J20002Pfix: the orientation matrix of planet-fixed relative to inertial
    \item  J20002Pfix\_dot: the derivative of the orientation matrix of planet-fixed relative to inertial
    \item  computeOrient: a flag indicating whether the reference should be computed
    \item  PlanetName
\end{itemize}

Only a the position and velocity vectors are transferred to the output ephemeris message, which therefore contains:
\begin{itemize}
    \item r\_BdyZero\_N[3]: the position of orbital body 
    \item v\_BdyZero\_N[3]: the velocity of orbital body
    \item timeTag: the vehicle Time-tag for state
\end{itemize}


 %This section includes mathematical models, code description, etc.


\section{Model Functions}

This module allows the creation of a map function that maps the input message to the output message in which the content will be copied.
The use of this can be seen in the user guide. %This includes a concise list of what the module does.


\section{Model Assumptions and Limitations}
The limitations of spherical harmonics gravity model are well-known and clearly explained in REFERENCE SCHAUB. The limitations include:
\begin{itemize}
	\item \textbf{Coefficient Accuracy}: The coefficients used in the spherical harmonics equations are typically calculated based on gravitational data gathered by satellites. Therefore, the accuracy of the model is determined by the accuracy of the satellite instrumentation and precision of the stored data. Furthermore, for some bodies, there may not be sufficient information available to provide accurate coefficients or higher-degree coefficients.
	\item \textbf{Maximum Degree}: The spherical sarmonics equation is a series expansion. Therefore, any implementation must truncate the equation at some point. The truncated portion of the equation necessarily defines some amount of error in the final calculation. This error is, however, small after the first handful of terms. Additionally, a larger distance between gravity body and spacecraft requires fewer terms of the series to achieve equal accuracy as compared to a case with less distance. This code allows the user to request a maximum number of terms to evaluate rather than a specific accuracy. This code lead to less-than-desirable accuracy with small separation distances and greater-than-necessary run times with large separation distances.
	\item \textbf{Planetary Ephemeris Data}: This code generally relies on an external package for planetary ephemeris information. Errors included in this package will translate into error in the gravity calculations, but those errors should be small.
\end{itemize} %This explains the assumptions made to reach the final mathematical implementation of the model and how those assumptions limit the model's usefulness.


\section{Test Description and Success Criteria}


\subsection{Test location}

The unit test for the simple\_nav module is located in:\\

\noindent
{\tt SimCode/navigation/simple\_nav/UnitTest/SimpleNavUnitTest.py} \\
\\

\subsection{Subtests}

\noindent This unit test is designed to functionally test the simulation model 
outputs as well as get complete code path coverage.  The test design is broken 
up into three main parts:\\
\begin{enumerate}
\item{Error Bound Enforcement: The simulation is run for 2.4 hours and the 
   error bounds for all of the signals }
\item{Error Bound Usage: The error signals are checked for all of the model 
   parameters over the course of the simulation to ensure that the error gets 
   to at least 75\% of its maximum error bound at some point to make sure that 
   it isn't stuck around zero.}
\item{Corner Case Check: The simulation is intentionally given bad inputs to 
   ensure that it alerts the user and does not crash.}
\end{enumerate}

\subsection{Test success criteria}

These tests are considered to pass if during the whole simulation time of 144 minutes,
all the variables need to say within an allowable statistical error. This means that they
must stay within their 1-sigma bounds $30\%$ of the time.

These sigma bounds are defined in Table~\ref{tab:sigmas}. These are chosen in regard to
the simulation's parameters and their orders of magnitude.

\begin{table}[htbp]
    \caption{Sigma Values}
\label{tab:sigmas}
    \centering \fontsize{10}{10}\selectfont
\begin{tabular}{|c||c|c|c|c|c|c|}
\hline
Variable & Position & Velocity & Attitude & Rates & $\Delta$ V & Sun Position \\ \hline \hline
Associated $\sigma$ & 5 & 0.05 & $\frac{5}{3600}$ (rad) & 0.05 (rad/s) & 0.053 & 5 \\ \hline 
\end{tabular}
\end{table}

 %This explains the unit test for the model. I.e. what features are tested and how. It may also include test tolerances, etc.


\section{Test Parameters}

Test parameters and inputs go here. I think that success criteria would work better here than in the test description section.


\subsection{Test Results}

All checks within test\_gravityDynEffector.py passed as expected. Table \ref{tab:results} shows the test results.\\

\begin{table}[htbp]
	\caption{Test results.}
	\label{tab:results}
	\centering \fontsize{10}{10}\selectfont
	\begin{tabular}{c | c | c  } % Column formatting, 
		\hline
		\textbf{Test} 				    & \textbf{Pass/Fail} 						   			           & \textbf{Notes} 									\\ \hline
%		Setup Test 		   			  	& \input{AutoTex/sphericalHarmonicsPassFail}     & \input{AutoTex/sphericalHarmonicsFailMsg}			 \\ \hline
%		Single-Body Gravity		   	& \input{AutoTex/singleBodyPassFail}                 & \input{AutoTex/singleBodyFailMsg} \\ \hline
%		Multi-Body Gravity			 &\input{AutoTex/multiBodyPassFail}  			 	 &  \input{AutoTex/multiBodyFailMsg} 			   \\ \hline
	\end{tabular}
\end{table} %This displays the test results. This includes both pass/fail statements as well as visual outputs via AutoTeX


\section{User Guide}

In order to have the spice\_interface module write out it's ephemeris messages, it must be set up from python. The main steps are listed below:

\begin{itemize}
 \item[-]   \texttt{SpiceObject = spice\_interface.SpiceInterface()}: Construct a spice\_interface object
  \item[-]  \texttt{SpiceObject.ModelTag = "SpiceInterfaceData"}: Define a model tag for the object
 \item[-]  \texttt{SpiceObject.SPICEDataPath = splitPath[0] + '/External/EphemerisData/'}: Give the path to the Spice Kernels
  \item[-]  \texttt{SpiceObject.OutputBufferCount = 10000}: Define a buffer count
 \item[-]  \texttt{SpiceObject.PlanetNames = spice\_interface.StringVector(["earth", "mars barycenter", "sun"])}: Give the celestial body names that which to be tracked
\item[-]  \texttt{SpiceObject.UTCCalInit = "2016 October 20, 00:00:00.0 TDB"}: Give a start date to the object
\end{itemize}

The default frame is \texttt{J2000}.
 %This section is to provide advice to users on necessary/useful inputs and best practices.


\bibliography{bibliography.bib} %This includes references used and mentioned.

\end{document}
