\section{Test Results}

\subsection{Pass/Fail}

\begin{center}
\begin{tabular}{|c|c|c|}
\hline
Parameter Sets & Test Result & Error Tolerance \\ \hline \hline
1  & \textcolor{ForestGreen}{Passed} & $10^{-9}$ \\ \hline
2  & \textcolor{ForestGreen}{Passed} & $10^{-9}$ \\ \hline
3  & \textcolor{ForestGreen}{Passed} & $10^{-9}$ \\ \hline
4  & \textcolor{ForestGreen}{Passed} & $10^{-9}$ \\ \hline
5  &\textcolor{ForestGreen}{Passed}& $10^{-9}$ \\ \hline
6  & \textcolor{ForestGreen}{Passed} & $10^{-9}$ \\ \hline
7  & \textcolor{ForestGreen}{Passed} & $10^{-9}$ \\ \hline
8  & \textcolor{ForestGreen}{Passed} & $10^{-9}$ \\ \hline
9  & \textcolor{ForestGreen}{Passed} & $10^{-9}$ \\ \hline
10  &\textcolor{ForestGreen}{Passed}& $10^{-9}$ \\ \hline
11  & \textcolor{ForestGreen}{Passed} & $10^{-9}$ \\ \hline
12  &\textcolor{ForestGreen}{Passed} & $10^{-9}$ \\ 
\hline

\end{tabular}
\end{center}

\begin{enumerate}
\item{\underline{ Instantaneous On/Off Factor}:} 

\input{AutoTex/Snippet1Thrusters_5s_30deg_Loc2_Rate10.tex}

 Figures \ref{fig:Force_1Thrusters_5s_30deg_Loc2_Rate10} and \ref{fig:Torque_1Thrusters_5s_30deg_Loc2_Rate10} show the force and torque behaviors (respectfully) for the thruster unit test.
    
    \input{AutoTex/Force_1Thrusters_5s_30deg_Loc2_Rate10.tex}
    \input{AutoTex/Torque_1Thrusters_5s_30deg_Loc2_Rate10.tex}
    \input{AutoTex/1Thrusters_5s_30deg_Loc2_Rate10.tex}
    
    As Figure \ref{fig:1Thrusters_5s_30deg_Loc2_Rate10} shows, the desired behavior is captured exactly for each 
    firing in the test for all of the forces and torques. This is validated by the exact same predicted and simulated thrust arrays.  \textcolor{ForestGreen}{Test successful.}

\item{\underline{Short Instantaneous Firing: }}

\input{AutoTex/Snippet1Thrusters_0s_30deg_Loc2_Rate10.tex}

 Figure \ref{fig:Force_1Thrusters_0s_30deg_Loc2_Rate10} shows the force behavior given this short input. We see that the test rate begin small next to the thrust duration, doesn't capture the jump quite well.
    
    \input{AutoTex/Force_1Thrusters_0s_30deg_Loc2_Rate10.tex}
    \input{AutoTex/1Thrusters_0s_30deg_Loc2_Rate10.tex}
    
    As Figure \ref{fig:1Thrusters_0s_30deg_Loc2_Rate10} shows, the desired behavior is captured exactly for each 
    firing in the test for all of the forces and torques. Despite the lower test rate, the forces and torques behave appropriately. This is validated by the exact same predicted and simulated thrust arrays.\textcolor{ForestGreen}{Test successful.}

 \item{\underline{Short Instantaneous Firing with faster test rate: }}
 
 \input{AutoTex/Snippet1Thrusters_0s_30deg_Loc2_Rate1000.tex}

 Figure \ref{fig:Force_1Thrusters_0s_30deg_Loc2_Rate1000} shows the force behavior given the same short input as previously. We now see that the jump is well resolved.
    
    \input{AutoTex/Force_1Thrusters_0s_30deg_Loc2_Rate1000.tex}
    \input{AutoTex/1Thrusters_0s_30deg_Loc2_Rate1000.tex}
    
    As Figure \ref{fig:1Thrusters_0s_30deg_Loc2_Rate1000} shows, the desired behavior is captured exactly for each 
    firing in the test for all of the forces and torques. This is validated by the exact same predicted and simulated thrust arrays. \textcolor{ForestGreen}{Test successful.}
 
 \item{\underline{Instantaneous On/Off Factor with faster test rate:} }
 
  \input{AutoTex/Snippet1Thrusters_5s_30deg_Loc2_Rate100.tex}

 The thrust command given is now 5 seconds long, as in the base test. The difference is that the test rate is now augmented in order to guarantee that it does not affect the test.
     
    \input{AutoTex/1Thrusters_5s_30deg_Loc2_Rate100.tex}
    
    As Figure \ref{fig:1Thrusters_5s_30deg_Loc2_Rate100} shows, the desired behavior is captured exactly for each 
    firing in the test for all of the forces and torques. This is validated by the exact same predicted and simulated thrust arrays. \textcolor{ForestGreen}{Test successful.}
 
 \item{\underline{Thruster Angle Test:}}
 
   \input{AutoTex/Snippet1Thrusters_5s_10deg_Loc2_Rate10.tex}

 The test now shows that the simulation still behaves with different thruster orientations.
     
    \input{AutoTex/1Thrusters_5s_10deg_Loc2_Rate10.tex}
    
    As Figure \ref{fig:1Thrusters_5s_10deg_Loc2_Rate10} shows, the desired behavior is captured exactly for each 
    firing in the test for all of the forces and torques. This is validated by the exact same predicted and simulated thrust arrays. \textcolor{ForestGreen}{Test successful.}
 
 
   \item{\underline{Thruster Position Test:} }
   
      \input{AutoTex/Snippet1Thrusters_5s_30deg_Loc0_Rate10.tex}

 This test shows that different locations still give correct values for forces and torques.
      
    \input{AutoTex/1Thrusters_5s_30deg_Loc0_Rate10.tex}
    
    As Figure \ref{fig:1Thrusters_5s_30deg_Loc0_Rate10} shows, the desired behavior is captured exactly for each 
    firing in the test for all of the forces and torques. This is validated by the exact same predicted and simulated thrust arrays. \textcolor{ForestGreen}{Test successful.}
   
   \item{\underline{Thruster Number Test:} }
   
         \input{AutoTex/Snippet2Thrusters_5s_30deg_Loc2_Rate10.tex}

 This test shows that the thruster model can incorporate several thrusters correctly. We add a second thruster and use the modified truth function for the forces and torques.
      
    \input{AutoTex/2Thrusters_5s_30deg_Loc2_Rate10.tex}
    
    As Figure \ref{fig:2Thrusters_5s_30deg_Loc2_Rate10} shows, the desired behavior is captured exactly for each 
    firing in the test for all of the forces and torques. This is validated by the exact same predicted and simulated thrust arrays.  \textcolor{ForestGreen}{Test successful.}
   
\item{\underline{Ramp On/Ramp Off Firing:} }

         \input{AutoTex/SnippetRamp_10steps_5s_CutoffOFF_Rate10_CutoffOFF.tex}

 This test now ramps the thrust up and down. We use a 10 step ramp that takes $0.1$s to climb and fall. This ramp time is slightly exaggerated in order to see the ramp clearly.      
    \input{AutoTex/Ramp_10steps_CutoffOFF_5s_testRate10.tex}
    
    As Figure \ref{fig:Ramp_10steps_CutoffOFF_5s_testRate10} shows, the desired behavior is captured exactly for each 
    firing in the test for all of the forces and torques. This is validated by the exact same predicted and simulated thrust arrays. \textcolor{ForestGreen}{Test successful.}
   

  \item{\underline{Short ramped firing:} }
  
    \input{AutoTex/SnippetRamp_10steps_0s_CutoffOFF_Rate10_CutoffOFF.tex}

Using the same ramp, the thruster fires for $0.5$s. We expect to see a climb and immediate fall of the thrust factor.
       
    \input{AutoTex/Ramp_10steps_CutoffOFF_0s_testRate10.tex}
    
    As Figure \ref{fig:Ramp_10steps_CutoffOFF_0s_testRate10} shows, the desired behavior is captured exactly for each 
    firing in the test for all of the forces and torques. This is validated by the exact same predicted and simulated thrust arrays. \textcolor{ForestGreen}{Test successful.}
  
\item{\underline{Ramp On/Ramp Off Firing with faster test rate:} }

    \input{AutoTex/SnippetRamp_10steps_5s_CutoffOFF_Rate100_CutoffOFF.tex}

 Using once again the same ramp, we run the test for $5$ seconds with a faster test rate. We seek to validate that the test rate has no impact on the simulation.
      
    \input{AutoTex/Ramp_10steps_CutoffOFF_5s_testRate100.tex}
    
    As Figure \ref{fig:Ramp_10steps_CutoffOFF_5s_testRate100} shows, the desired behavior is captured exactly for each 
    firing in the test for all of the forces and torques. This is validated by the exact same predicted and simulated thrust arrays. \textcolor{ForestGreen}{Test successful.}

\item{\underline{Cutoff firing:}}

 \input{AutoTex/SnippetRamp_10steps_CutoffON_Rate10_CutoffON.tex}

 Using the same ramp, we start firing the thruster with an initial command of 5 seconds. After just $0.2$ seconds of thrust ramping, we change the test command and thrust for $0.3$ seconds. This leads to a total thrust of $0.5$ seconds, and validates the fact that the command was properly overridden.
      
    \input{AutoTex/Ramp_10steps_CutoffON_5s_testRate10.tex}
    
    As Figure \ref{fig:Ramp_10steps_CutoffON_5s_testRate10} shows, the desired behavior is captured exactly for each 
    firing in the test for all of the forces and torques. This is validated by the exact same predicted and simulated thrust arrays. \textcolor{ForestGreen}{Test successful.}

\item{\underline{Ramp down firing:}}

 \input{AutoTex/SnippetRamp_10steps_CutoffON_Rate10rampDownON.tex}

 In this test, the initial command is of $0.5$ seconds. On the falling side of the ramp, a new command is given for $1.5$s. We expect to see the thrust climb again to steady state and last for the expected command time.   
      
    \input{AutoTex/Ramp_10steps_CutoffONrampDownON_testRate10.tex}
    
    As Figure \ref{fig:Ramp_10steps_CutoffONrampDownON_testRate10} shows, the desired behavior is captured exactly for each 
    firing in the test for all of the forces and torques. This is validated by the exact same predicted and simulated thrust arrays.  \textcolor{ForestGreen}{Test successful.}

\end{enumerate}




\subsection{Test Coverage}
The method coverage for all of the methods included in the spice\_interface 
module are tabulated in Table~\ref{tab:cov_met}

\begin{table}[htbp]
    \caption{Test Analysis Results}
   \label{tab:cov_met}
        \centering \fontsize{10}{10}\selectfont
   \begin{tabular}{c | r | r | r} % Column formatting, 
      \hline
      Method Name    & Unit Test Coverage (\%) & Runtime Self (\%) & Runtime Children (\%) \\
      \hline
      SelfInit & 100.0 & 0.0 & 0.0 \\
      CrossInit & 100.0 & 0.0 & 0.0 \\
      AddThruster & 100.0 & 0.0 & 0.0 \\
      UpdateState & 100.0 & 0.0 & 0.0 \\
      WriteOutputMessages & 100.0 & 0.0 & 0.0 \\
      ReadInputs & 100.0 & 0.0 & 0.0 \\
      ConfigureThrustRequests & 100.0 & 0.0 & 0.0 \\
      ComputeDynamics & 100.0 & 0.0 & 9.8 \\
      ComputeThrusterFire & 100.0 & 0.0 & 0.0 \\
      ComputeThrusterShut & 100.0 & 0.0 & 0.0 \\
      updateMassProperties & 100.0 & 0.0 & 0.6 \\
      \hline
   \end{tabular}
\end{table}

For all of the methods in the spice\_interface modules, the code coverage 
percentage is 100\% which meets our test requirements.  Additionally, 100\% of 
all code branches in the thruster\_dynamics source code were executed by this 
test.

%The test that was run to calculate thruster CPU usage was deliberately selected as 
%a stressing case for the thruster model.  The MOI burn was executed 9000 seconds 
%after the simulation was initialized and that maneuver takes 2000 seconds, so 
%approximately 20\% of the simulation was run with the vehicle under thrust.  
%With this stressing case, the ThrusterDynamics model accounted for 10\% of the 
%overall processing, which is certainly acceptable at this time.

The main CPU usage of the thruster\_dynamics source code occurs in the 
ComputeDynamics method that is called by the dynamics source.  The 
ThrusterDynamics methods themselves account for very little of the processing 
and it is the vector/matrix manipulation utilities called from the source that 
are the main culprits.  While the thruster model's ComputeDynamics function is 
using 50\% of the dynamics processing, that is only amounting to 10\% of the 
overall simulation processing.  The rest of the architecture in Basilisk should 
allow us to take the processing hit that we are getting from the 
ThrusterDynamics module without issue.

\subsection{Conclusions}
The thruster module has sufficient fidelity to accomplish the analysis 
that we need to perform thrust maneuvers.  All model capabilities were 
tested and analyzed in this document with all observed performance being nominal 
compared to the going-in expectation. Every line of source code was successfully tested and the integrated model 
performance was analyzed and is acceptable. Furthermore many thrust scenarios were tested in order to cover all outcomes of a maneuver and the robustness of the simulation.


