\section{Model Description}

\section{Introduction}
The Thruster model in the AVS Basilisk simulation is used to emulate the effect 
of a vehicle's thrusters on the overall system.  Its primary use is to generate 
realistic forces/torques on the vehicle structure and body.  This is 
accomplished by applying a force at a specified location/direction in the body and 
using the current vehicle center of mass to calculate the resultant torque.  
Each individual thruster in a given model has its own ramp-up/ramp-down profile 
specified as part of its initialization data and it follows those profiles during 
start-up and shutdown.

The thruster model also contains a mechanism that is used to change the current 
vehicle mass properties as the thruster fires propellant overboard.  This model 
uses the thruster ISP (specific impulse, also specified with configuration data) 
to calculate how much mass is being removed during a given thruster firing and 
decrements the mass properties included in the thruster linearly as a function 
of mass.  

\subsection{Output}

This module computes forces and torques, and is called as a Dynamic Effector to integrate the state.