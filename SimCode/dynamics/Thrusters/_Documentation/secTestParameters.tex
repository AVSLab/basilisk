\section{Test Parameters}


In order to have a rigorous automated test, we have to predict the forces and torques that will apply on the spacecraft. We use the following equations to compute the thrust at each time step. We call $\alpha$ the angle in which the thruster is pointing, $\bm r = r \hat{\bm e_r}= \left(r_x, r_y, r_z \right)$ it's position vector, 

\begin{enumerate}

\item{\underline{Mass flow rate}}: 

We compute the mass flow rate using the following equation:

\begin{equation}
\dot{m} = \frac{F}{g I_{sp}}
\end{equation}

Since we are using constant $I_{sp}$, thrust, and $g$ is constant, this is the same value throughout the tests. $\dot{m}$ is therefore either zero (when the thruster is not firing) or set to the previous value. 

\item{\underline{With one thruster}}: The forces are simply the projections of the thrust force on the axes of interest. The torque along $x$ is the arm along $z$ times the projection of the force along $y$, the torque along $z$ is the arm along $x$ times the projection of the force along $y$, the torque along $z$ is the arm along $x$ times the projection of the force along $y$. 

\begin{align}
F_x =F_{\mathrm{max}} \cos \alpha &\hspace{1cm} T_x = - F_{\mathrm{max}}\sin(\alpha) r_z \\ 
F_y = F_{\mathrm{max}} \sin \alpha &\hspace{1cm} T_y = F_{\mathrm{max}} \cos(\alpha) r \sin( \arctan(r_z/r_x)) \\ 
F_z = 0 &\hspace{1cm} T_z =  F_{\mathrm{max}} \sin(\alpha) r_x 
\end{align}


\item{\underline{With two thrusters}}: By giving indices $1$ and $2$ for each of the thrusters, we just need to add the Forces and torques defined above to get the total Forces and Torques:

\begin{align}
F_x = F_{x1} + F_{x2} &\hspace{1cm} T_x = T_{x1} + T_{x2} \\ 
F_y =  F_{y1} + F_{y2} &\hspace{1cm} T_y =  T_{y1} + T_{y2}\\ 
F_z =  F_{z1} + F_{z2} &\hspace{1cm} T_z =  T_{z1} + T_{z2}
\end{align}

\item{\underline{With ramps thruster}}: When the thrusters now ramp up and down, we create a normalized ramp function $\rho$. An example is given in \ref{fig:Ramp_function} in the case of a cutoff fire and renewed fire. \par

 \input{AutoTex/Ramp_function.tex}

We then prolong the force and torque end times as a function of the ramp slope, and multiply the initial functions by this ramping function:

\begin{align}
\tilde{F_x} = \rho F_{x} &\hspace{1cm} \tilde{T_x} =\rho T_{x}  \\ 
\tilde{F_y} =  \rho F_{y}  &\hspace{1cm} \tilde{T_y} =\rho  T_{y} \\ 
\tilde{F_z} = \rho F_{z} &\hspace{1cm} \tilde{T_z} =\rho  T_{z} 
\end{align}

\end{enumerate}