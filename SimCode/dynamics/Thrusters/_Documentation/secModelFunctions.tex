\section{Model Functions}

The Thrusters module contains methods allowing it to perform several tasks:

\begin{itemize}
	\item \textbf{Set thrusters}: Define and set several thrusters with different parameters, and locations. Parameters include: Minimum On-Time, $I_{sp}$, direction of thrust...
	\item \textbf{Ramp On/Off}: Define ramps that model the imperfect on time and off time for a thruster
	\item \textbf{Compute Forces and Torques}: Gets the forces and torques on the SpaceCraft given the previous definitions
	\item \textbf{Compute mass flow rate }: Computes the time derivative of the mass using $I_{sp}$ and Earth's gravity
\end{itemize}




\section{Model Assumptions and Limitations}

\subsection{Assumptions}

The Thruster module assumes that the thruster is thrusting exactly along it's position axis. Even if the position is dispersed, the thrust will be constant along that dispersed axis and will not vary.

The mass flow rate is considered to be constant as long as the thrusters are activated. Even in a ramp scenario, we assume that the mass is lost, without contributing to the thrust.

\subsection{Limitations}

The ramps in the thrusters modules are made by defining a set of elements to the ramp. They therefore form, by definition, a piecewise-linear ramp. If enough points are added, this will strongly ressemble a polynomial, but the ramps are in essence piecewise constant.

The $I_{sp}$ used for each thruster is considered constant throughout a simulation. This means there is no practical way of doing variable $I_{sp}$ simulations. It can nevertheless be done by stopping the simulation and modifying the values.

