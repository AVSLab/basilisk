\documentclass[]{BasiliskReportMemo}
\usepackage{AVS}


\newcommand{\submiterInstitute}{Autonomous Vehicle Simulation (AVS) Laboratory,\\ University of Colorado}

\newcommand{\ModuleName}{gravityEffector}
\newcommand{\subject}{Gravity Effector Test Report}
\newcommand{\status}{Initial document draft}
\newcommand{\preparer}{S. Carnahan}
\newcommand{\summary}{This unit test validates the internal aspects of the Basilisk spherical harmonics gravity effector module {\tt test\_gravityDynEffector.py} by comparing module output to expected output. The Basilisk gravity effector module is responsible for calculating the effects of gravity on a spacecraft orbiting a body. It utilizes spherical harmonics for calculation given the gravitation parameter for the body, a reference radius, and the maximum degree of spherical harmonics to be used. The unit test verifies basic set-up, single-body gravitational acceleration, and multi-body gravitational acceleration.}


\begin{document}


\makeCover


%
%	enter the revision documentation here
%	to add more lines, copy the table entry and the \hline, and paste after the current entry.
%
\pagestyle{empty}
{\renewcommand{\arraystretch}{1.1}
\noindent
\begin{longtable}{|p{0.5in}|p{4.5in}|p{1.14in}|}
\hline
{\bfseries Rev}: & {\bfseries Change Description} & {\bfseries By} \\
\hline
Draft & Initial document creation & S. Carnahan \\
\hline

\end{longtable}
}

\newpage
\setcounter{page}{1}
\pagestyle{fancy}

\tableofcontents
~\\ \hrule ~\\


\section{Introduction}
The Basilisk gravity effector module (gravityEffector.cpp) is responsible for calculating the effects of gravity on a spacecraft orbiting a body. This unit test contains three tests of increasing complexity that verify input/ouput, single-body calculations, and multi-body calculations.

\section{{\tt test\_gravityDynEffector} Test Description}

This test is located in {\tt SimCode/dynamics/gravityEffector/\_UnitTest/test\_gravityDynEffector.py}. In order to get good coverage of all the aspects of the module, the test is broken up into three sub-tests: \par

\begin{enumerate}
	\item \underline{Model Set-up Verification} This test verifies, via three checks, that the model is appropriately initialized when called. The first check verifies that the normalized coefficient matrix for the spherical harmonics calculations is initialized appropriately. The second check verifies that the magnitude of the gravity being calculated is reasonable. The final check ensures that the maximum degrees value is truly acting as a ceiling on the maximum number of degrees being used in the spherical harmonics algorithms.
	\item \underline{Single-Body Gravity Calculations} This test compares calculated gravity values around the Earth with ephemeris data from the Hubble telescope.
	\item \underline{Multi-Body Gravity Calculations} This test includes gravity effects from Earth, Mars, Jupiter, and the Sun. Calculations are verified by comparing against ephemeris data from the DAWN mission.
\end{enumerate} 

\section{Test Parameters}

This section summarizes the test input/output for each of the checks. 
\begin{itemize}
\item \underline{Error Tolerance}

There are specific error tolerances for each test. Error tolerances are determined based on whether the test results comparison should be exact or approximate due to integration or other reasons. Error tolerances for each test are summarized in table \ref{tab:errortol}. 

\begin{table}[htbp]
	\caption{Error tolerance for each test.}
	\label{tab:errortol}
	\centering \fontsize{10}{10}\selectfont
	\begin{tabular}{ c | c } % Column formatting, 
		\hline
		\textbf{Test}   							& \textbf{Tolerated Error} 						  \\ \hline
		Gravity Magnitude (In first test) & \input{AutoTex/sphericalHarmonicsAccuracy}		   \\ \hline
		Single-Body Gravity						& \input{AutoTex/singleBodyAccuracy}														   \\ \hline
		Multi-Body Gravity 						& \input{AutoTex/multiBodyAccuracy}	 		       \\ \hline
	\end{tabular}
\end{table}
\end{itemize}

\section{Test Results}

All checks within test\_gravityDynEffector.py passed as expected. Table \ref{tab:results} shows the test results.

\begin{table}[htbp]
	\caption{Test results.}
	\label{tab:results}
	\centering \fontsize{10}{10}\selectfont
	\begin{tabular}{c | c | c  } % Column formatting, 
		\hline
		\textbf{Test} 				    & \textbf{Pass/Fail} 						   			           & \textbf{Notes} 									\\ \hline
		Setup Test 		   			  	& \input{AutoTex/sphericalHarmonicsPassFail}     & \input{AutoTex/sphericalHarmonicsFailMsg}			 \\ \hline
		Single-Body Gravity		   	& \input{AutoTex/singleBodyPassFail}                 & \input{AutoTex/singleBodyFailMsg} \\ \hline
		Multi-Body Gravity			 &\input{AutoTex/multiBodyPassFail}  			 	 &  \input{AutoTex/multiBodyFailMsg} 			   \\ \hline
	\end{tabular}
\end{table}


\end{document}
