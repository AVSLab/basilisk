\section{Model Assumptions and Limitations}
The limitations of spherical harmonics gravity model are well-known and clearly explained in Schaub and Junkins' book\cite{schaub2014}. The limitations include:
\begin{itemize}
	\item \textbf{Coefficient Accuracy}: The coefficients used in the spherical harmonics equations are typically calculated based on gravitational data gathered by satellites. Therefore, the accuracy of the model is determined by the accuracy of the satellite instrumentation and precision of the stored data. Furthermore, for some bodies, there may not be sufficient information available to provide accurate coefficients or higher-degree coefficients.
	\item \textbf{Maximum Degree}: The spherical harmonics equation is a series expansion. Therefore, any implementation must truncate the equation at some point. The truncated portion of the equation necessarily defines some amount of error in the final calculation. This error is, however, small after the first handful of terms. Additionally, a larger distance between gravity body and spacecraft requires fewer terms of the series to achieve equal accuracy as compared to a case with less distance. This code allows the user to request a maximum number of terms to evaluate rather than a specific accuracy. This could lead to less-than-desirable accuracy with small separation distances and greater-than-necessary run times with large separation distances.
	\item \textbf{Planetary Ephemeris Data}: This code generally relies on an external package for planetary ephemeris information. Errors included in this package will translate into error in the gravity calculations, but those errors should be small. Because the ephemeris data is tabulated, this code should not be used to try to project the orbits of the celestial bodies in question. This could be done, though, by treating any celestial body as a "spacecraft".
\end{itemize}