\section{Model Functions}
The mathematical description of gravity effects are implemented in gravityEffector.cpp. This code performs the following primary functions
\begin{itemize}
	\item \textbf{GravBody Creation}: The code creates gravity bodies which are capable of affecting spacecraft. It does not effect a spacecraft unless that spacecraft explicitly adds the body as a gravity effector.
	\item \textbf{Orbital Energy}: The code can calculate the total orbital energy as well as orbital kinetic and orbital potential energy of a spacecraft about a gravity body.
	\item \textbf{Simple Gravity}: The code can compute a gravity acceleration between two bodies according to Newton's law of universal gravitation given $\mu$ and the distance between the bodies.
	\item \textbf{Spherical Harmonics}: The code can compute gravity acceleration between two bodies using the more-complex method of spherical harmonics. To do this, it must be provided with the same inputs as for calculating simple gravity. In addition, it needs to be provided a "degree" of spherical harmonics to be used and spherical harmonics coefficients useful up to that degree.
	\item \textbf{Multiple Body Effects}: The code can stack the effects of multiple gravity bodies on top of each other to determine the net effect on a spacecraft. The user must indicate in the spacecraft set-up which gravitational bodies should be taken into account.
	\item \textbf{Interface: Spacecraft States}: The code sends and receives spacecraft state information via the DynParamManager.
	\item \textbf{Interface: Energy Contributions}: The code sends spacecraft energy contributions via \\updateEnergyContributions() which is called by the spacecraft.
	\item \textbf{Interface: GravBody States}: The code outputs GravBody states(ephemeris information) via the Basilisk messaging system.
	
\end{itemize}