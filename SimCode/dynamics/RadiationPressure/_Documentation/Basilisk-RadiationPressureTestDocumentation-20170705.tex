\documentclass[]{BasiliskReportMemo}
\usepackage{AVS}
\usepackage{float}
\usepackage{array}

\newcommand{\submiterInstitute}{Autonomous Vehicle Simulation (AVS) Laboratory,\\ University of Colorado}

\newcommand{\ModuleName}{RadiationPressure}
\newcommand{\subject}{Radiation Pressure Model}
\newcommand{\status}{Initial document draft}
\newcommand{\preparer}{S. Carnahan}
\newcommand{\summary}{The radiation pressure module is responsible for calculating the dyamic effects of radiation pressure on a spacecraft. Effects can be calculated either by use of a simplified "cannonball" method or table look-up. A unit test has been written which checks both calculation methods against expected output values.}
\newcommand{\testname}{test\_radiationPressure.py }


\begin{document}


\makeCover


%
%	enter the revision documentation here
%	to add more lines, copy the table entry and the \hline, and paste after the current entry.
%
\pagestyle{empty}
{\renewcommand{\arraystretch}{1.1}
\noindent
\begin{longtable}{|p{0.5in}|p{4.5in}|p{1.14in}|}
\hline
{\bfseries Rev}: & {\bfseries Change Description} & {\bfseries By} \\
\hline
Draft & Initial document creation & \preparer\\
\hline

\end{longtable}
}

\newpage
\setcounter{page}{1}
\pagestyle{fancy}

\tableofcontents
~\\ \hrule ~\\


\section{Introduction}
The Basilisk radiation pressure module (radiation\_pressure.cpp) is responsible for calculating the effects of radiation pressure on a spacecraft. The mathematical models used are described below. The code unit tests are then presented and discussed.

\section{Mathematical Model}
Maths are modeled.
\subsection{Radiation Model}
Radiation is modeled by...
\subsection{Radiation Pressure Model}
Radiation Pressure is modeled by...
\subsection{Recursion Formulas}
The above formulas for the radiation pressure are broken down into iterative steps in order to be modeled in the simulation. The recursive formulas follow.
\subsection{Variable Definitions}
The variables in Table \ref{tabular:vars} are available for user input. Variables used by the module but not available to the user are not mentioned here. Variables with default settings do not necessarily need to be changed by the user, but may be.
	\begin{table}[H]
		\caption{Definition and Explanation of Variables Used.}
		\label{tab:errortol}
		\centering \fontsize{10}{10}\selectfont
		\begin{tabular}{ | m{3cm}| m{3cm} | m{3cm} | m{6cm} |} % Column formatting, 
			\hline
			\textbf{Variable}   		& \textbf{LaTeX Equivalent} 	&		\textbf{Variable Type} & \textbf{Notes} \\ \hline
			stateInMsgName			&N/A		   							    & string 								& Default setting: "inertial\_state\_output". This is the message from which the radiation pressure module receives spacecraft inertial data.\\ \hline
			sunEphmInMsgName	& N/A 										& string 								& Default setting: "sun\_planet\_data". This is the message through which radiation pressure gets information about the sun and planets.\\ \hline
			area 						  	  & N/A yet 							    & double 								& [m2] Default setting: 0.0f. Required input for cannonball method to get any real output. This is the area to use when approximating the surface area of the spacecraft.\\ \hline
			coefficientReflection 	  & N/A yet 								& double 								& Default setting: 1.2. This is a factor applied to the radiation pressure based on spacecraft surface properties.\\ \hline
			useCannonballModel	   & N/A 									   & bool 									& Default setting: True. To switch cannonball mode off, use the call setUseCanonballModel("False")\\ \hline
		\end{tabular}
	\label{tabular:vars}
	\end{table}
\section{Library}
This section means nothing to me.

\section{Unit Test}

This test is located at {\tt SimCode/dynamics/RadiationPressure/\_UnitTest/\testname} In order to get good coverage of all the aspects of the module, the test is broken up into two sub-tests. In each sub-test, a spacecraft is placed in the solar system and acted upon by the Sun. \par

\subsection{"Cannonball" Method} This test utilizes the "cannonball" method to calculate the effects of radiation pressure on spacecraft dynamics. The cannonball method approximates the spacecraft as a sphere. External forces in the inertial and body frame, as well as external torques in the body frame, are checked against known values.
\subsection{Table Look-up Method} This test uses a stored table of known effects of radiation pressure. It looks up values and compares them to the expected result to validate radiation pressure table look-up capabilities.


\subsection{Test Parameters}

This section summarizes the test input/output for each of the checks. 
\begin{itemize}
\item \underline{Error Tolerance}

There are specific error tolerances for each test. Error tolerances are determined based on whether the test results comparison should be exact or approximate due to integration or other reasons. Error tolerances for each test are summarized in table \ref{tab:errortol}. 

\begin{table}[htbp]
	\caption{Error tolerance for each test.}
	\label{tab:errortol}
	\centering \fontsize{10}{10}\selectfont
	\begin{tabular}{ c | c } % Column formatting, 
		\hline
		\textbf{Test}   							& \textbf{Tolerated Error} 						  \\ \hline
		"Cannonball" &\input{AutoTex/cannonballAccuracy}		   \\ \hline
		Look-up					& \input{AutoTex/lookupAccuracy}														   \\ \hline
	\end{tabular}
\end{table}
\end{itemize}

\subsection{Test Results}

All checks within \testname passed as expected. Table \ref{tab:results} shows the test results.

\begin{table}[htbp]
	\caption{Test results.}
	\label{tab:results}
	\centering \fontsize{10}{10}\selectfont
	\begin{tabular}{c | c | c  } % Column formatting, 
		\hline
		\textbf{Test} 				      & \textbf{Pass/Fail} 						   		   & \textbf{Notes} 									\\ \hline
		"Cannonball"	   			  	&\input{AutoTex/cannonballPassFail}      &\input{AutoTex/cannonballFailMsg} 	 \\ \hline
		Look-up	   	                     &\input{AutoTex/lookupPassFail}              &\input{AutoTex/lookupFailMsg}            \\ \hline
	\end{tabular}
\end{table}

\subsection{Test Coverage}
The test covers 2000\% of the code.
\section{Conclusion}
What a great piece of code.

\end{document}
