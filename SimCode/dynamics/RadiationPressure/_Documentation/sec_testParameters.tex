\section{Test Parameters}

This section summarizes the error tolerances for each test. Error tolerances are determined based on whether the test results comparison should be exact or approximate due to integration or other reasons. Error tolerances for each test are summarized in table \ref{tab:errortol}. 

\begin{table}[htbp]
	\caption{Error tolerance for each test.}
	\label{tab:errortol}
	\centering \fontsize{10}{10}\selectfont
	\begin{tabular}{ c | c } % Column formatting, 
		\hline
		\textbf{Test}   	      	               & \textbf{Tolerated Error} 						           \\ \hline
		``Cannonball"                           &\input{AutoTex/cannonballAccuracy}	 			  \\ \hline
		Look-up						                & \input{AutoTex/lookupAccuracy}		   				\\ \hline
		Look-up With Eclipse	             & \input{AutoTex/lookupWithEclipseAccuracy}    \\ \hline
		Look-up	(torque)			               & \input{AutoTex/lookupTorqueAccuracy}		   				\\ \hline
	    Look-up With Eclipse (torque)	& \input{AutoTex/lookupWithEclipseTorqueAccuracy}    \\ \hline
	\end{tabular}
\end{table}

\noindent Note that the lookup model tests utilize more stringent tolerances for torques. This is because the torque values are too small to use the same tolerances as the force calculations.