\section{Model Functions}
The mathematical description of gravity effects are implemented in gravityEffector.cpp. This code performs the following primary functions
\begin{itemize}
	\item \textbf{Cannonball Method}: The code calculates the force on a spacecraft due to solar radiation pressure.
	\item \textbf{Look-up Method}: The code calculates both the force and torque on a spacecraft due to solar radiation pressure. It uses user-provided tabulated data to do so.
	\item \textbf{Solar Eclipse}: The code takes solar eclipses into account via a "shadow factor". This shadow factor is output from the Basilisk solar eclipse module and can include the effects of multiple planets. It is applied to the force/torque outputs.
	\item \textbf{Interface: Spacecraft States}: The code receives spacecraft state information via the DynParamManager.
	\item \textbf{Interface: Forces and Torques}: The code sends spacecraft force and torque contributions via computeBodyForceTorque() which is called by the spacecraft.	
	%\item \textbf{Interface: Energy Contributions}: The code sends spacecraft energy contributions via updateEnergyContributions() which is called by the spacecraft.
	\item \textbf{Interface: Sun Ephemeris}: The code receives Sun states (ephemeris information) via the Basilisk messaging system.
	\item \textbf{Interface: Solar Eclipse}: The code receives solar eclipse (shadow factor) information via the Basilisk messaging system.
	
\end{itemize}



\section{Model Assumptions and Limitations}
The two methods of calculation used in this code have their own sets of assumptions and limitations. There are some assumptions which are common to both methods.
\begin{itemize}
	\item \textbf{Cannonball Model}: This model assumes that the radiation pressure will act normal to some equivalent surface area, $A_{\odot}$. While this could be a good assumption, $A_{\odot}$ would have to be time-varying with spacecraft attitude and incorporate spacecraft self-shadowing. In general, the code does not do this. This limits the cannonball method to being most accurate in relatively mundane simulations (no rapid rotations or varying self-shadowing). Additionally, this method does not calculate torques on the spacecraft, so it is limited to cases where high precision is not needed with regards to spacecraft attitude.
	\item \textbf{Look-up Method}: This method utilizes tabulated data. Therefore, there will be error associated with whatever method was used to generate and record the data, but those errors are outside of Basilisk. Furthermore, the algorithm selects the data which \textit{most closely} matches the current position of the spacecraft relative to the sun and does not interpolate between data points.  This method also is limited to users who have external models or real data to use to describe their spacecraft.
	\item \textbf{Radiation}: The radiation model is hard-coded to assume that the radiation comes from the Sun. It is not possible to model radiation pressure from other sources with this code. This applies to both the cannonball and look-up methods. The model has no time-varying radiation effects (solar storms, etc.). A more in-depth radiation model would be need if high-accuracy radiation pressure effect calculations are needed.
	\item \textbf{Eclipse}: The shadow factor applies a simple scaling factor to the output forces and torques. This assumes that all portions of the spacecraft are affected equally by the eclipse. This should, in most circumstances, be highly accurate. For exceptionally large $A_{\odot}$ spacecraft which also need highly accurate state calculations, this assumption could fail.
	\item \textbf{Tabulated Data Import} Currently, Basilisk includes a utility script to import data from XML files for use in radiation pressure calculations. While some users could learn to load data in other formats, this is currently a limitation to most users who have data in other forms.
\end{itemize}