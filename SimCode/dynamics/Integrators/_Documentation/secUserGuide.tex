% !TEX root = ./Basilisk-Integrators20170724.tex

\section{User Guide}

\subsection{Not Specifying an Integration Method}
If a Python BSK simulation is setup without specifying an integration method, then the default behavior is to load the fixed time step 4th order Runge-Kutta (RK4) method.  No additional code is required by the user.


\subsection{Selecting Alternate Integration Methods}
Assume the {\tt DynamicObject} class is the {\tt SpacecraftPlus()} object, declared through
\begin{verbatim}
	scObject = spacecraftPlus.SpacecraftPlus()
\end{verbatim}
To invoke the Euler's  integration scheme, the corresponding integration module is created using
\begin{verbatim}
	integratorObject = svIntegrators.svIntegratorEuler(scObject)
\end{verbatim}
If the 2nd order Heun's integration method is desired, use instead
\begin{verbatim}
	integratorObject = svIntegrators.svIntegratorRK2(scObject)
\end{verbatim}
To force the default RK4 method, use
\begin{verbatim}
	integratorObject = svIntegrators.svIntegratorRK4(scObject)
\end{verbatim}

Next, to connect this integrator module to the {\tt DynamicObject} instance (i.e. {\tt SpacecraftPlus()} called {\tt scObject}, use the following code
\begin{verbatim}
	scObject.setIntegrator(integratorObject)
\end{verbatim}
That is it, the Basilisk simulation is now setup to use the desired numerical  integration method.


\subsection{Creating New Integration Methods}
New integration modules can readily be created for Basilisk. They are all stored in the folder
\begin{verbatim}
Basilisk/SimCode/dynamics/Integrators/
\end{verbatim}
The integrators must be created to function on a general state vector and be independent of the particular dynamics being integrated. Note that the default integrator is placed inside the {\tt\_GeneralModulesFiles} folder within the dynamics folder.