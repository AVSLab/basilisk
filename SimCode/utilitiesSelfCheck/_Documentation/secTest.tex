% !TEX root = ./Basilisk-avsLibrary20170812.tex

\section{Test Description and Success Criteria}
The AVS support libraries undergo a series of unit test within C-functions that are called by the Swig'd interface {\tt avsLibrarySelfCheck}.  The automated unit test is performed via {\tt pytest} through the script:
\begin{verbatim}
     SimCode/utilitiesSelfCheck/\_UnitTest/test\_avsLibrarySelfCheck.py
\end{verbatim}
In each unit test the C-code library function output is compared to a externally hand-computed solution.  The success criteria is when the pre-computed solution and the C-code computed solution match to within a certain numerical tolerance.  The success criteria absolute value tolerances are shown in Table~\ref{tbl:tolerance}.



\begin{table}[htbp]
	\caption{Unit Test Absolute Difference Tolerances}
	\label{tbl:tolerance}
	\centering \fontsize{10}{10}\selectfont
	\begin{tabular}{c | c} % Column formatting, 
		\hline 
		\hline 
		Test    & Absolute Tolerance \\
		\hline 
		  Linear Algebra    & $10^{-10}$ \\
		  Rigid Body Kinematics    & $10^{-10}$ \\
		  Orbital Motion: Anomalies    & $10^{-10}$ \\
		  Orbital Motion: Element conversion    & $10^{-10}$ \\
		  Orbital Motion: Environment    & $10^{-10}$ \\
		\hline
		\hline
	\end{tabular}
\end{table}





\section{Test Parameters}

\subsection{Linear Algebra}
In these unit tests each function is provided non-trivial inputs, such as {\tt v1 = [1,2,3]} to evaluate the associate matrix math.  For specific information on the values used, look at the source code for the function {\tt testLinearAlgebra()} in the file {\tt avsLibrarySelfCheck.c}.  

\subsection{Rigid Body Kinematics}
The attitude kinematics sub-routines are provided with non-trivial inputs to test the functions.   For specific information on the values used, look at the source code for the function {\tt testRigidBodyKinematics()} in the file {\tt avsLibrarySelfCheck.c}.  


\subsection{Orbital Motion}
The orbital anomaly sub-routines are provided with non-trivial inputs to test the functions.   For specific information on the values used, look at the source code for the function {\tt testOrbitalAnomalies()} in the file {\tt avsLibrarySelfCheck.c}.  

Regarding the orbit element to Cartesian coordinate conversion routines, an Earth orbit is simulated that is either non-equatorial and non-circular (case 1), non-circular equatorial (case 2), circular and inclined (case 3) and circular equatorial (case 4).  The resulting orbit elements used are shown in Table~\ref{tbl:orbitValues}.  


\begin{table}[htbp]
	\caption{Orbit Elements Used in Orbit Coordinate Conversion Routine Checking}
	\label{tbl:orbitValues}
	\centering \fontsize{10}{10}\selectfont
	\begin{tabular}{c || c | c | c | c | c | c} % Column formatting, 
		\hline 
		\hline 
		Case    & SMA  [km]& $e$ & $i$[\dg] & $\Omega$[\dg\ & $\omega$[\dg\ & $f$[\dg] \\
		\hline 
		  1   &  7500 & 0.5 & 40 & 133 & 113 & 123 \\
		  2   &  7500 & 0.5 & 0 & 133 & 113 & 123 \\
		  3   &  7500 & 0.0 & 40 & 133 & 113 & 123 \\
		  4   &  7500 & 0.0 & 0 & 133 & 113 & 123 \\
		\hline
		\hline
	\end{tabular}
\end{table}





\section{Test Results}
An automated suite of tests are run to perform unit tests on all the AVS support library functions.  The results are shown in Table~\ref{tbl:avsLibraryResults}.  


\begin{table}[h]
	\caption{Integration test results.}
	\label{tbl:avsLibraryResults}
	\centering \fontsize{10}{10}\selectfont
	\begin{tabular}{c | c | l } % Column formatting, 
		\hline\hline
		\textbf{Test} 			& \textbf{Pass/Fail} 	 & \textbf{BSK Error Notes} 									        
		\\ \hline
		{\tt rigidBodyKinematics}		  	& 
		\input{AutoTex/avsLibrarySelfCheckTestMsg-TrueFalseFalseFalseFalse}      	  &
		\input{AutoTex/avsLibrarySelfCheckMsg-TrueFalseFalseFalseFalse}
		\\ 
		{\tt orbitalElements}		  	& 
		\input{AutoTex/avsLibrarySelfCheckTestMsg-FalseTrueFalseFalseFalse}      	  &
		\input{AutoTex/avsLibrarySelfCheckMsg-FalseTrueFalseFalseFalse}
		\\ 
		{\tt orbitalAnomalies}		  	& 
		\input{AutoTex/avsLibrarySelfCheckTestMsg-FalseFalseTrueFalseFalse}      	  &
		\input{AutoTex/avsLibrarySelfCheckMsg-FalseFalseTrueFalseFalse}
		\\ 
		{\tt linearAlgebra}		  	& 
		\input{AutoTex/avsLibrarySelfCheckTestMsg-FalseFalseFalseTrueFalse}      	  &
		\input{AutoTex/avsLibrarySelfCheckMsg-FalseFalseFalseTrueFalse}
		\\ 
		{\tt environment}		  	& 
		\input{AutoTex/avsLibrarySelfCheckTestMsg-FalseFalseFalseFalseTrue}      	  &
		\input{AutoTex/avsLibrarySelfCheckMsg-FalseFalseFalseFalseTrue}
		\\ 
		\hline\hline
	\end{tabular}
\end{table}

