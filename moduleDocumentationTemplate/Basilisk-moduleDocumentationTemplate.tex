\documentclass[]{BasiliskReportMemo}

\usepackage{cite}
\usepackage{AVS}
\usepackage{float} %use [H] to keep tables where you put them
\usepackage{array} %easy control of text in tables
\bibliographystyle{plain}


\newcommand{\submiterInstitute}{Autonomous Vehicle Simulation (AVS) Laboratory,\\ University of Colorado}


\newcommand{\ModuleName}{SphericalHarmonics}
\newcommand{\subject}{Spherical Harmonics C++ model}
\newcommand{\status}{Tested}
\newcommand{\preparer}{S. Carnahan}
\newcommand{\summary}{The gravity effector module is responsible for calculating the effects of gravity from a body on a spacecraft. A spherical harmonics model and implementation is developed and described. A unit test has been written and run which test basic input/output, single-body gravitational acceleration, and multi-body gravitational acceleration.}
\newcommand{\testname}{test\_gravityDynEffector.py } %include packages, styles, and new commands in this file.

\begin{document}

\makeCover

%
%	enter the revision documentation here
%	to add more lines, copy the table entry and the \hline, and paste after the current entry.
%
\pagestyle{empty}
{\renewcommand{\arraystretch}{2}
\noindent
\begin{longtable}{|p{0.5in}|p{4.5in}|p{1.14in}|}
\hline
{\bfseries Rev}: & {\bfseries Change Description} & {\bfseries By} \\
\hline
1.0 & First version - Mathematical formulation and implementation & M. Diaz Ramos \\
\hline
1.1 & Added test documentation & \preparer \\
\hline

\end{longtable}
} %when you update the file, update the revision table.

\newpage
\setcounter{page}{1}
\pagestyle{fancy}

\tableofcontents %Autogenerate the table of contents
~\\ \hrule ~\\ %Makes the line under table of contents
	
\section{Model Description}

Describe the module including mathematics, implementation, etc.

\subsection{Mathematical model}
There can be subsections, like this one.

\subsubsection{Gravity models}
Even subsubsections.  %This section includes mathematical models, code description, etc.
\clearpage

\section{Model Functions}
This section will contain a bullet-list and descriptions of what functions this module performs. For example:
\begin{itemize}
	\item \textbf{computeField()}: The gravity effector module computes the gravity field from a celestial body including spherical harmonics.
	\item \textbf{computeField()}: The gravity effector module computes the gravity field from a celestial body including spherical harmonics.
\end{itemize} %This includes a concise list of what the module does.
\clearpage

\section{Model Assumptions and Limitations}
This section should describe the assumptions used in formulating the mathematical model and how those assumptions limit the usefulness of the module. %This explains the assumptions made to reach the final mathematical implementation of the model and how those assumptions limit the model's usefulness.
\clearpage

\section{Test Description and Success Criteria}
The unit test, \testname,  validates the internal aspects of the Basilisk spherical harmonics gravity effector module by comparing module output to expected output. It utilizes spherical harmonics for calulcation given the gravitation parameter for the massive body, a reference radius, and the maximum degree of spherical harmonics to be used. The unit test verifies basic set-up, single-body gravitational acceleration, and multi-body gravitational acceleration.

\subsection{Model Set-Up Verification}
This test verifies, via three checks, that the model is appropriately initialized when called.
		\begin{itemize}
			\item \underline{1.1} The first check verifies that the normalized coefficient matrix for the spherical harmonics calculations is initialized appropriately as a $3\times3$ identity matrix. \\
			\item \underline{1.2} The second check verifies that the magnitude of the gravity being calculated is reasonable(i.e. between $9.7$ and $9.9$ $m/s^2$). \\
			\item \underline{1.3} The final check ensures that the maximum degrees value is truly acting as a ceiling on the maximum number of degrees being used in the spherical harmonics algorithms. For example, if the maximum degrees value is set to 20, an attempt is made to call the spherical harmonics algorithms with a degrees value of 100 and another attempt is made with a degrees value of 20. The results are compared and should be equal due to the enforcement of the maximum degrees value.\\
		\end{itemize}
\subsection{Single-Body Gravity Calculations} This test compares calculated gravity values around the Earth with ephemeris data from the Hubble telescope. The simulation begins shortly after 0200 UTC May 1, 2012 and carries on for two hours, evaluating the gravitational acceleration at two  second intervals.
\subsection{Multi-Body Gravity Calculations} This test includes gravity effects from Earth, Mars, Jupiter, and the Sun. Calculations are verified by comparing against ephemeris data from the DAWN mission. The simulation begins at midnight, October 24th, 2008 and carries on for just under 22 minutes, evaluating the gravitational acceleration at two second intervals. %This explains the unit test for the model. I.e. what features are tested and how. It may also include test tolerances, etc.
\clearpage

\section{Test Parameters}

Test parameters and inputs go here. I think that success criteria would work better here than in the test description section.
\clearpage

\subsection{Test Results}

All checks within test\_gravityDynEffector.py passed as expected. Table \ref{tab:results} shows the test results.\\

\begin{table}[htbp]
	\caption{Test results.}
	\label{tab:results}
	\centering \fontsize{10}{10}\selectfont
	\begin{tabular}{c | c | c  } % Column formatting, 
		\hline
		\textbf{Test} 				    & \textbf{Pass/Fail} 						   			           & \textbf{Notes} 									\\ \hline
%		Setup Test 		   			  	& \input{AutoTex/sphericalHarmonicsPassFail}     & \input{AutoTex/sphericalHarmonicsFailMsg}			 \\ \hline
%		Single-Body Gravity		   	& \input{AutoTex/singleBodyPassFail}                 & \input{AutoTex/singleBodyFailMsg} \\ \hline
%		Multi-Body Gravity			 &\input{AutoTex/multiBodyPassFail}  			 	 &  \input{AutoTex/multiBodyFailMsg} 			   \\ \hline
	\end{tabular}
\end{table} %This displays the test results. This includes both pass/fail statements as well as visual outputs via AutoTeX
\clearpage

\section{User Guide}
This section contains information directed specifically to users. It contains clear descriptions of what inputs are needed and what effect they have. It should also help the user be able to use the model for the first time.
 %This section is to provide advice to users on necessary/useful inputs and best practices.
\clearpage

\bibliography{bibliography.bib} %This includes references used and mentioned.

\end{document}
