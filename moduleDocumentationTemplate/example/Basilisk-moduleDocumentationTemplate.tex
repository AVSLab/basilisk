\documentclass[]{BasiliskReportMemo}

\usepackage{cite}
\usepackage{AVS}
\usepackage{float} %use [H] to keep tables where you put them
\usepackage{array} %easy control of text in tables
\usepackage{graphicx}
\usepackage{hyperref}
\bibliographystyle{plain}


\newcommand{\submiterInstitute}{Autonomous Vehicle Simulation (AVS) Laboratory,\\ University of Colorado}


\newcommand{\ModuleName}{spice\_interface}
\newcommand{\subject}{Spice data interface module}
\newcommand{\status}{Initial Document}
\newcommand{\preparer}{T. Teil}
\newcommand{\summary}{This unit test compares the results of the Spice data within the AVS Basilisk simulation with outside data. Spice generates information on universal time (UTC/GPS...) as well as ephemeris information for the bodies of the Solar System. The time information creates accurate time tags to be used by others on the project, and the planet ephemeris gives the location of the bodies of interest, and therefore their gravitational effects on the spacecraft. In this test, we compare the values given by Spice values to the actual expected values in order to validate the code.} %include packages, styles, and new commands in this file.

\begin{document}

\makeCover

%
%	enter the revision documentation here
%	to add more lines, copy the table entry and the \hline, and paste after the current entry.
%
\pagestyle{empty}
{\renewcommand{\arraystretch}{2}
\noindent
\begin{longtable}{|p{0.5in}|p{3.5in}|p{1.07in}|p{0.9in}|}
\hline
{\bfseries Rev} & {\bfseries Change Description} & {\bfseries By}& {\bfseries Date} \\
\hline
1.0 & Revision Description & F. Last1 & YYYYMMDD\\
\hline
1.1 & Revision Description& F. Last2 & YYYYMMDD\\
\hline

\end{longtable}
} %when you update the file, update the revision table.

\newpage
\setcounter{page}{1}
\pagestyle{fancy}

\tableofcontents %Autogenerate the table of contents
~\\ \hrule ~\\ %Makes the line under table of contents
	
\section{Model Description}
The radiation pressure module contains two different models for calculating the effects of solar radiation pressure on spacecraft state. Both methods of calculating solar radiation pressure have simple implementations in Basilisk using basic coefficients and assumptions. The methods of arriving at these coefficients can be complex and making the coefficients time-varying to improve accuracy can greatly increase complexity. The cannonball method used here essentially follows the mathematics described by Vallado\cite{vallado2001}.
\subsection{Radiation Model}
Radiation is modeled by using the solar flux at one astronomical unit and scaling by distance from the sun relative to 1 AU. The solar flux at one AU is taken as :
\begin{equation}
	SF_{\mathrm{AU}} = 1372.5398    [W/m^2]
\end{equation}
\subsubsection{Solar Eclipses}
Solar eclipses are are detected by the basilisk eclipse module. The effects of the eclipse are calculated into a shadow factor, $F_{\mathrm{s}}$, which is applied to the output forces and torques. 
\begin{equation}
	\mathbf{F}_{\mathrm{out}} = F_{\mathrm{s}}\mathbf{F}_{\mathrm{full\_sun}}
\end{equation}
\begin{equation}
\bm{\tau}_{\mathrm{out}} = F_{\mathrm{s}}\bm{\tau}_{\mathrm{full\_sun}}
\end{equation}
Then, calculations are carried on as seen below.

\subsection{Radiation Pressure Model}
\subsubsection{Cannonball Method}
The radiation pressure at 1AU, $p_{SR}$, can be taken as the solar flux divided by the speed of light. 
\begin{equation}
	p_{SR} = \frac{SF_{\mathrm{AU}}}{c}
\end{equation}
Then, a ``scaling factor'' can be determined. This ``scaling factor" is equivalent to the magnitude of the solar radiation force divided by the distance between the spacecraft and the sun:
\begin{equation}
	\frac{|\mathbf{F}_{\textrm{radiation}}|}{|\mathbf{r}_{\textrm{sun}}|} = \frac{-c_{R}p_{SR}A_{\odot}{AU}^2}{c |\mathbf{r}_{\textrm{sun}}|^3}
\end{equation}
This factor is then multiplied by the position vector from the spacecraft to the sun to get the force on the spacecraft due to solar radiation pressure.
\begin{equation}
	{\mathbf{F}_{\textrm{radiation}}} = \frac{|\mathbf{F}_{\textrm{radiation}}|}{|\mathbf{r}_{\textrm{sun}}|}  \mathbf{r}_{\textrm{sun}}
\end{equation}
The user must provide the coefficient of reflection and the equivalent area of the spacecraft to use this method.\\
\subsubsection{Table Look-up Method}
For the table look-up method, pre-determined values of torque and force acting on the spacecraft due to radiation pressure are given. It is required that these values be given at 1AU from the sun and with a corresponding direction vector form the spacecraft to the sun.\\\\
The look-up works by finding the direction vector in the given tables which most closely matches the spacecraft's current position vector. Then, the corresponding force and torque values are taken from the table and scaled according to the magnitude of the spacecraft-sun position vector.\\\\
Most important to the user of the table look-up method is the required input and format of data. Data must be recorded in XML format. As an example, see ../cube\_lookup.xml (in the radiation pressure folder). Additionally, a utility script called parseSRPLookup.py is provided there to read the XML input into numpy arrays. Experienced users are welcome to store their data in their own format and load it into equivalent numpy arrays as they see fit.\\\\
An example of using the provided python script to load data is shown in test\_radiationPressure.py. Note that this also requires import of the unitTestSupport library.\\ %This section includes mathematical models, code description, etc.


\section{Model Functions}
The mathematical description of gravity effects are implemented in gravityEffector.cpp. This code performs the following primary functions
\begin{itemize}
	\item \textbf{Cannonball Method}: The code calculates the force on a spacecraft due to solar radiation pressure.
	\item \textbf{Look-up Method}: The code calculates both the force and torque on a spacecraft due to solar radiation pressure. It uses user-provided tabulated data to do so.
	\item \textbf{Solar Eclipse}: The code takes solar eclipses into account via a "shadow factor". This shadow factor is output from the Basilisk solar eclipse module and can include the effects of multiple planets. It is applied to the force/torque outputs.
	\item \textbf{Interface: Spacecraft States}: The code receives spacecraft state information via the DynParamManager.
	\item \textbf{Interface: Forces and Torques}: The code sends spacecraft force and torque contributions via computeBodyForceTorque() which is called by the spacecraft.	
	%\item \textbf{Interface: Energy Contributions}: The code sends spacecraft energy contributions via updateEnergyContributions() which is called by the spacecraft.
	\item \textbf{Interface: Sun Ephemeris}: The code receives Sun states (ephemeris information) via the Basilisk messaging system.
	\item \textbf{Interface: Solar Eclipse}: The code receives solar eclipse (shadow factor) information via the Basilisk messaging system.
	
\end{itemize} %This includes a concise list of what the module does.


\section{Model Assumptions and Limitations}

\subsection{Assumptions}

The Thruster module assumes that the thruster is thrusting exactly along it's position axis. Even if the position is dispersed, the thrust will be constant along that dispersed axis and will not vary.

\subsection{Limitations}

The ramps in the thrusters modules are made by defining a set of elements to the ramp. They therefore form, by definition, a piecewise-linear ramp. If enough points are added, this will strongly ressemble a polynomial, but the ramps are in essence piecewise constant. %This explains the assumptions made to reach the final mathematical implementation of the model and how those assumptions limit the model's usefulness.


\section{Test Description and Success Criteria}

The unit test for the thruster\_dynamics module is located in:\\

\noindent
{\tt SimCode/dynamics/Thrusters/$\_$UnitTest/unit$\_$ThrusterDynamicsUnit.py} \\

\subsection{Test inputs}

The thruster inputs that were used are:

\begin{itemize}
\item The specific impulse: $I_{sp} = 266.7$s 

A thrusters potential to deliver force per mass flow rate. 
\item The maximum thrust: $F_{\mathrm{max}} = 1.0$ N

The scaling factor yielding the thrust.
\item The minimum On time: $t_{\mathrm{min}} = 0.006$s

The minimum time that the thruster can be fired.
\end{itemize}

\subsection{Test sections}

\noindent This unit test is designed to functionally test the simulation model 
outputs as well as get complete code path coverage.  The test design is broken 
up into several parts:\\
\begin{enumerate}
\item{\underline{Instantaneous On/Off Factor:} The thrusters are fired with an 
  instantaneous ramp to ensure that the firing is correct. This gives us a base case.}
\item{\underline{Short Instantaneous Firing:} A "short" firing that still respects the 
  rules of thumb above is fired to ensure that it is still correct.}
 \item{\underline{Short Instantaneous Firing with faster test rate:} A "short" firing that still respects the 
  rules of thumb above but with a faster test rate to see the jump.}
 \item{\underline{Instantaneous On/Off Factor with faster test rate:} The thrusters are fired with an 
  instantaneous ramp to ensure that the firing is correct given a different test rate. This shouldn't modify the physics.}
 \item{\underline{Thruster Angle Test:} The output forces/torques from the simulation 
  are checked with a thruster pointing in a different direction.}
   \item{\underline{Thruster Position Test:} The output forces/torques from the simulation 
  are checked with a thruster in a different position.}
   \item{\underline{Thruster Number Test:} The output forces/torques from the simulation 
  are checked with two thruster in different positions, with different angles.}
\item{\underline{Ramp On/Ramp Off Firing:} A set of ramps are set for the thruster to ensure 
  that the ramp configuration is respected during a ramped firing.}
  \item{\underline{Short ramped firing:} A thruster is fired for less than the amount of time it 
   takes to reach the end of its ramp.}
\item{\underline{Ramp On/Ramp Off Firing with faster test rate:} A set of ramps are set for the thruster to ensure 
  that the ramp configuration is respected during a ramped firing with different test rate.}
\item{\underline{Cutoff firing:} A thruster is commanded off (zero firing time) in the middle 
   of its ramp up to ensure that it correctly cuts off the current firing}
\item{\underline{Ramp down firing:} A thruster is fired during the middle of its ramp down 
   to ensure that it picks up at that point in its ramp-up curve and reaches 
   steady state correctly.}
\end{enumerate}

These scenarios create a set of different runs. These are run in the same test using pytest parameters. Therefore 12 different parameter sets were created to cover all of the listed parts.

\subsection{Test success criteria}

This thrusters test is considered successful if, for each of the scenarios, the output data matches exactly the truth data that is computed in python. This means that at every time step, the thrust is the one that is expected down to near machine precision ($\epsilon = 10^{-9}$). 

This leave no slack for uncertainty in the thrusters module.
 %This explains the unit test for the model. I.e. what features are tested and how. It may also include test tolerances, etc.


\section{Test Parameters}
This section summarizes the specific error tolerances for each test. Error tolerances are determined based on whether the test results comparison should be exact or approximate due to integration or other reasons. Error tolerances for each test are summarized in table \ref{tab:errortol}. 
	
	\begin{table}[H]
		\caption{Error tolerance for each test.}
		\label{tab:errortol}
		\centering \fontsize{10}{10}\selectfont
		\begin{tabular}{ c | c } % Column formatting, 
			\hline
			\textbf{Test}   									& \textbf{Tolerated Error} 						  \\ \hline
			Gyro/Accelerometer I/O 						& \input{AutoTex/gyroIOAccuracy}		   \\ \hline
			MRP Switching 									& - 														   \\ \hline
			Static Bias 										& \input{AutoTex/biasAccuracy} 	 		       \\ \hline
			Process Noise 									& \input{AutoTex/noiseAccuracy}			      \\ \hline
			Discretization 				  					   & \input{AutoTex/discretizationAccuracy}  \\ \hline
			Saturation 											& \input{AutoTex/saturationAccuracy} 	  \\ \hline
			Accelerometer Center of Mass Offset & \input{"AutoTex/COM offsetAccuracy"} \\ \hline
			IMU Misalignment 								& \input{AutoTex/misalignmentAccuracy} \\ \hline
			Bias Walk Bounds 								& - 														   \\ \hline
		\end{tabular}
	\end{table}


\section{Test Results}

\subsection{Pass/Fail}
The test results are explained below and summarized in Table~\ref{tab:results}.

\begin{enumerate}
\item{Error Bound Enforcement: We do want to violate the error bound a 
   statistically small number of times as most bounds are specified 3-sigma 
   and we'll need to be at the spec to make sure it works.  All signals remained 
   inside their bounds greater than 1-sigma (~30\%) of the time.  }
\item{Error Bound Usage: As stated above, we want to ensure that the random 
   walk process is effectively utilizing the error bound that it has been 
   given and not remaining mired near zero.  All error signals cross up above 
   75\% of their error bound at least once.}
\item{Corner Case Usage: All errors/warnings were stimulated and the simulation 
   still ran without incident. }
\end{enumerate}

\begin{table}[htbp]
    \caption{Test Results}
\label{tab:results}
    \centering \fontsize{10}{10}\selectfont
\begin{tabular}{|c||c|}
\hline
SubTest & Result \\ \hline \hline
Bound Enforcement& \textcolor{ForestGreen}{Passed} \\ \hline
Bound Usage &  \textcolor{ForestGreen}{Passed}\\ \hline
Corner Case &  \textcolor{ForestGreen}{Passed}\\ \hline
\end{tabular}
\end{table}



\subsection{Test Coverage}
The method coverage for all of the methods included in the simple\_nav 
module are tabulated in Tables~\ref{tab:cov_met} and \ref{tab:cov_met2}.

\begin{table}[htbp]
    \caption{Simple Navigation Test Analysis Results}
   \label{tab:cov_met}
        \centering \fontsize{10}{10}\selectfont
   \begin{tabular}{c | r | r | r} % Column formatting, 
      \hline
      Method Name    & Unit Test Coverage (\%) & Runtime Self (\%) & Runtime Children (\%) \\
      \hline
      UpdateState & 100.0 & 0.08 & 15.0 \\
      SelfInit & 100.0 & 0.0 & 0.0 \\
      CrossInit & 100.0 & 0.0 & 0.0 \\
      computeOutput & 100.0 & 0.0 & 0.0 \\
      \hline
   \end{tabular}
\end{table}

\begin{table}[htbp]
    \caption{GaussMarkov Test Analysis Results}
   \label{tab:cov_met2}
        \centering \fontsize{10}{10}\selectfont
   \begin{tabular}{c | r | r | r} % Column formatting, 
      \hline
      Method Name    & Unit Test Coverage (\%) & Runtime Self (\%) & Runtime Children (\%) \\
      \hline
      computeNextState & 100.0 & 0.71 & 12.4 \\
      setRNGSeed & 100.0 & 0.0 & 0.0 \\
      setPropMatrix & 100.0 & 0.0 & 0.0 \\
      getCurrentState & 100.0 & 0.0 & 0.0 \\
      setUpperBounds & 100.0 & 0.0 & 0.0 \\
      setNoiseMatrix & 100.0 & 0.0 & 0.0 \\
      setPropMatrix & 100.0 & 0.0 & 0.0 \\
      \hline
   \end{tabular}
\end{table}
For all of the code this test was designed for, the coverage percentage is 
100\%.  The CPU usage of the model is higher than would be ideal although this 
might just be a symptom of the level of simplicity present in the overall 
simulation.  The majority of the computations are coming from two pieces of the 
GaussMarkov code.  

The first is the random number generator.  The model is using 
one of the simplest random number generators in the standard template library.  
That is still a relatively expensive operation as random numbers are costly and 
we generate a new random number for each state.  The second factor is in the 
state and noise propagation.  Those are being performed with a matrix 
multiplication that is an $n^2$ operation.  We could save some computations 
here in the future if we took away the cross-correlation capability from some 
of the states which would definitely be easy and accurate.  It would just take 
some more code.
 %This displays the test results. This includes both pass/fail statements as well as visual outputs via AutoTeX


\input{sec_user_guide.tex} %This section is to provide advice to users on necessary/useful inputs and best practices.


\bibliography{bibliography.bib} %This includes references used and mentioned.

\end{document}
