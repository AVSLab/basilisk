% !TEX root = ./Basilisk-avsLibrary20170812.tex

\section{Library Functions}

The AVS astrodynamics support library is developed to facility writing C-code that must perform the associated mathematical functions.  It is not a Basilisk module, but included with the main Basilisk distribution as a support library.  The library goals are

\begin{itemize}
	\item \textbf{C-Code implementation}: All function calls only use C-code.
	\item \textbf{Linear Algebra}: Basic matrix math function that are common in Matlab and Python environment quickly become tedious when programming in C.  To avoid this, a custom written C library is developed to perform specialized 2, 3, 4, and 6 dimensional matrix math, but some linear algebra math on matrices of general size.
	\item \textbf{Rigid Body Kinematics}: Provides a comprehensive C-based software library to convert between a broad range of rigid body attitude coordinates, as well as add and subtract orientations.  This library can also compute the differential kinematic equations of select coordinate types. The textbook  Reference~\citenum{schaub} contains a complete listing of all the attitude library function in Appendix E.
	\item \textbf{Orbital Motion Library}:  Provides a comprehensive C-based software library to convert between orbit anomaly angles, as well as orbit elements and Cartesian coordinates.  This library also contains some simple space environment modeling functions.  
	
\end{itemize}



\section{Library Assumptions and Limitations}

\subsection{Assumptions}
\subsubsection{Linear Algebra Library}
The vector dimension can be declared either explicitly through 
\begin{verbatim}
     double vec[3]
\end{verbatim}
or implicitly through 
\begin{verbatim} 
     double *vec
\end{verbatim} 
and allocating the required memory dynamically.   However, with the specialized matrix dimensions the matrices are assumed to be defined through commands like 
\begin{verbatim}
     double m33[3][3]
\end{verbatim}
 for a $3\times 3$ matrix. 

\subsubsection{Rigid Body Kinematics}
These attitude kinematic relationships are general enough that no assumptions are made on the mathematics.  However, when the $3\times 3$ DCM are defined, they must be of type
\begin{verbatim}
     double BN[3][3]
\end{verbatim}


\subsubsection{Orbital Motion}
This orbital motion library accepts the inertial position and velocity vectors as pointers to a 3-dimensional array.  As with the linear algebra library, these can be declared explicitly or implicitly.    


\subsection{Limitations}
\subsubsection{Linear Algebra Library}
There are no limitations on the provided linear algebra routines.  They are written in a general manner. 

\subsubsection{Rigid Body Kinematics}
There are no limitations on the provided linear algebra routines.  They are written in a general manner.  

\subsubsection{Orbital Motion}
The orbital motion support library has the following limitations.
\begin{itemize}
	\item While the {\tt elem2rv()} support rectilinear cases, the inverse mapping in {\tt rv2elem()} does not.
	\item The gravitational zonal harmonics for $J_{3}$-$J_{6}$ are only implemented for Earth, and not other celestial objects.
	\item The atmospheric density and drag, as well as the Debye length models are only implemented for Earth orbiting scenarios. 
\end{itemize}













