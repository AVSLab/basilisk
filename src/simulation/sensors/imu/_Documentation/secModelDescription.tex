\section{Model Description}
The Basilisk IMU module imu\_sensor.cpp is responsible for producing sensed body rates and acceleration from simulation truth values. It also provides a change in velocity and change in attitude value for the time between IMU calls. Each check within test\_imu\_sensor.py sets initial attitude MRP, body rates, and accumulated Delta V and validates output for a range of time.

There is a large variation throughout the industry as to what constitutes and IMU.  Some manufacturers offer IMUs which output only acceleration and angular rate while others include accumulated change in velocity in attitude. For Basilisk, the IMU is defined as a device which outputs all four values.

\subsection{Mathematical Model}
\subsubsection{Platform Frame and Sensor Labels}
It will be helpful to note for the following descriptions that the sensor is labeled with a capital S while the sensor platform frame is labeled with a capital P. To be more explicit, There is a coordinate frame, P, the platform frame, in which is the IMU is defined. In all cases so far, the IMU sits at the platform frame origin and its axes are aligned with the platform frame axes. So, any position or velocity vector describing the sensor also describes the platform frame origin. With that in mind, in this report, it has been attempted to track the kinematics of the sensor, while reporting values in the platform frame, rather than tracking the kinematics of the platform form.

\subsubsection{Frame Depedent Derivatives}
Inertial time derivatives are marked with a dot (i.e. $\frac{\mathcal{N}d}{dt}x = \dot{x}$). Body frame time derivatives are marked with a prime (i.e. $\frac{\mathcal{B}d}{dt}x = x^\prime$)
\subsubsection{Angular Rates}
The angular rate of the sensor in platform frame coordinates is output as:
\begin{equation}
	 \leftexp{P}{\bm{\omega}_{S/N}} =  \leftexp{P}{\bm{\omega}_{P/N}} =[PB] \leftexp{B}{\bm{\omega}_{B/N}}
\end{equation}

Where $\cal{P}$ is the sensor platform frame, $\cal{B}$ is the vehicle body frame, and $\cal{N}$ is the inertial frame. [PB] is the direction cosine matrix from $\cal{B}$ to $\cal{P}$. This allows for an arbitrary angular offset between $\cal{B}$ and $\cal{P}$ and allows for that offset to be time-varying. $\leftexp{B}{\bm{\omega}_{B/N}}$ is provided by the spacecraftPlus output message from the most recent dynamics integration.

\subsubsection{Angular Displacement}
The IMU also outputs the angular displacement accumulated between IMU calls. In order to avoid complexities having to do with the relative timestep between the dynamics process and the IMU calls, this is not calculated in the same way as an IMU works physically. In this way, also, the dynamics do not have to be run at a fast enough rate for a physical IMU angular accumulation to be simulated. 
The modified Rodriguez parameter (MRP) is recorded for the last time (1) the IMU was called. Once the new MRP is received, both are converted to DCMs and the step-PRV is computed as follows The current MRP is always provided by the spacecraftPlus message from the most recent dynamics integration.
\begin{equation}
	[PN]_2 = [PB][BN]_2
\end{equation}
\begin{equation}
	[PN]_1 = [PB][BN]_1
\end{equation}
\begin{equation}
	[NP]_1 = [PN]_1^T
\end{equation}
\begin{equation}
	[P_2P_1] = [PN]_2[NP]_1
\end{equation}
\begin{equation}
\bm{q} = \verb|C2PRV(|[P_2P_1]\verb|)|
\end{equation}
where $\bm{q}$ above is the principal rotation vector for the sensor from timestep 1 to timestep 2. The functions used in conversion from the DCM to PRV are part of the Basilisk Rigid Body Kinematics library. The double conversion is used to avoid singularities.

\subsubsection{Linear Acceleration}
The sensor is assumed to have an arbitrary offset from the center of mass of the spacecraft. However, because of the completely coupled nature of the Basilisks dynamics framework, the center of mass does not need to be present explicitly in equations of motion for the sensor. It is implicit in the motion of the body frame. With that in mind, the equation for the acceleration of the sensor is derived below:

\begin{equation}
\bm{r}_{S/N} = \bm{r}_{B/N} + \bm{r}_{S/B}
\end{equation}

Using the transport theorem for $\dot{\bm{r}}_{S/B}$:
\begin{equation}
	\dot{\bm{r}}_{S/N} = \dot{\bm{r}}_{B/N} + \bm{r}'_{S/B} + \bm{\omega}_{B/N} \times \bm{r}_{S/B}
	\label{eq:rDot}
\end{equation}

But $\bm{r}'_{S/B}$ is $0$ because the sensor is assumed to be fixed relative to the body frame. Then,
\begin{equation}
\ddot{\bm{r}}_{S/N} = \ddot{\bm{r}}_{B/N} + \dot{\bm{\omega}}_{B/N} \times \bm{r}_{S/B} +  \bm{\omega}_{B/N} \times (\bm{\omega}_{B/N} \times \bm{r}_{S/B})
\end{equation}
The equation above is the equation for the inertial acceleration of the sensor, but the sensor will only measure the non-conservative accelerations. To account for this, the equation is modified to be:
\begin{equation}
\ddot{\bm{r}}_{S/N, \textrm{sensed}} = (\ddot{\bm{r}}_{B/N} - \bm{a}_\textrm{g}) + \dot{\bm{\omega}}_{B/N} \times \bm{r}_{S/B} +  \bm{\omega}_{B/N} \times (\bm{\omega}_{B/N} \times \bm{r}_{S/B})
\end{equation}
where $\bm{a}_\textrm{g}$ is the instantaneous acceleration due to gravity. Conveniently, $(\ddot{\bm{r}}_{B/N} - \bm{a}_\textrm{g})$ is available from the spacecraft, but in the body frame. The acceleration provided, though, is the time-averaged acceleration between the last two dynamics integration calls and not the instantaneous acceleration. $\bm{r}_{S/B}$ is also available in the body frame. $\dot{\bm{\omega}}_{B/N}$ is given by the spacecraft in body frame coordinates as well. Again, this is a time-averaged value output by the spacecraft, rather than an instantaneous value. Because all values are given in the body frame, the above equation is calulated in the body frame and then converted as seen below:
\begin{equation}
	\leftexp{P}{\ddot{\bm{r}}_{S/N, \textrm{sensed}}} = [PB] \leftexp{B}{ \ddot{\bm{r}}_{S/N, \textrm{sensed}}}
\end{equation}

\subsubsection{Change In Velocity}
The IMU also outputs the velocity accumulated between IMU calls. In order to avoid complexities having to do with the relative time step between the dynamics process and the IMU calls, this is not calculated in the same way as an IMU works physically. In this way, also, the dynamics do not have to be run at a fast enough rate for a physical IMU velocity accumulation to be simulated.

Differencing Eq. \ref{eq:rDot} with itself from time 1 to time 2 gives the equation:
\begin{equation}
	\Delta_{2/1} 	\dot{\bm{r}}_{S/N} = \Delta_{2/1} \dot{\bm{r}}_{B/N} + \Delta_{2/1} (\bm{\omega}_{B/N} \times \bm{r}_{S/B})
	\label{eq:DeltaVelocity}
\end{equation}
$\Delta_{2/1} \dot{\bm{r}}_{B/N}$ is calculated as the difference between the total change in velocity accumulated by the spacecraft body frame at time 2 minus the total change in velocity accumulated by the spacecraft body frame at time 1:
\begin{equation}
\Delta_{2/1} \dot{\bm{r}}_{B/N} = DV_{\textrm{body\_non-conservative},2} - DV_{\textrm{body\_non-conservative},1}
\end{equation}
The above $DV$ values are given by the spacecraft module in body frame coordinates but used in inertial coordinates. They are computed by accumulating the velocity after each dynamics integration and subtracting out the time-averaged gravitational acceleration multiplied by the dynamics time step. Then,
\begin{equation}
	\Delta_{2/1} (\bm{\omega}_{B/N} \times \bm{r}_{S/B}) = \bm{\omega}_{{B/N}_2} \times \bm{r}_{{S/B}_2} - \bm{\omega}_{{B/N}_1} \times \bm{r}_{{S/B}_1}
\end{equation}
$\bm{\omega}_{{B/N}}$ output by the spacecraft is the angular rate from the most recent dynamics integration. Again, the above equation is calculated in the inertial frame and then converted to platform frame coordinates. this means that the values given by spacecraft plus are first converted into inertial frame coordinates, including the location of the sensor in the body frame. At this point, Eq. \ref{eq:DeltaVelocity} is evaluated in the body frame and converted to the sensor platform frame:
\begin{equation}
\leftexp{P} {\Delta_{2/1}} 	\dot{\bm{r}}_{S/N} = [PN] ^{\mathcal{N}} \Delta_{2/1} 	\dot{\bm{r}}_{S/N}
\end{equation}
This, the change in velocity sensed by the IMU between IMU calls in platform frame coordinates, is the change in velocity output from the model. To be clear, this is the sensed inertial velocity change in platform frame coordinates. This makes the assumption that the IMU is tracking its attitude and performing the calculations internally to return this value correctly. It is not simply the integral of the acceleration value above in the body frame coordinates.

\subsubsection{Error Modeling}
The state which the simulation records for the spacecraft prior to sending that state to the IMU module is considered to be "truth". So, to simulate the errors found in real instrumentation, errors are added to the "truth" values for acceleration and angular velocity:

\begin{equation}
\mathbf{a}_{\mathrm{measured}} = \mathbf{a}_{\mathrm{truth}} + \mathbf{e}_{\mathrm{a,noise}} + \mathbf{e}_{\mathrm{a, bias}}
\end{equation}
\begin{equation}
\bm{\omega}_{\mathrm{measured}} = \bm{\omega}_{\mathrm{truth}} + \mathbf{e}_{\mathrm{\omega,noise}} + \mathbf{e}_{\mathrm{\omega, bias}}
\end{equation}
Then, these error values are "integrated" over the IMU timestep and applied to the $\Delta v$ and $PRV$ values:
\begin{equation}
\mathbf{DV}_{\mathrm{measured}} = \mathbf{DV}_{\mathrm{truth}} + (\mathbf{e}_{\mathrm{a,noise}} + \mathbf{e}_{\mathrm{a, bias}})\Delta t
\end{equation}
\begin{equation}
\mathbf{q}_{\mathrm{measured}} = \mathbf{q}_{\mathrm{truth}} + (\mathbf{e}_{\mathrm{\omega,noise}} + \mathbf{e}_{\mathrm{\omega, bias}})\Delta t
\end{equation}
This convenient approximation that $\bm{q$} = $\bm{\omega}\Delta t$ for a given timestep proves useful through the application of IMU errors.


\subsubsection{Data Discretization}
Because sensors record data digitally, that data can only be recorded in discrete chunks, rather than the (relatively) continuous values that the computer calculates at each time steps. In order to simulate real IMU behavior in this way, a least significant bit (LSB) value is accepted for both the gyro and the accelerometer. This LSB is applied in the following way:

\begin{equation}
\mathbf{a}_{\mathrm{discretized}} = \verb|sign|(\mathbf{a}_\mathrm{measured})(\mathrm{LSB})\Biggl\lfloor\Biggl|\frac{\mathbf{a}_{\mathrm{measured}}}{(\mathrm{LSB})}\Biggr|\Biggr\rfloor
\end{equation}
\begin{equation}
\mathbf{e}_{\mathrm{d},\mathrm{a}} = \mathbf{a}_{\mathrm{measured}} - \mathbf{a}_{\mathrm{discretized}}
\end{equation}
\begin{equation}
\mathbf{DV}_{\mathrm{discretized}} = \mathbf{DV}_{\mathrm{measured}} -\mathbf{e}_{\mathrm{d},\mathrm{a}} \Delta t
\end{equation}

\begin{equation}
\bm{\omega}_{\mathrm{discretized}} = \verb|sign|(\bm{\bm{\omega}_\mathrm{measured}})(\mathrm{LSB})\Biggl\lfloor\Biggl|\frac{\bm{\omega}_{\mathrm{measured}}}{(\mathrm{LSB})}\Biggr|\Biggr\rfloor
\end{equation}
\begin{equation}
\mathbf{e}_{\mathrm{d},\omega} =\bm{\omega}_{\mathrm{measured}} - \bm{\omega}_{\mathrm{discretized}}
\end{equation}
\begin{equation}
\mathbf{q}_{\mathrm{discretized}} = \mathbf{q}_{\mathrm{measured}} - \mathbf{e}_{\mathrm{d},\mathrm{\omega}}\Delta t
\end{equation}
Where $\lfloor$  $\rfloor$ indicate the \textbf{floor()} function and LSB can be either the accelerometer or gyro least significant bit as appropriate.

\subsubsection{Saturation}
Real sensors can also become saturated. Saturation is the last effect implemented on the IMU, \textit{in an elementwise manner}:

\begin{equation}
	\bm{a}_{\mathrm{sat}} = \mathrm{max}\big(a_{\mathrm{min}}, \mathrm{min}\big(    \bm{a}_{\mathrm{discretized}}, a_{\mathrm{max}}    \big)   \big)
\end{equation}
\begin{equation}
\bm{\omega}_{\mathrm{sat}} = \mathrm{max}\big(\omega_{\mathrm{min}}, \mathrm{min}\big(    \bm{\omega}_{\mathrm{discretized}}, \omega_{\mathrm{max}}    \big)   \big)
\end{equation}
The above operations are performed element-wise. This is only computed if the values are found to be outside of the max-min range. Now, along each axis that was saturated:

\begin{equation}
	DV_{\mathrm{sat},i} = a_{\mathrm{sat},i} \Delta t
\end{equation}
\begin{equation}
q_{\mathrm{sat},i} = \omega_{\mathrm{sat},i}  \Delta t
\end{equation}
The above is calculated any time that $a_i$ or $\omega_i$ are found to be outside of their max-min bounds. Note again the use of the approximation of the PRV as the integral of the angular rates.
