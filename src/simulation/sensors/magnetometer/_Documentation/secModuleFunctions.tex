% !TEX root = ./Basilisk-magnetometer-20190926.tex


\section{Module Functions}
The magnetometer functions are described below:
\begin{itemize}
	\item \textbf{Noise}: The module can apply Gaussian noise to the measurements.
	\item \textbf{Bias}: The module can apply a bias to the measurements.
	\item \textbf{Saturation}: The module bounds the output signal according to user-specified maximum and minimum saturation values.
    \item \textbf{Scale Factor}: The module can apply a scale factor to the measured value (truth + noise + bias). This is a linear scaling of the output.
	\item \textbf{Interface: Spacecraft State}: The module receives spacecraft state information from the messaging system.
	\item \textbf{Interface: Magnetic Field Vector}: The module receives magnetic field model information through the messaging system
\end{itemize}

\section{Module Assumptions and Limitations}
Assumptions made in TAM module and the corresponding limitations are shown below:
\begin{itemize}
    \item \textbf{Magnetic Field Model Inputs}: The magnetometer sensor is limited with the used magnetic field model which are individual magnetic field models complex and having their own assumptions. The reader is referred to the cited literature to learn more about the model limitations and challenges.
	\item \textbf{Error Inputs}: Since the error models rely on user inputs, these inputs are the most likely source of error in TAM output. Instrument bias would have to be measured experimentally or an educated guess would have to be made. The Gauss-Markov noise model has well-known assumptions and is generally accepted to be a good model for this application.
	\item \textbf{External Disturbances}: Currently, the module does not consider the external magnetic field, so it is limited to the cases where this effect is not significant. This can be overcome by using magnetic field models taking into these effects account or adding it as an additional term.
\end{itemize}