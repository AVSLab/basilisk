% !TEX root = ./Basilisk-magFieldWMM-20190618.tex

\section{Test Description and Success Criteria}
The WMM software provides a PDF with 12 sample locations and magnetic field values that should be returned in the NED frame.  
The unit test runs the magnetic field Basilisk module with two fixed spacecraft state input messages.  Their locations are identical and set to the WMM test locations.  The simulation option {\tt useDefault} checks if the module epoch time value default settings are used, or if the epoch time in terms of a decimal year is specified directly. The option {\tt useMsg} determines if the epoch time is read in through a message.  If this message is available, it is supposed to over-rule the epochYear variable.  The option {\tt useMinReach} dictates if the minimum orbit radius check is performed, while the option {\tt useMaxReach} checks if the maximum reach check is performed.  The option {\tt usePlanetEphemeris} checks if a planet state input message should be created.  All permutations are checked.

\section{Test Parameters}
The simulation tolerances are shown in Table~\ref{tab:errortol}.  In each simulation the neutral density output message is checked relative to python computed true values.  
\begin{table}[htbp]
	\caption{Error tolerance for each test.}
	\label{tab:errortol}
	\centering \fontsize{10}{10}\selectfont
	\begin{tabular}{ c | c } % Column formatting, 
		\hline\hline
		\textbf{Output Value Tested}  & \textbf{Tolerated Error}  \\ 
		\hline
		{\tt magneticField vector}        & \input{AutoTeX/unitTestToleranceValue} (nT, relative) \\ 
		\hline\hline
	\end{tabular}
\end{table}




\section{Test Results}
Over 200 test permutations are run by the unit test.  All are expected to pass.





