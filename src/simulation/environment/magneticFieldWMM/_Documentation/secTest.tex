% !TEX root = ./Basilisk-magFieldDipole-20190309.tex

\section{Test Description and Success Criteria}
This section describes the specific unit tests conducted on this module. 
This unit test only runs the magnetic field Basilisk module with two fixed spacecraft state input messages.  The simulation option {\tt useDefault} checks if the module default settings are used that lead to a zero magnetic field vector, or if the centered dipole parameters are setup manually.  The option {\tt useMinReach} dictates if the minimum orbit radius check is performed, while the option {\tt useMaxReach} checks if the maximum reach check is performed.  The option {\tt usePlanetEphemeris} checks if a planet state input message should be created.  All permutations are checked.

\section{Test Parameters}
The simulation tolerances are shown in Table~\ref{tab:errortol}.  In each simulation the neutral density output message is checked relative to python computed true values.  
\begin{table}[htbp]
	\caption{Error tolerance for each test.}
	\label{tab:errortol}
	\centering \fontsize{10}{10}\selectfont
	\begin{tabular}{ c | c } % Column formatting, 
		\hline\hline
		\textbf{Output Value Tested}  & \textbf{Tolerated Error}  \\ 
		\hline
		{\tt magneticField vector}        & 1e-05 (relative) \\ 
		\hline\hline
	\end{tabular}
\end{table}




\section{Test Results}
The following two tables show the test results.  All tests are expected to pass.



\begin{table}[H]
	\caption{Test result for {\tt test\_unitTestMagneticField.py}}
	\label{tab:results}
	\centering \fontsize{10}{10}\selectfont
	\begin{tabular}{c | c | c | c | c } % Column formatting, 
		\hline\hline
		{\tt useDefault} & {\tt useMinReach} & {\tt useMaxReach} & {\tt usePlanetEphemeris} & \textbf{Pass/Fail} \\ 
		\hline
	   False & False & False & False	& \textcolor{ForestGreen}{PASSED} \\ 
	   False & False & False & True	& \textcolor{ForestGreen}{PASSED} \\ 
	   False & False & True & False	& \textcolor{ForestGreen}{PASSED} \\ 
	   False & False & True & True	& \textcolor{ForestGreen}{PASSED} \\ 
	   False & True & False & False	& \textcolor{ForestGreen}{PASSED} \\ 
	   False & True & False & True	& \textcolor{ForestGreen}{PASSED} \\ 
	   False & True & True & False	& \textcolor{ForestGreen}{PASSED} \\ 
	   False & True & True & True	& \textcolor{ForestGreen}{PASSED} \\ 
	   True & False & False & False	& \textcolor{ForestGreen}{PASSED} \\ 
	   True & False & False & True	& \textcolor{ForestGreen}{PASSED} \\ 
	   True & False & True & False	& \textcolor{ForestGreen}{PASSED} \\ 
	   True & False & True & True	& \textcolor{ForestGreen}{PASSED} \\ 
	   True & True & False & False	& \textcolor{ForestGreen}{PASSED} \\ 
	   True & True & False & True	& \textcolor{ForestGreen}{PASSED} \\ 
	   True & True & True & False	& \textcolor{ForestGreen}{PASSED} \\ 
	   True & True & True & True	& \textcolor{ForestGreen}{PASSED} \\ 
	   \hline\hline
	\end{tabular}
\end{table}



