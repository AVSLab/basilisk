% !TEX root = ./Basilisk-magFieldWMM-20190618.tex

\section{User Guide}

\subsection{General Module Setup}
This section outlines the steps needed to add a {\tt MagneticField} module to a sim.
First, the planet magnetic field model must be imported and initialized:
\begin{verbatim}
     from Basilisk.simulation import magneticFieldWMM
     magModule = magneticFieldWMM.MagneticFieldWMM()
     magModule.ModelTag = "WMM"
     magModule.dataPath = splitPath[0] + 'Basilisk/supportData/MagneticField/'
\end{verbatim}
By default the model assumes the BSK epoch date and time.  To set a common epoch time across various BSK modules, then the {\tt EpochMsgPayload} must be created and written.
\begin{verbatim}
     epochMsgData = messaging2.EpochMsgPayload()
     dt = unitTestSupport.decimalYearToDateTime(decimalYear)
     epochMsgData.year = year
     epochMsgData.month = month
     epochMsgData.day = day
     epochMsgData.hours = hour
     epochMsgData.minutes = minute
     epochMsgData.seconds = second
     epMsg = messaging2.EpochMsg().write(epochMsgData)
     magModule.epochInMsg.subscribeTo(epMsg)
\end{verbatim}
If the user wants to set the WMM epoch directly, this is done by defining the {\tt epochDate} variable in terms of a decimal year format required by WMM. 
\begin{verbatim}
	magModule.epochDate = decimalYear
\end{verbatim}
Not that the epoch message, if available, over-writes the information of setting {\tt epochDate}.  

The model can  be added to a task like other simModels. 
\begin{verbatim}
unitTestSim.AddModelToTask(unitTaskName, testModule)
\end{verbatim}

Each {\tt MagneticField} module calculates the magnetic field based on the output state messages for a set of spacecraft.
To add spacecraft to the model the spacecraft state output message name is sent to the \verb|addScToModel| method:
\begin{verbatim}
scObject = spacecraftPlus.SpacecraftPlus()
scObject.ModelTag = "spacecraftBody"
magModule.addSpacecraftToModel(scObject.scStateOutMsg)
\end{verbatim}

\subsection{Planet Ephemeris Information}
The optional planet state message name can be set by directly adjusting that attribute of the class:
\begin{verbatim}
magModule.planetPosInMsg.subscribeTo(planetMsg)
\end{verbatim}
If SPICE is not being used, the planet is assumed to reside at the origin and $\bm r_{P/O} = \bm 0$.

\subsection{Setting the Model Reach}
By default the model doesn't perform any checks on the altitude to see if the specified magnetic field model should be used.  This is set through the parameters {\tt envMinReach} and {\tt envMaxReach}.  Their default values are -1.  If these are set to positive values, then if the spacecraft orbit radius is smaller than {\tt envMinReach} or larger than {\tt envMaxReach}, the magnetic field vector is set to zero.


