% !TEX root = ./Basilisk-atmosphere-20190221.tex


\section{Module Functions}
This module will:
\begin{itemize}
	\item \textbf{Compute atmospheric density and temperature}: Each of the provided models is fundamentally intended to compute the neutral atmospheric density and temperature for a spacecraft relative to a body. These parameters are stored in the AtmoPropsSimMsg struct. Supporting parameters needed by each model, such as planet-relative position, are also computed.
	\item \textbf{Communicate neutral density and temperature}: This module interfaces with modules that subscribe to neutral density messages via the messaging system.
	\item \textbf {Subscribe to model-relevant information:} Each provided atmospheric model requires different input information to operate, such as current space weather conditions and spacecraft positions. This module automatically attempts to subscribe to the relevant messages for a specified model. 
	\item \textbf{Support for multiple spacecraft} Only one NRLMSISE-00 atmosphere model is needed to compute densities for several spacecraft.
	\item \textbf{Support dynamic space-weather coupled density forecasting}: A primary benefit of the NRLMSISE-00 model is its ability to provide forecasts of neutral density in response to space weather events, changing atmospheric conditions, and the like that are not captured in simpler models.
\end{itemize}

\section{Module Assumptions and Limitations}
Individual atmospheric models are complex and have their own assumptions. At present, all non-exponential models are Earth-specific. For details about tradeoffs in atmospheric modeling, the reader is pointed to ~\citenum{vallado2013}. 

NRLMSISE-00, specifically, is highly dependent on ``space weather parameters'' such as $Ap$, $Kp$, $F10.7$, among others. The outputs of this model can vary greatly with the selected parameters.