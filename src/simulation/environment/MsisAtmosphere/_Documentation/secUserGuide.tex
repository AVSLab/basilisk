% !TEX root = ./Basilisk-atmosphere-20190221.tex

\section{User Guide}

\subsection{General Module Setup}
This section outlines the steps needed to add an MsisAtmosphere module to a sim.
First, the atmosphere must be imported and initialized:
\begin{verbatim}
from Basilisk.simulation import msisAtmosphere
newAtmo = msisAtmosphere.MsisAtmosphere()
newAtmo.ModelTag = "MsisAtmo"
\end{verbatim}

By default, the module assumes no planet radius or date. These can be set by calling
\begin{verbatim}
newAtmo.setEpoch(julian_date)
newAtmo.planetRadius = r_eq
\end{verbatim}
The model can then be added to a task like other simModels. Each Atmosphere calculates atmospheric parameters based on the output state messages for a set of spacecraft.

To add spacecraft to the model the spacecraft state output message name is sent to the \verb|addScToModel| method common to environmental models:
\begin{verbatim}
scObject = spacecraftPlus.SpacecraftPlus()
scObject.ModelTag = "spacecraftBody"
newAtmo.addSpacecraftToModel(scObject.scStateOutMsgName)
\end{verbatim}

\subsection{Planet Ephemeris Information}
The optional planet state message name can be set by directly adjusting that attribute of the class:
\begin{verbatim}
newAtmo.planetPosInMsgName = "PlanetSPICEMsgName"
\end{verbatim}
If SPICE is not being used, the planet is assumed to reside at the origin.

\subsection{Setting the Model Reach}
By default the model doesn't perform any checks on the altitude to see if the specified atmosphere model should be used.  This is set through the parameters {\tt envMinReach} and {\tt envMaxReach}.  Their default values are -1.  If these are set to positive values, then if the altitude is smaller than {\tt envMinReach} or larger than {\tt envMaxReach}, the density is set to zero.


\subsection{NRLMSISE-00 atmosphere user guide}
NRLMSISE-00 is dependent on a variety of space weather indexes, times, and locations. During initialization, a starting date must be set; this will be updated as the sim progresses using the simulation time. NRLMSISE-00 will attempt to subscribe to a standard set of message names that can be produced by the WIP space-weather data factory module, or set by hand. These messages are
\begin{verbatim}
sw_msg_names = [
"ap_24_0", "ap_3_0", "ap_3_-3","ap_3_-6","ap_3_-9",
"ap_3_-12","ap_3_-15","ap_3_-18","ap_3_-21","ap_3_-24",
"ap_3_-27", "ap_3_-30","ap_3_-33","ap_3_-36","ap_3_-39",
"ap_3_-42", "ap_3_-45", "ap_3_-48","ap_3_-51","ap_3_-54",
"ap_3_-57","f107_1944_0","f107_24_-24"
]
\end{verbatim}