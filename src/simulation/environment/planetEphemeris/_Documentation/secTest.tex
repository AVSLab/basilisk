% !TEX root = ./Basilisk-planetEphemeris-20190422.tex

\section{Test Description and Success Criteria}
The unit test configures the module to model general orbital motions of Earth and Venus.  In each case the translational motion is compute for 3 times steps from 0 to 1 second using a 0.5 second time step.  The planet orientation information is only set if the appropriate simulation parameter is set.  The following sub-sections discuss these flags.  If none of these flags are set, then the module default orientation behavior is expected.  If all the flags are set, then a constant rotation is set.  If only a partial set of orientation information is provided then the reset routine will force the module to throw an error message and only output a default constant zero orientation of the planet. 

\subsection{{\tt setRAN}}
This flag specifies if a set of right ascension angles are specified in the unit test.

\subsection{{\tt setDEC}}
This flag specifies if a set of declination angles are specified in the unit test.

\subsection{{\tt setLST}}
This flag specifies if a set of local sidereal time angles are specified in the unit test.

\subsection{{\tt setRate}}
This flag specifies if a set of planetary polar rotation rates are specified in the unit test.




\section{Test Parameters}
The unit test verify that the module output guidance message vectors match expected values.
\begin{table}[htbp]
	\caption{Error tolerance for each test.}
	\label{tab:errortol}
	\centering \fontsize{10}{10}\selectfont
	\begin{tabular}{ c | c } % Column formatting, 
		\hline\hline
		\textbf{Output Value Tested}  & \textbf{Tolerated Error}  \\ 
		\hline
		{\tt J2000Current}        & 1e-05 s   \\ 
		{\tt PositionVector}        & 1e-05 m   \\ 
		{\tt VelocityVector}        & 1e-05 m/s   \\ 
		{\tt J20002Pfix}        & 1e-05    \\ 
		{\tt J20002Pfix\_dot}        & $10^{-10}$ rad/s   \\ 
		{\tt computeOrient}        & 1e-05    \\ 
		\hline\hline
	\end{tabular}
\end{table}




\section{Test Results}
All orientation flag permutations are tested and are expected to pass.

\begin{table}[H]
	\caption{Test results}
	\label{tab:results}
	\centering \fontsize{10}{10}\selectfont
	\begin{tabular}{c | c  | c | c | c } % Column formatting, 
		\hline\hline
		{\tt setRAN} & {\tt setDEC} & {\tt setLST} & {\tt setRate} &\textbf{Pass/Fail} \\ 
		\hline
	   True & True & True & True & \input{AutoTeX/passFailTrueTrueTrueTrue} \\ 
	   True & True & True & False & \input{AutoTeX/passFailTrueTrueTrueFalse} \\ 
	   True & True & False & True & \input{AutoTeX/passFailTrueTrueFalseTrue} \\ 
	   True & True & False & False & \input{AutoTeX/passFailTrueTrueFalseFalse} \\ 
	   True & False & True & True & \input{AutoTeX/passFailTrueFalseTrueTrue} \\ 
	   True & False & True & False & \input{AutoTeX/passFailTrueFalseTrueFalse} \\ 
	   True & False & False & True & \input{AutoTeX/passFailTrueFalseFalseTrue} \\ 
	   True & False & False & False & \input{AutoTeX/passFailTrueFalseFalseFalse} \\ 
	   False & True & True & True & \input{AutoTeX/passFailFalseTrueTrueTrue} \\ 
	   False & True & True & False & \input{AutoTeX/passFailFalseTrueTrueFalse} \\ 
	   False & True & False & True & \input{AutoTeX/passFailFalseTrueFalseTrue} \\ 
	   False & True & False & False & \input{AutoTeX/passFailFalseTrueFalseFalse} \\ 
	   False & False & True & True & \input{AutoTeX/passFailFalseFalseTrueTrue} \\ 
	   False & False & True & False & \input{AutoTeX/passFailFalseFalseTrueFalse} \\ 
	   False & False & False & True & \input{AutoTeX/passFailFalseFalseFalseTrue} \\ 
	   False & False & False & False & \input{AutoTeX/passFailFalseFalseFalseFalse} \\ 
	   \hline\hline
	\end{tabular}
\end{table}

