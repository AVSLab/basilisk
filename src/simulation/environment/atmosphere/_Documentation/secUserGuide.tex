% !TEX root = ./Basilisk-atmosphere-20190221.tex

\section{User Guide}

\subsection{General Module Setup}
This section outlines the steps needed to add an Atmosphere module to a sim.

First, the atmosphere must be imported and initialized:
\begin{verbatim}
from Basilisk.simulation import atmosphere

newAtmo = atmosphere.Atmosphere()
\end{verbatim}

Next, the desired model type must be set by calling "\verb|setEnvType|":
\begin{verbatim}
atmoTaskName = "atmosphere"
newAtmo.ModelTag = "ExpAtmo"
newAtmo.setEnvType("exponential")
\end{verbatim}

The model can then be added to a task like other simModels. Each Atmosphere calculates atmospheric parameters based on the output state messages for a set of spacecraft; to add spacecraft to the model the \verb|addScToModel| method is invoked:

\begin{verbatim}
scObject = spacecraftPlus.SpacecraftPlus()
scObject.ModelTag = "spacecraftBody"
newAtmo.addSpacecraftToModel(scObject.scStateOutMsgName)
\end{verbatim}

Finally, the planet state message name can be set by directly adjusting that attribute of the class:
\begin{verbatim}
newAtmo.planetPosInMsgName = "PlanetSPICEMsgName"
\end{verbatim}
If SPICE is not being used, the planet is assumed to reside at the origin.

\subsection{Exponential atmosphere user guide}
If the module's \verb|envType| is set to \verb|"exponential"|, the parameters of the exponential atmosphere can be set by calling
\begin{verbatim}
    newAtmo.exponentialParams.baseDensity = baseDensityValue
    newAtmo.exponentialParams.scaleHeight = scaleHeightValue
    newAtmo.exponentialParams.planetRadius = planetRadiusValue
\end{verbatim}