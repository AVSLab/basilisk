\section{Model Functions}

The spice\_interface module contains methods allowing it to perform several tasks:

\begin{itemize}
	\item \textbf{Load Various Spice Kernels}: Spice interface can load a variety of available kernels.
	\item \textbf{Get GPS Time}: Computes GPS time from the Spice kernel.
	\item \textbf{Get Planet Data}: Pulls the desired planet's ephemeris from the Spice kernel.
	\item \textbf{Get current time string}: Pull the current time string.
	\item \textbf{Write Spice message}: This module then writes out the spice ephemeris message
	\item \textbf{Read Optional Epoch Message}:  This this message name is specified, then the epoch date and time information is pulled from this message.  This makes it simple to synchronize the epoch information across multiple modules using this one epoch message. 
\end{itemize}


\section{Model Assumptions and Limitations}

\subsection{Assumptions}

Spice interface reads extremely precise ephemeris data, which we compare with JPL's Horizons data. Depending on the celestial body, the accuracy may vary. 
There are no direct assumptions made while using this module. A user must simply make sure to be comparing the write data by assuring the the frames, times, and loaded kernels (mars vs mars barycenter) are the same.

\subsection{Limitations}

The limitations come directly from the kernels that are available to be loaded. These will limit the planets that can be tracked.