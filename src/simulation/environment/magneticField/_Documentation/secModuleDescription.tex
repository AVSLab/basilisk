% !TEX root = ./Basilisk-magField-20190309.tex

\section{Model Description}
The purpose of this module is to implement a magnetic field model that rotates with a planet fixed frame \frameDefinition{E}.  Here $\hat{\bm e}_{3}$ is the typical positive rotation axis and $\hat{\bm e}_{1}$ and $\hat{\bm e}_{2}$ span the planet's equatorial plane. 

{\tt MagneticField} represents a single interface to a variety of magnetic field models within the BSK framework. These are briefly summarized here for reference.
By invoking the magnetic field module, the default values are set such that the centered dipole model is simulated about Earth.
The reach of the model controlled by setting the variables {\tt envMinReach} and {\tt envMaxReach} to positive values.  These values are the radial distance from the planet center.  The default values are -1 which turns off this checking where the atmosphere model as unbounded reach.  

There are a multitude of magnetic field models.\footnote{\url { https://geomag.colorado.edu/geomagnetic-and-electric-field-models.html}} The goal with Basilisk is to provide a simple and consistent interface to a range of models.  The list of models is expected to grow over time.


\subsection{Planet Centric Spacecraft Position Vector}

For the following developments, the spacecraft location relative to the planet frame is required.  Let $\bm r_{B/P}$ be the spacecraft position vector relative to the planet center.  In the simulation the spacecraft location is given relative to an inertial frame origin $O$.  The planet centric position vector is computed using
\begin{equation}
	\bm r_{B/P} = \bm r_{B/O} - \bm r_{P/O}
\end{equation}
If not planet ephemeris message is specified, then the planet position vector $\bm r_{P/O}$ is set to zero.  

Let $[EN]$ be the direction cosine matrix\cite{schaub} that relates the rotating Earth-fixed frame relative to an inertial frame \frameDefinition{N}.  The simulation provides the spacecraft position vector in inertial frame components.  The planet centric position vector is then written in Earth-fixed frame components using
\begin{equation}
	\leftexp{E}{\bm r}_{B/P} = [EN] \ \leftexp{N}{\bm r}_{B/P}
\end{equation}




\subsection{Centered Dipole Magnetic Field Model}
The centered dipole model is a first order result of the more complex spherical harmonic modeling of the planet's magnetic field.\cite{Markley:2014lj}  There are several solution that provide an answer in the local North-Earth-Down or NED frame,\cite{Grifin:2005hx} or in the local  spherical coordinates.  Let  $\bm m$ be the magnetic dipole vector which is then defined as\cite{Markley:2014lj}
\begin{equation}
	V(\bm r_{B/P}) = \frac{\bm m \cdot \bm r_{B/P}}{|\bm r_{B/P}|^{3}}
\end{equation}
with the dipole vector being defined in Earth fixed frame $\cal E$ coordinates as 
\begin{equation}
	\bm m = \leftexp{E}{\begin{bmatrix}
		g_{1}^{1} \\ h_{1}^{1} \\ g_{1}^{0}
	\end{bmatrix}}
\end{equation}
The magnetic field vector $\bm B$ is  expressed at the spacecraft location as
\begin{equation}
	\bm B(\bm r_{B/P}) = - \nabla V(\bm r_{B/P}) = \frac{3 (\bm m \cdot \bm r_{B/P}) \bm r_{B/P} - \bm r_{B/P} \cdot \bm r_{B/P} \bm m}{|\bm r_{B/P}|^{5}}
	= \frac{1}{|\bm r_{B/P}|^{3}} \left ( 3 (\bm m \cdot \hat{\bm r}) \hat{\bm r} - \bm m \right)
\end{equation}
where 
\begin{equation}
	\hat{\bm r} = \frac{\bm r_{B/P}}{|\bm r_{B/P}|}
\end{equation}

The above vector equation is evaluated in $\cal E$-frame components, while the output is mapped into $\cal N$-frame components by returning
\begin{equation}
	\leftexp{N}{\bm B} = [EN]^{T} \ \leftexp{E}{\bm B}
\end{equation}
