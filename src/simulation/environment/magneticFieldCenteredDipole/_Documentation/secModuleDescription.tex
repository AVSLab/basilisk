% !TEX root = ./Basilisk-magFieldDipole-20190309.tex

\section{Model Description}
\subsection{General Module Behavior}
The purpose of this module is to implement a magnetic field model that rotates with a planet fixed frame \frameDefinition{P}.  Here $\hat{\bm p}_{3}$ is the typical positive rotation axis and $\hat{\bm p}_{1}$ and $\hat{\bm p}_{2}$ span the planet's equatorial plane. 

{\tt MagneticFieldCenteredDipole} is a child of the {\tt MagneticFieldBase} base class and provides a simple centered dipole magnetic field model. 
By invoking the magnetic field module, the default values are set such that the dipole parameters are zeroed and the magnetic field output is a zero vector.
The reach of the model controlled by setting the variables {\tt envMinReach} and {\tt envMaxReach} to positive values.  These values are the radial distance from the planet center.  The default values are -1 which turns off this checking where the atmosphere model as unbounded reach.  

There are a multitude of magnetic field models.\footnote{\url { https://geomag.colorado.edu/geomagnetic-and-electric-field-models.html}} The goal with Basilisk is to provide a simple and consistent interface to a range of models.  The list of models is expected to grow over time.


\subsection{Planet Centric Spacecraft Position Vector}

For the following developments, the spacecraft location relative to the planet frame is required.  Let $\bm r_{B/P}$ be the spacecraft position vector relative to the planet center.  In the simulation the spacecraft location is given relative to an inertial frame origin $O$.  The planet centric position vector is computed using
\begin{equation}
	\bm r_{B/P} = \bm r_{B/O} - \bm r_{P/O}
\end{equation}
If no planet ephemeris message is specified, then the planet position vector $\bm r_{P/O}$ is set to zero.  

Let $[PN]$ be the direction cosine matrix\cite{schaub} that relates the rotating planet-fixed frame relative to an inertial frame \frameDefinition{N}.  The simulation provides the spacecraft position vector in inertial frame components.  The planet centric position vector is then written in Earth-fixed frame components using
\begin{equation}
	\leftexp{P}{\bm r}_{B/P} = [PN] \ \leftexp{N}{\bm r}_{B/P}
\end{equation}




\subsection{Centered Dipole Magnetic Field Model}
The centered dipole model is a first order result of the more complex spherical harmonic modeling of the planet's magnetic field.\cite{Markley:2014lj}  There are several solutions that provide an answer in the local North-Earth-Down or NED frame,\cite{Grifin:2005hx} or in the local  spherical coordinates.  Let  $\bm m$ be the magnetic dipole vector which is then defined as\cite{Markley:2014lj}
\begin{equation}
	V(\bm r_{B/P}) = \frac{\bm m \cdot \bm r_{B/P}}{|\bm r_{B/P}|^{3}}
\end{equation}
with the dipole vector being defined in Earth fixed frame $\cal E$ coordinates as 
\begin{equation}
	\bm m = \leftexp{E}{\begin{bmatrix}
		g_{1}^{1} \\ h_{1}^{1} \\ g_{1}^{0}
	\end{bmatrix}}
\end{equation}
The magnetic field vector $\bm B$ is  expressed at the spacecraft location as
\begin{equation}
	\bm B(\bm r_{B/P}) = - \nabla V(\bm r_{B/P}) = \frac{3 (\bm m \cdot \bm r_{B/P}) \bm r_{B/P} - \bm r_{B/P} \cdot \bm r_{B/P} \bm m}{|\bm r_{B/P}|^{5}}
	= \frac{1}{|\bm r_{B/P}|^{3}} \left ( 3 (\bm m \cdot \hat{\bm r}) \hat{\bm r} - \bm m \right)
\end{equation}
where 
\begin{equation}
	\hat{\bm r} = \frac{\bm r_{B/P}}{|\bm r_{B/P}|}
\end{equation}

The above vector equation is evaluated in $\cal E$-frame components, while the output is mapped into $\cal N$-frame components by returning
\begin{equation}
	\leftexp{N}{\bm B} = [PN]^{T} \ \leftexp{P}{\bm B}
\end{equation}
