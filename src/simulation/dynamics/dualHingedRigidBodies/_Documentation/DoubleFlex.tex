\documentclass[paper]{aiaaNew}
% cover, paper, article, note,submit

%\usepackage[notref,notcite]{showkeys}


\newenvironment{proof}
{}{}


%\documentstyle[10pt,draft,fancyheadings]{AIAAtran}
%\documentstyle[9pt,twocolumn,technote,twoside]{AIAAtran}


\SubmitName{Schaub}

% for conference paper:
\usepackage{AVS}
\usepackage{amsmath,amsfonts,amssymb}
\usepackage{subfigure}
\usepackage{multirow}

\PaperNumber{xxx}

\CoverFigure{}

\Conference{{\bfseries AIAA Guidance, Navigation and \\ Control 
		Conference} \\
	August 10--12,~1998 / Boston, MA}

% for a journal simulation cover page:

\JournalName{Journal of Guidance, Navigation and Control}
\JournalIssue{Volume~xx, Number~xx, Jan.--Feb., 2001, Pages xx--xx}

% journal article simulation:

\ArticleIssue{Vol.~24, No.~1, Jan.--Feb., 2001}% first page
\ArticleHeader{Schaub Et Al: New Penalty Functions}% subsequent pages

% journal note simulation:

\NoteHeader{J.Guidance, Vol.~20, No.~13: Engineering Notes}

% set copyright and other notices to appear
% as a footnote at the bottom of the first page:

%\PaperNotice{\CopyrightB{1998}{Hanspeter Schaub}}

\JournalNotice{Presented as Paper~06--3792 at the AIAA
	Guidance, Navigation and Control Conference, San 
	Diego,~CA,
	July~29--31,~1996.
	\CopyrightB{1996}{the authors}}

% load the title, author, and abstract for use with the \maketitle command

\title{Spacecraft Dynamical Model with Dual-Connected Hinged Rigid Bodies}

\author{
	Cody Allard,%
	\thanks{Graduate Student, Aerospace Engineering Sciences, University of Colorado Boulder, AIAA Student Member.}
	\ and  Hanspeter Schaub%
	\thanks{Alfred T. and Betty E. Look Professor of Engineering, Department of Aerospace Engineering Sciences, University of Colorado, 431 UCB, Colorado Center for Astrodynamics Research, Boulder, CO 80309-0431. AIAA Associate Fellow.}\\
	{\normalsize\itshape
		University of Colorado, Boulder, Colorado, 80309, US}\\
%	\and
%	Scott Piggott%
%	\thanks{ADCS Integrated Simulation Software Lead, Laboratory for Atmospheric and Space Physics, University of Colorado Boulder.}\\
%	{\normalsize\itshape
%		Laboratory for Atmospheric and Space Physics, Boulder, Colorado, 80309, US}
}


\abstract{
	A large portion of spacecraft missions have stringent pointing, attitude knowledge, and control requirements. This results in the necessity of high fidelity dynamics modeled in the numerical simulation of the spacecraft. A crucial aspect of this high fidelity is modeling the components susceptible to flexing. One method is to model flexible behavior as rigid bodies connected to the hub by a single degree of freedom torsional hinge. Although this an approximation to flexing behavior, it allows for simulations to be computationally efficient and does not require FEA or solving partial differential equations. A draw back from this approach is only one mode can be present and is a first order approximation to bending. In this paper, the equations of motion for dual connected hinged rigid bodies are developed and allows for a closer approximation to bending and gives more opportunity to exam the different effects of modal frequencies. In addition, computational efficiency, is considered while deriving the EOMs and the back-substitution method is implemented for this dynamical system. This gives insight into how interconnected bodies effect the dynamic structure of the system.
}

\begin{document}
	
	
	\maketitle

	\section*{Nomenclature}
	\noindent\begin{tabular}{@{}lcl@{}}
		$\mathcal{N}$ &=& inertial reference frame \\
		$\mathcal{B}$ &=& body reference frame \\
		$\mathcal{H}_{i1}, \mathcal{H}_{i2}$ &=& first and second hinge reference frames for the $i^\text{th}$ panel pair \\
		$\mathcal{S}_{i1}, \mathcal{S}_{i2}$ &=& first and second solar panel reference frame for the $i^\text{th}$ panel pair \\
		$\{\hat{\bm b}_1,\hat{\bm b}_2,\hat{\bm b}_3\}$  &=&  body frame basis vectors\\
		$\{\hat{\bm n}_1,\hat{\bm n}_2,\hat{\bm n}_3\}$  &=&  inertial frame basis vectors\\
		$\{\hat{\bm s}_{i1\_1},\hat{\bm s}_{i1\_2},\hat{\bm s}_{i1\_3}\}$, $\{\hat{\bm s}_{i1\_1},\hat{\bm s}_{i1\_2},\hat{\bm s}_{i1\_3}\}$  &=&  solar panel frame basis vectors for first and second panel of the $i^\text{th}$ panel pair \\
		$\{\hat{\bm h}_{i1\_1},\hat{\bm h}_{i1\_2},\hat{\bm h}_{i1\_3}\}$, $\{\hat{\bm h}_{i1\_1},\hat{\bm h}_{i1\_2},\hat{\bm h}_{i1\_3}\}$ &=& hinge frame basis vectors for the first and second hinge of the $i^\text{th}$ panel pair \\
		$\mathcal{B}_{c}$ &=& center of mass of the spacecraft hub \\
		$\mathcal{C}$ &=& center of mass of the entire spacecraft \\
		$c$ &=& vector to spacecraft center of mass  \\
		$S_{c,i1}$,  $S_{c,i2}$ &=& center of mass of the first and second panels of the $i^\text{th}$ panel pair \\
		$d_{i1}$, $d_{i2}$ &=& distance from hinge to panel center of mass for first, second panels of the $i^\text{th}$ pair \\
		$m_{sc}$ &=& mass of the spacecraft \\
		$N_{S}$ &=& number of solar panel pairs \\
		$\bm r$ &=& position vector \\
		$\bm \omega $ &=& angular rate \\
		$\dot{\bm{v}}$ &=& time derivative of any vector, v, with respect to the inertial frame \\
		$\bm{v}'$ &=& time derivative of any vector, v, with respect to the body frame \\
		$\theta_{i1}$, $\theta_{i2}$ &=& angular displacement of panel 1 and 2 of the $i^\text{th}$ panel pair \\
		$l_{i1}$, $l_{i2}$ &=& the length of the first and second panel of the $i^\text{th}$ panel pair \\
		$I$ &=& moment of inertia \\
		$\bm L_B$ &=& total external torque about any point B \\
		$\tilde{m}$ &=& tilde matrix of any matrix m. See reference 11 \\
	    $\tau$ &=& torque \\
	    $k_{i1}$, $k_{i2}$ &=& rotational spring constant about hinges 1 and 2 of the $i^\text{th}$ panel pair \\
	    $c_{i1}$, $c_{i2}$ &=& damping constant about hinges 1 and 2 of the $i^\text{th}$ panel pair \\  
		
\end{tabular} 
	
	\section{Introduction}
	Spacecraft come in many shapes and sizes and some spacecraft have large appended solar panels or antennas. Typically these objects are connected to the spacecraft as cantilevered elements, therefore they are susceptible to flexing behavior. This behavior needs to be included in the dynamics. Often the spacecraft is assumed to be a rigid body, but this assumption will degrade the fidelity of the simulation if there are certain components that will flex. Flexing will impact both the translational and rotational motion (and associated stability margins) of the spacecraft, as well as sensor modeling such as accelerometers and rate gyros. For simulation and analysis purposes, flexing is very important because it can impact performance, requirements and success of the mission.

There are many different ways to model flexible dynamics. One method is to assume that the primary impact will be on the attitude dynamics of the spacecraft so the translational motion coupling can be ignored. Also, in some scenarios the effects of flexing can be assumed to only impact one plane of rotation, therefore one method is to constrain the motion to 1D rotational motion. \cite{sidi1997spacecraft}. This approach allows the flexing body to be modeled as a finite number of masses on a cantilevered beam and allows for different frequency modes to be present. \cite{sidi1997spacecraft}. This derivation results in a transfer function that is useful in determining the stability and frequency response due to different inputs. However, it neglects the cross coupling affect on the other rotational axes, and the effect on translational motion. This method is helpful in the early stages of a mission, but lacks fidelity and is limited in its application.

In contrast, the field of multi-body dynamics has extensive research on modeling flexible dynamics and the equations of motion presented are generalized for complex and diverse problems. This results in requiring derivation of equations because of generality. \cite{angeles2014kinematics, fleischer1974attitude,jerkovsky1978structure} These methods are required for unique and complex systems because the equations of motion depend on how many joints that are interconnected. For example, in robotic systems, the number of interconnected joints varies widely, and the equations of motion are specific to that system. \cite{zarafshan2010manipulation,moosavian2004explicit} Since there are many spacecraft that have similar designs with appended solar panels, there is a need to develop equations of motion that could be applied to these spacecraft. 

Similar to this paper, multiple publications present models of spacecraft dynamics with appended solar panels. \cite{kuang2004nonlinear,wallrapp2002simulation,wie1986modeling} However, this previous research is mainly focused on the deployment of solar panels and how the deployment affects the dynamics of the spacecraft. \cite{kuang2004nonlinear,wallrapp2002simulation,wie1986modeling}.

This paper can be seen as a follow on to Reference\cite{Allard2016rz} which introduces a method of modeling the flexible dynamics of the solar panels by assuming that the hub of the spacecraft and the solar panels are rigid bodies, but the solar panels are connected to the hub by single degree-of-freedom torsional springs. This approach does not allow for interconnected solar panels, only ones that are directly connected to the hub. The work presented in this paper, follows a similar path, but is considering dual-linked solar panels. This can give a closer approximation to flexing behavior and gives the ability to analyze two natural frequencies at one time, instead of being restricted to one. In addition, the back substitution method is expanded to include this dual-linked spacecraft system which allow for computational efficiency modularity in software architecture design.
		
	\section{Problem Statement}
	
	The formulation assumes that there is a rigid hub, with $N_S$ dual-linked solar panels (or appended rigid bodies) and Subscript $i$ is used to indicated the $i_\text{th}$ pair of solar panels. Figure~\ref{fig:Flex_Slosh_Figure} displays the frame and variable definitions used for this formulation.

	\begin{figure}
		\centering
		\includegraphics[]{Figures/Flex_Slosh_Figure}
		\caption{Frame and variable definitions used for formulation}
		\label{fig:Flex_Slosh_Figure}
	\end{figure} 
	
	There are six coordinate frames defined for this formulation. The inertial reference frame is indicated by \frameDefinition{N}. The body fixed coordinate frame, \frameDefinition{B}, which is anchored to the hub and can be oriented in any direction. The first solar panel frame, $\mathcal{S}_{i1}:\{\hat{\bm s}_{i1,1},\hat{\bm s}_{i1,2},\hat{\bm s}_{i1,3}\}$, is a frame with its origin located at its corresponding hinge location, $H_{i1}$. The $\mathcal{S}_{i1}$ frame is oriented such that $\hat{\bm{s}}_{i1,1}$ points antiparallel to the center of mass of the first solar panel, $S_{c,i1}$. The $\hat{\bm{s}}_{i1,2}$ axis is defined as the rotation axis that would yield a positive $\theta_{i1}$ using the right-hand rule. The distance from point $H_{i1}$ to point $S_{c,i1}$ is defined as $d_{i1}$. The total length of the first panel is $l_{i1}$ The hinge frame, $\mathcal{H}_{i1}:\{\hat{\bm h}_{i1,1}, \hat{\bm h}_{i1,2}, \hat{\bm h}_{i1,3} \}$, is a frame fixed with respect to the body frame, and is equivalent to the respective $\mathcal{S}_{i1}$ frame when the corresponding solar panel is undeflected.

	The other two frames $\mathcal{S}_{i2}$ and $\mathcal{H}_{i2}$ are frames attached to the second solar panel. The $\mathcal{H}_{i2}$ frame is located at the joint between the two solar panels and $\hat{\bm h}_{i1,2} = \hat{\bm h}_{i2,2}$. The $\hat{\bm h}_{i2,1}$ completes the definition of the $\mathcal{H}_{i2}$ frame and can be oriented in any direction while orthogonal to the $\hat{\bm h}_{i2,2}$ axis. This allows for the simulation to model undeployed solar panels for example and defines the undeflected direction of the second solar panel. The $\mathcal{S}_{i2}$ by being equal to the $\mathcal{H}_{i2}$ when the second solar panel is undeflected from its equilibrium point and rotates about the $\hat{\bm h}_{i2,2}$ axis.
	
	There are a few more key locations that need to be defined. Point $B$ is the origin of the body frame, and can have any location with respect to the hub. Point $B_c$ is the location of the center of mass of the rigid hub.
	
	Using the variables and frames defined, the following section outlines the derivation of equations of motion for the spacecraft. 
	
	\section{Derivation of Equations of Motion}
	
	\subsection{Rigid Spacecraft Hub Translational Motion}
	
	Following a similar derivation as in previous work \cite{Allard2016rz}, the derivation begins with Newton's first law for the center of mass of the spacecraft.
	\begin{equation}
	\ddot{\bm r}_{C/N} = \frac{\bm{F}}{m_{\text{\text{sc}}}}
	\label{eq:Newtons1Law}
	\end{equation}
	Ultimately the acceleration of the body frame or point $B$ is desired
	\begin{equation}
	\ddot{\bm r}_{B/N} = \ddot{\bm r}_{C/N}-\ddot{\bm c}
	\label{eq:RcRbacc}
	\end{equation}
	The definition of $\bm{c}$ the location of the center of mass of the entire spacecraft, can be seen in Eq. (\ref{eq:c}).
	\begin{equation}
	\bm{c} = \frac{1}{m_{\text{sc}}}\Big[m_{\text{\text{hub}}}\bm{r}_{B_{c}/B} +\sum_{i=1}^{N_{S}}\big(m_{\text{sp}_{i1}}\bm{r}_{S_{c,i1}/B}+m_{\text{sp}_{i2}}\bm{r}_{S_{c,i2}/B}\big)\Big]
	\label{eq:c} 
	\end{equation}
	To find the inertial time derivative of $\bm{c}$, it is first necessary to find the time derivative of $\bm{c}$ with respect to the body frame. A time derivative of any vector, $\bm{v}$, with respect to the body frame is denoted by $\bm{v}'$; the inertial time derivative is labeled as $\dot{\bm{v}}$. The first and second body-relative time derivatives of $\bm{c}$ can be seen in Eqs. (\ref{eq:cprime}) and (\ref{eq:cdprime}).
	\begin{align}
	\bm{c}' &= \frac{1}{m_{\text{sc}}}\sum_{i=1}^{N_{S}}\big(m_{\text{sp}_{i1}}\bm{r}'_{S_{c,i1}/B}+m_{\text{sp}_{i2}}\bm{r}'_{S_{c,i2}/B}\big)
	\label{eq:cprime}
\\
	\bm{c}'' &= \frac{1}{m_{\text{sc}}}\sum_{i=1}^{N_{S}}\big(m_{\text{sp}_{i1}}\bm{r}''_{S_{c,i1}/B}+m_{\text{sp}_{i2}}\bm{r}''_{S_{c,i2}/B}\big)
	\label{eq:cdprime}
	\end{align}
	The vector $\bm{r}_{S_{c,i1}/B}$ is readily defined using the $\hat{\bm{s}}_{i,1}$ axis
	\begin{equation}
	\bm{r}_{S_{c,{i1}}/B} = 	\bm{r}_{H_{i1}/B} -d_{i1} \bm{\hat{s}}_{i1,1}
	\label{eq:rcgspi1}
	\end{equation}
	The vector $\bm{r}_{S_{c,i2}/B}$ is defined similarly
	\begin{equation}
	\bm{r}_{S_{c,{i2}}/B} = 	\bm{r}_{H_{i1}/B} -l_{i1} \bm{\hat{s}}_{i1,1} - d_{i2}\bm{\hat{s}}_{i2,1}
	\label{eq:rcgspi2}
	\end{equation}
	Now the first and second time derivatives with respect to the body frame of $\bm{r}_{S_{c,i1}/B}$ are taken
	\begin{align}
	\bm{r}'_{S_{c,i1}/B} &= d_{i1} \dot{\theta}_{i1} \bm{\hat{s}}_{i1,3}
	\label{eq:drcgspi1}
\\
	\bm{r}''_{S_{c,i1}/B} &= d_{i1} \left(\ddot{\theta}_{i1} \bm{\hat{s}}_{i1,3} + \dot{\theta}_{i1}^2 \bm{\hat{s}}_{i1,1}\right)
	\label{eq:ddrcgspi1}
	\end{align}
	Similarly the body time derivatives of $\bm{r}_{S_{c,i2}/B}$ are defined in the following
	\begin{align}
	\bm{r}'_{S_{c,i2}/B} &= l_{i1} \dot{\theta}_{i1} \bm{\hat{s}}_{i1,3} + d_{i2}\big(\dot{\theta}_{i1} + \dot{\theta}_{i2}\big)\bm{\hat{s}}_{i2,3}
	\label{eq:drcgspi2}
\\
	\bm{r}''_{S_{c,i2}/B} &= l_{i1} \left(\ddot{\theta}_{i1} \bm{\hat{s}}_{i1,3} + \dot{\theta}_{i1}^2 \bm{\hat{s}}_{i1,1}\right) + d_{i2}\big(\ddot{\theta}_{i1} + \ddot{\theta}_{i2}\big)\bm{\hat{s}}_{i2,3} + d_{i2}\big(\dot{\theta}_{i1} + \dot{\theta}_{i2}\big)^2\bm{\hat{s}}_{i2,1}
	\label{eq:ddrcgspi2}
	\end{align}
	Eqs.~\eqref{eq:cprime} and ~\eqref{eq:cdprime} are next reformulated to include these new definitions:
	\begin{equation}
	\bm{c}' = \frac{1}{m_{\text{sc}}}\sum_{i=1}^{N_{S}}\bigg(m_{\text{sp}_{i1}}\Big[d_{i1} \dot{\theta}_{i1} \bm{\hat{s}}_{i1,3}\Big]+m_{\text{sp}_{i2}}\Big[l_{i1} \dot{\theta}_{i1} \bm{\hat{s}}_{i1,3} + d_{i2}\big(\dot{\theta}_{i1} + \dot{\theta}_{i2}\big)\bm{\hat{s}}_{i2,3}\Big]\bigg)
	\label{eq:cprime2}
	\end{equation}
	\begin{multline}
	\bm{c}'' = \frac{1}{m_{\text{sc}}}\sum_{i=1}^{N_{S}}\bigg(m_{\text{sp}_{i1}}d_{i1} \big(\ddot{\theta}_{i1} \bm{\hat{s}}_{i1,3} + \dot{\theta}_{i1}^2 \bm{\hat{s}}_{i1,1}\big)\\
	+m_{\text{sp}_{i2}}\Big[l_{i1} \left(\ddot{\theta}_{i1} \bm{\hat{s}}_{i1,3} + \dot{\theta}_{i1}^2 \bm{\hat{s}}_{i1,1}\right) + d_{i2}\big(\ddot{\theta}_{i1} + \ddot{\theta}_{i2}\big)\bm{\hat{s}}_{i2,3} + d_{i2}\big(\dot{\theta}_{i1} + \dot{\theta}_{i2}\big)^2\bm{\hat{s}}_{i2,1}\Big]\bigg)
	\label{eq:cdprime2}
	\end{multline}
	Using the transport theorem\cite{schaub} yields the following definition for $\ddot{\bm c}$
	\begin{equation}
	\ddot{\bm c} = \bm{c}'' + 2\bm\omega_{\cal B/N}\times\bm{c}'+\dot{\bm\omega}_{\cal B/N}\times\bm{c}+\bm\omega_{\cal B/N}\times\left(\bm\omega_{\cal B/N}\times\bm{c}\right)
	\label{eq:cddot}
	\end{equation}
	Eq.~\eqref{eq:RcRbacc} is updated to include Eq.~\eqref{eq:cddot}
	\begin{equation}
	\ddot{\bm r}_{B/N} = \ddot{\bm r}_{C/N}-\bm{c}'' - 2\bm\omega_{\cal B/N}\times\bm{c}'-\dot{\bm\omega}_{\cal B/N}\times\bm{c}-\bm\omega_{\cal B/N}\times\left(\bm\omega_{\cal B/N}\times\bm{c}\right)
	\label{eq:Rbddot}
	\end{equation}
	Substituting Eq.\eqref{eq:cdprime2} into Eq.\eqref{eq:Rbddot} and moving the second order state variables to the left hand side results in
	\begin{multline}
	\ddot{\bm r}_{B/N} + \dot{\bm\omega}_{\cal B/N}\times\bm{c} +  \frac{1}{m_{\text{sc}}}\sum_{i=1}^{N_{S}}\bigg(\Big[m_{\text{sp}_{i1}}d_{i1} \bm{\hat{s}}_{i1,3} +m_{\text{sp}_{i2}}l_{i1} \bm{\hat{s}}_{i1,3}+m_{\text{sp}_{i2}} d_{i2}\bm{\hat{s}}_{i2,3}\Big]\ddot{\theta}_{i1} +m_{\text{sp}_{i2}} d_{i2} \bm{\hat{s}}_{i2,3}\ddot{\theta}_{i2}\bigg) \\
	= \ddot{\bm r}_{C/N}-\frac{1}{m_{\text{sc}}}\sum_{i=1}^{N_{S}}\bigg(m_{\text{sp}_{i1}}d_{i1} \dot{\theta}_{i1}^2 \bm{\hat{s}}_{i1,1} +m_{\text{sp}_{i2}}\Big[l_{i1} \dot{\theta}_{i1}^2 \bm{\hat{s}}_{i1,1} + d_{i2}\big(\dot{\theta}_{i1} + \dot{\theta}_{i2}\big)^2\bm{\hat{s}}_{i2,1}\Big]\bigg) \\
	- 2\bm\omega_{\cal B/N}\times\bm{c}'-\bm\omega_{\cal B/N}\times\left(\bm\omega_{\cal B/N}\times\bm{c}\right)
	\label{eq:Rbddot2}
	\end{multline}
	Introducing the tilde matrix\cite{schaub} to replace the cross product operators and multiplying both sides by $m_\text{sc}$ simplifies the equation to
		\begin{multline}
	m_\text{sc} \ddot{\bm r}_{B/N} -m_\text{sc} [\tilde{\bm{c}}] \dot{\bm\omega}_{\cal B/N} +  \sum_{i=1}^{N_{S}}\bigg(\Big[m_{\text{sp}_{i1}}d_{i1} \bm{\hat{s}}_{i1,3} +m_{\text{sp}_{i2}}l_{i1} \bm{\hat{s}}_{i1,3}+m_{\text{sp}_{i2}} d_{i2}\bm{\hat{s}}_{i2,3}\Big]\ddot{\theta}_{i1} +m_{\text{sp}_{i2}} d_{i2} \bm{\hat{s}}_{i2,3}\ddot{\theta}_{i2}\bigg) \\
	= \bm F - 2m_\text{sc} [\tilde{\bm\omega}_{\cal B/N}] \bm c'- m_\text{sc} [\tilde{\bm\omega}_{\cal B/N}][\tilde{\bm\omega}_{\cal B/N}]\bm{c}\\
	-\sum_{i=1}^{N_{S}}\bigg(m_{\text{sp}_{i1}}d_{i1} \dot{\theta}_{i1}^2 \bm{\hat{s}}_{i1,1} +m_{\text{sp}_{i2}}\Big[l_{i1} \dot{\theta}_{i1}^2 \bm{\hat{s}}_{i1,1} + d_{i2}\big(\dot{\theta}_{i1} + \dot{\theta}_{i2}\big)^2\bm{\hat{s}}_{i2,1}\Big]\bigg) 
	\label{eq:Rbddot3}
	\end{multline}
	
	Equation~\eqref{eq:Rbddot3} is the translational motion equation and is the first EOM needed to describe the motion of the spacecraft. The following section develops the rotational EOM.
	
	\subsection{Rigid Spacecraft Hub Rotational Motion}
	
	Starting with Euler's equation when the body fixed coordinate frame origin is not coincident with the center of mass of the body\cite{schaub}
	\begin{equation}
	\bm{\dot{H}}_{\text{sc},B} = \bm{L}_B+m_{\text{\text{sc}}}\ddot{\bm r}_{B/N}\times\bm{c}
	\label{eq:Euler}
	\end{equation}
	where $\bm{L}_B$ is the total external torque about point $B$. The definition of the angular momentum vector of the spacecraft about point $B$ is
	\begin{multline}
	\bm{H}_{\text{sc},B} = [I_{\text{hub},B_c}] \bm\omega_{\cal B/N} + m_{\text{hub}} \bm{r}_{B_c/B}\times\bm{\dot{r}}_{B_c/B} \\ +\sum\limits_{i=1}^{N_S}\Big( [I_{\text{sp}_{i1},S_{c,i1}}] \bm\omega_{\cal B/N} + \dot{\theta}_{i1} I_{s_{i1,2}}\bm{\hat{s}}_{i1,2}+m_{\text{sp}_{i1}} \bm{r}_{S_{c,i1}/B} \times \dot{\bm r}_{S_{c,i1}/B}\\
	+ [I_{\text{sp}_{i2},S_{c,i2}}] \bm\omega_{\cal B/N} + \big(\dot{\theta}_{i1} + \dot{\theta}_{i2}\big) I_{s_{i2,2}}\bm{\hat{s}}_{i2,2}+m_{\text{sp}_{i2}} \bm{r}_{S_{c,i2}/B} \times \dot{\bm r}_{S_{c,i2}/B}\Big)
	\label{eq:Hb2}
	\end{multline}
	Both solar panel inertia's about their center of masses' are assumed to be defined along principal inertia axes and are of the form
	\begin{equation}
	[I_{\text{sp}_{i1},S_{c,i1}}] = \leftexp{S}{\begin{bmatrix}
		I_{s_{i1,1}} & 0 & 0 \\
		0 & I_{s_{i1,2}} & 0 \\
		0 & 0 & I_{s_{i1,3}}
		\end{bmatrix}}
	\label{eq:IspMatrix}
	\end{equation}
	
	\begin{equation}
	[I_{\text{sp}_{i2},S_{c,i2}}] = \leftexp{S}{\begin{bmatrix}
		I_{s_{i2,1}} & 0 & 0 \\
		0 & I_{s_{i2,2}} & 0 \\
		0 & 0 & I_{s_{i2,3}}
		\end{bmatrix}}
	\end{equation}
	Now the inertial time derivative of Eq. \eqref{eq:Hb2} is taken and yields
	\begin{multline}
	\dot{\bm{H}}_{\text{sc},B} = [I_{\text{hub},B_c}] \dot{\bm\omega}_{\cal B/N} + \bm\omega_{\cal B/N} \times [I_{\text{hub},B_c}] \bm\omega_{\cal B/N} + m_{\text{hub}} \bm{r}_{B_c/B}\times\ddot{\bm r}_{B_c/B}\\ +\sum\limits_{i=1}^{N_S} \biggl( [I'_{\text{sp}_{i1},S_{c,i1}}] \bm\omega_{\cal B/N} + [I_{\text{sp}_{i1},S_{c,i1}}] \dot{\bm\omega}_{\cal B/N} + \bm\omega_{\cal B/N} \times [I_{\text{sp}_{i1},S_{c,i1}}] \bm\omega_{\cal B/N}\\ 
	+ \ddot{\theta}_{i1} I_{s_{i1,2}}\bm{\hat{s}}_{i1,2}+ \bm\omega_{\cal B/N} \times \dot{\theta}_{i1} I_{s_{i1,2}}\bm{\hat{s}}_{i1,2} +m_{\text{sp}_{i1}} \bm{r}_{S_{c,i1}/B} \times \ddot{\bm r}_{S_{c,i1}/B}\\
	+ [I'_{\text{sp}_{i2},S_{c,i2}}] \bm\omega_{\cal B/N} + [I_{\text{sp}_{i2},S_{c,i2}}] \dot{\bm\omega}_{\cal B/N} + \bm\omega_{\cal B/N} \times [I_{\text{sp}_{i2},S_{c,i2}}] \bm\omega_{\cal B/N}\\ 
	+ \big(\ddot{\theta}_{i1} + \ddot{\theta}_{i2}\big) I_{s_{i2,2}}\bm{\hat{s}}_{i2,2}+ \bm\omega_{\cal B/N} \times \big(\dot{\theta}_{i1} + \dot{\theta}_{i2}\big) I_{s_{i2,2}}\bm{\hat{s}}_{i2,2} +m_{\text{sp}_{i1}} \bm{r}_{S_{c,i2}/B} \times \ddot{\bm r}_{S_{c,i2}/B}\biggr)
	\label{eq:Hbdot}
	\end{multline}
	The terms $\ddot{\bm r}_{B_c/B}$, $\ddot{\bm r}_{S_{c,i1}/B}$ and $\ddot{\bm r}_{S_{c,i2}/B}$ are found using the transport theorem and knowing that $\bm{r}_{B_c/B}$ is fixed with respect to the body frame.
	\begin{align}
	\ddot{\bm r}_{B_c/B} &= \bm{\dot{\omega}}_{\cal B/N} \times \bm{r}_{B_c/B} + \bm\omega_{\cal B/N} \times (\bm\omega_{\cal B/N} \times \bm{r}_{B_c/B})
	\label{eq:rbddot}
	\\
	\ddot{\bm r}_{S_{c,i1}/B} &= \bm{r}''_{S_{c,i1}/B} + 2 \bm\omega_{\cal B/N} \times \bm{r}'_{S_{c,i1}/B} +  \dot{\bm\omega}_{\cal B/N} \times \bm{r}_{S_{c,i1}/B} + \bm\omega_{\cal B/N} \times (\bm\omega_{\cal B/N} \times \bm{r}_{S_{c,i1}/B})
	\label{eq:rsddot}
    \\
	\ddot{\bm r}_{S_{c,i2}/B} &= \bm{r}''_{S_{c,i2}/B} + 2 \bm\omega_{\cal B/N} \times \bm{r}'_{S_{c,i2}/B} +  \dot{\bm\omega}_{\cal B/N} \times \bm{r}_{S_{c,i2}/B} + \bm\omega_{\cal B/N} \times (\bm\omega_{\cal B/N} \times \bm{r}_{S_{c,i2}/B})
	\label{eq:rsddot2}
	\end{align}
	Incorporating Eqs.~\eqref{eq:rbddot} -~\eqref{eq:rsddot2} into Eq.~\eqref{eq:Hbdot} results in

	\begin{multline}
	\dot{\bm{H}}_{\text{sc},B} = [I_{\text{hub},B_c}] \dot{\bm\omega}_{\cal B/N} + \bm\omega_{\cal B/N} \times [I_{\text{hub},B_c}] \bm\omega_{\cal B/N} + m_{\text{hub}} \bm{r}_{B_c/B}\times ( \dot{\bm\omega}_{\cal B/N}\times \bm{r}_{B_c/B}) \\+ m_{\text{hub}} \bm{r}_{B_c/B}\times\Big[\bm\omega_{\cal B/N} \times (\bm\omega_{\cal B/N} \times \bm{r}_{B_c/B})\Big] +\sum\limits_{i=1}^{N_S} \biggl( [I'_{\text{sp}_{i1},S_{c,i1}}] \bm\omega_{\cal B/N} + [I_{\text{sp}_{i1},S_{c,i1}}] \dot{\bm\omega}_{\cal B/N} \\+ \bm\omega_{\cal B/N} \times [I_{\text{sp}_{i1},S_{c,i1}}] \bm\omega_{\cal B/N} 
	+ \ddot{\theta}_{i1} I_{s_{i1,2}}\bm{\hat{s}}_{i1,2}+ \bm\omega_{\cal B/N} \times \dot{\theta}_{i1} I_{s_{i1,2}}\bm{\hat{s}}_{i1,2} 
	+m_{\text{sp}_{i1}} \bm{r}_{S_{c,i1}/B} \times \bm{r}''_{S_{c,i1}/B}\\ + 2 m_{\text{sp}_{i1}} \bm{r}_{S_{c,i1}/B} \times \Big( \bm\omega_{\cal B/N} \times \bm{r}'_{S_{c,i1}/B}\Big)
	+m_{\text{sp}_{i1}} \bm{r}_{S_{c,i1}/B} \times \Big( \dot{\bm\omega}_{\cal B/N} \times \bm{r}_{S_{c,i1}/B}\Big) \\+m_{\text{sp}_{i1}} \bm{r}_{S_{c,i1}/B} \times \Big[ \bm\omega_{\cal B/N} \times (\bm\omega_{\cal B/N} \times \bm{r}_{S_{c,i1}/B})\Big]
	+ [I'_{\text{sp}_{i2},S_{c,i2}}] \bm\omega_{\cal B/N} + [I_{\text{sp}_{i2},S_{c,i2}}] \dot{\bm\omega}_{\cal B/N} \\+ \bm\omega_{\cal B/N} \times [I_{\text{sp}_{i2},S_{c,i2}}] \bm\omega_{\cal B/N} 
	+ \big(\ddot{\theta}_{i1} + \ddot{\theta}_{i2}\big) I_{s_{i2,2}}\bm{\hat{s}}_{i2,2}+ \bm\omega_{\cal B/N} \times \big(\dot{\theta}_{i1} + \dot{\theta}_{i2}\big) I_{s_{i2,2}}\bm{\hat{s}}_{i2,2} \\
	+m_{\text{sp}_{i2}} \bm{r}_{S_{c,i2}/B} \times \bm{r}''_{S_{c,i2}/B} + 2 m_{\text{sp}_{i2}} \bm{r}_{S_{c,i2}/B} \times \Big(\bm\omega_{\cal B/N} \times \bm{r}'_{S_{c,i2}/B}\Big)\\
	 +m_{\text{sp}_{i2}} \bm{r}_{S_{c,i2}/B} \times \Big(\dot{\bm\omega}_{\cal B/N} \times \bm{r}_{S_{c,i2}/B}\Big) +m_{\text{sp}_{i2}} \bm{r}_{S_{c,i2}/B} \times \Big[\bm\omega_{\cal B/N} \times (\bm\omega_{\cal B/N} \times \bm{r}_{S_{c,i2}/B})\Big]\biggr)
	\label{eq:Hbdot3}
\end{multline}	

	Applying the parallel axis theorem the following inertia tensor terms are defined as
	\begin{align}
	[I_{\text{hub},B}] &= [I_{\text{hub},B_c}] + m_{\text{hub}}[\bm{\tilde{r}}_{B_c/B}] [\bm{\tilde{r}}_{B_c/B}]^T
	\label{eq:IHubB}
	\\
	[I_{\text{sp}_{i1},B}] &= [I_{\text{sp}_{i1},S_{c,i1}}] + m_{\text{sp}_{i1}}[\bm{\tilde{r}}_{S_{c,i1}/B}] [\bm{\tilde{r}}_{S_{c,{i1}}/B}]^T
	\\
	[I_{\text{sp}_{i2},B}] &= [I_{\text{sp}_{i2},S_{c,i2}}] + m_{\text{sp}_{i2}}[\bm{\tilde{r}}_{S_{c,i2}/B}] [\bm{\tilde{r}}_{S_{c,{i2}}/B}]^T
	\\
	[I_{\text{sc},B}] &= [I_{\text{hub},B}] + \sum\limits_{i=1}^{N_S}\Big( [I_{\text{sp}_{i1},B}] + [I_{\text{sp}_{i2},B}]\Big)
	\label{eq:IscB}
	\end{align}
	Because the tilde matrices are skew-symmetric, taking the body-relative time derivative of Equation~\eqref{eq:IscB} yields
	\begin{multline}
		[I'_{\text{sc},B}] = \sum\limits_{i=1}^{N_S} \Big[[I'_{\text{sp}_{i1},S_{c,i1}}] - m_{\text{sp}_{i1}}\left([\bm{\tilde{r}}'_{S_{c,{i1}}/B}] [\bm{\tilde{r}}_{S_{c,{i1}}/B}] + [\bm{\tilde{r}}_{S_{c,{i1}}/B}] [\bm{\tilde{r}}'_{S_{c,{i1}}/B}]\right)\\
		+[I'_{\text{sp}_{i2},S_{c,i2}}] - m_{\text{sp}_{i2}}\left([\bm{\tilde{r}}'_{S_{c,{i2}}/B}] [\bm{\tilde{r}}_{S_{c,{i2}}/B}] + [\bm{\tilde{r}}_{S_{c,{i2}}/B}] [\bm{\tilde{r}}'_{S_{c,{i2}}/B}]\right)\Big]
		\label{eq:IprimeScB}
	\end{multline}
	$[I'_{\text{sp}_{i1},S_{c,i1}}]$ needs to be defined and can be conveniently expressed by leveraging the assumption that the inertia matrix is diagonal (as seen in Eq. \eqref{eq:IspMatrix}) and is written in terms of its base vectors:
	\begin{equation}
	[I_{\text{sp}_{i1},S_{c,i1}}] = I_{s_{i1,1}}\hat{\bm s}_{i1,1}\hat{\bm s}_{i1,1}^{T}+I_{s_{i1,2}}\hat{\bm s}_{i1,2}\hat{\bm s}_{i1,2}^{T}+I_{s_{i1,3}}\hat{\bm s}_{i1,3}\hat{\bm s}_{i1,3}^{T}
	\label{eq:iprime}
	\end{equation}
	Taking the body time derivative of Eq.~\eqref{eq:iprime} results in
	\begin{multline}
	[I'_{\text{sp}_{i1},S_{c,i1}}] = I_{s_{i1,1}}\hat{\bm s}'_{i1,1}\hat{\bm s}_{i1,1}^{T}+I_{s_{i1,1}}\hat{\bm s}_{i1,1}\hat{\bm s}_{i1,1}'^{T}+I_{s_{i1,2}}\hat{\bm s}'_{i1,2}\hat{\bm s}_{i1,2}^{T}\\
	+I_{s_{i1,2}}\hat{\bm s}_{i1,2}\hat{\bm s}_{i1,2}'^{T}+I_{s_{i1,3}}\hat{\bm s}'_{i1,3}\hat{\bm s}_{i1,3}^{T}+I_{s_{i1,3}}\hat{\bm s}_{i1,3}\hat{\bm s}_{i1,3}'^{T}
	\label{eq:iprime2}
	\end{multline}
	Using the transport theorem for each basis vector, j, in the $\bm{\hat{s}}_{i1}$ frame: $\hat{\bm s}'_{i1,j} = \bm\omega_{\cal{S}_{\textit{i1}}/\cal{B}}\times\hat{\bm s}_{i1,j}=\dot{\theta}_{i1}\hat{\bm s}_{i1,2}\times\hat{\bm s}_{i1,j}$, applying this to Eq.~\eqref{eq:iprime2}, evaluating the cross products, and simplifying results in
	\begin{equation}
	[I'_{\text{sp}_{i1},S_{c,i1}}] = \dot{\theta}_{i1}(I_{s_{i1,3}}-I_{s_{i1,1}})(\hat{\bm s}_{i1,1}\hat{\bm s}_{i1,3}^{T}+\hat{\bm s}_{i1,3}\hat{\bm s}_{i1,1}^{T})
	\label{eq:iprime3}
	\end{equation}
	
	Applying the same methodology for $[I'_{\text{sp}_{i2},S_{c,i2}}]$ and using the following definition: $\hat{\bm s}'_{i2,j} = \bm\omega_{\cal{S}_{\textit{i2}}/\cal{B}}\times\hat{\bm s}_{i2,j}=\big(\dot{\theta}_{i1} + \dot{\theta}_{i2} \big)\hat{\bm s}_{i2,2}\times\hat{\bm s}_{i2,j}$ results in 
	\begin{equation}
	[I'_{\text{sp}_{i2},S_{c,i2}}] = \big(\dot{\theta}_{i1}+\dot{\theta}_{i2}\big)(I_{s_{i2,3}}-I_{s_{i2,1}})(\hat{\bm s}_{i2,1}\hat{\bm s}_{i2,3}^{T}+\hat{\bm s}_{i2,3}\hat{\bm s}_{i2,1}^{T})
	\label{eq:iprime4}
	\end{equation}
	Substituting Eq.~\eqref{eq:iprime3} and Eq.~\eqref{eq:iprime4} into Eq.~\eqref{eq:Hbdot3} and using Eq.~\eqref{eq:IscB} to simplify results in Eq.~\eqref{eq:Hbdot4}. The Jacobi Identity, $(\bm a \times \bm b)\times \bm c = \bm a \times (\bm b\times \bm c) - \bm b \times (\bm a\times \bm c)$, is used to combine terms.\\
	Factoring out $\dot{\bm\omega}_{\cal B/N}$ and, selectively, $\bm{\omega}_{\cal B/N}$ and utilizing the tilde matrix transforms Eq. \ref{eq:Hbdot3} into Eq. \ref{eq:Hbdot17} so that $[I_{\text{sc},B}]$ can be extracted.
	\begin{multline}
	\dot{\bm{H}}_{\text{sc},B} = \bigg([I_{\text{hub},B_c}]  - m_{\text{hub}} [\tilde{\bm{r}}_{B_c/B}] [\tilde{\bm{r}}_{B_c/B}] + 
	\sum\limits_{i=1}^{N_S} \Big([I_{\text{sp}_{i1},S_{c,i1}}]+ [I_{\text{sp}_{i2},S_{c,i2}}] - m_{\text{sp}_{i1}} [\tilde{\bm{r}}_{S_{c,i1}/B}] [\tilde{\bm{r}}_{S_{c,i1}/B}] \\
	- m_{\text{sp}_{i2}} [\tilde{\bm{r}}_{S_{c,i2}/B}] [\tilde{\bm{r}}_{S_{c,i2}/B}]   \Big)\bigg)\dot{\bm\omega}_{\cal B/N}	+ \bm\omega_{\cal B/N} \times \bigg([I_{\text{hub},B_c}] - m_{\text{hub}} [\tilde{\bm{r}}_{B_c/B}] [\tilde{\bm{r}}_{B_c/B}] + \\
	\sum\limits_{i=1}^{N_S}\Big(  [I_{\text{sp}_{i1},S_{c,i1}}]+ [I_{\text{sp}_{i2},S_{c,i2}}] - m_{\text{sp}_{i1}} [\tilde{\bm{r}}_{S_{c,i1}/B}] [\tilde{\bm{r}}_{S_{c,i1}/B}] - m_{\text{sp}_{i2}} [\tilde{\bm{r}}_{S_{c,i2}/B}] [\tilde{\bm{r}}_{S_{c,i2}/B}]  \Big) \bigg) \bm\omega_{\cal B/N} \\ 
	+\sum\limits_{i=1}^{N_S} \biggl( [I'_{\text{sp}_{i1},S_{c,i1}}] \bm\omega_{\cal B/N} 
	+ \ddot{\theta}_{i1} I_{s_{i1,2}}\bm{\hat{s}}_{i1,2}+ \bm\omega_{\cal B/N} \times \dot{\theta}_{i1} I_{s_{i1,2}}\bm{\hat{s}}_{i1,2}	+m_{\text{sp}_{i1}} \bm{r}_{S_{c,i1}/B} \times \bm{r}''_{S_{c,i1}/B} \\
	+ 2 m_{\text{sp}_{i1}} \bm{r}_{S_{c,i1}/B} \times \Big( \bm\omega_{\cal B/N} \times \bm{r}'_{S_{c,i1}/B}\Big)	+ [I'_{\text{sp}_{i2},S_{c,i2}}] \bm\omega_{\cal B/N} 
	+ \big(\ddot{\theta}_{i1} + \ddot{\theta}_{i2}\big) I_{s_{i2,2}}\bm{\hat{s}}_{i2,2}+ \\
	\bm\omega_{\cal B/N} \times \big(\dot{\theta}_{i1} + \dot{\theta}_{i2}\big) I_{s_{i2,2}}\bm{\hat{s}}_{i2,2}
	+m_{\text{sp}_{i2}} \bm{r}_{S_{c,i2}/B} \times \bm{r}''_{S_{c,i2}/B} + 2 m_{\text{sp}_{i2}} \bm{r}_{S_{c,i2}/B} \times \Big(\bm\omega_{\cal B/N} \times \bm{r}'_{S_{c,i2}/B}\Big)\biggr)
	\label{eq:Hbdot17}
	\end{multline}
	$[I_{\text{sc},B}]$ is substituted in from Eq. \ref{eq:IHubB} through Eq. \ref{eq:IscB}:
	\begin{multline}
	\dot{\bm{H}}_{\text{sc},B} = [I_{\text{sc},B}]\dot{\bm\omega}_{\cal B/N}	+ \bm\omega_{\cal B/N} \times [I_{\text{sc},B}] \bm\omega_{\cal B/N} \\ 
	+\sum\limits_{i=1}^{N_S} \biggl( [I'_{\text{sp}_{i1},S_{c,i1}}] \bm\omega_{\cal B/N} 
	+ \ddot{\theta}_{i1} I_{s_{i1,2}}\bm{\hat{s}}_{i1,2}+ \bm\omega_{\cal B/N} \times \dot{\theta}_{i1} I_{s_{i1,2}}\bm{\hat{s}}_{i1,2}	+m_{\text{sp}_{i1}} \bm{r}_{S_{c,i1}/B} \times \bm{r}''_{S_{c,i1}/B} \\
	+ 2 m_{\text{sp}_{i1}} \bm{r}_{S_{c,i1}/B} \times \Big( \bm\omega_{\cal B/N} \times \bm{r}'_{S_{c,i1}/B}\Big)	+ [I'_{\text{sp}_{i2},S_{c,i2}}] \bm\omega_{\cal B/N} 
	+ \big(\ddot{\theta}_{i1} + \ddot{\theta}_{i2}\big) I_{s_{i2,2}}\bm{\hat{s}}_{i2,2}+ \\
	\bm\omega_{\cal B/N} \times \big(\dot{\theta}_{i1} + \dot{\theta}_{i2}\big) I_{s_{i2,2}}\bm{\hat{s}}_{i2,2}
	+m_{\text{sp}_{i2}} \bm{r}_{S_{c,i2}/B} \times \bm{r}''_{S_{c,i2}/B} + 2 m_{\text{sp}_{i2}} \bm{r}_{S_{c,i2}/B} \times \Big(\bm\omega_{\cal B/N} \times \bm{r}'_{S_{c,i2}/B}\Big)\biggr)
	\label{eq:Hbdot18}
	\end{multline}
	Splitting the doubled terms:
	\begin{multline}
	\dot{\bm{H}}_{\text{sc},B} = [I_{\text{sc},B}]\dot{\bm\omega}_{\cal B/N}	+ \bm\omega_{\cal B/N} \times [I_{\text{sc},B}] \bm\omega_{\cal B/N} \\ 
	+\sum\limits_{i=1}^{N_S} \biggl( [I'_{\text{sp}_{i1},S_{c,i1}}] \bm\omega_{\cal B/N} + m_{\text{sp}_{i1}} \bm{r}_{S_{c,i1}/B} \times \Big( \bm\omega_{\cal B/N} \times \bm{r}'_{S_{c,i1}/B}\Big)
	+ \ddot{\theta}_{i1} I_{s_{i1,2}}\bm{\hat{s}}_{i1,2}+ \bm\omega_{\cal B/N} \times \dot{\theta}_{i1} I_{s_{i1,2}}\bm{\hat{s}}_{i1,2}	\\
	+m_{\text{sp}_{i1}} \bm{r}_{S_{c,i1}/B} \times \bm{r}''_{S_{c,i1}/B}
	+ m_{\text{sp}_{i1}} \bm{r}_{S_{c,i1}/B} \times \Big( \bm\omega_{\cal B/N} \times \bm{r}'_{S_{c,i1}/B}\Big)	+ [I'_{\text{sp}_{i2},S_{c,i2}}] \bm\omega_{\cal B/N} \\
	+ m_{\text{sp}_{i2}} \bm{r}_{S_{c,i2}/B} \times \Big(\bm\omega_{\cal B/N} \times \bm{r}'_{S_{c,i2}/B}\Big)
	+ \big(\ddot{\theta}_{i1} + \ddot{\theta}_{i2}\big) I_{s_{i2,2}}\bm{\hat{s}}_{i2,2} + 
	\bm\omega_{\cal B/N} \times \big(\dot{\theta}_{i1} + \dot{\theta}_{i2}\big) I_{s_{i2,2}}\bm{\hat{s}}_{i2,2}\\
	+m_{\text{sp}_{i2}} \bm{r}_{S_{c,i2}/B} \times \bm{r}''_{S_{c,i2}/B} + m_{\text{sp}_{i2}} \bm{r}_{S_{c,i2}/B} \times \Big(\bm\omega_{\cal B/N} \times \bm{r}'_{S_{c,i2}/B}\Big)\biggr)
	\label{eq:Hbdot19}
	\end{multline}
	Using the Jacobi Identity again, followed by tilde matrix substitution:
	\begin{multline}
	\dot{\bm{H}}_{\text{sc},B} = [I_{\text{sc},B}]\dot{\bm\omega}_{\cal B/N}	+ \bm\omega_{\cal B/N} \times [I_{\text{sc},B}] \bm\omega_{\cal B/N} \\ 
	+\sum\limits_{i=1}^{N_S} \biggl( [I'_{\text{sp}_{i1},S_{c,i1}}] \bm\omega_{\cal B/N}  - m_{\text{sp}_{i1}}\Big([\tilde{\bm{r}}_{S_{c,i1}/B}][\tilde{\bm{r'}}_{S_{c,i1}/B}] + [\tilde{\bm{r'}}_{S_{c,i1}/B}][\tilde{\bm{r}}_{S_{c,i1}/B}]   \Big)\bm\omega_{\cal B/N}	
	+ \ddot{\theta}_{i1} I_{s_{i1,2}}\bm{\hat{s}}_{i1,2}\\
	+ \bm\omega_{\cal B/N} \times \dot{\theta}_{i1} I_{s_{i1,2}}\bm{\hat{s}}_{i1,2}	+m_{\text{sp}_{i1}} \bm{r}_{S_{c,i1}/B} \times \bm{r}''_{S_{c,i1}/B} 
	+ m_{\text{sp}_{i1}} \bm{r}_{S_{c,i1}/B} \times \Big( \bm\omega_{\cal B/N} \times \bm{r}'_{S_{c,i1}/B}\Big)	\\
	+ [I'_{\text{sp}_{i2},S_{c,i2}}] \bm\omega_{\cal B/N}  - m_{\text{sp}_{i2}}\Big([\tilde{\bm{r}}_{S_{c,i2}/B}] [\tilde{\bm{r'}}_{S_{c,i2}/B}] + [\tilde{\bm{r'}}_{S_{c,i2}/B}] [\tilde{\bm{r}}_{S_{c,i2}/B}]                 \Big)  \bm\omega_{\cal B/N}      
	+ \big(\ddot{\theta}_{i1} + \ddot{\theta}_{i2}\big) I_{s_{i2,2}}\bm{\hat{s}}_{i2,2} \\
	+ \bm\omega_{\cal B/N} \times \big(\dot{\theta}_{i1} + \dot{\theta}_{i2}\big) I_{s_{i2,2}}\bm{\hat{s}}_{i2,2}
	+m_{\text{sp}_{i2}} \bm{r}_{S_{c,i2}/B} \times \bm{r}''_{S_{c,i2}/B} + m_{\text{sp}_{i2}} \bm{r}_{S_{c,i2}/B} \times \Big(\bm\omega_{\cal B/N} \times \bm{r}'_{S_{c,i2}/B}\Big)\biggr)
	\label{eq:Hbdot20}
	\end{multline}
	Factoring out $\bm\omega_{\cal B/N}$, and substituting in from Eq. \ref{eq:IprimeScB} leaves:	
	\begin{multline}
	\dot{\bm{H}}_{\text{sc},B} = [I_{\text{sc},B}] \dot{\bm\omega}_{\cal B/N} + \bm\omega_{\cal B/N} \times [I_{\text{sc},B}] \bm\omega_{\cal B/N} + [I'_{\text{sc},B}] \bm\omega_{\cal B/N}
	 +  \sum\limits_{i=1}^{N_S} \bigg[ \ddot{\theta}_{i1} I_{s_{i1,2}}\bm{\hat{s}}_{i1,2}\\
	+ \bm\omega_{\cal B/N} \times \dot{\theta}_{i1} I_{s_{i1,2}}\bm{\hat{s}}_{i1,2} 
	+m_{\text{sp}_{i1}} \bm{r}_{S_{c,i1}/B} \times \bm{r}''_{S_{c,i1}/B}
	 +m_{\text{sp}_{i1}} \bm\omega_{\cal B/N} \times \left(\bm{r}_{S_{c,i1}/B} \times \bm{r}'_{S_{c,i1}/B}\right)\\
	  +\big(\ddot{\theta}_{i1}+\ddot{\theta}_{i2}\big) I_{s_{i2,2}}\bm{\hat{s}}_{i2,2}
	 + \bm\omega_{\cal B/N} \times \big(\dot{\theta}_{i1}+\dot{\theta}_{i2}\big) I_{s_{i2,2}}\bm{\hat{s}}_{i2,2} \\
	 +m_{\text{sp}_{i2}} \bm{r}_{S_{c,i2}/B} \times \bm{r}''_{S_{c,i2}/B}
	 +m_{\text{sp}_{i2}} \bm\omega_{\cal B/N} \times \left(\bm{r}_{S_{c,i2}/B} \times \bm{r}'_{S_{c,i2}/B}\right)\bigg]
	\label{eq:Hbdot4}
	\end{multline}
	Eqs. (\ref{eq:Euler}) and (\ref{eq:Hbdot4}) are equated and yield
	\begin{multline}
	\bm{L}_B+m_{\text{sc}}\ddot{\bm r}_{B/N}\times\bm{c} = [I_{\text{sc},B}] \dot{\bm\omega}_{\cal B/N} + \bm\omega_{\cal B/N} \times [I_{\text{sc},B}] \bm\omega_{\cal B/N} + [I'_{\text{sc},B}] \bm\omega_{\cal B/N} 
	+  \sum\limits_{i=1}^{N_S} \bigg[ \ddot{\theta}_{i1} I_{s_{i1,2}}\bm{\hat{s}}_{i1,2}\\
	+ \bm\omega_{\cal B/N} \times \dot{\theta}_{i1} I_{s_{i1,2}}\bm{\hat{s}}_{i1,2} 
	+m_{\text{sp}_{i1}} \bm{r}_{S_{c,i1}/B} \times \bm{r}''_{S_{c,i1}/B}
	+m_{\text{sp}_{i1}} \bm\omega_{\cal B/N} \times \left(\bm{r}_{S_{c,i1}/B} \times \bm{r}'_{S_{c,i1}/B}\right)\\
	+\big(\ddot{\theta}_{i1}+\ddot{\theta}_{i2}\big) I_{s_{i2,2}}\bm{\hat{s}}_{i2,2}
	+ \bm\omega_{\cal B/N} \times \big(\dot{\theta}_{i1}+\dot{\theta}_{i2}\big) I_{s_{i2,2}}\bm{\hat{s}}_{i2,2} \\
	+m_{\text{sp}_{i2}} \bm{r}_{S_{c,i2}/B} \times \bm{r}''_{S_{c,i2}/B}
	+m_{\text{sp}_{i2}} \bm\omega_{\cal B/N} \times \left(\bm{r}_{S_{c,i2}/B} \times \bm{r}'_{S_{c,i2}/B}\right)\bigg]
	\label{eq:Hbdot5}
	\end{multline}
	Finally, using tilde matrix and simplifying yields the modified Euler equation, which is the second EOM necessary to describe the motion of the spacecraft.
	\begin{multline}
	[I_{\text{sc},B}] \dot{\bm\omega}_{\cal B/N} = -[\bm{\tilde{\omega}}_{\cal B/N}] [I_{\text{sc},B}] \bm\omega_{\cal B/N} - [I'_{\text{sc},B}] \bm\omega_{\cal B/N} -  \sum\limits_{i=1}^{N_S} \bigg[ \ddot{\theta}_{i1} I_{s_{i1,2}}\bm{\hat{s}}_{i1,2}\\
	+ [\bm{\tilde{\omega}}_{\cal B/N}] \dot{\theta}_{i1} I_{s_{i1,2}}\bm{\hat{s}}_{i1,2} 
	+m_{\text{sp}_{i1}} [\tilde{\bm{r}}_{S_{c,i1}/B}] \bm{r}''_{S_{c,i1}/B}
	+m_{\text{sp}_{i1}} [\bm{\tilde{\omega}}_{\cal B/N}] [\tilde{\bm{r}}_{S_{c,i1}/B}] \bm{r}'_{S_{c,i1}/B}\\
	+\big(\ddot{\theta}_{i1}+\ddot{\theta}_{i2}\big) I_{s_{i2,2}}\bm{\hat{s}}_{i2,2}
	+ [\bm{\tilde{\omega}}_{\cal B/N}] \big(\dot{\theta}_{i1}+\dot{\theta}_{i2}\big) I_{s_{i2,2}}\bm{\hat{s}}_{i2,2} \\
	+m_{\text{sp}_{i2}} [\tilde{\bm{r}}_{S_{c,i2}/B}] \bm{r}''_{S_{c,i2}/B}
	+m_{\text{sp}_{i2}} [\bm{\tilde{\omega}}_{\cal B/N}] [\tilde{\bm{r}}_{S_{c,i2}/B}] \bm{r}'_{S_{c,i2}/B}\bigg]
	+ \bm{L}_B - m_{\text{sc}} [\tilde{\bm{c}}] \ddot{\bm r}_{B/N}
	\label{eq:Final5}
	\end{multline}
	However, it is desirable to place the second order state variables on the left hand side of the equation. Performing some rearranging of Eq.~\eqref{eq:Final5} results in an intermediate step. 
	\begin{multline}
	m_{\text{sc}} [\tilde{\bm{c}}] \ddot{\bm r}_{B/N} + [I_{\text{sc},B}] \dot{\bm\omega}_{\cal B/N} + \sum\limits_{i=1}^{N_S} \bigg[ \ddot{\theta}_{i1} I_{s_{i1,2}}\bm{\hat{s}}_{i1,2} 
	+m_{\text{sp}_{i1}} [\tilde{\bm{r}}_{S_{c,i1}/B}] \bm{r}''_{S_{c,i1}/B}\\
	+\big(\ddot{\theta}_{i1}+\ddot{\theta}_{i2}\big) I_{s_{i2,2}}\bm{\hat{s}}_{i2,2} 
	+m_{\text{sp}_{i2}} [\tilde{\bm{r}}_{S_{c,i2}/B}] \bm{r}''_{S_{c,i2}/B}\bigg] = -[\bm{\tilde{\omega}}_{\cal B/N}] [I_{\text{sc},B}] \bm\omega_{\cal B/N} - [I'_{\text{sc},B}] \bm\omega_{\cal B/N} \\
	-  \sum\limits_{i=1}^{N_S} \bigg[
	[\bm{\tilde{\omega}}_{\cal B/N}] \dot{\theta}_{i1} I_{s_{i1,2}}\bm{\hat{s}}_{i1,2} 
	+m_{\text{sp}_{i1}} [\bm{\tilde{\omega}}_{\cal B/N}] [\tilde{\bm{r}}_{S_{c,i1}/B}] \bm{r}'_{S_{c,i1}/B}\\
	+ [\bm{\tilde{\omega}}_{\cal B/N}] \big(\dot{\theta}_{i1}+\dot{\theta}_{i2}\big) I_{s_{i2,2}}\bm{\hat{s}}_{i2,2}
	+m_{\text{sp}_{i2}} [\bm{\tilde{\omega}}_{\cal B/N}] [\tilde{\bm{r}}_{S_{c,i2}/B}] \bm{r}'_{S_{c,i2}/B}\bigg]
	+ \bm{L}_B 
	\label{eq:Finalint}
	\end{multline}
Then, the second order terms are factored out:
\begin{multline}
m_{\text{sc}} [\tilde{\bm{c}}] \ddot{\bm r}_{B/N} + [I_{\text{sc},B}] \dot{\bm\omega}_{\cal B/N} + \sum\limits_{i=1}^{N_S} \bigg[\Big(
I_{s_{i1,2}}\bm{\hat{s}}_{i1,2} 
+m_{\text{sp}_{i1}} d_{i1} [\tilde{\bm{r}}_{S_{c,i1}/B}] \bm{\hat{s}}_{i1,3} + I_{s_{i2,2}}\bm{\hat{s}}_{i2,2} \\
+m_{\text{sp}_{i2}} l_{i1} [\tilde{\bm{r}}_{S_{c,i2}/B}] \bm{\hat{s}}_{i1,3}
+m_{\text{sp}_{i2}} d_{i2} [\tilde{\bm{r}}_{S_{c,i2}/B}] \bm{\hat{s}}_{i2,3}\Big) \ddot{\theta}_{i1}
+\Big(I_{s_{i2,2}}\bm{\hat{s}}_{i2,2} 
+m_{\text{sp}_{i2}} [\tilde{\bm{r}}_{S_{c,i2}/B}] d_{i2}\bm{\hat{s}}_{i2,3}\Big) \ddot{\theta}_{i2} \bigg]\\
 = -[\bm{\tilde{\omega}}_{\cal B/N}] [I_{\text{sc},B}] \bm\omega_{\cal B/N} - [I'_{\text{sc},B}] \bm\omega_{\cal B/N} \\
-  \sum\limits_{i=1}^{N_S} \bigg[
[\bm{\tilde{\omega}}_{\cal B/N}] \dot{\theta}_{i1} I_{s_{i1,2}}\bm{\hat{s}}_{i1,2} 
+m_{\text{sp}_{i1}} d_{i1} \dot{\theta}_{i1}^2 [\tilde{\bm{r}}_{S_{c,i1}/B}] \bm{\hat{s}}_{i1,1}
+m_{\text{sp}_{i2}} l_{i1} \dot{\theta}_{i1}^2 [\tilde{\bm{r}}_{S_{c,i2}/B}] \bm{\hat{s}}_{i1,1}
+m_{\text{sp}_{i1}} [\bm{\tilde{\omega}}_{\cal B/N}] [\tilde{\bm{r}}_{S_{c,i1}/B}] \bm{r}'_{S_{c,i1}/B}\\
+ [\bm{\tilde{\omega}}_{\cal B/N}] \big(\dot{\theta}_{i1}+\dot{\theta}_{i2}\big) I_{s_{i2,2}}\bm{\hat{s}}_{i2,2}
+m_{\text{sp}_{i2}} d_{i2}\big(\dot{\theta}_{i1} + \dot{\theta}_{i2}\big)^2 [\tilde{\bm{r}}_{S_{c,i2}/B}] \bm{\hat{s}}_{i2,1}
+m_{\text{sp}_{i2}} [\bm{\tilde{\omega}}_{\cal B/N}] [\tilde{\bm{r}}_{S_{c,i2}/B}] \bm{r}'_{S_{c,i2}/B}\bigg]
+ \bm{L}_B
\label{eq:Finalint}
\end{multline}

	The terms $\bm{r}''_{S_{c,i1}/B}$ and $\bm{r}''_{S_{c,i2}/B}$ contain second order state variables, therefore replacing their definition seen in Eqs.~\eqref{eq:ddrcgspi1} and ~\eqref{eq:ddrcgspi2} and simplifying the expression yields
	\begin{multline}
	m_{\text{sc}} [\tilde{\bm{c}}] \ddot{\bm r}_{B/N} + [I_{\text{sc},B}] \dot{\bm\omega}_{\cal B/N} + \sum\limits_{i=1}^{N_S} \bigg[  \big(I_{s_{i1,2}}\bm{\hat{s}}_{i1,2}+m_{\text{sp}_{i1}}d_{i1} [\tilde{\bm{r}}_{S_{c,i1}/B}]   \bm{\hat{s}}_{i1,3} + I_{s_{i2,2}}\bm{\hat{s}}_{i2,2}\\
	+m_{\text{sp}_{i2}}l_{i1} [\tilde{\bm{r}}_{S_{c,i2}/B}]  \bm{\hat{s}}_{i1,3}+m_{\text{sp}_{i2}}d_{i2} [\tilde{\bm{r}}_{S_{c,i2}/B}] \bm{\hat{s}}_{i2,3}\big) \ddot{\theta}_{i1}
	+\big( I_{s_{i2,2}}\bm{\hat{s}}_{i2,2}+m_{\text{sp}_{i2}} d_{i2} [\tilde{\bm{r}}_{S_{c,i2}/B}] \bm{\hat{s}}_{i2,3}\big)\ddot{\theta}_{i2}\bigg] \\
	= -[\bm{\tilde{\omega}}_{\cal B/N}] [I_{\text{sc},B}] \bm\omega_{\cal B/N} - [I'_{\text{sc},B}] \bm\omega_{\cal B/N} \\
	-  \sum\limits_{i=1}^{N_S} \bigg[
	\dot{\theta}_{i1} I_{s_{i1,2}} [\bm{\tilde{\omega}}_{\cal B/N}] \bm{\hat{s}}_{i1,2} 
	+m_{\text{sp}_{i1}} [\bm{\tilde{\omega}}_{\cal B/N}] [\tilde{\bm{r}}_{S_{c,i1}/B}] \bm{r}'_{S_{c,i1}/B} +m_{\text{sp}_{i1}}d_{i1}\dot{\theta}_{i1}^2  [\tilde{\bm{r}}_{S_{c,i1}/B}] \bm{\hat{s}}_{i1,1}\\
	+ \big(\dot{\theta}_{i1}+\dot{\theta}_{i2}\big) I_{s_{i2,2}}[\bm{\tilde{\omega}}_{\cal B/N}]\bm{\hat{s}}_{i2,2}
	+m_{\text{sp}_{i2}} [\bm{\tilde{\omega}}_{\cal B/N}] [\tilde{\bm{r}}_{S_{c,i2}/B}] \bm{r}'_{S_{c,i2}/B} 	\\+m_{\text{sp}_{i2}} [\tilde{\bm{r}}_{S_{c,i2}/B}] \big(l_{i1} \dot{\theta}_{i1}^2 \bm{\hat{s}}_{i1,1} + d_{i2}\big(\dot{\theta}_{i1} + \dot{\theta}_{i2}\big)^2\bm{\hat{s}}_{i2,1}\big)\bigg]
	+ \bm{L}_B 
	\label{eq:Final6}
	\end{multline}
	
	
	\subsection{Dual Linked Solar Panel Motion}
	
	The following section follows the same derivation seen in previous work\cite{Allard2016rz} and is summarized here for convenience. 
	Let $\bm L_{H_{i1}} = L_{i1,1} \hat{\bm s}_{i1,1} + L_{i1,2} \hat{\bm s}_{i1,2} + L_{i1,3} \hat{\bm s}_{i1,3}$ be the total torque acting on the first solar panel at point $H_{i1}$. The corresponding hinge torque is given through
	\begin{equation}
	L_{i1,2} = - k_{i1} \theta_{i1} - c_{i1}\dot{\theta}_{i1} +  k_{i2} \theta_{i2} + c_{i2} \dot\theta_{i2} + \hat{\bm s}_{i1,2} \cdot \bm \tau_{\text{ext}_{i1},H_{i1}} + \hat{\bm s}_{i1,2} \cdot \bm{r}_{H_{i2}/H_{i1}} \times \bm F_{1/2i}
	\label{eq:hingeTorque1}
	\end{equation}
	Where $\bm F_{1/2i}$ is the reaction of solar panel 2 acting on solar panel 1. It is important to point out that $\bm F_{1/2i} = - \bm F_{2/1i}$. 
	
	To define the $\bm F_{1/2i}$, $\bm F_{2/1i}$ needs to be defined. This is done performing the super particle theorem on the second solar panel:
	\begin{equation}
	\bm F_{2/1i} + \bm F_{\text{ext}_{i2}} = m_{sp_{i2}} \ddot{\bm{r}}_{S_{c,i2}/N}
	\end{equation}
	The sum of the external forces on solar panel 2, $\bm F_{\text{ext}_{i2}}$, is separate because it does not contribute to the reaction force at the joint. With this definition $\bm F_{1/2i}$ is defined as
	\begin{equation}
	\bm F_{1/2i} = \bm F_{\text{ext}_{i2}}  - m_{sp_{i2}} \ddot{\bm{r}}_{S_{c,i2}/N} 
	\end{equation}
	Plugging this definition into Eq.~\eqref{eq:hingeTorque1} yields
	\begin{equation}
	L_{i1,2} = - k_{i1} \theta_{i1} - c_{i1}\dot{\theta}_{i1} +  k_{i2} \theta_{i2} + c_{i2} \dot\theta_{i2} + \hat{\bm s}_{i1,2} \cdot \bm \tau_{\text{ext}_{i1},H_{i1}} + \hat{\bm s}_{i1,2} \cdot \Big[\bm{r}_{H_{i2}/H_{i1}} \times \big(\bm F_{\text{ext}_{i2}}  - m_{sp_{i2}} \ddot{\bm{r}}_{S_{c,i2}/N}\big)\Big]
	\label{eq:hingeTorque2}
	\end{equation}
	The hinge structure produces the other two torques $L_{i1,1}$ and $L_{i1,3}$. $\bm \tau_{\text{ext}_{i1},H_{i1}}$ is the external torque on the solar panel and is projected onto the $\hat{\bm s}_{i,2}$ direction to find its contribution to $L_{i1,2}$. Gravity, for example would apply the following torque on the solar panel about point $H_{i1}$
	\begin{equation}
	\bm \tau_{g,H_{i1}} = \bm r_{S_{c,i1}/H_{i1}} \times \bm F_g
	\end{equation}
	
	The inertial angular velocity vector for the solar panel frame is
	\begin{equation}
	\bm\omega_{\mathcal{S}_{i1}/\mathcal{N}} = \bm\omega_{\mathcal{S}_{i1}/\mathcal{H}_{i1}} + \bm\omega_{\mathcal{H}_{i1}/\mathcal{B}} + \bm\omega_{\cal B/N}
	\end{equation}
	where $\bm\omega_{\mathcal{S}_{i1}/\mathcal{H}_{i1}} = \dot\theta_{i1} \hat{\bm s}_{i1,2}$.  
	Because the hinge frame $\mathcal{H}_{i1}$ is fixed relative to the body frame $\mathcal{B}$ the relative angular velocity vector is $\bm\omega_{\mathcal{H}_{i1}/\mathcal{B}} = \bm 0$.  The body angular velocity vector is written in $\mathcal{S}_{i1}$-frame components as
	\begin{align}
	\bm\omega_{\cal B/N} &= ( \hat{\bm s}_{i1,1} \cdot \bm\omega_{\cal B/N}) \hat{\bm s}_{i1,1}
	+ (\hat{\bm s}_{i1,2} \cdot\bm\omega_{\cal B/N}) \hat{\bm s}_{i1,2}
	+ (\hat{\bm s}_{i1,3} \cdot\bm\omega_{\cal B/N}) \hat{\bm s}_{i1,3}
	\\
	&= \omega_{s_{i1,1}} \hat{\bm s}_{i1,1} + \omega_{s_{i1,2}} \hat{\bm s}_{i1,2} + \omega_{s_{i1,3}}\hat{\bm s}_{i1,3}
	\end{align}
	Using this definition greatly simplifies the following algebraic development.  Finally, the inertial solar panel angular velocity vector is written as
	\begin{equation}
	\bm\omega_{\mathcal{S}_{i1}/\mathcal{N}}  = \omega_{s_{i1,1}} \hat{\bm s}_{{i1},1} + (\omega_{s_{i1,2}} + \dot\theta_{i1})\hat{\bm s}_{i1,2} + \omega_{s_{i1,3}} \hat{\bm s}_{i1,3}
	\end{equation}
	As $\hat{\bm s}_{i1,2}$ is a body-fixed vector, note that
	\begin{equation}
	\dot\omega_{s_{i1,2}} = \frac{\leftexp{B}\D}{\D t} \left( \bm\omega_{\cal B/N} \cdot \hat{\bm s}_{i1,2} \right)
	= \frac{\leftexp{B}\D}{\D t} \left( \bm\omega_{\cal B/N}\right) \cdot \hat{\bm s}_{i1,2}  = 
	\dot{\bm\omega}_{\cal B/N} \cdot \hat{\bm s}_{i1,2}
	\end{equation}
	
	Substituting these angular velocity components into the rotational equations of motion of a rigid body with torques taken about its center of mass\cite{schaub}, the general solar panel equations of motion are written as
	\begin{align}
	I_{s_{i1,1}} \dot\omega_{s_{i1,1}} &= - (I_{s_{i1,3}} - I_{s_{i1,2}}) (\omega_{s_{i1,2}}+\dot\theta_{i1})\omega_{s_{i1,3}} + L_{s_{i1,1}}
	\\
	I_{s_{i1,2}} ( \dot\omega_{s_{i1,2}} + \ddot\theta_{i1}) &= - (I_{s_{i1,1}} - I_{s_{i1,3}}) \omega_{s_{i1,3}} \omega_{s_{i1,1}} + L_{s_{i1,2}}
	\\
	I_{s_{i1,3}} \dot\omega_{s_{i1,3}} &= - (I_{s_{i1,2}} - I_{s_{i1,1}}) \omega_{s_{i1,1}}(\omega_{s_{i1,2}}+ \dot\theta_{i1}) + L_{s_{i1,3}}
	\end{align}
	where $\bm L_{S_{c,i1}} = L_{s_{i1,1}} \hat{\bm s}_{i1,1} + L_{s_{i1,2}} \hat{\bm s}_{i1,2} + L_{s_{i1,3}} \hat{\bm s}_{i1,3}$ is the net torque acting on the solar panel about its center of mass.  The second differential equation is used to get the equations of motion of $\theta_{i1}$.  The first and third equation could be used to back-solve for the structural hinge torques embedded in $L_{s_{i1,1}}$ and $L_{s_{i1,3}}$ if needed.  
	
	Let $\bm F_{S_{c,i1}}$ be the net force acting on the first solar panel.  Using the superparticle theorem\cite{schaub} yields
	\begin{equation}
	\bm F_{S_{c,i1}} = m_{\text{sp}_{i1}} \ddot{\bm r}_{S_{c,i1}/N}
	\end{equation}
	The torque about the solar panel center of mass can be related to the torque about the hinge point $H_{i1}$ using
	\begin{equation}
	\bm L_{H_i1} = \bm L_{S_{c,i1}} + \bm r_{S_{c,i1}/H_{i1}} \times \bm F_{S_{c,i1}} 
	\end{equation}
	Solving for the torque about $S_{c,i}$ yields
	\begin{equation}
	\bm L_{S_{c,i1}} = \bm L_{H_i1} - \bm r_{S_{c,i1}/H_{i1}} \times m_{\text{sp}_{i1}} \ddot{\bm r}_{S_{c,i1}/N}
	\end{equation}
	Taking the vector dot product with $\hat{\bm s}_{i1,2}$ and using $\bm r_{S_{c,i1}/H_{i1}} = -d_{i1} \hat{\bm s}_{i1,1}$ results in
	\begin{equation}
	L_{s_{i1,2}} = \hat{\bm s}_{i1,2} \cdot \bm L_{S_{c,i1}} =  \underbrace{\hat{\bm s}_{i1,2} \cdot \bm L_{H_{i1}}}_{L_{i1,2}}  -  \hat{\bm s}_{i1,2} \cdot \left(
	\bm r_{S_{c,i1}/H_{i1}} \times m_{\text{sp}_{i1}} \ddot{\bm r}_{S_{c,i1}/N} \right)
	\end{equation}
	\begin{multline}
	L_{s_{i1,2}} = - k_{i1} \theta_{i1} - c_{i1}\dot{\theta}_{i1} +  k_{i2} \theta_{i2} + c_{i2} \dot\theta_{i2} + \hat{\bm s}_{i1,2} \cdot \bm \tau_{\text{ext}_{i1},H_{i1}} + \hat{\bm s}_{i1,2} \cdot \Big[\bm{r}_{H_{i2}/H_{i1}} \times \big(\bm F_{\text{ext}_{i2}}  - m_{sp_{i2}} \ddot{\bm{r}}_{S_{c,i2}/N}\big)\Big]\\
	 + m_{\text{sp}_{i1}} d_{i1} \hat{\bm s}_{i1,2} \cdot \left(\hat{\bm s}_{i1,1} \times  \ddot{\bm r}_{S_{c,i1}/N} \right)
	\end{multline}
	Expanding a couple of definitions
	\begin{multline}
	L_{s_{i1,2}} = - k_{i1} \theta_{i1} - c_{i1}\dot{\theta}_{i1} +  k_{i2} \theta_{i2} + c_{i2} \dot\theta_{i2} + \hat{\bm s}_{i1,2} \cdot \bm \tau_{\text{ext}_{i1},H_{i1}} - l_{i1} \hat{\bm s}_{i1,2} \cdot \big(\hat{\bm s}_{i1,1} \times \bm F_{\text{ext}_{i2}}\big) \\
	+ m_{sp_{i2}} l_{i1} \hat{\bm s}_{i1,2} \cdot \Big[\hat{\bm s}_{i1,1} \times \big( \ddot{\bm{r}}_{S_{c,i2}/N}\big)\Big]
	+ m_{\text{sp}_{i1}} d_{i1} \hat{\bm s}_{i1,2} \cdot \left(\hat{\bm s}_{i1,1} \times  \ddot{\bm r}_{S_{c,i1}/N} \right)
	\end{multline}
	Using the double vector cross product identity results in:
	\begin{multline}
	L_{s_{i1,2}} = - k_{i1} \theta_{i1} - c_{i1}\dot{\theta}_{i1} +  k_{i2} \theta_{i2} + c_{i2} \dot\theta_{i2} + \hat{\bm s}_{i1,2} \cdot \bm \tau_{\text{ext}_{i1},H_{i1}} + l_{i1} \hat{\bm s}_{i1,3} \cdot \bm F_{\text{ext}_{i2}} \\
	- m_{sp_{i2}} l_{i1} \hat{\bm s}_{i1,3} \cdot \ddot{\bm{r}}_{S_{c,i2}/N}
	- m_{\text{sp}_{i1}} d_{i1} \hat{\bm s}_{i1,3} \cdot  \ddot{\bm r}_{S_{c,i1}/N}
	\end{multline}
	The following definitions need to be defined:
	\begin{multline}
	\ddot{\bm r}_{S_{c,i1}/N} = \ddot{\bm{r}}_{B/N} + \ddot{\bm r}_{S_{c,i1}/B} \\ 
	= \ddot{\bm{r}}_{B/N} + \bm{r}''_{S_{c,i1}/B} + 2 \bm\omega_{\cal B/N} \times \bm{r}'_{S_{c,i1}/B} +  \dot{\bm\omega}_{\cal B/N} \times \bm{r}_{S_{c,i1}/B} + \bm\omega_{\cal B/N} \times (\bm\omega_{\cal B/N} \times \bm{r}_{S_{c,i1}/B})
	\end{multline}
	\begin{multline}
	\ddot{\bm r}_{S_{c,i2}/N} = \ddot{\bm{r}}_{B/N} + \ddot{\bm r}_{S_{c,i2}/B} \\
	= \ddot{\bm{r}}_{B/N} + \bm{r}''_{S_{c,i2}/B} + 2 \bm\omega_{\cal B/N} \times \bm{r}'_{S_{c,i2}/B} +  \dot{\bm\omega}_{\cal B/N} \times \bm{r}_{S_{c,i2}/B} + \bm\omega_{\cal B/N} \times (\bm\omega_{\cal B/N} \times \bm{r}_{S_{c,i2}/B})
	\end{multline}

	Substituting these definitions into the torque equation results in:

	
	\begin{multline}
L_{s_{i1,2}} = - k_{i1} \theta_{i1} - c_{i1}\dot{\theta}_{i1} +  k_{i2} \theta_{i2} + c_{i2} \dot\theta_{i2} + \hat{\bm s}_{i1,2} \cdot \bm \tau_{\text{ext}_{i1},H_{i1}} + l_{i1} \hat{\bm s}_{i1,3} \cdot \bm F_{\text{ext}_{i2}} \\
- m_{\text{sp}_{i1}} d_{i1} \hat{\bm s}_{i1,3} \cdot \Big[\ddot{\bm{r}}_{B/N} + \bm{r}''_{S_{c,i1}/B} + 2 \bm\omega_{\cal B/N} \times \bm{r}'_{S_{c,i1}/B} +  \dot{\bm\omega}_{\cal B/N} \times \bm{r}_{S_{c,i1}/B}
 + \bm\omega_{\cal B/N} \times (\bm\omega_{\cal B/N} \times \bm{r}_{S_{c,i1}/B})\Big]\\
 - m_{sp_{i2}} l_{i1} \hat{\bm s}_{i1,3} \cdot \Big[\ddot{\bm{r}}_{B/N} + \bm{r}''_{S_{c,i2}/B} + 2 \bm\omega_{\cal B/N} \times \bm{r}'_{S_{c,i2}/B} +  \dot{\bm\omega}_{\cal B/N} \times \bm{r}_{S_{c,i2}/B} + \bm\omega_{\cal B/N} \times (\bm\omega_{\cal B/N} \times \bm{r}_{S_{c,i2}/B})\Big]
\end{multline}

	Substituting this torque into the earlier differential equation
\begin{equation}
I_{s_{i1,2}} ( \dot\omega_{s_{i1,2}} + \ddot\theta_{i1}) = - (I_{s_{i1,1}} - I_{s_{i1,3}}) \omega_{s_{i1,3}} \omega_{s_{i1,1}} + L_{s_{i1,2}}
\end{equation}
leads to the desired scalar hinged solar panel equation of motion

\begin{multline}
I_{s_{i1,2}} ( \hat{\bm s}_{i1,2}^T \dot{\bm\omega}_{\cal B/N} + \ddot\theta_{i1}) = - (I_{s_{i1,1}} - I_{s_{i1,3}}) \omega_{s_{i1,3}} \omega_{s_{i1,1}} - k_{i1} \theta_{i1} - c_{i1}\dot{\theta}_{i1} +  k_{i2} \theta_{i2} + c_{i2} \dot\theta_{i2} \\
+ \hat{\bm s}_{i1,2} \cdot \bm \tau_{\text{ext}_{i1},H_{i1}} + l_{i1} \hat{\bm s}_{i1,3} \cdot \bm F_{\text{ext}_{i2}}
- m_{\text{sp}_{i1}} d_{i1} \hat{\bm s}_{i1,3} \cdot \Big[\ddot{\bm{r}}_{B/N} + \bm{r}''_{S_{c,i1}/B} + 2 \bm\omega_{\cal B/N} \times \bm{r}'_{S_{c,i1}/B} \\
+  \dot{\bm\omega}_{\cal B/N} \times \bm{r}_{S_{c,i1}/B}
+ \bm\omega_{\cal B/N} \times (\bm\omega_{\cal B/N} \times \bm{r}_{S_{c,i1}/B})\Big]
- m_{sp_{i2}} l_{i1} \hat{\bm s}_{i1,3} \cdot \Big[\ddot{\bm{r}}_{B/N} + \bm{r}''_{S_{c,i2}/B} + 2 \bm\omega_{\cal B/N} \times \bm{r}'_{S_{c,i2}/B} \\
+  \dot{\bm\omega}_{\cal B/N} \times \bm{r}_{S_{c,i2}/B} + \bm\omega_{\cal B/N} \times (\bm\omega_{\cal B/N} \times \bm{r}_{S_{c,i2}/B})\Big]
\end{multline}

Moving second order variable to the left hand side of the equation yields:

\begin{multline}
\Big[m_{\text{sp}_{i1}} d_{i1} \hat{\bm s}_{i1,3}^T + m_{sp_{i2}} l_{i1} \hat{\bm s}_{i1,3}^T \Big] \ddot{\bm{r}}_{B/N} + \Big[I_{s_{i1,2}} \hat{\bm s}_{i1,2}^T - m_{\text{sp}_{i1}} d_{i1} \hat{\bm s}_{i1,3}^T [\tilde{\bm{r}}_{S_{c,i1}/B}] - m_{sp_{i2}} l_{i1} \hat{\bm s}_{i1,3}^T [\tilde{\bm{r}}_{S_{c,i2}/B}] \Big]\dot{\bm\omega}_{\cal B/N} \\
+ I_{s_{i1,2}} \ddot\theta_{i1} + m_{\text{sp}_{i1}} d_{i1} \hat{\bm s}_{i1,3}^T \bm{r}''_{S_{c,i1}/B}
+ m_{sp_{i2}} l_{i1} \hat{\bm s}_{i1,3}^T \bm{r}''_{S_{c,i2}/B} 
  = - (I_{s_{i1,1}} - I_{s_{i1,3}}) \omega_{s_{i1,3}} \omega_{s_{i1,1}} - k_{i1} \theta_{i1} - c_{i1}\dot{\theta}_{i1} \\
  +  k_{i2} \theta_{i2} + c_{i2} \dot\theta_{i2} 
+ \hat{\bm s}_{i1,2} \cdot \bm \tau_{\text{ext}_{i1},H_{i1}} + l_{i1} \hat{\bm s}_{i1,3} \cdot \bm F_{\text{ext}_{i2}}
- m_{\text{sp}_{i1}} d_{i1} \hat{\bm s}_{i1,3} \cdot \Big[2 \bm\omega_{\cal B/N} \times \bm{r}'_{S_{c,i1}/B} \\
+ \bm\omega_{\cal B/N} \times (\bm\omega_{\cal B/N} \times \bm{r}_{S_{c,i1}/B})\Big]
- m_{sp_{i2}} l_{i1} \hat{\bm s}_{i1,3} \cdot \Big[ 2 \bm\omega_{\cal B/N} \times \bm{r}'_{S_{c,i2}/B} 
 + \bm\omega_{\cal B/N} \times (\bm\omega_{\cal B/N} \times \bm{r}_{S_{c,i2}/B})\Big]
\end{multline}

Expanding the $\bm{r}''_{S_{c,i1}/B}$ and $\bm{r}''_{S_{c,i2}/B}$ terms, replacing cross products with the tilde matrix and again isolating the second order variables results in: 

\begin{multline}
\Big[m_{\text{sp}_{i1}} d_{i1} \hat{\bm s}_{i1,3}^T + m_{sp_{i2}} l_{i1} \hat{\bm s}_{i1,3}^T \Big] \ddot{\bm{r}}_{B/N} + \Big[I_{s_{i1,2}} \hat{\bm s}_{i1,2}^T - m_{\text{sp}_{i1}} d_{i1} \hat{\bm s}_{i1,3}^T [\tilde{\bm{r}}_{S_{c,i1}/B}] - m_{sp_{i2}} l_{i1} \hat{\bm s}_{i1,3}^T [\tilde{\bm{r}}_{S_{c,i2}/B}] \Big]\dot{\bm\omega}_{\cal B/N} \\
+ \Big[I_{s_{i1,2}}+ m_{\text{sp}_{i1}} d_{i1}^2 + m_{sp_{i2}} l_{i1}^2 + m_{sp_{i2}} l_{i1} d_{i2} \hat{\bm s}_{i1,3}^T \bm{\hat{s}}_{i2,3} \Big] \ddot\theta_{i1} + \Big[m_{sp_{i2}} l_{i1} d_{i2} \hat{\bm s}_{i1,3}^T \bm{\hat{s}}_{i2,3} \Big] \ddot{\theta}_{i2}\\
= - (I_{s_{i1,1}} - I_{s_{i1,3}}) \omega_{s_{i1,3}} \omega_{s_{i1,1}} - k_{i1} \theta_{i1} - c_{i1}\dot{\theta}_{i1} 
+  k_{i2} \theta_{i2} + c_{i2} \dot\theta_{i2} 
+ \hat{\bm s}_{i1,2}^T \bm \tau_{\text{ext}_{i1},H_{i1}} + l_{i1} \hat{\bm s}_{i1,3}^T \bm F_{\text{ext}_{i2}}\\ 
- m_{\text{sp}_{i1}} d_{i1} \hat{\bm s}_{i1,3}^T \Big[2 [\tilde{\bm\omega}_{\cal B/N}] \bm{r}'_{S_{c,i1}/B}
+ [\tilde{\bm\omega}_{\cal B/N}] [\tilde{\bm\omega}_{\cal B/N}] \bm{r}_{S_{c,i1}/B}\Big]
\\
- m_{sp_{i2}} l_{i1} \hat{\bm s}_{i1,3}^T \Big[ 2 [\tilde{\bm\omega}_{\cal B/N}] \bm{r}'_{S_{c,i2}/B} 
+ [\tilde{\bm\omega}_{\cal B/N}] [\tilde{\bm\omega}_{\cal B/N}] \bm{r}_{S_{c,i2}/B} + l_{i1} \dot{\theta}_{i1}^2 \bm{\hat{s}}_{i1,1} + d_{i2}\big(\dot{\theta}_{i1} + \dot{\theta}_{i2}\big)^2\bm{\hat{s}}_{i2,1}\Big]
\label{eq:solar_panel_final10}
\end{multline}

Eq.~\eqref{eq:solar_panel_final10} is the EOM that describes the motion of the first solar panel with a linked secondary panel attached at the end. The final step is to find the EOM of the secondary panel. Following a very similar pattern the EOM for the second panel is found. First the torque about point $H_{i2}$ is defined as:
	\begin{equation}
L_{i2,2} = - k_{i2} \theta_{i2} - c_{i2} \dot{\theta}_{i2} + \hat{\bm s}_{i2,2} \cdot \bm \tau_{\text{ext}_{i2},H_{i2}}
\label{eq:hingeTorque3}
\end{equation}
The relationship between the torque about the center of mass of the solar panel and about the hinge point is defined as:
\begin{equation}
\bm L_{H_i2} = \bm L_{S_{c,i2}} + \bm r_{S_{c,i2}/H_{i2}} \times \bm F_{S_{c,i2}} 
\end{equation}
The torque about $\hat{\bm s}_{i2,2}$ is the only torque that is required:
\begin{equation}
L_{s_{i2,2}} = \hat{\bm s}_{i2,2} \cdot \bm L_{S_{c,i2}} =  \underbrace{\hat{\bm s}_{i2,2} \cdot \bm L_{H_{i2}}}_{L_{i2,2}}  -  \hat{\bm s}_{i2,2} \cdot \left(
\bm r_{S_{c,i2}/H_{i2}} \times m_{\text{sp}_{i2}} \ddot{\bm r}_{S_{c,i2}/N} \right)
\end{equation}
Substituting Eq.~\eqref{eq:hingeTorque3} into the previous equation yields
\begin{equation}
L_{s_{i2,2}} = - k_{i2} \theta_{i2} - c_{i2} \dot{\theta}_{i2} + \hat{\bm s}_{i2,2} \cdot \bm \tau_{\text{ext}_{i2},H_{i2}}  -  m_{\text{sp}_{i2}} d_{i2} \hat{\bm s}_{i2,3} \cdot \ddot{\bm r}_{S_{c,i2}/N}
\end{equation}

	Substituting this torque into the modified Euler's equation for the second panel
\begin{equation}
	I_{s_{i2,2}} ( \dot\omega_{s_{i2,2}} + \ddot\theta_{i1} + \ddot\theta_{i2}) = - (I_{s_{i2,1}} - I_{s_{i2,3}}) \omega_{s_{i2,3}} \omega_{s_{i2,1}} + L_{s_{i2,2}}
\end{equation}

Results in:

\begin{multline}
	I_{s_{i2,2}} ( \dot\omega_{s_{i2,2}} + \ddot\theta_{i1} + \ddot\theta_{i2}) = - (I_{s_{i2,1}} - I_{s_{i2,3}}) \omega_{s_{i2,3}} \omega_{s_{i2,1}} - k_{i2} \theta_{i2} - c_{i2} \dot{\theta}_{i2} + \hat{\bm s}_{i2,2}^T \bm \tau_{\text{ext}_{i2},H_{i2}}  \\
	-  m_{\text{sp}_{i2}} d_{i2} \hat{\bm s}_{i2,3}^T \ddot{\bm r}_{S_{c,i2}/N}
\end{multline}

Substituting the definition of $\ddot{\bm r}_{S_{c,i2}/N}$ yields:

\begin{multline}
I_{s_{i2,2}} ( \dot\omega_{s_{i2,2}} + \ddot\theta_{i1} + \ddot\theta_{i2}) = - (I_{s_{i2,1}} - I_{s_{i2,3}}) \omega_{s_{i2,3}} \omega_{s_{i2,1}} - k_{i2} \theta_{i2} - c_{i2} \dot{\theta}_{i2} + \hat{\bm s}_{i2,2}^T \bm \tau_{\text{ext}_{i2},H_{i2}}  \\
-  m_{\text{sp}_{i2}} d_{i2} \hat{\bm s}_{i2,3}^T \Big[ \ddot{\bm{r}}_{B/N} + \bm{r}''_{S_{c,i2}/B} + 2 \bm\omega_{\cal B/N} \times \bm{r}'_{S_{c,i2}/B} +  \dot{\bm\omega}_{\cal B/N} \times \bm{r}_{S_{c,i2}/B} + \bm\omega_{\cal B/N} \times (\bm\omega_{\cal B/N} \times \bm{r}_{S_{c,i2}/B})\Big]
\end{multline}

Moving the second order state variables to the left hand side of the equation yields:

\begin{multline}
\Big[m_{\text{sp}_{i2}} d_{i2} \hat{\bm s}_{i2,3}^T\Big] \ddot{\bm{r}}_{B/N} + \Big[I_{s_{i2,2}} \hat{\bm s}_{i2,2}^T + m_{\text{sp}_{i2}} d_{i2} \hat{\bm s}_{i2,3}^T [\tilde{\bm{r}}_{S_{c,i2}/B}]\Big] \dot{\bm\omega}_{\cal B/N} + \Big[I_{s_{i2,2}}\Big] \ddot\theta_{i1} + \Big[I_{s_{i2,2}}\Big] \ddot\theta_{i2} \\
+ m_{\text{sp}_{i2}} d_{i2} \hat{\bm s}_{i2,3}^T  \bm{r}''_{S_{c,i2}/B} = - (I_{s_{i2,1}} - I_{s_{i2,3}}) \omega_{s_{i2,3}} \omega_{s_{i2,1}} - k_{i2} \theta_{i2} - c_{i2} \dot{\theta}_{i2} + \hat{\bm s}_{i2,2}^T \bm \tau_{\text{ext}_{i2},H_{i2}}  \\
-  m_{\text{sp}_{i2}} d_{i2} \hat{\bm s}_{i2,3}^T \Big[ 2 \bm\omega_{\cal B/N} \times \bm{r}'_{S_{c,i2}/B} + \bm\omega_{\cal B/N} \times (\bm\omega_{\cal B/N} \times \bm{r}_{S_{c,i2}/B})\Big]
\end{multline}

Expanding $\bm{r}''_{S_{c,i2}/B}$, isolating second order state variables to the left hand side and introducing the skew symmetric matrix:

\begin{multline}
\Big[m_{\text{sp}_{i2}} d_{i2} \hat{\bm s}_{i2,3}^T\Big] \ddot{\bm{r}}_{B/N} + \Big[I_{s_{i2,2}} \hat{\bm s}_{i2,2}^T + m_{\text{sp}_{i2}} d_{i2} \hat{\bm s}_{i2,3}^T [\tilde{\bm{r}}_{S_{c,i2}/B}]\Big] \dot{\bm\omega}_{\cal B/N} + \Big[I_{s_{i2,2}} + m_{\text{sp}_{i2}} d_{i2}^2 + m_{\text{sp}_{i2}} l_{i1} d_{i2} \hat{\bm s}_{i2,3}^T \bm{\hat{s}}_{i1,3} \Big] \ddot\theta_{i1} \\
+ \Big[I_{s_{i2,2}} + m_{\text{sp}_{i2}} d_{i2}^2 \Big] \ddot\theta_{i2} 
= - (I_{s_{i2,1}} - I_{s_{i2,3}}) \omega_{s_{i2,3}} \omega_{s_{i2,1}} - k_{i2} \theta_{i2} - c_{i2} \dot{\theta}_{i2} + \hat{\bm s}_{i2,2}^T \bm \tau_{\text{ext}_{i2},H_{i2}}  \\
-  m_{\text{sp}_{i2}} d_{i2} \hat{\bm s}_{i2,3}^T \Big[ 2 [\tilde{\bm\omega}_{\cal B/N}] \bm{r}'_{S_{c,i2}/B} + [\tilde{\bm\omega}_{\cal B/N}] [\tilde{\bm\omega}_{\cal B/N}] \bm{r}_{S_{c,i2}/B} + l_{i1} \dot{\theta}_{i1}^2 \bm{\hat{s}}_{i1,1} \Big]
\label{eq:sp2final}
\end{multline}

Eq.~\eqref{eq:sp2final} is the last EOM needed to describe the motion of the spacecraft. The next section develops the back substitution method for interconnected panels and gives meaningful insight on how effectors connected to other effectors dynamically couple to the spacecraft. 

	\subsection{Back Substitution Method}
	The dynamical coupling of this complex system can be visualized in the following equation:
	
	\begin{equation}
	\begin{bmatrix}
	[3\times 3] & [3\times 3] & 3\times 1 & 3\times 1 & 3\times 1 & 3\times 1 & . & 3\times 1 & 3\times 1\\
	[3\times 3] & [3\times 3] & 3\times 1 & 3\times 1 & 3\times 1 & 3\times 1 & . & 3\times 1 & 3\times 1\\
	[1\times 3] & [1\times 3] & 1\times 1 & 1\times 1  & 0 & 0 & . & 0 & 0\\
	[1\times 3] & [1\times 3] & 1\times 1 & 1\times 1  & 0 & 0 & . & 0 & 0\\
	[1\times 3] & [1\times 3] & 0 & 0  & 1\times 1 & 1\times 1 & . & 0 & 0 \\
	[1\times 3] & [1\times 3] & 0 & 0  & 1\times 1 & 1\times 1 & . & 0 & 0\\
	. & . & . & . & . & . & . & . & .\\
	[1\times 3] & [1\times 3] & 0 & 0  & 0 & 0 & . & 1\times 1 & 1\times 1\\
	[1\times 3] & [1\times 3] & 0 & 0  & 0 & 0 & . & 1\times 1 & 1\times 1\\
	\end{bmatrix}
	\begin{bmatrix}
	\ddot{\boldsymbol{r}}_{B/N}\\
	\dot{\boldsymbol{\omega}}_{B/N}\\
	\ddot{\theta}_{11}\\
	\ddot{\theta}_{12}\\
	\ddot{\theta}_{21}\\
	\ddot{\theta}_{22}\\
	.\\
	\ddot{\theta}_{N1}\\
	\ddot{\theta}_{N2}
	\end{bmatrix}
	=
	\begin{bmatrix}
	3\times 1\\
	3\times 1\\
	1\times 1\\
	1\times 1\\
	1\times 1\\
	1\times 1\\
	.\\
	1\times 1\\
	1\times 1
	\end{bmatrix}
	\end{equation}
	This system mass matrix shows that the all of the solar panel modes are fully coupled with the hub, and that the pairs of solar panels are fully coupled with one another. However, the pairs of solar panels are not directly coupled with other pairs of solar panels. To utilize this pattern in the system mass matrix, the following back-substitution is developed. 
	
	First, Eq.~\eqref{eq:solar_panel_final10} and ~\eqref{eq:sp2final} are rearranged so that the second order state variables for the solar panel motions are isolated on the left hand side:
\begin{multline}
\Big[I_{s_{i1,2}}+ m_{\text{sp}_{i1}} d_{i1}^2 + m_{sp_{i2}} l_{i1}^2 + m_{sp_{i2}} l_{i1} d_{i2} \hat{\bm s}_{i1,3}^T \bm{\hat{s}}_{i2,3} \Big] \ddot\theta_{i1} + \Big[m_{sp_{i2}} l_{i1} d_{i2} \hat{\bm s}_{i1,3}^T \bm{\hat{s}}_{i2,3} \Big] \ddot{\theta}_{i2}=\\
-\Big[m_{\text{sp}_{i1}} d_{i1} \hat{\bm s}_{i1,3}^T + m_{sp_{i2}} l_{i1} \hat{\bm s}_{i1,3}^T \Big] \ddot{\bm{r}}_{B/N} - \Big[I_{s_{i1,2}} \hat{\bm s}_{i1,2}^T - m_{\text{sp}_{i1}} d_{i1} \hat{\bm s}_{i1,3}^T [\tilde{\bm{r}}_{S_{c,i1}/B}] - m_{sp_{i2}} l_{i1} \hat{\bm s}_{i1,3}^T [\tilde{\bm{r}}_{S_{c,i2}/B}] \Big]\dot{\bm\omega}_{\cal B/N} \\
- (I_{s_{i1,1}} - I_{s_{i1,3}}) \omega_{s_{i1,3}} \omega_{s_{i1,1}} - k_{i1} \theta_{i1} - c_{i1}\dot{\theta}_{i1} 
+  k_{i2} \theta_{i2} + c_{i2} \dot\theta_{i2} 
+ \hat{\bm s}_{i1,2}^T \bm \tau_{\text{ext}_{i1},H_{i1}} + l_{i1} \hat{\bm s}_{i1,3}^T \bm F_{\text{ext}_{i2}}\\ 
- m_{\text{sp}_{i1}} d_{i1} \hat{\bm s}_{i1,3}^T \Big[2 [\tilde{\bm\omega}_{\cal B/N}] \bm{r}'_{S_{c,i1}/B}
+ [\tilde{\bm\omega}_{\cal B/N}] [\tilde{\bm\omega}_{\cal B/N}] \bm{r}_{S_{c,i1}/B}\Big]
\\
- m_{sp_{i2}} l_{i1} \hat{\bm s}_{i1,3}^T \Big[ 2 [\tilde{\bm\omega}_{\cal B/N}] \bm{r}'_{S_{c,i2}/B} 
+ [\tilde{\bm\omega}_{\cal B/N}] [\tilde{\bm\omega}_{\cal B/N}] \bm{r}_{S_{c,i2}/B} + l_{i1} \dot{\theta}_{i1}^2 \bm{\hat{s}}_{i1,1} + d_{i2}\big(\dot{\theta}_{i1} + \dot{\theta}_{i2}\big)^2\bm{\hat{s}}_{i2,1}\Big]
\label{eq:spMotion1}
\end{multline}
	
\begin{multline}
 \Big[I_{s_{i2,2}} + m_{\text{sp}_{i2}} d_{i2}^2 + m_{\text{sp}_{i2}} l_{i1} d_{i2} \hat{\bm s}_{i2,3}^T \bm{\hat{s}}_{i1,3} \Big] \ddot\theta_{i1} 
+ \Big[I_{s_{i2,2}} + m_{\text{sp}_{i2}} d_{i2}^2 \Big] \ddot\theta_{i2} 
= \\
-\Big[m_{\text{sp}_{i2}} d_{i2} \hat{\bm s}_{i2,3}^T\Big] \ddot{\bm{r}}_{B/N} - \Big[I_{s_{i2,2}} \hat{\bm s}_{i2,2}^T + m_{\text{sp}_{i2}} d_{i2} \hat{\bm s}_{i2,3}^T [\tilde{\bm{r}}_{S_{c,i2}/B}]\Big] \dot{\bm\omega}_{\cal B/N} - (I_{s_{i2,1}} - I_{s_{i2,3}}) \omega_{s_{i2,3}} \omega_{s_{i2,1}} \\
- k_{i2} \theta_{i2} - c_{i2} \dot{\theta}_{i2} + \hat{\bm s}_{i2,2}^T \bm \tau_{\text{ext}_{i2},H_{i2}}  
-  m_{\text{sp}_{i2}} d_{i2} \hat{\bm s}_{i2,3}^T \Big[ 2 [\tilde{\bm\omega}_{\cal B/N}] \bm{r}'_{S_{c,i2}/B} + [\tilde{\bm\omega}_{\cal B/N}] [\tilde{\bm\omega}_{\cal B/N}] \bm{r}_{S_{c,i2}/B} + l_{i1} \dot{\theta}_{i1}^2 \bm{\hat{s}}_{i1,1} \Big]
\label{eq:sp3final}
\end{multline}
Now, defining the elements of a matrix $[A_i]$ as:
\begin{subequations}
	\begin{align}
	a_{i1,1} &= I_{s_{i1,2}} + m_{\text{sp}_{i1}} d^2_{i1} + m_{sp_{i2}} l^2_{i1}+ m_{sp_{i2}} l_{i1} d_{i2}\hat{\bm s}^T_{i1,3}  \bm{\hat{s}}_{i2,3} \\
	a_{i1,2} &= m_{sp_{i2}} l_{i1} d_{i2} \hat{\bm s}^T_{i1,3}  \bm{\hat{s}}_{i2,3}\\
	a_{i2,1} &= I_{s_{i2,2}} + m_{\text{sp}_{i2}} d^2_{i2}  +  m_{\text{sp}_{i2}} l_{i1} d_{i2}\hat{\bm s}^T_{i2,3}  \bm{\hat{s}}_{i1,3} \\
	a_{i2,2} &= I_{s_{i2,2}} +  m_{\text{sp}_{i2}} d^2_{i2}
	\end{align}
\end{subequations}
And defining the row elements of a matrix $[F_i]$ as:
\begin{subequations}
	\begin{align}
	\bm f_{i1} &= -\big(m_{sp_{i2}} l_{i1} + m_{\text{sp}_{i1}} d_{i1} \big) \hat{\bm s}^T_{i1,3}\\
	\bm f_{i2} &= -m_{\text{sp}_{i2}} d_{i2} \hat{\bm s}_{i2,3}^T
	\end{align}
\end{subequations}
With a $2\times 3$ matrix $[G_i]$ which has row elements defined as:
\begin{subequations}
	\begin{align}
	\bm g_{i1} &= - \Big[I_{s_{i1,2}} \hat{\bm s}_{i1,2}^T - m_{\text{sp}_{i1}} d_{i1} \hat{\bm s}_{i1,3}^T [\tilde{\bm{r}}_{S_{c,i1}/B}] - m_{sp_{i2}} l_{i1} \hat{\bm s}_{i1,3}^T [\tilde{\bm{r}}_{S_{c,i2}/B}] \Big]^T\\
	\bm g_{i2} &= -  \Big[I_{s_{i2,2}} \hat{\bm s}_{i2,2}^T + m_{\text{sp}_{i2}} d_{i2} \hat{\bm s}_{i2,3}^T [\tilde{\bm{r}}_{S_{c,i2}/B}]\Big]^T
	\end{align}
\end{subequations}
Also defining the vector $\bm v_i$ as  as $2\times1$ with the following components:
	\begin{multline}
v_{i1} = - (I_{s_{i1,1}} - I_{s_{i1,3}}) \omega_{s_{i1,3}} \omega_{s_{i1,1}} - k_{i1} \theta_{i1} - c_{i1}\dot{\theta}_{i1} 
+  k_{i2} \theta_{i2} + c_{i2} \dot\theta_{i2} 
+ \hat{\bm s}_{i1,2}^T \bm \tau_{\text{ext}_{i1},H_{i1}} + l_{i1} \hat{\bm s}_{i1,3}^T \bm F_{\text{ext}_{i2}}\\ 
- m_{\text{sp}_{i1}} d_{i1} \hat{\bm s}_{i1,3}^T \Big[2 [\tilde{\bm\omega}_{\cal B/N}] \bm{r}'_{S_{c,i1}/B}
+ [\tilde{\bm\omega}_{\cal B/N}] [\tilde{\bm\omega}_{\cal B/N}] \bm{r}_{S_{c,i1}/B}\Big]
\\
- m_{sp_{i2}} l_{i1} \hat{\bm s}_{i1,3}^T \Big[ 2 [\tilde{\bm\omega}_{\cal B/N}] \bm{r}'_{S_{c,i2}/B} 
+ [\tilde{\bm\omega}_{\cal B/N}] [\tilde{\bm\omega}_{\cal B/N}] \bm{r}_{S_{c,i2}/B} + l_{i1} \dot{\theta}_{i1}^2 \bm{\hat{s}}_{i1,1} + d_{i2}\big(\dot{\theta}_{i1} + \dot{\theta}_{i2}\big)^2\bm{\hat{s}}_{i2,1}\Big]
\label{eq:solar_panel_final8}
\end{multline}
\begin{multline}
v_{i2} = - (I_{s_{i2,1}} - I_{s_{i2,3}}) \omega_{s_{i2,3}} \omega_{s_{i2,1}} \\
- k_{i2} \theta_{i2} - c_{i2} \dot{\theta}_{i2} + \hat{\bm s}_{i2,2}^T \bm \tau_{\text{ext}_{i2},H_{i2}}  
-  m_{\text{sp}_{i2}} d_{i2} \hat{\bm s}_{i2,3}^T \Big[ 2 [\tilde{\bm\omega}_{\cal B/N}] \bm{r}'_{S_{c,i2}/B} + [\tilde{\bm\omega}_{\cal B/N}] [\tilde{\bm\omega}_{\cal B/N}] \bm{r}_{S_{c,i2}/B} + l_{i1} \dot{\theta}_{i1}^2 \bm{\hat{s}}_{i1,1} \Big]
\end{multline}

Eqs. \eqref{eq:spMotion1} and \eqref{eq:sp3final} can now be re-written as:
\begin{equation}
a_{i1,1} \ddot\theta_{i1} +  a_{i1,2} \ddot{\theta}_{i2}=\bm f_{i1} \ddot{\bm{r}}_{B/N} + \bm g_{i1}\dot{\bm\omega}_{\cal B/N} 
+v_{i1}
\label{eq:spMotion1Simple}
\end{equation}

\begin{equation}
a_{i2,1} \ddot\theta_{i1} 
+ a_{i2,2}\ddot\theta_{i2} = \bm f_{i2}\ddot{\bm{r}}_{B/N} + \bm g_{i2}\dot{\bm\omega}_{\cal B/N} + v_{i2}
\label{eq:sp3finalSimple}
\end{equation}

	Eqs.~\eqref{eq:spMotion1Simple} and ~\eqref{eq:sp3finalSimple} are combined and written in matrix form to utilize some linear algebra techniques.
	\begin{equation}
	[A_i]\begin{bmatrix}
	\ddot \theta_{i1}\\
	\ddot \theta_{i2}
	\end{bmatrix}
	= [F_i] \ddot{\bm r}_{B/N} + [G_i]\dot{\bm\omega}_{\cal B/N} + \bm v_i
	\label{eq:thetadot}
	\end{equation}
	Eq.~\eqref{eq:thetadot} can now be solved by inverting matrix $[A_i]$. Note the definition $[E_i] = [A_i]^{-1}$.
	\begin{equation}
	\begin{bmatrix}
	\ddot \theta_{i1}\\
	\ddot \theta_{i2}
	\end{bmatrix}
	= [E_i][F_i] \ddot{\bm r}_{B/N} + [E_i][G_i]\dot{\bm\omega}_{\cal B/N} + [E_i]\bm v_i
\label{eq:thetadot2}
\end{equation}
And the subcomponents of $[E]$ are defined as
\begin{equation}
[E] = \begin{bmatrix}
\bm e_{i1}^T\\
\bm e_{i2}^T
\end{bmatrix}
\label{eq:E}
\end{equation}
	Since the modified Euler's equation, Eq.~\eqref{eq:Final6}, has $\ddot \theta_{i1}$ and $\ddot \theta_{i2}$ terms, it is more convenient to use the expression for $\ddot \theta_i$ as
	\begin{equation}
	\ddot \theta_{i1}
	= e_{i1}^T[F_i] \ddot{\bm r}_{B/N} + e_{i1}^T[G_i]\dot{\bm\omega}_{\cal B/N} + e_{i1}^T\bm v_i
	\label{eq:thetadot4}
	\end{equation}
	\begin{equation}
\ddot \theta_{i2}
= e_{i2}^T[F_i] \ddot{\bm r}_{B/N} + e_{i2}^T[G_i]\dot{\bm\omega}_{\cal B/N} + e_{i2}^T\bm v_i
\label{eq:thetadot5}
\end{equation}

	
The next step in the back substitution method is to analytically substitute Eqs.~\eqref{eq:thetadot4} and ~\eqref{eq:thetadot5} into the translational and rotational EOMs repeated here for clarity:
\begin{multline}
m_\text{sc} \ddot{\bm r}_{B/N} -m_\text{sc} [\tilde{\bm{c}}] \dot{\bm\omega}_{\cal B/N} +  \sum_{i=1}^{N_{S}}\bigg(\Big[m_{\text{sp}_{i1}}d_{i1} \bm{\hat{s}}_{i1,3} +m_{\text{sp}_{i2}}l_{i1} \bm{\hat{s}}_{i1,3}+m_{\text{sp}_{i2}} d_{i2}\bm{\hat{s}}_{i2,3}\Big]\ddot{\theta}_{i1} +m_{\text{sp}_{i2}} d_{i2} \bm{\hat{s}}_{i2,3}\ddot{\theta}_{i2}\bigg) \\
= \bm F - 2m_\text{sc} [\tilde{\bm\omega}_{\cal B/N}] \bm c'- m_\text{sc} [\tilde{\bm\omega}_{\cal B/N}][\tilde{\bm\omega}_{\cal B/N}]\bm{c}\\
-\sum_{i=1}^{N_{S}}\bigg(m_{\text{sp}_{i1}}d_{i1} \dot{\theta}_{i1}^2 \bm{\hat{s}}_{i1,1} +m_{\text{sp}_{i2}}\Big[l_{i1} \dot{\theta}_{i1}^2 \bm{\hat{s}}_{i1,1} + d_{i2}\big(\dot{\theta}_{i1} + \dot{\theta}_{i2}\big)^2\bm{\hat{s}}_{i2,1}\Big]\bigg) 
\label{eq:Rbddot4}
\end{multline}

	\begin{multline}
m_{\text{sc}} [\tilde{\bm{c}}] \ddot{\bm r}_{B/N} + [I_{\text{sc},B}] \dot{\bm\omega}_{\cal B/N} + \sum\limits_{i=1}^{N_S} \bigg[  \big(I_{s_{i1,2}}\bm{\hat{s}}_{i1,2}+m_{\text{sp}_{i1}}d_{i1} [\tilde{\bm{r}}_{S_{c,i1}/B}]   \bm{\hat{s}}_{i1,3} + I_{s_{i2,2}}\bm{\hat{s}}_{i2,2}\\
+m_{\text{sp}_{i2}}l_{i1} [\tilde{\bm{r}}_{S_{c,i2}/B}]  \bm{\hat{s}}_{i1,3}+m_{\text{sp}_{i2}}d_{i2} [\tilde{\bm{r}}_{S_{c,i2}/B}] \bm{\hat{s}}_{i2,3}\big) \ddot{\theta}_{i1}
+\big( I_{s_{i2,2}}\bm{\hat{s}}_{i2,2}+m_{\text{sp}_{i2}} d_{i2} [\tilde{\bm{r}}_{S_{c,i2}/B}] \bm{\hat{s}}_{i2,3}\big)\ddot{\theta}_{i2}\bigg] \\
= -[\bm{\tilde{\omega}}_{\cal B/N}] [I_{\text{sc},B}] \bm\omega_{\cal B/N} - [I'_{\text{sc},B}] \bm\omega_{\cal B/N} \\
-  \sum\limits_{i=1}^{N_S} \bigg[
\dot{\theta}_{i1} I_{s_{i1,2}} [\bm{\tilde{\omega}}_{\cal B/N}] \bm{\hat{s}}_{i1,2} 
+m_{\text{sp}_{i1}} [\bm{\tilde{\omega}}_{\cal B/N}] [\tilde{\bm{r}}_{S_{c,i1}/B}] \bm{r}'_{S_{c,i1}/B} +m_{\text{sp}_{i1}}d_{i1}\dot{\theta}_{i1}^2  [\tilde{\bm{r}}_{S_{c,i1}/B}] \bm{\hat{s}}_{i1,1}\\
+ \big(\dot{\theta}_{i1}+\dot{\theta}_{i2}\big) I_{s_{i2,2}}[\bm{\tilde{\omega}}_{\cal B/N}]\bm{\hat{s}}_{i2,2}
+m_{\text{sp}_{i2}} [\bm{\tilde{\omega}}_{\cal B/N}] [\tilde{\bm{r}}_{S_{c,i2}/B}] \bm{r}'_{S_{c,i2}/B} 	\\+m_{\text{sp}_{i2}} [\tilde{\bm{r}}_{S_{c,i2}/B}] \big(l_{i1} \dot{\theta}_{i1}^2 \bm{\hat{s}}_{i1,1} + d_{i2}\big(\dot{\theta}_{i1} + \dot{\theta}_{i2}\big)^2\bm{\hat{s}}_{i2,1}\big)\bigg]
+ \bm{L}_B 
\label{eq:Final7}
\end{multline}

Performing this substitution for translation yields:
\begin{multline}
m_\text{sc} \ddot{\bm r}_{B/N} -m_\text{sc} [\tilde{\bm{c}}] \dot{\bm\omega}_{\cal B/N} +  \sum_{i=1}^{N_{S}}\bigg(\Big[m_{\text{sp}_{i1}}d_{i1} \bm{\hat{s}}_{i1,3} +m_{\text{sp}_{i2}}l_{i1} \bm{\hat{s}}_{i1,3}+m_{\text{sp}_{i2}} d_{i2}\bm{\hat{s}}_{i2,3}\Big]\Big(e_{i1}^T[F_i] \ddot{\bm r}_{B/N} + e_{i1}^T[G_i]\dot{\bm\omega}_{\cal B/N} + e_{i1}^T\bm v_i\Big) \\+m_{\text{sp}_{i2}} d_{i2} \bm{\hat{s}}_{i2,3}\Big(e_{i2}^T[F_i] \ddot{\bm r}_{B/N} + e_{i2}^T[G_i]\dot{\bm\omega}_{\cal B/N} + e_{i2}^T\bm v_i\Big)\bigg) 
= \bm F - 2m_\text{sc} [\tilde{\bm\omega}_{\cal B/N}] \bm c'- m_\text{sc} [\tilde{\bm\omega}_{\cal B/N}][\tilde{\bm\omega}_{\cal B/N}]\bm{c}\\
-\sum_{i=1}^{N_{S}}\bigg(m_{\text{sp}_{i1}}d_{i1} \dot{\theta}_{i1}^2 \bm{\hat{s}}_{i1,1} +m_{\text{sp}_{i2}}\Big[l_{i1} \dot{\theta}_{i1}^2 \bm{\hat{s}}_{i1,1} + d_{i2}\big(\dot{\theta}_{i1} + \dot{\theta}_{i2}\big)^2\bm{\hat{s}}_{i2,1}\Big]\bigg) 
\label{eq:Rbddot5}
\end{multline}
Combining like terms yields:
\begin{multline}
\Bigg \lbrace m_\text{sc} [I_{3\times 3}] +  \sum_{i=1}^{N_{S}}\bigg[\Big(m_{\text{sp}_{i1}}d_{i1} \bm{\hat{s}}_{i1,3} +m_{\text{sp}_{i2}}l_{i1} \bm{\hat{s}}_{i1,3}+m_{\text{sp}_{i2}} d_{i2}\bm{\hat{s}}_{i2,3}\Big)e_{i1}^T[F_i] +m_{\text{sp}_{i2}} d_{i2} \bm{\hat{s}}_{i2,3}e_{i2}^T[F_i] \bigg] \Bigg \rbrace \ddot{\bm r}_{B/N} \\
+\Bigg \lbrace-m_\text{sc} [\tilde{\bm{c}}] +  \sum_{i=1}^{N_{S}}\bigg[\Big(m_{\text{sp}_{i1}}d_{i1} \bm{\hat{s}}_{i1,3} +m_{\text{sp}_{i2}}l_{i1} \bm{\hat{s}}_{i1,3}+m_{\text{sp}_{i2}} d_{i2}\bm{\hat{s}}_{i2,3}\Big) e_{i1}^T[G_i] +m_{\text{sp}_{i2}} d_{i2} \bm{\hat{s}}_{i2,3} e_{i2}^T[G_i]\bigg] \Bigg \rbrace \dot{\bm\omega}_{\cal B/N} \\
= \bm F - 2m_\text{sc} [\tilde{\bm\omega}_{\cal B/N}] \bm c'- m_\text{sc} [\tilde{\bm\omega}_{\cal B/N}][\tilde{\bm\omega}_{\cal B/N}]\bm{c}
-\sum_{i=1}^{N_{S}}\bigg(m_{\text{sp}_{i1}}d_{i1} \dot{\theta}_{i1}^2 \bm{\hat{s}}_{i1,1} +m_{\text{sp}_{i2}}\Big[l_{i1} \dot{\theta}_{i1}^2 \bm{\hat{s}}_{i1,1} + d_{i2}\big(\dot{\theta}_{i1} + \dot{\theta}_{i2}\big)^2\bm{\hat{s}}_{i2,1}\Big]\\
 +\Big[m_{\text{sp}_{i1}}d_{i1} \bm{\hat{s}}_{i1,3} +m_{\text{sp}_{i2}}l_{i1} \bm{\hat{s}}_{i1,3}+m_{\text{sp}_{i2}} d_{i2}\bm{\hat{s}}_{i2,3}\Big]e_{i1}^T\bm v_i 
+m_{\text{sp}_{i2}} d_{i2} \bm{\hat{s}}_{i2,3}e_{i2}^T\bm v_i \bigg) 
\label{eq:Rbddot6}
\end{multline}
Substitution into the rotational equation of motion:
	\begin{multline}
m_{\text{sc}} [\tilde{\bm{c}}] \ddot{\bm r}_{B/N} + [I_{\text{sc},B}] \dot{\bm\omega}_{\cal B/N} + \sum\limits_{i=1}^{N_S} \bigg[  \big[I_{s_{i1,2}}\bm{\hat{s}}_{i1,2}+m_{\text{sp}_{i1}}d_{i1} [\tilde{\bm{r}}_{S_{c,i1}/B}]   \bm{\hat{s}}_{i1,3} + I_{s_{i2,2}}\bm{\hat{s}}_{i2,2}\\
+m_{\text{sp}_{i2}}l_{i1} [\tilde{\bm{r}}_{S_{c,i2}/B}]  \bm{\hat{s}}_{i1,3}+m_{\text{sp}_{i2}}d_{i2} [\tilde{\bm{r}}_{S_{c,i2}/B}] \bm{\hat{s}}_{i2,3}\big] \Big(e_{i1}^T[F_i] \ddot{\bm r}_{B/N} + e_{i1}^T[G_i]\dot{\bm\omega}_{\cal B/N} + e_{i1}^T\bm v_i\Big)\\
+\big[ I_{s_{i2,2}}\bm{\hat{s}}_{i2,2}+m_{\text{sp}_{i2}} d_{i2} [\tilde{\bm{r}}_{S_{c,i2}/B}] \bm{\hat{s}}_{i2,3}\big]\Big(e_{i2}^T[F_i] \ddot{\bm r}_{B/N} + e_{i2}^T[G_i]\dot{\bm\omega}_{\cal B/N} + e_{i2}^T\bm v_i\Big)
\bigg] \\
= -[\bm{\tilde{\omega}}_{\cal B/N}] [I_{\text{sc},B}] \bm\omega_{\cal B/N} - [I'_{\text{sc},B}] \bm\omega_{\cal B/N} \\
-  \sum\limits_{i=1}^{N_S} \bigg[
\dot{\theta}_{i1} I_{s_{i1,2}} [\bm{\tilde{\omega}}_{\cal B/N}] \bm{\hat{s}}_{i1,2} 
+m_{\text{sp}_{i1}} [\bm{\tilde{\omega}}_{\cal B/N}] [\tilde{\bm{r}}_{S_{c,i1}/B}] \bm{r}'_{S_{c,i1}/B} +m_{\text{sp}_{i1}}d_{i1}\dot{\theta}_{i1}^2  [\tilde{\bm{r}}_{S_{c,i1}/B}] \bm{\hat{s}}_{i1,1}\\
+ \big(\dot{\theta}_{i1}+\dot{\theta}_{i2}\big) I_{s_{i2,2}}[\bm{\tilde{\omega}}_{\cal B/N}]\bm{\hat{s}}_{i2,2}
+m_{\text{sp}_{i2}} [\bm{\tilde{\omega}}_{\cal B/N}] [\tilde{\bm{r}}_{S_{c,i2}/B}] \bm{r}'_{S_{c,i2}/B} 	\\+m_{\text{sp}_{i2}} [\tilde{\bm{r}}_{S_{c,i2}/B}] \big(l_{i1} \dot{\theta}_{i1}^2 \bm{\hat{s}}_{i1,1} + d_{i2}\big(\dot{\theta}_{i1} + \dot{\theta}_{i2}\big)^2\bm{\hat{s}}_{i2,1}\big)\bigg]
+ \bm{L}_B 
\label{eq:Final7sub}
\end{multline}
And combining like terms yields:
	\begin{multline}
\Bigg \lbrace m_{\text{sc}} [\tilde{\bm{c}}] + \sum\limits_{i=1}^{N_S} \bigg[  \big(I_{s_{i1,2}}\bm{\hat{s}}_{i1,2}+m_{\text{sp}_{i1}}d_{i1} [\tilde{\bm{r}}_{S_{c,i1}/B}]   \bm{\hat{s}}_{i1,3} + I_{s_{i2,2}}\bm{\hat{s}}_{i2,2}
+m_{\text{sp}_{i2}}l_{i1} [\tilde{\bm{r}}_{S_{c,i2}/B}]  \bm{\hat{s}}_{i1,3}+m_{\text{sp}_{i2}}d_{i2} [\tilde{\bm{r}}_{S_{c,i2}/B}] \bm{\hat{s}}_{i2,3}\big)e_{i1}^T[F_i]\\
 +\big( I_{s_{i2,2}}\bm{\hat{s}}_{i2,2}+m_{\text{sp}_{i2}} d_{i2} [\tilde{\bm{r}}_{S_{c,i2}/B}] \bm{\hat{s}}_{i2,3}\big)e_{i2}^T[F_i] \bigg]\Bigg \rbrace \ddot{\bm r}_{B/N}
+ \Bigg \lbrace [I_{\text{sc},B}] + \sum\limits_{i=1}^{N_S} \bigg[  \big(I_{s_{i1,2}}\bm{\hat{s}}_{i1,2}+m_{\text{sp}_{i1}}d_{i1} [\tilde{\bm{r}}_{S_{c,i1}/B}]   \bm{\hat{s}}_{i1,3} + I_{s_{i2,2}}\bm{\hat{s}}_{i2,2}\\
+m_{\text{sp}_{i2}}l_{i1} [\tilde{\bm{r}}_{S_{c,i2}/B}]  \bm{\hat{s}}_{i1,3}+m_{\text{sp}_{i2}}d_{i2} [\tilde{\bm{r}}_{S_{c,i2}/B}] \bm{\hat{s}}_{i2,3}\big)e_{i1}^T[G_i] +\big( I_{s_{i2,2}}\bm{\hat{s}}_{i2,2}+m_{\text{sp}_{i2}} d_{i2} [\tilde{\bm{r}}_{S_{c,i2}/B}] \bm{\hat{s}}_{i2,3}\big)e_{i2}^T[G_i] \bigg]\Bigg \rbrace \dot{\bm\omega}_{\cal B/N}\\
= -[\bm{\tilde{\omega}}_{\cal B/N}] [I_{\text{sc},B}] \bm\omega_{\cal B/N} - [I'_{\text{sc},B}] \bm\omega_{\cal B/N} 
-  \sum\limits_{i=1}^{N_S} \bigg[
\dot{\theta}_{i1} I_{s_{i1,2}} [\bm{\tilde{\omega}}_{\cal B/N}] \bm{\hat{s}}_{i1,2} 
\\+m_{\text{sp}_{i1}} [\bm{\tilde{\omega}}_{\cal B/N}] [\tilde{\bm{r}}_{S_{c,i1}/B}] \bm{r}'_{S_{c,i1}/B} +m_{\text{sp}_{i1}}d_{i1}\dot{\theta}_{i1}^2  [\tilde{\bm{r}}_{S_{c,i1}/B}] \bm{\hat{s}}_{i1,1}
+ \big(\dot{\theta}_{i1}+\dot{\theta}_{i2}\big) I_{s_{i2,2}}[\bm{\tilde{\omega}}_{\cal B/N}]\bm{\hat{s}}_{i2,2}
+m_{\text{sp}_{i2}} [\bm{\tilde{\omega}}_{\cal B/N}] [\tilde{\bm{r}}_{S_{c,i2}/B}] \bm{r}'_{S_{c,i2}/B} 	\\+m_{\text{sp}_{i2}} [\tilde{\bm{r}}_{S_{c,i2}/B}] \big(l_{i1} \dot{\theta}_{i1}^2 \bm{\hat{s}}_{i1,1} + d_{i2}\big(\dot{\theta}_{i1} + \dot{\theta}_{i2}\big)^2\bm{\hat{s}}_{i2,1}\big) +   \big(I_{s_{i1,2}}\bm{\hat{s}}_{i1,2}+m_{\text{sp}_{i1}}d_{i1} [\tilde{\bm{r}}_{S_{c,i1}/B}]   \bm{\hat{s}}_{i1,3} + I_{s_{i2,2}}\bm{\hat{s}}_{i2,2}\\
+m_{\text{sp}_{i2}}l_{i1} [\tilde{\bm{r}}_{S_{c,i2}/B}]  \bm{\hat{s}}_{i1,3}+m_{\text{sp}_{i2}}d_{i2} [\tilde{\bm{r}}_{S_{c,i2}/B}] \bm{\hat{s}}_{i2,3}\big)e_{i1}^T\bm v_i+\big( I_{s_{i2,2}}\bm{\hat{s}}_{i2,2}+m_{\text{sp}_{i2}} d_{i2} [\tilde{\bm{r}}_{S_{c,i2}/B}] \bm{\hat{s}}_{i2,3}\big)e_{i2}^T\bm v_i\bigg]
+ \bm{L}_B 
\label{eq:Final8}
\end{multline}

With the following definitions:
\begin{multline}
[A_{\text{contr}}] = \sum_{i=1}^{N_{S}}\bigg[\Big(m_{\text{sp}_{i1}}d_{i1} \bm{\hat{s}}_{i1,3} +m_{\text{sp}_{i2}}l_{i1} \bm{\hat{s}}_{i1,3}+m_{\text{sp}_{i2}} d_{i2}\bm{\hat{s}}_{i2,3}\Big)e_{i1}^T[F_i] +m_{\text{sp}_{i2}} d_{i2} \bm{\hat{s}}_{i2,3}e_{i2}^T[F_i] \bigg]
\end{multline}\begin{multline}
[B_{\text{contr}}] =  \sum_{i=1}^{N_{S}}\bigg[\Big(m_{\text{sp}_{i1}}d_{i1} \bm{\hat{s}}_{i1,3} +m_{\text{sp}_{i2}}l_{i1}\bm{\hat{s}}_{i1,3}+m_{\text{sp}_{i2}} d_{i2}\bm{\hat{s}}_{i2,3}\Big) e_{i1}^T[G_i] +m_{\text{sp}_{i2}} d_{i2} \bm{\hat{s}}_{i2,3} e_{i2}^T[G_i]\bigg]
\end{multline}\begin{multline}
\bm v_{\text{trans,contr}} = -\sum_{i=1}^{N_{S}}\bigg(m_{\text{sp}_{i1}}d_{i1} \dot{\theta}_{i1}^2 \bm{\hat{s}}_{i1,1} +m_{\text{sp}_{i2}}\Big[l_{i1} \dot{\theta}_{i1}^2 \bm{\hat{s}}_{i1,1} + d_{i2}\big(\dot{\theta}_{i1} + \dot{\theta}_{i2}\big)^2\bm{\hat{s}}_{i2,1}\Big]\\
+\Big[m_{\text{sp}_{i1}}d_{i1} \bm{\hat{s}}_{i1,3} +m_{\text{sp}_{i2}}l_{i1} \bm{\hat{s}}_{i1,3}+m_{\text{sp}_{i2}} d_{i2}\bm{\hat{s}}_{i2,3}\Big]e_{i1}^T\bm v_i 
+m_{\text{sp}_{i2}} d_{i2} \bm{\hat{s}}_{i2,3}e_{i2}^T\bm v_i \bigg) 
\end{multline}\begin{multline}
[C_{\text{contr}}] = \sum\limits_{i=1}^{N_S} \bigg[  \big(I_{s_{i1,2}}\bm{\hat{s}}_{i1,2}+m_{\text{sp}_{i1}}d_{i1} [\tilde{\bm{r}}_{S_{c,i1}/B}]   \bm{\hat{s}}_{i1,3} + I_{s_{i2,2}}\bm{\hat{s}}_{i2,2}
+m_{\text{sp}_{i2}}l_{i1} [\tilde{\bm{r}}_{S_{c,i2}/B}]  \bm{\hat{s}}_{i1,3}+m_{\text{sp}_{i2}}d_{i2} [\tilde{\bm{r}}_{S_{c,i2}/B}] \bm{\hat{s}}_{i2,3}\big)e_{i1}^T[F_i]\\
+\big( I_{s_{i2,2}}\bm{\hat{s}}_{i2,2}+m_{\text{sp}_{i2}} d_{i2} [\tilde{\bm{r}}_{S_{c,i2}/B}] \bm{\hat{s}}_{i2,3}\big)e_{i2}^T[F_i] \bigg]
\end{multline}\begin{multline}
[D_{\text{contr}}] = \sum\limits_{i=1}^{N_S} \bigg[  \big(I_{s_{i1,2}}\bm{\hat{s}}_{i1,2}+m_{\text{sp}_{i1}}d_{i1} [\tilde{\bm{r}}_{S_{c,i1}/B}]   \bm{\hat{s}}_{i1,3} + I_{s_{i2,2}}\bm{\hat{s}}_{i2,2}+m_{\text{sp}_{i2}}l_{i1} [\tilde{\bm{r}}_{S_{c,i2}/B}]  \bm{\hat{s}}_{i1,3}\\
+m_{\text{sp}_{i2}}d_{i2} [\tilde{\bm{r}}_{S_{c,i2}/B}] \bm{\hat{s}}_{i2,3}\big)e_{i1}^T[G_i] +\big( I_{s_{i2,2}}\bm{\hat{s}}_{i2,2}+m_{\text{sp}_{i2}} d_{i2} [\tilde{\bm{r}}_{S_{c,i2}/B}] \bm{\hat{s}}_{i2,3}\big)e_{i2}^T[G_i] \bigg]\
\end{multline}\begin{multline}
[v_{\text{rot,contr}}] = -\sum\limits_{i=1}^{N_S} \bigg[
\dot{\theta}_{i1} I_{s_{i1,2}} [\bm{\tilde{\omega}}_{\cal B/N}] \bm{\hat{s}}_{i1,2} 
\\+m_{\text{sp}_{i1}} [\bm{\tilde{\omega}}_{\cal B/N}] [\tilde{\bm{r}}_{S_{c,i1}/B}] \bm{r}'_{S_{c,i1}/B} +m_{\text{sp}_{i1}}d_{i1}\dot{\theta}_{i1}^2  [\tilde{\bm{r}}_{S_{c,i1}/B}] \bm{\hat{s}}_{i1,1}
+ \big(\dot{\theta}_{i1}+\dot{\theta}_{i2}\big) I_{s_{i2,2}}[\bm{\tilde{\omega}}_{\cal B/N}]\bm{\hat{s}}_{i2,2}
+m_{\text{sp}_{i2}} [\bm{\tilde{\omega}}_{\cal B/N}] [\tilde{\bm{r}}_{S_{c,i2}/B}] \bm{r}'_{S_{c,i2}/B} 	\\+m_{\text{sp}_{i2}} [\tilde{\bm{r}}_{S_{c,i2}/B}] \big(l_{i1} \dot{\theta}_{i1}^2 \bm{\hat{s}}_{i1,1} + d_{i2}\big(\dot{\theta}_{i1} + \dot{\theta}_{i2}\big)^2\bm{\hat{s}}_{i2,1}\big) +   \big(I_{s_{i1,2}}\bm{\hat{s}}_{i1,2}+m_{\text{sp}_{i1}}d_{i1} [\tilde{\bm{r}}_{S_{c,i1}/B}]   \bm{\hat{s}}_{i1,3} + I_{s_{i2,2}}\bm{\hat{s}}_{i2,2}\\
+m_{\text{sp}_{i2}}l_{i1} [\tilde{\bm{r}}_{S_{c,i2}/B}]  \bm{\hat{s}}_{i1,3}+m_{\text{sp}_{i2}}d_{i2} [\tilde{\bm{r}}_{S_{c,i2}/B}] \bm{\hat{s}}_{i2,3}\big)e_{i1}^T\bm v_i+\big( I_{s_{i2,2}}\bm{\hat{s}}_{i2,2}+m_{\text{sp}_{i2}} d_{i2} [\tilde{\bm{r}}_{S_{c,i2}/B}] \bm{\hat{s}}_{i2,3}\big)e_{i2}^T\bm v_i\bigg]
\end{multline}\begin{equation}
[A]  = m_\text{sc} [I_{3\times 3}] + [A_{\text{contr}}]
\end{equation}\begin{equation}
[B] = -m_\text{sc} [\tilde{\bm{c}}] + [B_{\text{contr}}]
\end{equation}\begin{equation}
\bm v_{\text{trans}} = \bm F - 2m_\text{sc} [\tilde{\bm\omega}_{\cal B/N}] \bm c'- m_\text{sc} [\tilde{\bm\omega}_{\cal B/N}][\tilde{\bm\omega}_{\cal B/N}]\bm{c} + \bm v_{\text{trans,contr}}
\end{equation}\begin{equation}
[C] = m_{\text{sc}} + [C_{\text{contr}}]
\end{equation}\begin{equation}
[D] =  [I_{\text{sc},B}] + [D_{\text{contr}}]
\end{equation}\begin{equation}
\bm v_{\text{rot}} = -[\bm{\tilde{\omega}}_{\cal B/N}] [I_{\text{sc},B}] \bm\omega_{\cal B/N} - [I'_{\text{sc},B}] \bm\omega_{\cal B/N} + \bm{L}_B + \bm v_{\text{rot,contr}}
\end{equation}


This produces the following simplified equations:

	\begin{equation}
\begin{bmatrix}
[A] & [B]\\
[C] & [D]
\end{bmatrix} \begin{bmatrix}
\ddot{\bm r}_{B/N}\\
\dot{\bm\omega}_{\cal B/N}
\end{bmatrix} = \begin{bmatrix}
\bm v_{\text{trans}}\\
\bm v_{\text{rot}}
\end{bmatrix}
\end{equation}

Solving the system-of-equations by

\begin{equation}
\dot{\bm\omega}_{\cal B/N} = \Big([D] - [C]][A]^{-1}[B]\Big)^{-1}(\bm v_{\text{rot}} - [C][A]^{-1}\bm v_{\text{trans}})
\label{eq:omegaDot}
\end{equation}

\begin{equation}
\ddot{\bm r}_{B/N} = [A]^{-1} (\bm v_{\text{trans}} - [B]\dot{\bm\omega}_{\cal B/N})
\label{eq:rBNDDot}
\end{equation}
	
	Now Eq.~\eqref{eq:omegaDot} and ~\eqref{eq:rBNDDot} can be used to solve for $\dot{\bm\omega}_{\cal B/N}$ and $\ddot{\bm r}_{B/N}$. Once these second order state variables are solved for, Eqs.~\eqref{eq:thetadot4} and ~\eqref{eq:thetadot5} can be used to directly solve for $\ddot \theta_{i1}$ and $\ddot \theta_{i2}$. This shows that the back substitution method can work seamlessly for interconnected bodies. For this problem the number of interconnected bodies was fixed to be 2, and resulted in an additional $2\times 2$ matrix inversion for each solar panel pair. This shows that for general interconnected bodies, this method would result in needing to invert a matrix based on the number of interconnected bodies. 
	
	In addition, this problem retains the ability for modularity in software architecture design while not sacrificing any accuracy. The dual-linked hinged rigid bodies were implemented in Basilisk and fit into the architecture quite easily. The code (the .cpp and .h files for the modules) is attached to the end of this document. It shows how smoothly this approach can integrate into the modular dynamics architecture. The code is on a branch and is currently being verified for energy and momentum validation. However, I know that I have a bug somewhere either in the code or in the derivation because the simulation is blowing up. 
		
	\section{Conclusion}
	
	Modeling the spacecraft as a rigid body can be a great assumption and applicable to many missions and scenarios. However, in some cases the rigid body assumption does not give enough fidelity of the actual dynamics of the spacecraft and analysis could suffer due to this. One major concern for spacecraft are flexing appended bodies. These could be solar panels, antennas, extended platforms, etc. These flexing bodies will influence the dynamics and can affect precision pointing, performance and general behaviors. Modeling this can be very helpful for spacecraft design and analysis. 
	
	In this paper the equations of motion for a spacecraft with dual-linked hinged rigid bodies is developed. This EOMs are fully coupled and do not make any approximations that would violate energy and momentum. This allows for a simulation to verify energy and momentum which is an essential tool when developing these complex systems. 
	
	A goal of this project was to determine the impact on the back-substitution method for interconnected bodies. The results confirm that the back-substitution method can be applied for interconnected bodies which is a great insight because the computational impact on interconnected systems without using the back substitution method would be great. With the current implementation, the dual-linked hinged rigid bodies requires a $2\times 2$ matrix inverse for every pair of hinged rigid bodies in the system. 
	
	Furthermore, a goal of the back-substitution method is to allow for the system to be modular from a computer science perspective. In prior work, there were no interconnected appended bodies and the back-substitution method resulted in a very modular way of computing the dynamics. The result presented in this paper shows that interconnected bodies also result in this modular design. 
	
	
	\bibliographystyle{aiaa}   % Number the references.
	\bibliography{references}   % Use references.bib to resolve the labels.
	
\end{document}

