% README file for moduleDocumentationTemplate TeX template.
% This template should be used to document all Basilisk modules.
% Updated 20170711 - S. Carnahan
%
%-Copy the contents of this folder to your own _Documentation folder
%
%-Rename the Basilisk-moduleDocumentationTemplate.tex appropriately
%
% All edits should be made in one of:
% sec_modelAssumptionsLimitations.tex
% sec_modelDescription.tex
% sec_modelFunctions.tex
% sec_revisionTable.tex
% sec_testDescription.tex
% sec_testParameters.tex
% sec_testResults.tex
% sec_user_guide.tex
%
%-Some rules about referencing within the document:
%1. If writing the suer guide, assume the module description is present
%2. If writing the validation section, assume the module features section is present
%3. Make no other assumptions about any sections being present. This allow for sections of the document to be used elsewhere without breaking.

%In order to import some of these sections into a document in a different directory:
%\usepackage{import}
%Then, the sections are called with \subimport{relative path}{file} in order to \input{file} using the right relative path.
%\import{full path}{file} can also be used if absolute paths are preferred over relative paths.

%%%%%%%%%%%%%%%%%%%%%%%%%%%%%%%%%%%%%%%%%%%%%%%%%




\documentclass[]{BasiliskReportMemo}

\usepackage{cite}
\usepackage{AVS}
\usepackage{float} %use [H] to keep tables where you put them
\usepackage{array} %easy control of text in tables
\usepackage{graphicx}
\bibliographystyle{plain}


\newcommand{\submiterInstitute}{Autonomous Vehicle Simulation (AVS) Laboratory,\\ University of Colorado}


\newcommand{\ModuleName}{spacecraftSystem}
\newcommand{\subject}{System of Spacecraft Equations of Motion Model}
\newcommand{\status}{To be Reviewed}
\newcommand{\preparer}{C. Allard}
\newcommand{\summary}{SpacecraftSystem is an instantiation of the dynamicObject abstract class. This abstract class is representing systems that have equations of motion that need to be integrated and therefore the main goal of this dynamic object is finding the state derivatives and interfacing with the integrator to integrate the state forward in time. The spacecraftSystem is specifically geared towards simulating multiple spacecraft at a time that will allow for docking and detaching of spacecraft. If spacecraft is ever removed then this will be the sole dynamicObject that will control the spacecraft dynamics of the system. This current module is IN PROGRESS and burning edge development in its current state.}

\begin{document}

\makeCover

%
%	enter the revision documentation here
%	to add more lines, copy the table entry and the \hline, and paste after the current entry.
%
\pagestyle{empty}
{\renewcommand{\arraystretch}{2}
\noindent
\begin{longtable}{|p{0.5in}|p{3.5in}|p{1.07in}|p{0.9in}|}
\hline
{\bfseries Rev} & {\bfseries Change Description} & {\bfseries By}& {\bfseries Date} \\
\hline
1.0 & Initial Draft (Module is still IN PROGRESS) & C. Allard & 20180613\\
\hline
1.1 & Renamed classes to {\tt spacecraftSystem} and {\tt spacecraftUnit} & H. Schaub & 2020-02-15\\
\hline
\end{longtable}
}



\newpage
\setcounter{page}{1}
\pagestyle{fancy}

\tableofcontents %Autogenerate the table of contents
~\\ \hrule ~\\ %Makes the line under table of contents










	
\section{Model Description}

This utility produces white noise. It can have a bias added to it to make a non-zero mean. It also generates "random walk" by treating the noise as additive. Walk bounds can be provided that push the random walk away from the boundary. Setting the bounds to zero or less disables the random walk, this is the default.
 %This section includes mathematical models, code description, etc.

\section{Model Functions}
The mathematical description of gravity effects are implemented in gravityEffector.cpp. This code performs the following primary functions
\begin{itemize}
	\item \textbf{Cannonball Method}: The code calculates the force on a spacecraft due to solar radiation pressure.
	\item \textbf{Look-up Method}: The code calculates both the force and torque on a spacecraft due to solar radiation pressure. It uses user-provided tabulated data to do so.
	\item \textbf{Solar Eclipse}: The code takes solar eclipses into account via a "shadow factor". This shadow factor is output from the Basilisk solar eclipse module and can include the effects of multiple planets. It is applied to the force/torque outputs.
	\item \textbf{Interface: Spacecraft States}: The code receives spacecraft state information via the DynParamManager.
	\item \textbf{Interface: Forces and Torques}: The code sends spacecraft force and torque contributions via computeForceTorque() which is called by the spacecraft.	If using the cannonball method, the returned torque values are zero.
	\item \textbf{Interface: Sun Ephemeris}: The code receives Sun states (ephemeris information) via the Basilisk messaging system.
	\item \textbf{Interface: Solar Eclipse}: The code receives solar eclipse (shadow factor) information via the Basilisk messaging system.
	
\end{itemize}



\section{Model Assumptions and Limitations}
The two methods of calculation used in this code have their own sets of assumptions and limitations. There are some assumptions which are common to both methods.
\begin{itemize}
	\item \textbf{Cannonball Model}: This default solar radiation pressure model assumes that the radiation pressure will act normal to some equivalent surface area, $A_{\odot}$. While this could be a good assumption, $A_{\odot}$ would have to be time-varying with spacecraft attitude and incorporate spacecraft self-shadowing. In general, the code does not do this. This limits the cannonball method to being most accurate in relatively mundane simulations (no rapid rotations or varying self-shadowing). Additionally, this method does not calculate torques on the spacecraft, so it is limited to cases where high precision is not needed with regards to spacecraft attitude.
	\item \textbf{Look-up Method}: This method utilizes tabulated data. Therefore, there will be error associated with whatever method was used to generate and record the data, but those errors are outside of Basilisk. Furthermore, the algorithm selects the data which \textit{most closely} matches the current position of the spacecraft relative to the sun and does not interpolate between data points.  This method also is limited to users who have external models or real data to use to describe their spacecraft.
	\item \textbf{Radiation}: The radiation model is hard-coded to assume that the radiation comes from the Sun. It is not possible to model radiation pressure from other sources with this code. This applies to both the cannonball and look-up methods. The model has no time-varying radiation effects (solar storms, etc.). A more in-depth radiation model would be need if high-accuracy radiation pressure effect calculations are needed.
	\item \textbf{Eclipse}: The shadow factor applies a simple scaling factor to the output forces and torques. This assumes that all portions of the spacecraft are affected equally by the eclipse. This should, in most circumstances, be highly accurate. For exceptionally large $A_{\odot}$ spacecraft which also need highly accurate state calculations, this assumption could fail.
	\item \textbf{Tabulated Data Import} Currently, Basilisk includes a utility script to import data from XML files for use in radiation pressure calculations. While some users could learn to load data in other formats, this is currently a limitation to most users who have data in other forms.
\end{itemize} %This includes a concise list of what the module does. It also includes model assumptions and limitations

% !TEX root = ./Basilisk-planetEphemeris-20190422.tex

\section{Test Description and Success Criteria}
The unit test configures the module to model general orbital motions of Earth and Venus.  In each case the translational motion is compute for 3 times steps from 0 to 1 second using a 0.5 second time step.  The planet orientation information is only set if the appropriate simulation parameter is set.  The following sub-sections discuss these flags.  If none of these flags are set, then the module default orientation behavior is expected.  If all the flags are set, then a constant rotation is set.  If only a partial set of orientation information is provided then the reset routine will force the module to throw an error message and only output a default constant zero orientation of the planet. 

\subsection{{\tt setRAN}}
This flag specifies if a set of right ascension angles are specified in the unit test.

\subsection{{\tt setDEC}}
This flag specifies if a set of declination angles are specified in the unit test.

\subsection{{\tt setLST}}
This flag specifies if a set of local sidereal time angles are specified in the unit test.

\subsection{{\tt setRate}}
This flag specifies if a set of planetary polar rotation rates are specified in the unit test.




\section{Test Parameters}
The unit test verify that the module output guidance message vectors match expected values.
\begin{table}[htbp]
	\caption{Error tolerance for each test.}
	\label{tab:errortol}
	\centering \fontsize{10}{10}\selectfont
	\begin{tabular}{ c | c } % Column formatting, 
		\hline\hline
		\textbf{Output Value Tested}  & \textbf{Tolerated Error}  \\ 
		\hline
		{\tt J2000Current}        & \input{AutoTeX/toleranceValue} s   \\ 
		{\tt PositionVector}        & \input{AutoTeX/toleranceValue} m   \\ 
		{\tt VelocityVector}        & \input{AutoTeX/toleranceValue} m/s   \\ 
		{\tt J20002Pfix}        & \input{AutoTeX/toleranceValue}    \\ 
		{\tt J20002Pfix\_dot}        & $10^{-10}$ rad/s   \\ 
		{\tt computeOrient}        & \input{AutoTeX/toleranceValue}    \\ 
		\hline\hline
	\end{tabular}
\end{table}




\section{Test Results}
All orientation flag permutations are tested and are expected to pass.

\begin{table}[H]
	\caption{Test results}
	\label{tab:results}
	\centering \fontsize{10}{10}\selectfont
	\begin{tabular}{c | c  | c | c | c } % Column formatting, 
		\hline\hline
		{\tt setRAN} & {\tt setDEC} & {\tt setLST} & {\tt setRate} &\textbf{Pass/Fail} \\ 
		\hline
	   True & True & True & True & \input{AutoTeX/passFailTrueTrueTrueTrue} \\ 
	   True & True & True & False & \input{AutoTeX/passFailTrueTrueTrueFalse} \\ 
	   True & True & False & True & \input{AutoTeX/passFailTrueTrueFalseTrue} \\ 
	   True & True & False & False & \input{AutoTeX/passFailTrueTrueFalseFalse} \\ 
	   True & False & True & True & \input{AutoTeX/passFailTrueFalseTrueTrue} \\ 
	   True & False & True & False & \input{AutoTeX/passFailTrueFalseTrueFalse} \\ 
	   True & False & False & True & \input{AutoTeX/passFailTrueFalseFalseTrue} \\ 
	   True & False & False & False & \input{AutoTeX/passFailTrueFalseFalseFalse} \\ 
	   False & True & True & True & \input{AutoTeX/passFailFalseTrueTrueTrue} \\ 
	   False & True & True & False & \input{AutoTeX/passFailFalseTrueTrueFalse} \\ 
	   False & True & False & True & \input{AutoTeX/passFailFalseTrueFalseTrue} \\ 
	   False & True & False & False & \input{AutoTeX/passFailFalseTrueFalseFalse} \\ 
	   False & False & True & True & \input{AutoTeX/passFailFalseFalseTrueTrue} \\ 
	   False & False & True & False & \input{AutoTeX/passFailFalseFalseTrueFalse} \\ 
	   False & False & False & True & \input{AutoTeX/passFailFalseFalseFalseTrue} \\ 
	   False & False & False & False & \input{AutoTeX/passFailFalseFalseFalseFalse} \\ 
	   \hline\hline
	\end{tabular}
\end{table}

 % This includes test description, test parameters, and test results

\section{User Guide}
When using this model, the user should follow the setup procedure corresponding to his or her desired conversion described below. The procedures outline the required inputs and for each conversion and some recommended parameter values.
\subsection{Element to Cartesian}
	\begin{itemize}
		\item \textit{Elements2Cart} = \textit{True}
		\item \textit{useEphemFormat}
		\begin{itemize}
			\item \textit{True} for planet state data
			\item \textit{False} for spacecraft state data
		\end{itemize}
		\item \textit{inputsGood} = True
		\item $\mu$ is recommended to be $3.86\text{e+}14$ $\frac{\text{m}^3}{\text{s}^2}$.
		\item Keplerian orbital elements should abide by the cases layed out in the Model Functions section.
	\end{itemize}
\subsection{Cartesian to Element}
	\begin{itemize}
		\item \textit{Elements2Cart} = \textit{False}
		\item \textit{useEphemFormat}
		\begin{itemize}
			\item \textit{True} for planet state data
			\item \textit{False} for spacecraft state data
		\end{itemize}
		\item \textit{inputsGood} = \textit{True}
		\item $\mu$ is recommended to be $3.86\text{e+}14$ $\frac{\text{m}^3}{\text{s}^2}$.
		\item Recommended Cartesian vectors can be obtained from Figures \ref{fig:2} through \ref{fig:13}, which correspond to the allowed orbit types.
	\end{itemize}

\subsection{Variable Definition and Code Description}
The variables in Table \ref{tabular:vars} are available for user input. Variables used by the module but not available to the user are not mentioned here. Variables with default settings do not necessarily need to be changed by the user, but may be.
\begin{table}[H]
	\caption{Definition and Explanation of Variables Used.}
	\label{tab:errortol}
	\centering \fontsize{10}{10}\selectfont
	\begin{tabular}{  m{3cm}| m{3cm} | m{3cm} | m{6cm} } % Column formatting, 
		\hline
		\textbf{Variable}   							& \textbf{LaTeX Equivalent} 	&		\textbf{Variable Type} & \textbf{Notes}			  \\ \hline
		r$_N$	&$\bm{r}$ & double & [km]Default setting: 0.0. Position vector either used as an input to or obtained as an output from the conversions.\\ \hline
		v$_N$	& $\bm{\dot{r}}$ & double & [km/s]Default setting: 0.0. Velocity vector either used as an input to or obtained as an output from the conversions.\\ 
		\hline
		$\mu$	& $\mu$ & double & [m3/s2] Required Input. This is the gravitational parameter of the body. Replaces the product of Earth's gravitational force and mass for this test..\\
		\hline
		a & $a$ & double & [km] Required Input. The semimajor axis of the body's orbit.\\ 
		\hline
		e & $e$ & double & Required Input. The eccentricity of the body's orbit.\\ 
		\hline
		i & $i$ & double & [rad] Required Input. The inclination of the body's orbit\\ 
		\hline
		Omega & $\Omega$ & double & [rad] Required Input. The ascending node of the body's orbit. \\ 
		\hline
		omega & $\omega$ & double & [rad] Required Input. The argument of periapses of the body's orbit. \\ 
		\hline
		f & $f$ & double & [rad] Required Input. The true anomaly of the body's orbit \\
		\hline
		Elements2Cart & N/A & bool & Default Setting: False. Identifies the desired conversion. \\ 
		\hline
		useEphemFormat & N/A & bool & Default Setting: False. Identifies whether the body is a spacecraft or a planet.\\
		\hline
		inputsGood & N/A & bool & Default Setting: False. Indicates whether the code reads valid state data for conversion.\\
		\hline
	\end{tabular}
	\label{tabular:vars}
\end{table}
\begin{thebibliography}{1}
	\bibitem{bib:1}
	Vallado, D. A., and McClain, W. D., \textit{Fundamentals of Astrodynamics and Applications, 4th ed}. Hawthorne, CA: Published by Microcosm Press, 2013.
	\bibitem{bib:2}
	Schaub, H., and Junkins, J. L., \textit{Analytical Mechanics of Space Systems, 3rd ed.}. Reston, VA: American Institute of Aeronautics and Astronautics.
\end{thebibliography} % Contains a discussion of how to setup and configure  the BSK module





\bibliography{bibliography} %This includes references used and mentioned.

\end{document}
