\section{Test Description and Success Criteria}
The tests are located in \texttt{simulation/dynamics/VSCMGs/\_UnitTest/\newline
test\_VSCMGStateEffector\_integrated.py} \textbf{and} \texttt{simulation/dynamics/VSCMGs/\newline\_UnitTest/
test\_VSCMGStateEffector\_ConfigureVSCMGRequests.py}. Depending on the test, there are different success criteria. These are outlined in the following subsections:
\subsection{Balanced Wheels Scenario - Integrated Test}
In this test the simulation is placed into orbit around Earth with point gravity, has 3 VSCMGs attached to the spacecraft, and they are in ``Balanced" mode. Each wheel is given a commanded torque for half the simulation and the rest of the simulation the torques are set to zero. The following parameters are being tested:
\begin{itemize}
	\item Conservation of orbital angular momentum
	\item Conservation of orbital energy
	\item Conservation of rotational angular momentum
	\item Conservation of rotational energy (second half of the simulation)
	\item Achieving the expected final attitude
	\item Achieving the expected final position
\end{itemize}

\subsection{Simple Jitter Scenario - Integrated Test}
In this test the simulation is placed into orbit around Earth with point gravity, has 3 VSCMGs attached to the spacecraft, and they are in ``Simple Jitter" mode. Each wheel is given a commanded torque for half the simulation and the rest of the simulation the torques are set to zero. The following parameters are being tested:
\begin{itemize}
\item Achieving the expected final attitude
\item Achieving the expected final position
\end{itemize}

\subsection{Fully Coupled Jitter Scenario - Integrated Test}
In this test the simulation is placed into orbit around Earth with point gravity, has 3 VSCMGs attached to the spacecraft, and they are in ``Fully Coupled Jitter" mode. Each wheel is given a commanded torque for half the simulation and the rest of the simulation the torques are set to zero. The following parameters are being tested:
\begin{itemize}
\item Conservation of orbital angular momentum
\item Conservation of orbital energy
\item Conservation of rotational angular momentum
\item Conservation of rotational energy (second half of the simulation)
\item Achieving the expected final attitude
\item Achieving the expected final position
\end{itemize}

\section{Test Parameters}

Since this is an integrated test, the inputs to the test are the physical parameters of the spacecraft along with the initial conditions of the states. These parameters are outlined in Tables~\ref{tab:hub}-~\ref{tab:initial}. Additionally, the error tolerances can be seen in Table~\ref{tab:errortol}. The error tolerances are different depending on the test. The energy-momentum conservation values will normally have an agreement down to 1e-14, but to ensure cross-platform agreement the tolerance was chose to be 1e-10. The position and attitude checks have a tolerance set to 1e-7 and is because 8 significant digits were chosen as the values being compared to. The BOE tests depend on the integration time step but as the time step gets smaller the accuracy gets better. So 1e-8 tolerance was chosen so that a larger time step could be used but still show agreement. The Friction tests give the same numerical outputs down to ~1e-15 between python and Basilisk, but 1e-10 was chosen to ensure cross platform agreement. Finally, the saturation and minimum torque tests have 1e-10 to ensure cross-platform success, but these values will typically agree to machine precision. 

\begin{table}[htbp]
	\caption{Spacecraft Hub Parameters for Energy Momentum Conservation Scenarios}
	\label{tab:hub}
	\centering \fontsize{10}{10}\selectfont
	\begin{tabular}{ c | c | c | c } % Column formatting, 
		\hline
		\textbf{Name}  & \textbf{Description}  & \textbf{Value} & \textbf{Units} \\
		\hline
		mHub  & mass & 750.0 & kg \\
		IHubPntBc\_B & Inertia in $\cal{B}$ frame & $\begin{bmatrix}
		900.0 & 0.0 & 0.0\\
		0.0 & 800.0 & 0.0\\
		0.0 & 0.0 & 600.0
		\end{bmatrix}$ & kg-m$^2$ \\
		r\_BcB\_B & CoM Location in $\cal{B}$ frame & $\begin{bmatrix}
		-0.0002 & 0.0001 & 0.1 \end{bmatrix}^T$ & m \\
		\hline
	\end{tabular}
\end{table}

\begin{table}[htbp]
	\caption{VSCMG 1 Parameters for Energy Momentum Conservation Scenarios}
	\label{tab:rw1}
	\centering \fontsize{10}{10}\selectfont
	\begin{tabular}{ c | c | c | c } % Column formatting, 
		\hline
		\textbf{Name}  & \textbf{Description}  & \textbf{Value} & \textbf{Units} \\
		\hline
		Js  & Spin Axis Inertia & 0.159 & kg-m$^2$ \\
		mass & mass & 12.0 & kg \\
		U\_s & Static Imbalance & 4.8E-6 & kg-m \\
		U\_d & Dynamic Imbalance & 15.4E-7 & kg-m$^2$ \\
		gsHat\_B & Spin Axis in $\cal{B}$ frame & $\begin{bmatrix}
		1.0 & 0.0 & 0.0 \end{bmatrix}^T$ & - \\
		rWB\_B & Location of Wheel in $\cal{B}$ frame & $\begin{bmatrix}
		0.1 & 0.0 & 0.0 \end{bmatrix}^T$ & m \\
		\hline
	\end{tabular}
\end{table}

\begin{table}[htbp]
	\caption{VSCMG 2 Parameters for Energy Momentum Conservation Scenarios}
	\label{tab:rw2}
	\centering \fontsize{10}{10}\selectfont
	\begin{tabular}{ c | c | c | c } % Column formatting, 
		\hline
		\textbf{Name}  & \textbf{Description}  & \textbf{Value} & \textbf{Units} \\
		\hline
		Js  & Spin Axis Inertia & 0.159 & kg-m$^2$ \\
		mass & mass & 12.0 & kg \\
		U\_s & Static Imbalance & 4.8E-6 & kg-m \\
		U\_d & Dynamic Imbalance & 15.4E-7 & kg-m$^2$ \\
		gsHat\_B & Spin Axis in $\cal{B}$ frame & $\begin{bmatrix}
		0.0 & 1.0 & 0.0 \end{bmatrix}^T$ & - \\
		rWB\_B & Location of Wheel in $\cal{B}$ frame & $\begin{bmatrix}
		0.0 & 0.1 & 0.0 \end{bmatrix}^T$ & m \\
		\hline
	\end{tabular}
\end{table}

\begin{table}[htbp]
	\caption{VSCMG 3 Parameters for Energy Momentum Conservation Scenarios}
	\label{tab:rw3}
	\centering \fontsize{10}{10}\selectfont
	\begin{tabular}{ c | c | c | c } % Column formatting, 
		\hline
		\textbf{Name}  & \textbf{Description}  & \textbf{Value} & \textbf{Units} \\
		\hline
		Js  & Spin Axis Inertia & 0.159 & kg-m$^2$ \\
		mass & mass & 12.0 & kg \\
		U\_s & Static Imbalance & 4.8E-6 & kg-m \\
		U\_d & Dynamic Imbalance & 15.4E-7 & kg-m$^2$ \\
		gsHat\_B & Spin Axis in $\cal{B}$ frame & $\begin{bmatrix}
		0.0 & 0.0 & 1.0 \end{bmatrix}^T$ & - \\
		rWB\_B & Location of Wheel in $\cal{B}$ frame & $\begin{bmatrix}
		0.0 & 0.0 & 0.1 \end{bmatrix}^T$ & m \\
		\hline
	\end{tabular}
\end{table}

\begin{table}[htbp]
	\caption{VSCMG 1 parameters for friction tests}
	\label{tab:rwFriction}
	\centering \fontsize{10}{10}\selectfont
	\begin{tabular}{ c | c | c | c } % Column formatting, 
		\hline
		\textbf{Name}  & \textbf{Description}  & \textbf{Value} & \textbf{Units} \\
		\hline
		Js  & Spin Axis Inertia & 0.159 & kg-m$^2$ \\
		mass & mass & 12.0 & kg \\
		gsHat\_B & Spin Axis in $\cal{B}$ frame & $\begin{bmatrix}
		\frac{\sqrt{3}}{3} & \frac{\sqrt{3}}{3} & \frac{\sqrt{3}}{3} \end{bmatrix}^T$ & - \\
		rWB\_B & Location of Wheel in $\cal{B}$ frame & $\begin{bmatrix}
		0.5 & -0.5 & 0.5 \end{bmatrix}^T$ & m \\
		\hline
	\end{tabular}
\end{table}

\begin{table}[htbp]
	\caption{VSCMG 2 parameters for friction tests}
	\label{tab:rwFriction2}
	\centering \fontsize{10}{10}\selectfont
	\begin{tabular}{ c | c | c | c } % Column formatting, 
		\hline
		\textbf{Name}  & \textbf{Description}  & \textbf{Value} & \textbf{Units} \\
		\hline
		Js  & Spin Axis Inertia & 0.159 & kg-m$^2$ \\
		mass & mass & 12.0 & kg \\
		gsHat\_B & Spin Axis in $\cal{B}$ frame & $\begin{bmatrix}
		\frac{\sqrt{3}}{3} & \frac{\sqrt{3}}{3} & \frac{\sqrt{3}}{3} \end{bmatrix}^T$ & - \\
		rWB\_B & Location of Wheel in $\cal{B}$ frame & $\begin{bmatrix}
		-0.5 & 0.5 & -0.5 \end{bmatrix}^T$ & m \\
		\hline
	\end{tabular}
\end{table}

\begin{table}[htbp]
	\caption{Initial Conditions for Energy Momentum Conservation Scenarios}
	\label{tab:initial}
	\centering \fontsize{10}{10}\selectfont
	\begin{tabular}{ c | c | c | c } % Column formatting, 
		\hline
		\textbf{Name}  & \textbf{Description}  & \textbf{Value} & \textbf{Units} \\
		\hline
		(RW 1) OmegaInit  & (RW 1) Initial $\Omega$ & 500 & RPM \\
		(RW 2) OmegaInit  & (RW 2) Initial $\Omega$ & 200 & RPM \\
		(RW 3) OmegaInit  & (RW 3) Initial $\Omega$ & -150 & RPM \\
		r\_CN\_NInit & Initial Position of S/C & $\begin{bmatrix}
		-4020339 &	7490567 & 5248299 
		\end{bmatrix}^T$ & m \\
		v\_CN\_NInit & Initial Velocity of S/C & $\begin{bmatrix}
		-5199.78 & -3436.68 & 1041.58
		\end{bmatrix}^T$ & m/s \\
		sigma\_BNInit & Initial MRP of $\cal{B}$ frame & $\begin{bmatrix}
		0.0 & 0.0 & 0.0
		\end{bmatrix}^T$ & - \\
		omega\_BN\_BInit & Initial Angular Velocity of $\cal{B}$ frame & $\begin{bmatrix}
		0.08 & & 0.01 & 0.0
		\end{bmatrix}^T$ & rad/s \\
		\hline
	\end{tabular}
\end{table}

\begin{table}[htbp]
	\caption{Error Tolerance - Note: Relative Tolerance is $\textnormal{abs}(\frac{\textnormal{truth} - \textnormal{value}}{\textnormal{truth}}$)}
	\label{tab:errortol}
	\centering \fontsize{10}{10}\selectfont
	\begin{tabular}{| c | c |} % Column formatting, 
		\hline
		Test   & Relative Tolerance \\
		\hline
		Energy and Momentum Conservation & 1e-10 \\
		\hline
		Position, Attitude Check & 1e-7 \\
		\hline
		BOE & 1e-8 \\
		\hline
		Friction Tests & 1e-10 \\
		\hline
		Saturation Tests & 1e-10 \\
		\hline
		Minimum Torque & 1e-10 \\
		\hline	
	\end{tabular}
\end{table}

\clearpage

\section{Test Results}

\subsection{Balanced Wheels Scenario - Integrated Test Results}

\input{AutoTex/ChangeInOrbitalAngularMomentumBalancedWheels}
\input{AutoTex/ChangeInOrbitalEnergyBalancedWheels}
\input{AutoTex/ChangeInRotationalAngularMomentumBalancedWheels}
\input{AutoTex/ChangeInRotationalEnergyBalancedWheels}

\clearpage

\subsection{Simple Jitter Scenario - Integrated Test Results}

\input{AutoTex/ChangeInOrbitalAngularMomentumJitterSimple}
\input{AutoTex/ChangeInOrbitalEnergyJitterSimple}
\input{AutoTex/ChangeInRotationalAngularMomentumJitterSimple}
\input{AutoTex/ChangeInRotationalEnergyJitterSimple}

\clearpage

\subsection{Fully Coupled Jitter Scenario - Integrated Test Results}

\input{AutoTex/ChangeInOrbitalAngularMomentumJitterFullyCoupled}
\input{AutoTex/ChangeInOrbitalEnergyJitterFullyCoupled}
\input{AutoTex/ChangeInRotationalAngularMomentumJitterFullyCoupled}
\input{AutoTex/ChangeInRotationalEnergyJitterFullyCoupled}

\clearpage

\subsection{Fully Coupled Jitter with Gravity Scenario - Integrated Test Results}

\input{AutoTex/ChangeInOrbitalAngularMomentumJitterFullyCoupledGravity}
\input{AutoTex/ChangeInOrbitalEnergyJitterFullyCoupledGravity}
\input{AutoTex/ChangeInRotationalAngularMomentumJitterFullyCoupledGravity}
\input{AutoTex/ChangeInRotationalEnergyJitterFullyCoupledGravity}

\clearpage

\subsection{Balanced Wheels, Simple Jitter, Fully Coupled Jitter and Fully Coupled Jitter with Gravity Tests Results}

\begin{table}[htbp]
	\caption{Test results.}
	\label{tab:results}
	\centering \fontsize{10}{10}\selectfont
	\begin{tabular}{c | c } % Column formatting, 
		\hline
		\textbf{Test} 				    & \textbf{Pass/Fail}  \\ \hline
		Balanced Wheels  & \input{AutoTeX/BalancedWheelsPassFail} \\
		Simple Jitter  & \input{AutoTeX/JitterSimplePassFail}   \\ 
		Fully Coupled Jitter & \input{AutoTeX/JitterFullyCoupledPassFail}  \\ 
		Fully Coupled Jitter + Gravity  & \input{AutoTeX/JitterFullyCoupledGravityPassFail}  \\ \hline
	\end{tabular}
\end{table}

\clearpage
