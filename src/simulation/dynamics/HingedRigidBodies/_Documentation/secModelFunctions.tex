\section{Model Functions}

This module is intended to be used an approximation to a flexing body attached to the spacecraft. Examples include solar arrays, antennas, and other appended bodies that would exhibit flexing behavior. Below is a list of functions that this model performs:

\begin{itemize}
	\item Compute it's contributions to the mass properties of the spacecraft
	\item Provides matrix contributions for the back substitution method
	\item Compute it's derivatives for $\theta$ and $\dot{\theta}$
	\item Adds energy and momentum contributions to the spacecraft
	\item create an output message with the panel inertial position and attitude states
\end{itemize}

\section{Model Assumptions and Limitations}
Below is a summary of the assumptions/limitations:

\begin{itemize}
	\item Is a first-order approximation to a flexing body
	\item Is developed in such a way that does not require constraints to be met
	\item The hinged rigid body must have a diagonal inertia tensor with respect the $\mathcal{S}_i$ frame as seen in Figure~\ref{fig:FlexFigure}
	\item Only linear spring and damping terms
	\item Will only approximate one flexing mode at a time
	\item Cannot simulate multiple interconnected panels
	\item The hinged rigid body will always stay attached to the hub (the hinge does not have torque limits)
	\item The hinge does not have travel limits, therefore if the spring is not stiff enough it will unrealistically travel through bounds such as running into the spacecraft hub
	\item The EOMs are nonlinear equations of motion, therefore there can be inaccuracies (and divergence) that result from integration. Having a time step of $<= 0.10\ \text{sec}$ is recommended, but this also depends on the natural frequency of the system
	\item When trying to match the frequency of a physical appended body, note that the natural frequency of the coupled system will be different than the appending body flexing by itself
\end{itemize}