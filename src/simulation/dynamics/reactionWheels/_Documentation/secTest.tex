\section{Test Description and Success Criteria}
The tests are located in \texttt{simulation/dynamics/reactionWheels/\_UnitTest/\newline
test\_reactionWheelStateEffector\_integrated.py} \textbf{,} \texttt{simulation/dynamics/reactionWheels/\newline\_UnitTest/
test\_reactionWheelStateEffector\_ConfigureRWRequests.py} \textbf{and} \texttt{simulation/\newline dynamics/reactionWheels/\_UnitTest/test\_reactionWheelStateEffector\_RWUpdate.py}. Depending on the test, there are different success criteria. These are outlined in the following subsections:
\subsection{Balanced Wheels Scenario - Integrated Test}
In this test the simulation is placed into orbit around Earth with point gravity, has 3 reaction wheels attached to the spacecraft, and the wheels are in ``Balanced" mode. Each wheel is given a commanded torque for half the simulation and the rest of the simulation the torques are set to zero. The following parameters are being tested:
\begin{itemize}
	\item Conservation of orbital angular momentum
	\item Conservation of orbital energy
	\item Conservation of rotational angular momentum
	\item Conservation of rotational energy (second half of the simulation)
	\item Achieving the expected final attitude
	\item Achieving the expected final position
\end{itemize}

\subsection{Simple Jitter Scenario - Integrated Test}
In this test the simulation is placed into orbit around Earth with point gravity, has 3 reaction wheels attached to the spacecraft, and the wheels are in ``Simple Jitter" mode. Each wheel is given a commanded torque for half the simulation and the rest of the simulation the torques are set to zero. The following parameters are being tested:
\begin{itemize}
\item Achieving the expected final attitude
\item Achieving the expected final position
\end{itemize}

\subsection{Fully Coupled Jitter Scenario - Integrated Test}
In this test the simulation is placed into orbit around Earth with point gravity, has 3 reaction wheels attached to the spacecraft, and the wheels are in ``Fully Coupled Jitter" mode. Each wheel is given a commanded torque for half the simulation and the rest of the simulation the torques are set to zero. The following parameters are being tested:
\begin{itemize}
\item Conservation of orbital angular momentum
\item Conservation of orbital energy
\item Conservation of rotational angular momentum
\item Conservation of rotational energy (second half of the simulation)
\item Achieving the expected final attitude
\item Achieving the expected final position
\end{itemize}

\subsection{BOE Calculation Scenario - Integrated Test}

The BOE for this scenario can be seen in Figure~\ref{fig:BOE}. This involves a rigid body hub connected to a reaction wheel with the spin axis being aligned with both the hub's center of mass and the reaction wheel's center of mass. This problem assumes the hub and reaction wheel are fixed to rotate about the the spin axis and so it is a two degree of freedom problem. The analytical expressions for the angular velocity of the hub, $\omega_1$, the angle of the hub, $\theta$ and the reaction wheel speed, $\Omega$ are shown in Figure~\ref{fig:BOE}. The test sets up Basilisk so that the initial conditions constrain the spacecraft to rotate about the spin axis. The results confirm that the analytical expressions agree with the Basilisk simulation. 

\begin{figure}[htbp]
	\centerline{
		\includegraphics[width=0.8\textwidth]{Figures/BOE}}
	\caption{Back of the envelope calculation for RWs}
	\label{fig:BOE}
\end{figure}

\clearpage

\subsection{Friction - Integrated Tests}

In this test the goal is to validate that the friction model is matching the desired static, Coulomb and linear friction model. This is done by setting a spacecraft with two identical reaction wheels with identical spin axes. There are two scenarios being tested: one when the reaction wheels start with a non-zero angular velocity and are spinning down to rest, and the other scenario is when both reaction wheels are starting from rest and applying torque to them to break away from the static friction. These two different friction models for spinning up and spinning down reaction wheels can be seen in Section~\ref{sec:Friction}.

\subsubsection{Spin Down Friction Test}

In this test, the spacecraft is set to have no initial rotation. The wheel speeds are equal in magnitude but opposite in direction. The expected results are that the spacecraft will not rotate and the wheel speeds will spin down to zero while matching the function seen in Eq.~\eqref{eq:friction2}. The test verifies that the math in Eq.~\eqref{eq:friction2} is being computed properly by calculating this in python and ensuring that the results match the Basilisk output.

\subsubsection{Spin Up Friction Test}

In this test, the spacecraft and wheel speeds are all have zero angular velocities. The reaction wheels are given equal but opposite applied torques that are greater in magnitude than the static friction torque. The expected results are that the spacecraft will not rotate and the wheel speeds will spin up while matching the function seen in Eq.~\eqref{eq:friction} and Figure~\ref{fig:Stribeck}. The test verifies that the math in Eq.~\eqref{eq:friction} is being computed properly by calculating this in python and ensuring that the results match the Basilisk output.

\subsection{Saturation - Unit Test}

This test is ensuring that when a commanded torque requests a torque above the max torque of the wheels, the applied torque is set to the max torque. The logic can be seen in the following equation:

\begin{equation}
\begin{gathered}
\begin{aligned}
&\textnormal{\textbf{if}}\quad u_{\textnormal{cmd}} > u_{\textnormal{max}}  \quad \textnormal{\textbf{then}}\\
&\quad u_s = u_{\textnormal{max}} \\
&\textnormal{\textbf{else if}}\quad u_{\textnormal{cmd}} < - u_{\textnormal{max}}  \quad \textnormal{\textbf{then}}\\
&\quad u_s =  - u_{\textnormal{max}} \\
&\textnormal{\textbf{else}}\\
&\quad u_s = u_{\textnormal{cmd}}\\
&\textnormal{\textbf{end if}}\\
\end{aligned}
\end{gathered}
\label{eq:ballin}
\end{equation}

The test gives two commanded torques, one above $u_{\textnormal{max}}$ and one below, and ensures that the correct values are being set for $u_s$.

\subsection{Minimum Torque - Unit Test}

This test is ensuring that when a commanded torque requests a torque below the minimum torque of the wheels, the applied torque is set to zero. The logic can be seen in the following equation:

\begin{equation}
\begin{gathered}
\begin{aligned}
&\textnormal{\textbf{if}}\quad \vert u_{\textnormal{cmd}} \vert < u_{\textnormal{min}}  \quad \textnormal{\textbf{then}}\\
&\quad u_s = 0.0 \\
&\textnormal{\textbf{end if}}\\
\end{aligned}
\end{gathered}
\label{eq:ballin2}
\end{equation}

The test gives two commanded torques, one above $u_{\textnormal{min}}$ and one below, and ensures that the correct values are being set for $u_s$.

\subsection{Speed Saturation - Unit Test}

This test is ensuring that when the current reaction wheel speed is greater than or equal to the maximum allowable wheel speed, and the applied torque is trying to increase the wheel speed further, the applied torque is set to zero. The logic can be seen in the following equation:

\begin{equation}
\begin{gathered}
\begin{aligned}
&\textnormal{\textbf{if}}\quad \vert \Omega \vert >= \Omega_{\text{max}} \textnormal{ and } \Omega u_s >= 0  \quad \textnormal{\textbf{then}}\\
&\quad u_s = 0.0 \\
&\textnormal{\textbf{end if}}\\
\end{aligned}
\end{gathered}
\label{eq:ballin3}
\end{equation}

The test requests a non-zero commanded torque $u_{\textnormal{cmd}}$, gives two possibilities for $\Omega$, one above $\Omega_{\textnormal{max}}$ and one below, and ensures that the correct values are being set for $u_s$.

\subsection{Power Saturation - Unit Test}

This test is ensuring that when the requested reaction wheel power is greater than or equal to the maximum allowable wheel power, the applied torque is limited based on the current wheel speed. The logic can be seen in the following equation:

\begin{equation}
\begin{gathered}
\begin{aligned}
&\textnormal{\textbf{if}}\quad \vert \Omega \cdot u_{\textnormal{cmd}} \vert >= P_{\text{max}} \textnormal \quad \textnormal{\textbf{then}}\\
&\quad u_s = \textnormal{sign}(u_{\textnormal{cmd}})\cdot \vert P_{\text{max}}/\Omega \vert \\
&\textnormal{\textbf{end if}}\\
\end{aligned}
\end{gathered}
\label{eq:ballin3}
\end{equation}

The test requests a non-zero commanded torque $u_{\textnormal{cmd}}$ and gives three possibilities for $\Omega$. The test requests one above positive $P_{\text{max}}$, one above negative $P_{\text{max}}$, one below, and ensures that the correct values are being set for $u_s$.

\subsection{Update - Unit Test}

This test is ensuring the functionality of changing the reaction wheel (RW) characteristics while the simulation is running. It starts by testing the initial setup, and then does four additional tests: the first two change the maximum allowed torque, the third one changes the maximum power limit, and the final one changes the current wheel speeds and maximum allowed wheel speeds.
All these tests rely on the fact that, when a maximum or minimum value is surpassed, the applied torque is capped accordingly. Successful change of parameters is confirmed by observing the corresponding change in supplied torque.

\section{Test Parameters}

Since this is an integrated test, the inputs to the test are the physical parameters of the spacecraft along with the initial conditions of the states. These parameters are outlined in Tables~\ref{tab:hub}-~\ref{tab:initial}. Additionally, the error tolerances can be seen in Table~\ref{tab:errortol}. The error tolerances are different depending on the test. The energy-momentum conservation values will normally have an agreement down to 1e-14, but to ensure cross-platform agreement the tolerance was chose to be 1e-10. The position and attitude checks have a tolerance set to 1e-7 and is because 8 significant digits were chosen as the values being compared to. The BOE tests depend on the integration time step but as the time step gets smaller the accuracy gets better. So 1e-8 tolerance was chosen so that a larger time step could be used but still show agreement. The Friction tests give the same numerical outputs down to ~1e-15 between python and Basilisk, but 1e-10 was chosen to ensure cross platform agreement. Finally, the saturation and minimum torque tests have 1e-10 to ensure cross-platform success, but these values will typically agree to machine precision. 

\begin{table}[htbp]
	\caption{Spacecraft Hub Parameters for Energy Momentum Conservation Scenarios}
	\label{tab:hub}
	\centering \fontsize{10}{10}\selectfont
	\begin{tabular}{ c | c | c | c } % Column formatting, 
		\hline
		\textbf{Name}  & \textbf{Description}  & \textbf{Value} & \textbf{Units} \\
		\hline
		mHub  & mass & 750.0 & kg \\
		IHubPntBc\_B & Inertia in $\cal{B}$ frame & $\begin{bmatrix}
		900.0 & 0.0 & 0.0\\
		0.0 & 800.0 & 0.0\\
		0.0 & 0.0 & 600.0
		\end{bmatrix}$ & kg-m$^2$ \\
		r\_BcB\_B & CoM Location in $\cal{B}$ frame & $\begin{bmatrix}
		-0.0002 & 0.0001 & 0.1 \end{bmatrix}^T$ & m \\
		\hline
	\end{tabular}
\end{table}

\begin{table}[htbp]
	\caption{Reaction Wheel 1 Parameters for Energy Momentum Conservation Scenarios}
	\label{tab:rw1}
	\centering \fontsize{10}{10}\selectfont
	\begin{tabular}{ c | c | c | c } % Column formatting, 
		\hline
		\textbf{Name}  & \textbf{Description}  & \textbf{Value} & \textbf{Units} \\
		\hline
		Js  & Spin Axis Inertia & 0.159 & kg-m$^2$ \\
		mass & mass & 12.0 & kg \\
		U\_s & Static Imbalance & 4.8E-6 & kg-m \\
		U\_d & Dynamic Imbalance & 15.4E-7 & kg-m$^2$ \\
		gsHat\_B & Spin Axis in $\cal{B}$ frame & $\begin{bmatrix}
		1.0 & 0.0 & 0.0 \end{bmatrix}^T$ & - \\
		rWB\_B & Location of Wheel in $\cal{B}$ frame & $\begin{bmatrix}
		0.1 & 0.0 & 0.0 \end{bmatrix}^T$ & m \\
		\hline
	\end{tabular}
\end{table}

\begin{table}[htbp]
	\caption{Reaction Wheel 2 Parameters for Energy Momentum Conservation Scenarios}
	\label{tab:rw2}
	\centering \fontsize{10}{10}\selectfont
	\begin{tabular}{ c | c | c | c } % Column formatting, 
		\hline
		\textbf{Name}  & \textbf{Description}  & \textbf{Value} & \textbf{Units} \\
		\hline
		Js  & Spin Axis Inertia & 0.159 & kg-m$^2$ \\
		mass & mass & 12.0 & kg \\
		U\_s & Static Imbalance & 4.8E-6 & kg-m \\
		U\_d & Dynamic Imbalance & 15.4E-7 & kg-m$^2$ \\
		gsHat\_B & Spin Axis in $\cal{B}$ frame & $\begin{bmatrix}
		0.0 & 1.0 & 0.0 \end{bmatrix}^T$ & - \\
		rWB\_B & Location of Wheel in $\cal{B}$ frame & $\begin{bmatrix}
		0.0 & 0.1 & 0.0 \end{bmatrix}^T$ & m \\
		\hline
	\end{tabular}
\end{table}

\begin{table}[htbp]
	\caption{Reaction Wheel 3 Parameters for Energy Momentum Conservation Scenarios}
	\label{tab:rw3}
	\centering \fontsize{10}{10}\selectfont
	\begin{tabular}{ c | c | c | c } % Column formatting, 
		\hline
		\textbf{Name}  & \textbf{Description}  & \textbf{Value} & \textbf{Units} \\
		\hline
		Js  & Spin Axis Inertia & 0.159 & kg-m$^2$ \\
		mass & mass & 12.0 & kg \\
		U\_s & Static Imbalance & 4.8E-6 & kg-m \\
		U\_d & Dynamic Imbalance & 15.4E-7 & kg-m$^2$ \\
		gsHat\_B & Spin Axis in $\cal{B}$ frame & $\begin{bmatrix}
		0.0 & 0.0 & 1.0 \end{bmatrix}^T$ & - \\
		rWB\_B & Location of Wheel in $\cal{B}$ frame & $\begin{bmatrix}
		0.0 & 0.0 & 0.1 \end{bmatrix}^T$ & m \\
		\hline
	\end{tabular}
\end{table}

\begin{table}[htbp]
	\caption{Reaction wheel 1 parameters for friction tests}
	\label{tab:rwFriction}
	\centering \fontsize{10}{10}\selectfont
	\begin{tabular}{ c | c | c | c } % Column formatting, 
		\hline
		\textbf{Name}  & \textbf{Description}  & \textbf{Value} & \textbf{Units} \\
		\hline
		Js  & Spin Axis Inertia & 0.159 & kg-m$^2$ \\
		mass & mass & 12.0 & kg \\
		gsHat\_B & Spin Axis in $\cal{B}$ frame & $\begin{bmatrix}
		\frac{\sqrt{3}}{3} & \frac{\sqrt{3}}{3} & \frac{\sqrt{3}}{3} \end{bmatrix}^T$ & - \\
		rWB\_B & Location of Wheel in $\cal{B}$ frame & $\begin{bmatrix}
		0.5 & -0.5 & 0.5 \end{bmatrix}^T$ & m \\
		\hline
	\end{tabular}
\end{table}

\begin{table}[htbp]
	\caption{Reaction wheel 2 parameters for friction tests}
	\label{tab:rwFriction2}
	\centering \fontsize{10}{10}\selectfont
	\begin{tabular}{ c | c | c | c } % Column formatting, 
		\hline
		\textbf{Name}  & \textbf{Description}  & \textbf{Value} & \textbf{Units} \\
		\hline
		Js  & Spin Axis Inertia & 0.159 & kg-m$^2$ \\
		mass & mass & 12.0 & kg \\
		gsHat\_B & Spin Axis in $\cal{B}$ frame & $\begin{bmatrix}
		\frac{\sqrt{3}}{3} & \frac{\sqrt{3}}{3} & \frac{\sqrt{3}}{3} \end{bmatrix}^T$ & - \\
		rWB\_B & Location of Wheel in $\cal{B}$ frame & $\begin{bmatrix}
		-0.5 & 0.5 & -0.5 \end{bmatrix}^T$ & m \\
		\hline
	\end{tabular}
\end{table}

\begin{table}[htbp]
	\caption{Initial Conditions for Energy Momentum Conservation Scenarios}
	\label{tab:initial}
	\centering \fontsize{10}{10}\selectfont
	\begin{tabular}{ c | c | c | c } % Column formatting, 
		\hline
		\textbf{Name}  & \textbf{Description}  & \textbf{Value} & \textbf{Units} \\
		\hline
		(RW 1) OmegaInit  & (RW 1) Initial $\Omega$ & 500 & RPM \\
		(RW 2) OmegaInit  & (RW 2) Initial $\Omega$ & 200 & RPM \\
		(RW 3) OmegaInit  & (RW 3) Initial $\Omega$ & -150 & RPM \\
		r\_CN\_NInit & Initial Position of S/C & $\begin{bmatrix}
		-4020339 &	7490567 & 5248299 
		\end{bmatrix}^T$ & m \\
		v\_CN\_NInit & Initial Velocity of S/C & $\begin{bmatrix}
		-5199.78 & -3436.68 & 1041.58
		\end{bmatrix}^T$ & m/s \\
		sigma\_BNInit & Initial MRP of $\cal{B}$ frame & $\begin{bmatrix}
		0.0 & 0.0 & 0.0
		\end{bmatrix}^T$ & - \\
		omega\_BN\_BInit & Initial Angular Velocity of $\cal{B}$ frame & $\begin{bmatrix}
		0.08 & & 0.01 & 0.0
		\end{bmatrix}^T$ & rad/s \\
		\hline
	\end{tabular}
\end{table}

\begin{table}[htbp]
	\caption{Error Tolerance - Note: Relative Tolerance is $\textnormal{abs}(\frac{\textnormal{truth} - \textnormal{value}}{\textnormal{truth}}$)}
	\label{tab:errortol}
	\centering \fontsize{10}{10}\selectfont
	\begin{tabular}{| c | c |} % Column formatting, 
		\hline
		Test   & Relative Tolerance \\
		\hline
		Energy and Momentum Conservation & 1e-10 \\
		\hline
		Position, Attitude Check & 1e-7 \\
		\hline
		BOE & 1e-8 \\
		\hline
		Friction Tests & 1e-10 \\
		\hline
		Saturation Tests & 1e-10 \\
		\hline
		Minimum Torque & 1e-10 \\
		\hline	
	\end{tabular}
\end{table}

\clearpage

\section{Test Results}

\subsection{Balanced Wheels Scenario - Integrated Test Results}

\input{AutoTex/ChangeInOrbitalAngularMomentumBalancedWheels}
\input{AutoTex/ChangeInOrbitalEnergyBalancedWheels}
\input{AutoTex/ChangeInRotationalAngularMomentumBalancedWheels}
\input{AutoTex/ChangeInRotationalEnergyBalancedWheels}

\clearpage

\subsection{Fully Coupled Jitter Scenario - Integrated Test Results}

\input{AutoTex/ChangeInOrbitalAngularMomentumJitterFullyCoupled}
\input{AutoTex/ChangeInOrbitalEnergyJitterFullyCoupled}
\input{AutoTex/ChangeInRotationalAngularMomentumJitterFullyCoupled}
\input{AutoTex/ChangeInRotationalEnergyJitterFullyCoupled}

\clearpage

\subsection{BOE Calculation Scenario - Integrated Test Results}

\input{AutoTex/ReactionWheelBOETheta}
\input{AutoTex/ReactionWheelBOEBodyRate}
\input{AutoTex/ReactionWheelBOERWRate}

\clearpage

\subsection{Friction Spin Down Integrated Test Results}

\input{AutoTex/ReactionWheelFrictionSpinDownTestBodyRates}
\input{AutoTex/ReactionWheelFrictionSpinDownTestFrictionTorque}
\input{AutoTex/ReactionWheelFrictionSpinDownTestWheelSpeed}

\clearpage

\subsection{Friction Spin Up Integrated Test Results}

\input{AutoTex/ReactionWheelFrictionSpinUpTestBodyRates}
\input{AutoTex/ReactionWheelFrictionSpinUpTestFrictionTorque}
\input{AutoTex/ReactionWheelFrictionSpinUpTestWheelSpeed}

\clearpage

\subsection{Simple Jitter, Saturation and Minimum Torque Tests Results}

\begin{table}[htbp]
	\caption{Test results.}
	\label{tab:results}
	\centering \fontsize{10}{10}\selectfont
	\begin{tabular}{c | c } % Column formatting, 
		\hline
		\textbf{Test} 				    & \textbf{Pass/Fail}  \\ \hline
		Simple Jitter  & \input{AutoTeX/JitterSimplePassFail} \\
		Saturation  & \input{AutoTeX/saturationPassFail}   \\ 
		Minimum Torque  & \input{AutoTeX/minimumPassFail}  \\ \hline
	\end{tabular}
\end{table}

\clearpage
