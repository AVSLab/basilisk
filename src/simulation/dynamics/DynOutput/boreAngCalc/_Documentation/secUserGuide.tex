\section{User Guide}

The module can use either a celestial body message or an inertial heading set by the user. It defaults to using the celestial body message if both the message and the heading are configured.

A common setup for both cases contains:

\begin{itemize}
  \item[-]      \texttt{BACObject = boreAngCalc.BoreAngCalc()}: Construct the boreAngCalc module
  \item[-]   \texttt{BACObject.ModelTag = "solarArrayBoresight"}: Set the model tag
  \item[-]   \texttt{BACObject.boreVec\_B = boreVec\_B}: Set the body frame boresight vector assuming it is assigned to variable \texttt{boreVec\_B}
  \item[-]   \texttt{BACObject.scStateInMsg.subscribeTo(scMsg)}: Attach the spacecraft's state message from \texttt{scMsg}
  \item[-]   \texttt{TotalSim.AddModelToTask(unitTaskName, BACObject)}: Attach the module to a task
\end{itemize}

If using the celestial body message, then:

\begin{itemize}
     \item[-]   \texttt{BACObject.celBodyInMsg.subscribeTo(celBodyMsg)}: Attach the celestial body message of type \texttt{SpicePlanetStateMsg()}
\end{itemize}

If using the inertial heading, then:

\begin{itemize}
     \item[-]   \texttt{BACObject.inertialHeadingVec\_N = inertialHeading}: Define the inertial heading
\end{itemize}
