\section{Model Functions}

This module is intended to be used as a model to represent a spacecraft that can be decomposed into a rigid body hub and has the ability to model state effectors such as reactions wheels, and flexing solar panels, etc attached to the hub.  

\begin{itemize}
	\item Updates the mass properties of the spacecraft by adding up all of the contributions to the mass properties from the hub and the state effectors
	\item Adds up all of the matrix contributions from the hub and state effectors for the back substitution method and gives this information to the hub effector
	\item Adds up all of the force and torque contributions from dynamicEffectors and gravityEffector
	\item Calls all of the \textit{computeDerivatives} methods for the hub and all of the state effectors which is essentially doing $\dot{\bm X} = \bm f(\bm X, t)$
	\item Integrates the states forward one time step using the selected integrator
	\item Calculates the total energy and momentum of the spacecraft by adding up contributions from the hub and the state effectors
\end{itemize}

\section{Model Assumptions and Limitations}
Below is a summary of the assumptions/limitations:

\begin{itemize}
	\item Translational only mode cannot have state effectors attached to it that change the mass properties of the spacecraft. However, a state effector like a battery could be used since this does not change the mass properties of the spacecraft.
	\item Rotational only mode can only have state effectors attached to it that do not change the mass properties of the spacecraft, i.e. balanced reaction wheels, and the $\cal{B}$ frame origin must coincident with the center of mass of the spacecraft. 
	\item State effectors that are changing the mass properties of the spacecraft are considered new bodies that are added to the mass properties of the spacecraft. For example, adding flexing solar panels to the sim would require subtracting the mass and inertia of the solar panels from the total spacecraft which would then represent the rigid body hub, and then when the solar panels are added to the sim, their mass inertia are programatically added back to the spacecraft.
	\item The limitations of the sim are primarily based on what configuration you have set the spacecraft in (i.e. what state effectors and dynamic effectors you have attached to the spacecraft). Additionally you are limited to the current capability of Basilisk in regards to state effectors and dynamic effectors. 
	\item The accuracy of the simulation is based upon the integrator and integrator step size
	\item As discussed in the description section, the body fixed frame $\cal{B}$ can be oriented generally and the origin of the $\cal{B}$ frame can be placed anywhere as long as it is fixed with respect to the body. This means that there no limitations from the rigid body hub perspective. 
\end{itemize}