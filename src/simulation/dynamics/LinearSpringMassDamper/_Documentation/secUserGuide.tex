\section{User Guide}

This section is to outline the steps needed to setup a Fuel Slosh State Effector in python using Basilisk.

\begin{enumerate}
	\item Import the linearSpringMassDamper class: \newline \textit{from Basilisk.simulation import linearSpringMassDamper}
	\item Create an instantiation of a linear spring mass damper particle: \newline \textit{particle1 = linearSpringMassDamper.LinearSpringMassDamper()}
	\item Define all physical parameters for a linear spring mass damper particle. For example: \newline
	\textit{particle1.r\_PB\_B = [[0.1], [0], [-0.1]]}
	Do this for all of the parameters for a linear spring mass damper particle seen in the public variables in the .h file.
	\item Define the initial conditions of the states:\newline
	\textit{particle1.rhoInit = 0.05 \quad particle1.rhoDotInit = 0.0}
	\item Define a unique name for each state:\newline
	\textit{particle1.nameOfRhoState = "linearSpringMassDamperRho" \quad particle1.nameOfRhoDotState = "linearSpringMassDamperRhoDot"}
	\item Finally, add the particle to your spacecraftPlus:\newline
	\textit{scObject.addStateEffector(particle1)}. See spacecraftPlus documentation on how to set up a spacecraftPlus object. 
\end{enumerate}
