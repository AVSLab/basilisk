\section{Model Functions}

This module is considered a fuel tank effector attached to a rigid body hub and has the following functions:

\begin{itemize}
	\item Compute tank properties depending on the tank model being used
	\item Provides its contributions to the mass properties of the spacecraft 
	\item Provides its contributions to the back-substitution matrices
	\item Computes its derivative for its mass flow rate using the vector of attached thrusters
	\item Provides its contributions to energy and momentum of the spacecraft
\end{itemize}

\section{Model Assumptions and Limitations}
Below is a summary of the assumptions/limitations:

\begin{itemize}
	\item There can be no relative rotational motion between the fuel in the tank and rigid body hub 
	\item The thrusters can be vectored but rather need to be fixed with respect to the rigid body hub
	\item There is no relative rotational motion between the mass leaving the spacecraft through the thrusters and the rigid body hub
	\item The movement of the mass between tanks and thrusters is not considered as a dynamical effect (i.e. the motion of the mass through piping in the spacecraft)
	\item Fuel slosh is not being considered by this module alone, however fuel slosh can be attached to the spacecraft by using the \textit{fuelSloshParticle} effector
	\item The mass and inertia of the fuel tank is not included in the rigid body mass when setting up a simulation, but rather the fuel tank mass properties is added to the spacecraft dynamically during the simulation
\end{itemize}