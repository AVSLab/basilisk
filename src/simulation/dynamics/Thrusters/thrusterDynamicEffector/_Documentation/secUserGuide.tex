\section{User Guide}

The model can be configured according to the user's wishes, but the following
rules of thumb should probably be respected unless the user is confident:
\begin{enumerate}
\item{The internal simulation dynamics step time should be less than or equal
     to the thruster ramp-up/ramp-down time steps}
\item{The internal simulation dynamics step time should be less than or equal to
     the desired thruster discretization level}
\item{The internal simulation dynamics step time should be less than one-tenth
    of the expected minimum allowable thruster firing duration}
\end{enumerate}

A common set up for thrusters, contains:

\begin{itemize}
  \item[-]      \texttt{thrusterSet = thrusterDynamicEffector.ThrusterDynamicEffector()}: Construct the Thruster Dyn Effector
  \item[-]   \texttt{thrusterSet.ModelTag = "ACSThrusterDynamics"}: Set the model tag
  \item[-]   \texttt{thruster1 = thrusterDynamicEffector.THRSimConfigMsgPayload()}: Create a individual thruster
  \item[-]   \texttt{thruster1.thrLoc\_B = [[1.0] ,[ 0.], [0.]] }: Set the thruster's location
  \item[-]   \texttt{thruster1.thrDir\_B = [[math.cos(anglerad)], [math.sin(anglerad)], [0.0]]}: Set the thruster thrust direction
  \item[-]   \texttt{thruster1.MaxThrust = 1.0}: Set the max thrust
  \item[-]   \texttt{thruster1.MaxSwirlTorque = 0.5}: Set the maximum swirl torque
  \item[-]   \texttt{thruster1.steadyIsp = 226.7}: Set the $I_{sp}$
  \item[-]   \texttt{thruster1.MinOnTime = 0.006}: Set the minimum on time
  \item[-]   \texttt{thrusterSet.addThruster(thruster1)}: Add thruster to the Dyn Effector
\end{itemize}

If attaching the thruster to a body, the last line is instead:

\begin{itemize}
     \item[-]   \texttt{thrusterSet.addThruster(thruster1, bodyStatesMsg)}: Add thruster to the Dyn Effector and attach it to a different body through the states message
\end{itemize}

If setting up a ramp, the user must also perform this:

\begin{itemize}
 \item[-]      \texttt{rampOnList = []}
 \item[-]      \texttt{rampOffList = []}

 \item[-]      \texttt{for i in range(rampsteps):}

\hspace{2cm}\texttt{fnewElement = thrusterDynamicEffector.THRTimePairSimMsg()}

\hspace{2cm}\texttt{fnewElement.TimeDelta = (i + 1.) * 0.1}

\hspace{2cm}\texttt{fnewElement.ThrustFactor = (i + 1.0) / 10.0}

\hspace{2cm}\texttt{fnewElement.IspFactor = (i + 1.0) / 10.0}

\hspace{2cm}\texttt{frampOnList.append(newElement)}

\hspace{2cm}\texttt{fnewElement = thrusterDynamicEffector.THRTimePairSimMsg()}

\hspace{2cm}\texttt{fnewElement.TimeDelta = (i + 1) * 0.1}

\hspace{2cm}\texttt{fnewElement.ThrustFactor = 1.0 - (i + 1.0) / 10.0}

 \hspace{2cm}\texttt{fnewElement.IspFactor = newElement.ThrustFactor}

 \hspace{2cm}\texttt{frampOffList.append(newElement)}

 \item[-]      \texttt{thrusterSet.thrusterData[0].ThrusterOnRamp =}

  \texttt{thrusterDynamicEffector.ThrusterTimeVector(rampOnList)}: Add the on ramp
 \item[-]      \texttt{thrusterSet.thrusterData[0].ThrusterOffRamp =}

 \texttt{thrusterDynamicEffector.ThrusterTimeVector(rampOffList)}: Add the off ramp
\end{itemize}

If setting up blow down effects, the user must also add these optional parameters to the thruster model before adding it to the set:

\begin{itemize}
  \item[-]   \texttt{thruster1.thrBlowDownCoeff = [1.0 , 2.0, 3.0] }: Set any number of polynomial coefficients for mass to thrust equation
  \item[-]   \texttt{thruster1.thrLoc\_B = [1.0 , 2.0, 3.0] }: Set any number of polynomial coefficients for mass to $I_{sp}$ equation
\end{itemize}
