% !TEX root = ./Basilisk-atmosphere-20190221.tex

\section{User Guide}

\subsection{General Module Setup}
This section outlines the steps needed to add an Atmosphere module to a sim.
First, the atmosphere must be imported and initialized:
\begin{verbatim}
from Basilisk.simulation import exponentialAtmosphere
newAtmo = exponentialAtmosphere.ExponentialAtmosphere()
newAtmo.ModelTag = "ExpAtmo"
\end{verbatim}
By default the module uses parameters such that there is no neutral density, and the output is zero'd.  To add specific exponential atmosphere model parameters, use:
\begin{verbatim}
    newAtmo.exponentialParams.baseDensity = baseDensityValue
    newAtmo.exponentialParams.scaleHeight = scaleHeightValue
    newAtmo.planetRadius = planetRadiusValue
\end{verbatim}
To specify a preset exponential model, such as for Earth, the following macro can be used:
\begin{verbatim}
	simSetPlanetEnvironment.exponentialAtmosphere(testModule, "earth")
\end{verbatim}

The model can then be added to a task like other simModels. Each Atmosphere calculates atmospheric parameters based on the output state messages for a set of spacecraft.

To add spacecraft to the model the spacecraft state output message name is sent to the \verb|addScToModel| method:
\begin{verbatim}
scObject = spacecraftPlus.SpacecraftPlus()
scObject.ModelTag = "spacecraftBody"
newAtmo.addSpacecraftToModel(scObject.scStateOutMsgName)
\end{verbatim}

\subsection{Planet Ephemeris Information}
The optional planet state message name can be set by directly adjusting that attribute of the class:
\begin{verbatim}
newAtmo.planetPosInMsgName = "PlanetSPICEMsgName"
\end{verbatim}
If SPICE is not being used, the planet is assumed to reside at the origin.

\subsection{Setting the Model Reach}
By default the model doesn't perform any checks on the altitude to see if the specified atmosphere model should be used.  This is set through the parameters {\tt envMinReach} and {\tt envMaxReach}.  Their default values are -1.  If these are set to positive values, then if the altitude is smaller than {\tt envMinReach} or larger than {\tt envMaxReach}, the density is set to zero.

