% !TEX root = ./Basilisk-atmosphere-20190221.tex

\section{User Guide}

\subsection{General Module Setup}
This section outlines the steps needed to add a facetDrag effector to a spacecraft, add facets to that module, and connect said module to an atmosphere module.

First, import the module and set its tag:
\begin{verbatim}
from Basilisk.simulation import facetDragDynamicEffector
newDrag = facetDragDynamicEffector.FacetDragDynamicEffector()
newDrag.ModelTag = "FacetDrag"
\end{verbatim}

Assuming an atmospheric density model has already been set up, the density message to be used by this drag effector can be set by calling
\begin{verbatim}
newDrag.setDensityMessage(atmoModule.envOutMsgs[0])
\end{verbatim}

By default, the module has no facets and drag calculation is skipped. To add a facet, call th \verb|addFacet| function with an area, drag coefficient, body-frame normal vector, and body-frame facet location. For example, to add multiple facets, call
\begin{verbatim}
scAreas = [1.0, 1.0]
scCoeff = np.array([2.0, 2.0])
B_normals = [np.array([1, 0, 0]), np.array([0, 1, 0])]
B_locations = [np.array([0.1,0,0]), np.array([0,0.1,0])]

for ind in range(0,len(scAreas)):
	newDrag.addFacet(scAreas[ind], scCoeff[ind], B_normals[ind], B_locations[ind])
\end{verbatim}


Finally, add the module to a spacecraft object using the \verb|addDynamicEffector| method:
\begin{verbatim}
scObject = spacecraft.Spacecraft()
scObject.ModelTag = "spacecraftBody"
scObject.addDynamicEffector(newDrag)
\end{verbatim}

