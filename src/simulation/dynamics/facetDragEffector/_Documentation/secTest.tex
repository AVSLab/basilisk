% !TEX root = ./Basilisk-atmosphere-20190221.tex

\section{Test Description and Success Criteria}
This section describes the specific unit tests conducted on this module.

\subsection{General Functionality}

\subsubsection{setDensityMessage}

This test verifies that the user can specify the atmospheric density message used by the module.

\subsubsection{testDragForce}

This test verifies that the module correctly calculates the drag force given the model's assumptions. It also implicitly tests the compatibility of facetDrag and exponentialAtmosphere. 

\subsubsection{testShadow}

This test verifies that panels that are not in the flow are correctly ignored for the purposes of drag calculation.

\subsection{Model-Specific Tests}

\subsubsection{test\_unitFacetDrag.py}

This unit test runs setDensityMessage, testDragForce, and testShadow to verify the functionality of the module. 

\section{Test Parameters}
The simulation tolerances are shown in Table~\ref{tab:errortol}.  In each simulation the neutral density output message is checked relative to python computed true values.  
\begin{table}[htbp]
	\caption{Error tolerance for each test.}
	\label{tab:errortol}
	\centering \fontsize{10}{10}\selectfont
	\begin{tabular}{ c | c } % Column formatting, 
		\hline\hline
		\textbf{Output Value Tested}  & \textbf{Tolerated Error}  \\ 
		\hline
		{\tt newDrag.forceExternal\_N}        & 1e-05 (relative)   \\ 		\hline\hline
	\end{tabular}
\end{table}




\section{Test Results}
The following table shows the test result.


\begin{table}[H]
	\caption{Test result for test\_unitFacetDrag.py}
	\label{tab:results}
	\centering \fontsize{10}{10}\selectfont
	\begin{tabular}{c  | c } % Column formatting, 
		\hline\hline
		\textbf{Check} &  \textbf{Pass/Fail} \\ 
		\hline
		1 &  \input{AutoTeX/unitTestPassFail} \\ 
		\hline
		\hline
	\end{tabular}
\end{table}



