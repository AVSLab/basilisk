\section{Test Description and Success Criteria}
The unit test, test\_gravityDynEffector.py,  validates the internal aspects of the Basilisk spherical harmonics gravity effector module by comparing module output to expected output. It utilizes spherical harmonics for calulcations given the gravitation parameter for the massive body, a reference radius, and the maximum degree of spherical harmonics to be used. The unit test verifies basic set-up, single-body gravitational acceleration, and multi-body gravitational acceleration.

\subsection{Model Set-Up Verification}
This test verifies, via three checks, that the model is appropriately initialized when called.
		\begin{itemize}
			\item \underline{1.1} The first check verifies that the normalized coefficient matrix for the spherical harmonics calculations is initialized appropriately as a $3\times3$ identity matrix. \\
			\item \underline{1.2} The second check verifies that the magnitude of the gravity being calculated is reasonable(i.e. between $9.7$ and $9.9$ $m/s^2$). \\
			\item \underline{1.3} The final check ensures that the maximum degrees value is truly acting as a ceiling on the maximum number of degrees being used in the spherical harmonics algorithms. For example, if the maximum degrees value is set to 20, an attempt is made to call the spherical harmonics algorithms with a degrees value of 100 and another attempt is made with a degrees value of 20. The results are compared and should be equal due to the enforcement of the maximum degrees value.\\
		\end{itemize}
\subsection{Independent Spherical Harmonics Check} This test compares the Basilisk gravity module spherical harmonics acceleration output to an independently formulated python solution. Gravity is measured at an arbitrary point. Note that the independent solution has singularities at the poles that lead to minor divergences in total acceleration.
\subsection{Single-Body Gravity Calculations} This test compares calculated gravity values around the Earth with ephemeris data from the Hubble telescope. The simulation begins shortly after 0200 UTC May 1, 2012 and carries on for two hours, evaluating the gravitational acceleration at two  second intervals.
\subsection{Multi-Body Gravity Calculations} This test checks whether gravity from multiple sources is correctly stacked when applied to a spacecraft. First, a planet is placed in space near a spacecraft. Second, a planet with half the mass of the first is placed the same distance from the spacecraft but in the opposite direction. The gravitational acceleration along that axis is seen to be cut in half for the spacecraft. Finally, a third planet identical to the second is added coincident with the second and the net gravitational acceleration on the spacecraft is seen to be zero. 

\section{Test Parameters}

This section summarizes the  specific error tolerances for each test. Error tolerances are determined based on whether the test results comparison should be exact or approximate due to integration or other reasons. Error tolerances for each test are summarized in table \ref{tab:errortol}. \\

\begin{table}[htbp]
	\caption{Error tolerance for each test. Note that relative tolerance = $\frac{truth - result}{truth}$}
	\label{tab:errortol}
	\centering \fontsize{10}{10}\selectfont
	\begin{tabular}{ c |  l } % Column formatting, 
		\hline
		\textbf{Test}   							    & \textbf{Tolerance} 						  \\ \hline
		Setup Test                           			 & \input{AutoTex/sphericalHarmonicsAccuracy}	(Absolute)	   \\ 
		Independent Spherical Harmonics Check  & \input{AutoTex/independentCheckAccuracy}	(Relative)	   \\ 
		Single-Body Gravity						   & \input{AutoTex/singleBodyAccuracy}						(Relative)								   \\ 
		Multi-Body Gravity 						   & \input{AutoTex/multiBodyAccuracy}	 (Relative for first check, absolute for second)		       \\ \hline
	\end{tabular}
\end{table}

The Setup Test has a large tolerance, which is acceptable because it is not trying to test exact values but only reasonableness. The Independent Spherical Harmonics check has tight tolerances because the results should be nearly identical for two different formulations of the same mathematics. Single Body Gravity also has relatively large tolerances because it is just trying to roughly match experimental data, but not all real effects are included in the model. Finally, the Multi-Body Gravity test has tight tolerances because it is set up in such a way that the results should be exact down to machine precision.

\section{Test Results}

All checks within test\_gravityDynEffector.py passed as expected. Table \ref{tab:results} shows the test results.\\

\begin{table}[htbp]
	\caption{Test results.}
	\label{tab:results}
	\centering \fontsize{10}{10}\selectfont
	\begin{tabular}{c | c | c  } % Column formatting, 
		\hline
		\textbf{Test} 				    & \textbf{Pass/Fail} 						   			           & \textbf{Notes} 									\\ \hline
		Setup Test       			  	& \input{AutoTex/sphericalHarmonicsPassFail}     & \input{AutoTex/sphericalHarmonicsFailMsg}			 \\
		Independent Spherical Harmonics Check  & \input{AutoTex/independentCheckPassFail}                 & \input{AutoTex/independentCheckFailMsg} \\ 
		Single-Body Gravity		   	& \input{AutoTex/singleBodyPassFail}                 & \input{AutoTex/singleBodyFailMsg} \\ 
		Multi-Body Gravity			 &\input{AutoTex/multiBodyPassFail}  			 	 &  \input{AutoTex/multiBodyFailMsg} 			   \\ \hline
	\end{tabular}
\end{table}