% !TEX root = ./Basilisk-Integrators20170724.tex

\section{Model Functions}

The Basilisk integrator functionality is not a regular BSK module, but rather works in conjunction with the {\tt DynamicalObject} class.    The integration functions and goals are:


\begin{itemize}
	\item \textbf{Default Integrator}: The dynamical object should have a default integrator assigned when created.  This avoids the user having to add an integrator in Python, unless they want to select an alternate integrator.
	\item \textbf{Works on any equations of motion}: The integrator needs to function on any dynamical system.
\end{itemize}



\section{Model Assumptions and Limitations}

\subsection{Assumptions}

The equations of motion class {\tt DynamicalObject} is assumed to respond to the method  {\tt equationsOfMotion}.  The integrator then integrates the system forward in time one BSK time step.  

\subsection{Limitations}

Currently only fixed-time step integrators have been implemented.  However, the architecture allows for variable time-step integrators to be implemented as long as their integration time step does not exceed the BSK time step setup for the dynamic equations of motion solving task. 
