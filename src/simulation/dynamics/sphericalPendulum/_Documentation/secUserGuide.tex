\section{User Guide}

This section is to outline the steps needed to setup a spherical pendulum in python using Basilisk.

\begin{enumerate}
	\item Import the sphericalPendulum class: \newline \textit{from Basilisk.simulation import sphericalPendulum}
	\item Create an instantiation of a spherical pendulum particle: \newline \textit{particle1 = sphericalPendulum.SphericalPendulum()}
	\item Define all physical parameters for a spherical pendulum particle. For example: \newline
	\textit{particle1.r\_PB\_B = [[0.1], [0], [-0.1]]}
	Do this for all of the parameters for a spherical pendulum seen in the public variables in the .h file.
	\item Define the initial conditions of the states:\newline
	\textit{particle1.phiInit = 0.05 \quad particle1.phiDotInit = 0.0}
	\item Define a unique name for each state:\newline
	\textit{particle1.nameOfPhiState = "sphericalPendulumPhi" \quad particle1.nameOfPhiDotState = "sphericalPendulumPhiDot"}
	\item Finally, add the particle to your spacecraftPlus:\newline
	\textit{scObject.addStateEffector(particle1)}. See spacecraftPlus documentation on how to set up a spacecraftPlus object. 
\end{enumerate}
