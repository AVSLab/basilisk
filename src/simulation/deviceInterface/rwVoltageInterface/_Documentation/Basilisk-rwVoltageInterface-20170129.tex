\documentclass[]{BasiliskReportMemo}

\newcommand{\submiterInstitute}{Autonomous Vehicle Simulation (AVS) Laboratory,\\ University of Colorado}

\newcommand{\ModuleName}{rwVoltageInterface}
\newcommand{\subject}{Module to convert a RW voltage input into a RW motor torque output}
\newcommand{\status}{First Draft}
\newcommand{\preparer}{H. Schuab}
\newcommand{\summary}{This module provides an analog voltage interface for the cluster of RW devices.  An input voltage is converted to a RW motor torque message.  This message is what drives the RW dynamics.     }


\begin{document}


\makeCover


%
%	enter the revision documentation here
%	to add more lines, copy the table entry and the \hline, and paste after the current entry.
%
\pagestyle{empty}
{\renewcommand{\arraystretch}{1.1}
\noindent
\begin{longtable}{|p{0.5in}|p{4.5in}|p{1.14in}|}
\hline
{\bfseries Rev}: & {\bfseries Change Description} & {\bfseries By} \\
\hline
Draft & Initial cut at this documentation & H. Schaub \\
\hline

\end{longtable}
}

\newpage
\setcounter{page}{1}
\pagestyle{fancy}

\tableofcontents
~\\ \hrule ~\\


\section{Description}
This module is a simulation environment module which simulates the analog voltage interface of a RW cluster.  The input is an array of voltages $V_{i}$.  The Reaction Wheel (RW) motor torque $u_{s_{i}}$ is evaluated using a linear mapping
\begin{equation}
	u_{s_{i}} = V_{i} \gamma
\end{equation}
where $\gamma$ is constant value.  The output of the module is an array of RW motor torques.  The deadband and saturation behavior of the RW speed is modeled inside the RW dynamics model.  



\section{Module Setup}
The interface module is created in python using:
\begin{verbatim}
    testModule = rwVoltageInterface.RWVoltageInterface()
    testModule.ModelTag = "rwVoltageInterface"
\end{verbatim}
The only parameter that must be set is the voltage to torque conversion gain $\gamma$.  This is done using
\begin{verbatim}
    testModule.voltage2TorqueGain = 1.32        # [Nm/V] conversion gain 
\end{verbatim}


\section{Unit Test Discussion}
A series of unit tests are performed to check the validity of this module's operation.  Three base voltages $V_{0}$  are test where $V_{0} \in (5.0, 7.5, 0.0)$.    The input voltages are then setup as 
\begin{equation}
	\bm V = V_{0} \begin{bmatrix} 1 \\ 1 \\ 1 \end{bmatrix}
	+ \begin{bmatrix} 0.0 \\ 1.0 \\ 1.5 \end{bmatrix}
\end{equation} 
The unit test results are down in the following Tables.  

\input{AutoTeX/baseVoltage5.0.tex}
\input{AutoTeX/baseVoltage-7.5.tex}
\input{AutoTeX/baseVoltage0.0.tex}









\end{document}
