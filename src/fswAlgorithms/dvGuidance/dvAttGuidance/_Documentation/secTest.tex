% !TEX root = ./Basilisk-dvGuidance-2019-03-28.tex

\section{Test Description and Success Criteria}
The unit test computes setups up a sample $\Delta v$ maneuver message and runs the module for 3 time steps.  This ensures that the $\theta(t)$ angle updates  are occurring correctly.  





\section{Test Parameters}
The test sets up $\leftexp{N}{\Delta\bm v} = [5,5,5]$ m/s while $\leftexp{N}{\hat{\bm r}} = [1,0,0]$.  The commanded reference frame rotation rate is 0.5 rad/sec, and the burn start time is set to 0.5 seconds.  

The unit test verify that the module output message vectors match expected values.
\begin{table}[htbp]
	\caption{Error tolerance for each test.}
	\label{tab:errortol}
	\centering \fontsize{10}{10}\selectfont
	\begin{tabular}{ c | c } % Column formatting, 
		\hline\hline
		\textbf{Output Value Tested}  & \textbf{Tolerated Error}  \\ 
		\hline
		{\tt sigma\_RN}        & 1e-05	   \\ 
		{\tt omega\_RN\_N}        & 1e-05	   \\ 
		{\tt domega\_RN\_N}        & 1e-05	   \\ 
		\hline\hline
	\end{tabular}
\end{table}




\section{Test Results}
All of the tests passed.
\begin{table}[H]
	\caption{Test results}
	\label{tab:results}
	\centering \fontsize{10}{10}\selectfont
	\begin{tabular}{c | c  } % Column formatting, 
		\hline\hline
		\textbf{Check} &\textbf{Pass/Fail} \\ 
		\hline
	   1	   			& \textcolor{ForestGreen}{PASSED} \\ 
	   \hline\hline
	\end{tabular}
\end{table}
