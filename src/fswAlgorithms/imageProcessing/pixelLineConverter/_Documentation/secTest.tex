% !TEX root = ./Basilisk-pixelLineConverter-20190524.tex

\section{Test Description and Success Criteria}

The unit test for this converter modules creates all three input messages. Using the same values added in those messages, it computes the expected position and covariance of the spacecraft.


\section{Test Parameters}

The unit test verify that the module output message states match expected values.
\begin{table}[htbp]
	\caption{Error tolerance for each test.}
	\label{tab:errortol}
	\centering \fontsize{10}{10}\selectfont
	\begin{tabular}{ c | c } % Column formatting, 
		\hline\hline
		\textbf{Output Value Tested}  & \textbf{Tolerated Error}  \\ 
		\hline
		{\tt r\_N}        & \input{AutoTeX/toleranceValuePos}	   \\ 
		{\tt covar\_N}        & \input{AutoTeX/toleranceValueVel}	   \\ 
		{\tt timeTag}        & \input{AutoTeX/toleranceValuePos}	   \\ 
		\hline\hline
	\end{tabular}
\end{table}

\section{Test Results}
The unit test is expected to pass.
\begin{table}[H]
	\caption{Test results}
	\label{tab:results}
	\centering \fontsize{10}{10}\selectfont
	\begin{tabular}{c | c  } % Column formatting, 
		\hline\hline
		\textbf{Check} &\textbf{Pass/Fail} \\ 
		\hline
	   1	   			& \textcolor{ForestGreen}{PASSED} \\ 
	   \hline\hline
	\end{tabular}
\end{table}


