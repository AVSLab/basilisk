% !TEX root = ./Basilisk-inertialUKF-20190402.tex

\section{User Guide}
\subsection{Filter Set-up, initialization, and I/O}

In order for the filter to run, the user must set a few parameters:

\begin{itemize}
\item The unscented filter has 3 parameters that need to be set, and are best as: \\
      \texttt{ filterObject.alpha = 0.02} \\
      \texttt{ filterObject.beta = 2.0} \\
      \texttt{ filterObject.kappa = 0.0} 
\item Initialize orbit: \\ 
\texttt{     mu = 42828.314} \\
 \texttt{    elementsInit = orbitalMotion.ClassicElements()} \\
 \texttt{    elementsInit.a = 4000} \\
 \texttt{    elementsInit.e = 0.2} \\
 \texttt{    elementsInit.i = 10} \\
  \texttt{   elementsInit.Omega = 0.001} \\
 \texttt{    elementsInit.omega = 0.01} \\
 \texttt{    elementsInit.f = 0.1} \\
 \texttt{    r, v = orbitalMotion.elem2rv(mu, elementsInit)} 
\item The initial covariance: \\
 \texttt{Filter.covar =} \\
  \texttt{[1000, 0.0, 0.0, 0.0, 0.0, 0.0,\\
                              0.0, 1000., 0.0, 0.0, 0.0, 0.0,\\
                              0.0, 0.0, 1000., 0.0, 0.0, 0.0,\\
                              0.0, 0.0, 0.0, 5., 0.0, 0.0,\\
                              0.0, 0.0, 0.0, 0.0, 5., 0.0,\\
                              0.0, 0.0, 0.0, 0.0, 0.0, 5.]}
 \item The initial state :\\
  \texttt{      filterObject.stateInit = r.tolist() + v.tolist()} 
    \item The process noise :\\
  \texttt{     qNoiseIn = np.identity(6)} \\
  \texttt{     qNoiseIn[0:3, 0:3] = qNoiseIn[0:3, 0:3]*0.00001*0.00001} \\
  \texttt{     qNoiseIn[3:6, 3:6] = qNoiseIn[3:6, 3:6]*0.0001*0.0001} \\
  \texttt{     filterObject.qNoise = qNoiseIn.reshape(36).tolist()}
\end{itemize}

The messages must also be set as such:

\begin{itemize}
\item    \texttt{    filterObject.navStateOutMsgName = "relod\_state\_estimate"}
 \item    \texttt{   filterObject.filtDataOutMsgName = "relod\_filter\_data"}
 \item    \texttt{   filterObject.opNavInMsgName = "reold\_opnav\_meas"}
\end{itemize}

