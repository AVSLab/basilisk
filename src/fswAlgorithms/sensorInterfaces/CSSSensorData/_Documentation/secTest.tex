% !TEX root = ./Basilisk-CSSSensorDataArray-20190207.tex

\section{Test Description and Success Criteria}
The unit test checks for proper functionality of the module for various CSS configurations and input values, both within and outside expected bounds.



\section{Test Parameters}

The three test cases run include:
\begin{enumerate}
\item Checking for appropriate correction to raw CSS sensor data.
\item Handle too many sensor inputs, i.e. enforcing the total number of sensors inputs to the maximum sensors defined
\item Ensuring the sim shuts down if zero or a negative number of sensors are configured.
\end{enumerate}

The unit test verify that the module output guidance message vectors match expected values.
\begin{table}[htbp]
	\caption{Error tolerance for each test.}
	\label{tab:errortol}
	\centering \fontsize{10}{10}\selectfont
	\begin{tabular}{ c | c } % Column formatting, 
		\hline\hline
		\textbf{Output Value Tested}  & \textbf{Tolerated Error}  \\ 
		\hline
		{\tt cosValues}        & 1e-05	   \\ 
		\hline\hline
	\end{tabular}
\end{table}




\section{Test Results}

All of the tests passed:
\begin{table}[H]
	\caption{Test results}
	\label{tab:results}
	\centering \fontsize{10}{10}\selectfont
	\begin{tabular}{c | c  } % Column formatting, 
		\hline\hline
		\textbf{Check} 						  		&\textbf{Pass/Fail} \\ 
		\hline
	   1	   			& \textcolor{ForestGreen}{PASSED} \\ 

	   \hline\hline
	\end{tabular}
\end{table}








