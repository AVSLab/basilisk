% !TEX root = ./Basilisk-CSSSensorDataArray-20190207.tex

\section{Test Description and Success Criteria}
The unit test checks for proper functionality of the module for various CSS configurations and input values, both within and outside expected bounds. The three test cases run include:
\begin{enumerate}
\item Checking for appropriate correction to raw CSS sensor data while also handling too many sensor inputs (i.e. enforcing the total number of sensors inputs to the maximum sensors defined).
\item Ensuring the simulation shuts down if zero sensors are configured.
\item Ensuring the simulation shuts down if a negative number of sensors are configured.
\end{enumerate}


\section{Test Parameters}

The unit test verify that the modules output cosine values match expectation.
\begin{table}[htbp]
	\caption{Error tolerance for each test.}
	\label{tab:errortol}
	\centering \fontsize{10}{10}\selectfont
	\begin{tabular}{ c | c } % Column formatting, 
		\hline\hline
		\textbf{Output Value Tested}  & \textbf{Tolerated Error}  \\ 
		\hline
		{\tt cosValues}        & \input{AutoTeX/toleranceValue}	   \\ 
		\hline\hline
	\end{tabular}
\end{table}




\section{Test Results}

Only the first test case produces a result, as the others cause the simulation to shut down. As such, the first test passed and the remaining two tests throw expected failures:
\begin{table}[H]
	\caption{Test results}
	\label{tab:results}
	\centering \fontsize{10}{10}\selectfont
	\begin{tabular}{c | c | c  } % Column formatting, 
		\hline\hline
		\textbf{Num Sensors} 		& \textbf{Num Inputs}				  		&\textbf{Pass/Fail} \\ 
		\hline
	   4	   			& 5 &\input{AutoTeX/passFail_4} \\ 
	   \hline\hline
	\end{tabular}
\end{table}








