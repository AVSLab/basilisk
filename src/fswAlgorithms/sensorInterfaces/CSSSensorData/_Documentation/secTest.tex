% !TEX root = ./Basilisk-CSSSensorDataModule-20190207.tex

\section{Test Description and Success Criteria}
The unit test checks for proper functionality of the module for various CSS configurations and input values, both within and outside expected bounds. The three test cases run include:
\begin{enumerate}
\item Checking for appropriate correction to raw CSS sensor data while also handling too many sensor inputs. The number of sensor values is larger than the number of sensor devices.  The extra data should be returned as a zero value.
\item Ensuring the a zero message is returned if zero sensors are configured.
\item Ensuring that if the number of sensors is set to a value larger than the allowable number of sensors, then this value is set to the maximum number of sensors.
\end{enumerate}


\section{Test Parameters}

The unit test verify that the modules output cosine values match expectation.
\begin{table}[htbp]
	\caption{Error tolerance for each test.}
	\label{tab:errortol}
	\centering \fontsize{10}{10}\selectfont
	\begin{tabular}{ c | c } % Column formatting, 
		\hline\hline
		\textbf{Output Value Tested}  & \textbf{Tolerated Error}  \\ 
		\hline
		{\tt cosValues}        & 1e-05	   \\ 
		\hline\hline
	\end{tabular}
\end{table}




\section{Test Results}
The unit test results are shown in Table~\ref{tab:results}.  All tests should be passing.
\begin{table}[H]
	\caption{Test results}
	\label{tab:results}
	\centering \fontsize{10}{10}\selectfont
	\begin{tabular}{c | c | c  } % Column formatting, 
		\hline\hline
		\textbf{Num Sensors} 		& \textbf{Num Inputs}				  		&\textbf{Pass/Fail} \\ 
		\hline
	   4	   			& 5 &\input{AutoTeX/passFail_4} \\ 
	   0	   			& 5 &\input{AutoTeX/passFail_0} \\ 
	   33	   			& 5 &\input{AutoTeX/passFail_33} \\ 
	   \hline\hline
	\end{tabular}
\end{table}








