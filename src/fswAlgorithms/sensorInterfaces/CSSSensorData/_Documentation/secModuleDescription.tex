% !TEX root = ./Basilisk-CSSSensorDataModule-20190207.tex

\section{Model Description}

\begin{figure}[H]
	\centerline{
		\includegraphics[scale=0.5]{Figures/CSSCalibration}
	}
	\caption{Example of CSS output (blue) relative to a cosine curve (red) \cite{CSS}.}
	\label{fig:CSSraw}
\end{figure}

This module reads in raw sensor data from the \verb. CSSArraySensorIntMsg., message type, iterates through each raw CSS measurement, normalizes the measurement, checks that the input is within sensible bounds, and then corrects the measurement based on pre-calibrated Chebyshev polynomial. The corrected cosine measurement value is then written out in a \verb. CSSArraySensorIntMsg.. 

\subsection{Equations}
The Chebyshev forumlation follows the standard formulation:
\begin{equation}
T_{i+1}(x) = 2xT_i(x) - T_{i-1}(x)
\end{equation}
where
\begin{equation}
T_0(x) = 1 
\end{equation}
\begin{equation}
T_1(x) = x
\end{equation}

As such, the algorithm to compute the Chebyshev polynomial approximation given by the following exerpt \cite{Chebyshev}:
\begin{enumerate}
\item Suppose we want to evaluate Chebyshev polynomial of order $i$ at $x_0$, $(T_i(x_0))$
\item The first two order of Chebyshev polynomials can be evaluated using the following form

\begin{equation}
T_0(x) = 1 
\end{equation}
\begin{equation}
T_1(x) = x
\end{equation}


\item The Chebyshev polynomial of order $i > 1$ can be computed using the values of Chebyshev polynomials of order $i-1$ and $i-2$ and the following recursive formula:
\begin{equation}
T_{i+1}(x) = 2xT_i(x) - T_{i-1}(x)
\end{equation}
\item Apply this formula up to the order $i$ to evaluate Chebyshev polynomial of order $i$ at $x_0$. 
\end{enumerate}

