% !TEX root = ./Basilisk-CSSSensorDataModule-20190207.tex

\section{Model Description}

\begin{figure}[H]
	\centerline{
		\includegraphics[scale=0.5]{Figures/CSSCalibration}
	}
	\caption{Example of \href{https://lang.adcole.com/aerospace/analog-sun-sensors/coarse-sun-sensor-detector/}{CSS output} (blue) relative to a cosine curve (red)\cite{CSS}.}
	\label{fig:CSSraw}
\end{figure}

This module 
\begin{itemize}
\item Reads in raw CSS measurement from the \verb.CSSArraySensorIntMsg. message type.
\item Iterates through each raw CSS measurement, normalizing the measurement and checking that the input is within sensible bounds.
\item Corrects the measurement based on calibrated residual function based on Chebyshev polynomials. 
\item Outputs a \verb.CSSArraySensorIntMsg. with the corrected CSS Measurements.
\end{itemize}


\subsection{Equations}
\subsubsection{Residual Function}
The correction applied to each CSS measurement is based on a function that maps a raw CSS measurement to the expected cosine response for that measurement (i.e. the distance from red curve to the blue curve in Fig.\ref{fig:CSSraw}). This function is modeled using a Chebyshev polynomial series to the $N$-th power represented by the following form: 
\begin{equation}
\delta x = \sum_{i=0}^{N} C_i*T_i(x_{meas})
\end{equation}
 where $T_i(x)$ represents the Chebyshev polynomials, and $C_i$ are the pre-determined scaling factors. 

This correction to the raw measurement is then applied using:
\begin{equation}
x_{corr} = x_{meas} + \delta x
\end{equation}

\subsubsection{Chebyshev Polynomial Computation\cite{Chebyshev}}
The procedure to compute the Chebyshev polynomials, $T_i(x)$, is as follows:
\begin{enumerate}
\item Suppose we want to evaluate Chebyshev polynomial of order $i$ at $x_0$, $(T_i(x_0))$
\item The first two order of Chebyshev polynomials can be evaluated using the following form

\begin{equation}
T_0(x) = 1 
\end{equation}
\begin{equation}
T_1(x) = x
\end{equation}

\item The Chebyshev polynomial of order $i > 1$ can be computed using the values of Chebyshev polynomials of order $i-1$ and $i-2$ and the following recursive formula:
\begin{equation}
T_{i+1}(x) = 2xT_i(x) - T_{i-1}(x)
\end{equation}
\item Apply this formula up to the order $i$ to evaluate Chebyshev polynomial of order $i$ at $x_0$. 
\end{enumerate}





