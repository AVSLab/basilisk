% !TEX root = ./Basilisk-thrFiringRemainder-2019-03-28.tex

\section{Test Description and Success Criteria}
The unit test creates a desired thruster force input vector and then runs the simulation for 3 seconds.  If the {\tt resetCheck} flag is true then a {\tt reset()} method is called and the simulation is repeated for another 2.5 seconds.  If the {\tt dvOn} flag is set then the off-pulsing mode is checked.  




\section{Test Parameters}

The simulation sets up 8 thrusters.  All permutations with the {\tt resetCheck} and {\tt dvOn} states are run.  The output is checked to the tolerance shown in Table~\ref{tab:errortol}.

\begin{table}[htbp]
	\caption{Error tolerance for each test.}
	\label{tab:errortol}
	\centering \fontsize{10}{10}\selectfont
	\begin{tabular}{ c | c } % Column formatting, 
		\hline\hline
		\textbf{Output Value Tested}  & \textbf{Tolerated Error}  \\ 
		\hline
		{\tt OnTimeRequest}        & 1e-05	   \\ 
		\hline\hline
	\end{tabular}
\end{table}




\section{Test Results}
All of the tests passed:
\begin{table}[H]
	\caption{Test results}
	\label{tab:results}
	\centering \fontsize{10}{10}\selectfont
	\begin{tabular}{c | c  | c } % Column formatting, 
		\hline\hline
		{\tt resetCheck} & {\tt dvOn} &\textbf{Pass/Fail} \\ 
		\hline
	   False & False	   			& \input{AutoTeX/passFailFalseFalse} \\ 
	   False & True	   			& \input{AutoTeX/passFailFalseTrue} \\ 
	   True & False	   			& \input{AutoTeX/passFailTrueFalse} \\ 
	   True & True	   			& \input{AutoTeX/passFailTrueTrue} \\ 
	   \hline\hline
	\end{tabular}
\end{table}



