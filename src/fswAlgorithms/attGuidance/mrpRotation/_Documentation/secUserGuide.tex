% !TEX root = ./Basilisk-MRPROTATION-20180522.tex

\section{User Guide}
\subsection{Specifying Desired Rotation}
If the {\tt mrpRotation} module is set directly with the desired rotation states, then the modules variables {\tt mrpSet} and {\tt omega\_RR0\_R} must be set.

If instead the desired rotation states are to be read in, then then input message name {\tt desiredAttInMsgName} must be specified, and a corresponding message of type {\tt AttStateFswMsg} created.  

\subsection{Required Input and Output Messages}
The $\mathcal{R}_{0}$ input reference frame state message is specified through {tt attRefInMsgName}.
The output message name is specified through the {\tt attRefOutMsgName}. 

\subsection{Optional Output Message}
If desired, the $\bm\sigma_{R/R_{0}}$ and $\leftexp{R}{\bm\omega}_{R/R_{0}}$ states can be written to the message {\tt attitudeOutMsgName}.  If this message name is not set, then this output message is not created.  

\subsubsection{Module Reset Behavior}
If the module is reset, then the {\tt priorTime} flag is reset, meaning it take another time step to compute the sampling period used to integrate the kinematic differential equations.  