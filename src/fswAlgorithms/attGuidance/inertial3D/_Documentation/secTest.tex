% !TEX root = ./Basilisk-Inertial3D-2016-01-15.tex

\section{Test Description and Success Criteria}
Describe the unit test(s) in here.

\subsection{Check 1}
There could be subsections for various checks done within the unit test.





\section{Test Parameters}

The unit test verifies that the module output guidance message vectors match expected values.  The simulation sets up the module to output a fixed MRP set.  

\begin{table}[htbp]
	\caption{Error tolerance for each test.}
	\label{tab:errortol}
	\centering \fontsize{10}{10}\selectfont
	\begin{tabular}{ c | c } % Column formatting, 
		\hline\hline
		\textbf{Output Value Tested}  & \textbf{Tolerated Error}  \\ 
		\hline
		{\tt sigma\_RN}        & 1e-05	   \\ 
		{\tt omega\_RN\_N}        & 1e-05	   \\ 
		{\tt domega\_RN\_N}        & 1e-05	   \\ 
		\hline\hline
	\end{tabular}
\end{table}




\section{Test Results}
All of the tests passed:
\begin{table}[H]
	\caption{Test results}
	\label{tab:results}
	\centering \fontsize{10}{10}\selectfont
	\begin{tabular}{c | c  } % Column formatting, 
		\hline\hline
		\textbf{Check} 						  		&\textbf{Pass/Fail} \\ 
		\hline
	   1	   			& \textcolor{ForestGreen}{PASSED} \\ 
	   \hline\hline
	\end{tabular}
\end{table}



