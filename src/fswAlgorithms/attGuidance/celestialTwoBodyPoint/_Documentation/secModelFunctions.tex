% !TEX root = ./Basilisk-celestialTwoBodyPoint-20190311.tex


\section{Module Functions}
\begin{itemize}
	\item \textbf{parseInputMessages}: This method takes the navigation translational info as well as the spice data of the primary celestial body and, if applicable, the second one, and computes the relative state vectors necessary to create the restricted 2-body pointing reference frame.
	\item \textbf{computeCelestialTwoBodyPoint}: This method takes the spacecraft and points a specified axis at a named
 celestial body specified in the configuration data.  It generates the commanded attitude and assumes that the control errors are computed downstream.	
 \end{itemize}

\section{Module Assumptions and Limitations}

The module assumes for now that the planetary acceleration vectors are zero.
Furthermore, if the spacecraft trying to navigate with this algorithm is an a rectilinear (impact) trajectory towards one of the planets used for pointing, the frames will not be defined. This module therefore assumes that the orbits are conics.  