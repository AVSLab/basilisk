% !TEX root = ./Basilisk-sunSafePoint-20180427.tex

\section{Model Description}

\begin{figure}[H]
	\centerline{
		\includegraphics{Figures/sunHeading}
	}
	\caption{Body Vector Illustrations.}
	\label{fig:Fig1}
\end{figure}

\subsection{Module Goal}
This attitude guidance module has the goal of aligning a commanded body-fixed spacecraft vector $\hat{\bm s}_{c}$ with another input vector $\bm s$.  If $\hat{\bm s}_{c}$ is for example the solar panel normal vector, and $\bm s$ is the current sun heading vector, this module will compute the attitude tracking errors to align the solar panels towards the sun, i.e. achieve sun pointing.  Sun pointing is a mode for general recharging the spacecraft, but is also a common guidance scenario with Safe Mode.  

Besides $\bm s$, the second input vector is the inertial body angular velocity vector $\bm\omega_{B/N}$.  The sun pointing frame is assumed to be at rest, thus the attitude rate tracking error is set equal to the body rates.  

As the desired sun pointing orientation is inertial, the inertial reference frame rate $\bm\omega_{R/N}$ and acceleration $\dot{\bm\omega}_{R/N}$ are set to zero. 

Note that this module does not establish a unique sun-pointing reference frame.  Rather, the pointing condition, align $\hat{\bm s}_{c}$ with $\bm s$ is an under-determined 2 degree of freedom condition.  Thus, the rotation angle about $\bm s$ is left to be arbitrary in this sun pointing module.  For the sun pointing applications this is a very practical result as the power generation does not depend on the orientation about $\bm s$.  



\subsection{Equations}
\subsubsection{Good Sun Direction Vector Case}
In the following mathematical developments all vectors are assumed to be taken with respect to a body-fixed frame $\cal B$.  The attitude of the body $\cal B$ relative to the reference frame $\cal R$ is written as a principal rotation from $\cal R$ to $\cal B$.  Thus, the associated principal rotation vector $\hat{\bm e}$ is
\begin{equation}
	\label{eq:ssp:1}
	\hat{\bm e} = \frac{\bm s \times \hat{\bm s}_{c}}{|\bm s \times \hat{\bm s}_{c}|}
\end{equation}
Note that the sun direction vector $\bm s$ does not have to be a normalized input vector.  

The principal rotation angle between the two vectors is given through
\begin{equation}
	\label{eq:ssp:2}
	\Phi = \arccos ( \bm s \cdot \hat{\bm s}_{c})
\end{equation}

Next, this rotation from $\cal R$ to $\cal B$ is written as a set of MRPs through
\begin{equation}
	\label{eq:ssp:3}
	\bm\sigma_{B/R} = \tan\left(\frac{\Phi}{4}\right) \hat{\bm e}
\end{equation}
The set $\bm\sigma_{B/R}$ is the attitude error of the output attitude guidance message.  


The tracking error angular velocity vector is set equal to the measure body rates to bring the spacecraft to rest when pointing in the desired direction.
\begin{equation}
	\label{eq:ssp:4}
	\bm\omega_{B/R} = \bm\omega_{B/N}
\end{equation}

Finally, the attitude guidance message must specify the inertial reference frame rates and acceleration vectors.  These are all set to zero.
\begin{align}
	\bm\omega_{R/N} &= \bm 0 \\
	\dot{\bm \omega}_{R/N} &= \bm 0
\end{align}

\subsubsection{No Sun Direction Vector Case}
 If $\Phi$ is less then then module parameter {\tt minUnitMag}, then it is assumed that no good sun heading vector is available and the attitude tracking error $\bm\sigma_{B/R}$ is set to zero.   
 
 Further, if the sun is not visible, the module allows for a non-zero body rate to be prescribed.  This allows the spacecraft to engage in a constant rate tumble specified through the module configuration vector {\tt omega\_RN\_B}.  In this case the tracking error rate is evaluate through
 \begin{equation}
 	\label{eq:ssp:6}
	\bm\omega_{B/R} = \bm\omega_{B/N} - \bm\omega_{R/N}
 \end{equation}
 and the output message reference rate is set equal to the prescribed $\bm\omega_{R/N}$

 
 
 
 
 
 