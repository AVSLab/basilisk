% !TEX root = ./Basilisk-sunSafePoint-20180427.tex

\section{Test Description and Success Criteria}
The mathematics in this module are straight forward and can be tested in a series of input and output evaluation tests.


\subsection{Check 1}
Here a check is performed where the sun vector measurement $\bm s$ has a non-zero length and is not aligned with $\hat{\bm s}_{c}$.  

\subsection{Check 2}
The sun direction vector $\bm s$ is given a norm value that is less than {\tt minUnitMag}.  In this case the attitude tracking $\bm\sigma_{B/R}$ should be set to zero.  Further, the body rate errors are now evaluated relative to a fixed $\bm\omega_{R/N}$ vector.  



\section{Test Parameters}
The unit test verify that the module output guidance message vectors match expected values.
\begin{table}[htbp]
	\caption{Error tolerance for each test.}
	\label{tab:errortol}
	\centering \fontsize{10}{10}\selectfont
	\begin{tabular}{ c | c } % Column formatting, 
		\hline\hline
		\textbf{Output Value Tested}  & \textbf{Tolerated Error}  \\ 
		\hline
		$\bm\sigma_{B/R}$        & \input{AutoTex/toleranceValue}	   \\ 
		$\bm\omega_{B/R}$        & \input{AutoTex/toleranceValue}\\ 
		$\bm\omega_{R/N}$        & \input{AutoTex/toleranceValue} \\ 
		$\dot{\bm\omega}_{R/N}$        & \input{AutoTex/toleranceValue}  \\ 
		\hline\hline
	\end{tabular}
\end{table}

The nominal module test input values are $\hat{\bm s}_{c} = (0,0,1)$, $\bm s = (1,0,0)$ and $\leftexp{B}{\bm\omega}_{B/N} = (0.01, 0.50, -0.20)$ rad/sec.  The nominal body-fixed search rate is set to $\leftexp{B}{\bm\omega}_{R/N} = (0.0, 0.0, 0.1)$ rad/sec.  This rate is only used if no sun direction vector is available.  



\section{Test Results}

All of the tests passed:
\begin{table}[H]
	\caption{Test results}
	\label{tab:results}
	\centering \fontsize{10}{10}\selectfont
	\begin{tabular}{c | c  } % Column formatting, 
		\hline\hline
		\textbf{Check} 						  		&\textbf{Pass/Fail} \\ 
		\hline
	   1	   			& \input{AutoTex/passFail1} \\ 
	   2	   			& \input{AutoTex/passFail2} \\ 
	   \hline\hline
	\end{tabular}
\end{table}



