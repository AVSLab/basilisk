% !TEX root = ./Basilisk-MODULENAME-yyyymmdd.tex

\section{User Guide}

The user only needs to setup the module and link the proper message names:

\begin{itemize}
\item Build the C-struct: \\
\texttt{    moduleConfig = attTrackingError.attTrackingErrorConfig()}
\item Wrap the module: \\
\texttt{  moduleWrap = alg\_contain.AlgContain(moduleConfig,} \\
\texttt{  attTrackingError.Update\_attTrackingError,} \\
\texttt{ attTrackingError.SelfInit\_attTrackingError,} \\
\texttt{ attTrackingError.CrossInit\_attTrackingError)} \\
\texttt{ moduleWrap.ModelTag = "attTrackingError"} 
\item Add the module to the task: \\
 \texttt{ unitTestSim.AddModelToTask(unitTaskName, moduleWrap, moduleConfig)}
\item Link message names. For instance: \\
  \texttt{  moduleConfig.inputNavName  = "inputNavName"} \\
     \texttt{   moduleConfig.inputRefName  = "inputRefName"} \\
     \texttt{   moduleConfig.outputDataName = "outputName"}
\item  Set the R0R vector: \\
\texttt{    moduleConfig.sigma\_R0R  = [0.01, 0.05, -0.55]}
\end{itemize}