% !TEX root = ./Basilisk-ephemDifference-2019-03-27.tex

\section{Test Description and Success Criteria}
The unit test creates  input ephemeris messages for Mars, Jupiter and Saturn relative to the sun.  The base ephemeris message is created for Earth relative to the sun.  The test evaluates the Mars, Jupiter and Saturn position and velocity vectors relative to the Earth.  

The module requires matching pairs of input and output ephemeris messages.  The body counting logic only adds adds messages if both input and output message is specified.  This is tested by including two more input messages names where only the output message defined, but not the input message.  This should terminate the message counting with a value of 3.

The test also checks that the output message has the time tag of the input message, not the base ephemeris message.

The simulation is run for a single time step to ensure the math is performed correctly.  



\section{Test Parameters}

The unit test verify that the module output guidance message vectors match expected values.
\begin{table}[htbp]
	\caption{Error tolerance for each test.}
	\label{tab:errortol}
	\centering \fontsize{10}{10}\selectfont
	\begin{tabular}{ c | c } % Column formatting, 
		\hline\hline
		\textbf{Output Value Tested}  & \textbf{Tolerated Error}  \\ 
		\hline
		{\tt r\_BdyZero\_N}        & \input{AutoTeX/toleranceValuePos}m 	   \\ 
		{\tt v\_BdyZero\_N}        & \input{AutoTeX/toleranceValueVel}m/s 	   \\ 
		\hline\hline
	\end{tabular}
\end{table}




\section{Test Results}
The test is expected to pass.
\begin{table}[H]
	\caption{Test results}
	\label{tab:results}
	\centering \fontsize{10}{10}\selectfont
	\begin{tabular}{c | c  } % Column formatting, 
		\hline\hline
		\textbf{Check} 	&\textbf{Pass/Fail} \\ 
		\hline
	   Mars	   			& \input{AutoTeX/passFail3} \\ 
	   Jupiter	   			& \input{AutoTeX/passFail3} \\ 
	   Saturn	   			& \input{AutoTeX/passFail3} \\ 
	   No Bodies			& \input{AutoTeX/passFail0} \\
	   \hline\hline
	\end{tabular}
\end{table}




