% !TEX root = ./Basilisk-ephemDifference-2019-03-27.tex

\section{Test Description and Success Criteria}
The unit test creates  input ephemeris messages for Mars, Jupiter and Saturn relative to the sun.  The base ephemeris message is created for Earth relative to the sun.  The test evaluates the Mars, Jupiter and Saturn position and velocity vectors relative to the Earth.  The simulation is run for a single time step to ensure the math is performed correctly.  



\section{Test Parameters}

Test and simulation parameters and inputs go here. Basically, describe your test in the section above, but put any specific numbers or inputs to the tests in this section.

The unit test verify that the module output guidance message vectors match expected values.
\begin{table}[htbp]
	\caption{Error tolerance for each test.}
	\label{tab:errortol}
	\centering \fontsize{10}{10}\selectfont
	\begin{tabular}{ c | c } % Column formatting, 
		\hline\hline
		\textbf{Output Value Tested}  & \textbf{Tolerated Error}  \\ 
		\hline
		{\tt r\_BdyZero\_N}        & 10m 	   \\ 
		{\tt v\_BdyZero\_N}        & 0.0001 m/s 	   \\ 
		\hline\hline
	\end{tabular}
\end{table}




\section{Test Results}
The test is expected to pass.
\begin{table}[H]
	\caption{Test results}
	\label{tab:results}
	\centering \fontsize{10}{10}\selectfont
	\begin{tabular}{c | c  } % Column formatting, 
		\hline\hline
		\textbf{Check} 						  		&\textbf{Pass/Fail} \\ 
		\hline
	   Mars	   			& \input{AutoTeX/passFail11} \\ 
	   Jupiter	   			& \input{AutoTeX/passFail11} \\ 
	   Saturn	   			& \input{AutoTeX/passFail11} \\ 
	   \hline\hline
	\end{tabular}
\end{table}




