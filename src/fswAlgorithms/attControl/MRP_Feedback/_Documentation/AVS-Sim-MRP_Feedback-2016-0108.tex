\documentclass[]{BasiliskReportMemo}

\newcommand{\submiterInstitute}{Autonomous Vehicle Simulation (AVS) Laboratory,\\ University of Colorado}

\newcommand{\ModuleName}{MRP\_Feedback}
\newcommand{\subject}{MRP Feedback ADCS Control Module}
\newcommand{\status}{Initial Documentation Draft}
\newcommand{\preparer}{H. Schaub}
\newcommand{\summary}{This module provides a general MRP feedback control law, applying to using $N$ reaction wheels with general orientation.     }


\begin{document}


\makeCover


%
%	enter the revision documentation here
%	to add more lines, copy the table entry and the \hline, and paste after the current entry.
%
\pagestyle{empty}
{\renewcommand{\arraystretch}{2}
\noindent
\begin{longtable}{|p{0.5in}|p{4.5in}|p{1.14in}|}
\hline
{\bfseries Rev}: & {\bfseries Change Description} & {\bfseries By} \\
\hline
Draft & Initial Draft document & H. Schaub \\
\hline

\end{longtable}
}

\newpage
\setcounter{page}{1}
\pagestyle{fancy}

\tableofcontents
~\\ \hrule ~\\


\section{Initialization}
Simply call the module reset function prior to using this control module.  This will reset the prior function call time variable, and reset the attitude error integral measure.  The control update period $\Delta t$ is evaluated automatically.  

\section{Algorithm}
		 This module employed the MRP feedback algorithm of Example 8.16 of Reference\citenum{schaub}.  This  nonlinear attitude tracking control includes an integral measure of the attitude error.  Further, we seek to avoid quadratic $\bm\omega$ terms to reduce the likelihood of control saturation during a detumbling phase.  Let the new nonlinear feedback control be expressed as
		\begin{multline}
			\label{eq:GusRW}
			[G_{s}]\bm u_{s} = K \bm\sigma + [P] \delta\bm\omega + [P][K_{I}] \bm z - [I_{\text{RW}}](\dot{\bm\omega}_{r} - [\tilde{\bm\omega}]\bm\omega_{r}) + \bm L
			\\
			- ([\tilde{\bm \omega}_{r}] + [\widetilde{K_{I}\bm z}])
			\left([I_{\text{RW}}]\bm\omega + [G_{s}]\bm h_{s} \right)
		\end{multline}
		The integral attitude error measure $\bm z$ is defined through
		\begin{equation*}
			\bm z = K \int_{t_{0}}^{t} \bm\sigma \text{d}t + [I_{\text{RW}}](\delta\bm\omega - \delta\bm\omega_{0})
		\end{equation*}
		
		The integral measure $\bm z$ must be computed to determine $[P][K_{I}] \bm z$, and the expression $[\widetilde{K_{I}\bm z}]$ is added to $[\widetilde{\bm\omega_{r}}]$ term.  

		To analyze the stability of this control, the following Lyapunov candidate function is used:
		\begin{equation*}
			V(\delta\bm\omega, \bm\sigma, \bm z) = \frac{1}{2} \delta\bm\omega^{T} [I_{\text{RW}}] \delta\bm\omega
			+ 2 K \ln ( 1 + \bm\sigma^{T} \bm\sigma) + \frac{1}{2} \bm z ^{T} [K_{I}]\bm z
		\end{equation*}
		provides a convenient positive definite attitude error function.  The attitude feedback gain $K$ is positive, while the integral feedback gain $[K_{I}]$ is a symmetric positive definite matrix.  
		The resulting Lyapunov rate expression, solved in Eq.~(8.101), is given by
		\begin{equation*}
			\dot V =  (\delta\bm\omega + [K_{I}]\bm z)^{T} \left ( [I_{\text{RW}}] \frac{{}^{\mathcal{B \!}}\text{d}}{\text{d}t} (\delta\bm\omega) + K \bm \sigma \right )
		\end{equation*}
		Substituting the equations of motion of a spacecraft with $N$ reaction wheels (see Eq.~(8.160) in Reference~\citenum{schaub}), results in
		\begin{equation*}
			\dot V =  (\delta\bm\omega + [K_{I}]\bm z )^{T} \left (
			 - [\tilde{\bm\omega}] ([I_{\text{RW}}] \bm\omega +[G_{s}]\bm h_{s}) - [G_{s}] \bm u_{s} + \bm L
			 - [I_{\text{RW}}] ( \dot{\bm \omega}_{r} - [\tilde{\bm\omega}]\bm\omega_{r}) + K \bm\sigma
			\right)
		\end{equation*}
		Substituting the control expression in Eq.~\eqref{eq:GusRW} and making use of $\bm \alpha = \bm\omega_{r} - [K_{I}]\bm z$ leads  to 
		\begin{align*}
			\begin{split}
			\dot V &=  (\delta\bm\omega + [K_{I}]\bm z )^{T} \Big (
			- ([\tilde{\bm\omega}] - [\tilde{\bm\omega}_{r}] + [\widetilde{K_{I}\bm z}]) ([I_{\text{RW}}] \bm\omega + [G_{s}]\bm h_{s})
			+( K \bm\sigma - K \bm\sigma) 
			\\
			& \quad - [P]\delta\bm\omega - [P][K_{I}]\bm z + [I_{\text{RW}}](\dot{\bm\omega}_{r} - [\tilde{\bm\omega}]\bm\omega_{r}) - [I_{\text{RW}}](\dot{\bm\omega}_{r} - [\tilde{\bm\omega}]\bm\omega_{r})
			+ ( \bm L - \bm L)
			\Big)
			\end{split}
			\\
			&=  (\delta\bm\omega + [K_{I}]\bm z )^{T} \Big (
			 - ([\widetilde{\delta\bm\omega}] + [\widetilde{K_{I}\bm z}] )  ([I_{\text{RW}}] \bm\omega + [G_{s}]\bm h_{s})
			 - [P] (\delta\bm\omega + [K_{I}]\bm z)
			\Big )
		\end{align*}
		Because $(\delta\bm\omega + [K_{I}]\bm z )^{T}  ([\widetilde{\delta\bm\omega}] + [\widetilde{K_{I}\bm z}] ) = 0$, the Lyapunov rate reduces the negative semi-definite expression
		\begin{equation*}
			\dot V = -  (\delta\bm\omega + [K_{I}]\bm z )^{T} [P]  (\delta\bm\omega + [K_{I}]\bm z )
		\end{equation*}
		This proves the new control is globally stabilizing.  Asymptotic stability is shown following the same steps as for the  nonlinear integral feedback control in Eq. (8.104) in Reference~\citenum{schaub}.  
		
		One of the goals set forth at the beginning of the example was avoiding quadratic $\bm\omega$ feedback terms to reduce the odds of control saturation during periods with large $\bm\omega$ values.  However, the control in Eq.~\eqref{eq:GusRW} contains a product of $\bm z$ and $\bm\omega$.  Let us study this term in more detail.  The $\bm\omega$ expression with this product terms is found to be
		\begin{equation*}
			[\widetilde{K_{I}\bm z}] ([I_{\text{RW}}]\bm \omega)
			 \quad \Rightarrow \quad 
			-  (
			[\widetilde{I_{\text{RW}} \bm \omega}] 
			 ) ([K_{I}] [I_{\text{RW}}] \bm \omega + \cdots )
		\end{equation*}
		If the integral feedback gain is a scalar $K_{I}$, rather than a symmetric positive definite matrix $[K_{I}]$, the quadratic $\bm\omega$ term vanishes.  If the full $3\times 3$ gain matrix is employed, then quadratic rate feedback terms are retained.  



\bibliographystyle{unsrt}   % Number the references.
\bibliography{references}   % Use references.bib to resolve the labels.


\end{document}
