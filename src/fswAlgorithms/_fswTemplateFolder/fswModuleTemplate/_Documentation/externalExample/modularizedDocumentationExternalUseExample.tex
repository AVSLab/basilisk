%This example shows how to load sections that are written elsewhere and look for their inputs in their own relative file path, rather than look from the document you are building.
%Here, the sec_header.tex file has \usepackage{import}
%Then, the sections are called with \subimport{relative path}{file} in order to \input{file} using the right relative path.
%\import{full path}{file} can also be used if absolute paths are preferred over relative paths.
%%%%%%%%%%%%%%%%%%%%%%%%%%%%%%%%%%%%%%%%%%%%%%%%%%%%%%%%%%%%%%%%%%%%%%%%%%%%%
%%%%%%%%%%%%%%%%%%%%%%%%%%%%%%%%%%%%%%%%%%%%%%%%%%%%%%%%%%%%%%%%%%%%%%%%%%%%%
%%%%%%%%%%%%%%%%%%%%%%%%%%%%%%%%%%%%%%%%%%%%%%%%%%%%%%%%%%%%%%%%%%%%%%%%%%%%%
\documentclass[]{BasiliskReportMemo}

\usepackage{cite}
\usepackage{AVS}
\usepackage{float} %use [H] to keep tables where you put them
\usepackage{array} %easy control of text in tables
\usepackage{import} %allows for importing from multiple sub-directories
\bibliographystyle{plain}


\newcommand{\submiterInstitute}{Autonomous Vehicle Simulation (AVS) Laboratory,\\ University of Colorado}


\newcommand{\ModuleName}{SphericalHarmonics}
\newcommand{\subject}{Spherical Harmonics C++ model}
\newcommand{\status}{Tested}
\newcommand{\preparer}{S. Carnahan}
\newcommand{\summary}{The gravity effector module is responsible for calculating the effects of gravity from a body on a spacecraft. A spherical harmonics model and implementation is developed and described. A unit test has been written and run which test basic input/output, single-body gravitational acceleration, and multi-body gravitational acceleration.}

\begin{document}
	
	\makeCover
	
	%
	%	enter the revision documentation here
	%	to add more lines, copy the table entry and the \hline, and paste after the current entry.
	%
	\pagestyle{empty}
	{\renewcommand{\arraystretch}{2}
		\noindent
		\begin{longtable}{|p{0.5in}|p{4.5in}|p{1.14in}|}
			\hline
			{\bfseries Rev}: & {\bfseries Change Description} & {\bfseries By} \\
			\hline
			1.0 & First version - Mathematical formulation and implementation & M. Diaz Ramos \\
			\hline
			1.1 & Added test documentation & S. Carnahan \\
			\hline
			
		\end{longtable}
	}
	
	\newpage
	\setcounter{page}{1}
	\pagestyle{fancy}
	
	\tableofcontents %Autogenerate the table of contents
	~\\ \hrule ~\\ %Makes the line under table of contents
	
\subimport{../example/}{secModelDescription.tex} %This section includes mathematical models, code description, etc.

\subimport{../example/}{secModelFunctions.tex} %This includes a concise list of what the module does.

\subimport{../example/}{secTest.tex} %This explains the unit test for the model. I.e. what features are tested and how. It may also include test tolerances, etc.

\subimport{../example/}{secUserGuide.tex} %This section is to provide advice to users on necessary/useful inputs and best practices.

\bibliography{../example/bibliography.bib} %This includes references used and mentioned.

\end{document}
