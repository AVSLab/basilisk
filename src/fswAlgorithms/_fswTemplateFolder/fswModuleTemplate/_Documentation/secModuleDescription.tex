% !TEX root = ./Basilisk-MODULENAME-yyyymmdd.tex

\section{Model Description}

\begin{figure}[H]
	\centerline{
		\includegraphics{Figures/Fig1}
	}
	\caption{Sample Figure Inclusion.}
	\label{fig:Fig1}
\end{figure}

Provide a brief introduction to the material being discussed in this report.  For example, include what the motivation is, maybe provide a supportive figure such as Figure~\ref{fig:Fig1}, reference earlier work if needed in a literature review.
Describe the module including mathematics, implementation, etc.

\subsection{Equations}
Equations are centered with the equation number flush to the right. In the text, these equations should be referenced by name as Eq.~\eqref{eq:ab} or Equation~\eqref{eq:ab} (e.g., not eq.  1, (1), or Equation 1).
\begin{equation}
	\label{eq:ab}
	a = b^{2}
\end{equation}

\subsection{Citation}
The citation of bibliographical references is indicated in the text by superscripted Arabic numerals, preferably at the end of a sentence.  This is the default style included in this report \LaTeX\ class.\cite{pines1973}

References listed at the end of the paper are indicated in the text by a superscript Arabic number. If this causes confusion in mathematics or if a superscript is not appropriate for other reasons, this can be expressed as (Ref.~1). 

\subsection{Figures}
Illustrations are referred to by name in the text as Figure~1, Figure 2, etc., or, Figures 3 and 4 (e.g., not figure 1, Fig. 1, or \emph{Figure} 1). Captions are in title case (miniscule lettering with the first letter of major words majuscule); they are 10-point serif font and centered below the figure as shown in Figure~\ref{fig:Fig1}. Each illustration should have a caption unless it is a mere sketch. An explanatory caption of several sentences is permissible. Each included illustration must be called out (mentioned) in the text. Ideally, figures should appear within the text just before they are called out. 

The figure files (PDF preferred) should be stored in a common ``Figure'' sub-folder.  If available, any drawing documents used to create this figure can be stored in a ``Support'' sub-folder.

\subsection{Tables}
Tables are referred to by name in the text as Table 1, or, Tables 2 and 3 (e.g., not table 1, Tbl. 1, or Table 1). The title is centered above the table, as shown in Table~\ref{tab:label}. The minimum number of lines needed for clarity is desired. The table font may be adjusted smaller than the body text as necessary.

\begin{table}[htbp]
	\caption{A Caption Goes Here}
	\label{tab:label}
	\centering \fontsize{10}{10}\selectfont
	\begin{tabular}{c | r | r } % Column formatting, 
		\hline 
		Animal    & Description & Price (\$)\\
		\hline 
		Gnat      & per gram & 13.65 \\
		& each     &  0.01 \\
		Gnu       & stuffed  & 92.50 \\
		Emu       & stuffed  & 33.33 \\
		Armadillo & frozen   &  8.99 \\
		\hline
	\end{tabular}
\end{table}

\subsection{Mathematical model}
There can be subsections, like this one.

\subsubsection{Gravity models}
Even subsubsections. 