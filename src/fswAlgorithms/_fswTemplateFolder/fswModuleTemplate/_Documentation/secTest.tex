% !TEX root = ./Basilisk-MODULENAME-yyyymmdd.tex

\section{Test Description and Success Criteria}
Describe the unit test(s) in here.

\subsection{Check 1}
There could be subsections for various checks done within the unit test.





\section{Test Parameters}

Test and simulation parameters and inputs go here. Basically, describe your test in the section above, but put any specific numbers or inputs to the tests in this section.






\section{Test Results}
The results of the unit test should be included in the documentation.  The results can be discussed verbally, but also included as tables and figures.  

\subsection{Unit Test Table Results}
To automatically create a unit test table to include in the documentation, use the command:
\begin{verbatim}
unitTestSupport.writeTableLaTeX(
tableName,
tableHeaders,
caption,
dataMatrix,
path)
\end{verbatim}




\subsection{Unit Test Figure Results}
If figures and plots are generated in the python unit tests, these can be also automatically included in the unit test documentation.  This is achieved with the command:
\begin{verbatim}
unitTestSupport.writeFigureLaTeX(
"testPlot",
"Illustration of Sample Plot",
plt,
"width=0.5\\textwidth",
path)
\end{verbatim}


