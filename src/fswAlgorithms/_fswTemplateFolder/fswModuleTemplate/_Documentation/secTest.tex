% !TEX root = ./Basilisk-MODULENAME-yyyymmdd.tex

\section{Test Description and Success Criteria}
Describe the unit test(s) in here.

\subsection{Check 1}
There could be subsections for various checks done within the unit test.





\section{Test Parameters}

Test and simulation parameters and inputs go here. Basically, describe your test in the section above, but put any specific numbers or inputs to the tests in this section.

The unit test verify that the module output guidance message vectors match expected values.
\begin{table}[htbp]
	\caption{Error tolerance for each test.}
	\label{tab:errortol}
	\centering \fontsize{10}{10}\selectfont
	\begin{tabular}{ c | c } % Column formatting, 
		\hline\hline
		\textbf{Output Value Tested}  & \textbf{Tolerated Error}  \\ 
		\hline
		{\tt outputVector}        & \input{AutoTex/toleranceValue}	   \\ 
		\hline\hline
	\end{tabular}
\end{table}




\section{Test Results}
The results of the unit test should be included in the documentation.  The results can be discussed verbally, but also included as tables and figures.  

All of the tests passed:
\begin{table}[H]
	\caption{Test results}
	\label{tab:results}
	\centering \fontsize{10}{10}\selectfont
	\begin{tabular}{c | c  } % Column formatting, 
		\hline\hline
		\textbf{Check} 						  		&\textbf{Pass/Fail} \\ 
		\hline
	   1	   			& \input{AutoTex/passFail11} \\ 
	   2	   			& \input{AutoTex/passFail13} \\ 
	   3	   			& \input{AutoTex/passFail22} \\ 
	   \hline\hline
	\end{tabular}
\end{table}



\subsection{Unit Test Table Results}
To automatically create a unit test table to include in the documentation, use the command:
\begin{verbatim}
unitTestSupport.writeTableLaTeX(
tableName,
tableHeaders,
caption,
dataMatrix,
path)
\end{verbatim}

Here are the sample \TeX\ table form the unit tests.

\begin{table}[htbp]
\caption{Sample output table for param1 = 1 and param2 = 1.}
\label{tbl:test11}
\centering
\begin{tabular}{ccccccc}
\hline
  time [s]  &  Output 1  &  Error  &  Output 2  &  Error  &  Output 3 $\bm r$  &  Error  \\
\hline
     0      &     2      &    0    &     1      &    0    &        0.7         &    0    \\
    0.5     &     3      &    0    &     1      &    0    &        0.7         &    0    \\
     1      &     4      &    0    &     1      &    0    &        0.7         &    0    \\
    1.5     &     2      &    0    &     1      &    0    &        0.7         &    0    \\
     2      &     3      &    0    &     1      &    0    &        0.7         &    0    \\
\hline
\end{tabular}\end{table}
\begin{table}[htbp]
\caption{Sample output table for param1 = 1 and param2 = 3.}
\label{tbl:test13}
\centering
\begin{tabular}{ccccccc}
\hline
  time [s]  &  Output 1  &  Error  &  Output 2  &  Error  &  Output 3 $\bm r$  &  Error  \\
\hline
     0      &     2      &    0    &     3      &    0    &        0.7         &    0    \\
    0.5     &     3      &    0    &     3      &    0    &        0.7         &    0    \\
     1      &     4      &    0    &     3      &    0    &        0.7         &    0    \\
    1.5     &     2      &    0    &     3      &    0    &        0.7         &    0    \\
     2      &     3      &    0    &     3      &    0    &        0.7         &    0    \\
\hline
\end{tabular}\end{table}
\begin{table}[htbp]
\caption{Sample output table for param1 = 2 and param2 = 2.}
\label{tbl:test22}
\centering
\begin{tabular}{ccccccc}
\hline
  time [s]  &  Output 1  &  Error  &  Output 2  &  Error  &  Output 3 $\bm r$  &  Error  \\
\hline
     0      &     3      &    0    &     2      &    0    &        0.7         &    0    \\
    0.5     &     4      &    0    &     2      &    0    &        0.7         &    0    \\
     1      &     5      &    0    &     2      &    0    &        0.7         &    0    \\
    1.5     &     3      &    0    &     2      &    0    &        0.7         &    0    \\
     2      &     4      &    0    &     2      &    0    &        0.7         &    0    \\
\hline
\end{tabular}\end{table}

\subsection{Unit Test Figure Results}
If figures and plots are generated in the python unit tests, these can be also automatically included in the unit test documentation.  This is achieved with the command:
\begin{verbatim}
unitTestSupport.writeFigureLaTeX(
"testPlot",
"Illustration of Sample Plot",
plt,
"width=0.5\\textwidth",
path)
\end{verbatim}


\input{AutoTex/testPlot11.tex}
\input{AutoTex/testPlot13.tex}
\input{AutoTex/testPlot22.tex}

