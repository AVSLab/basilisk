% README file for moduleDocumentationTemplate TeX template.
% This template should be used to document all Basilisk modules.
% Updated 20170711 - S. Carnahan
%
%-Copy the contents of this folder to your own _Documentation folder
%
%-Rename the Basilisk-moduleDocumentationTemplate.tex appropriately
%
% All edits should be made in one of:
% sec_modelAssumptionsLimitations.tex
% sec_modelDescription.tex
% sec_modelFunctions.tex
% sec_revisionTable.tex
% sec_testDescription.tex
% sec_testParameters.tex
% sec_testResults.tex
% sec_user_guide.tex
%
%-Some rules about referencing within the document:
%1. If writing the suer guide, assume the module description is present
%2. If writing the validation section, assume the module features section is present
%3. Make no other assumptions about any sections being present. This allow for sections of the document to be used elsewhere without breaking.

%In order to import some of these sections into a document in a different directory:
%\usepackage{import}
%Then, the sections are called with \subimport{relative path}{file} in order to \input{file} using the right relative path.
%\import{full path}{file} can also be used if absolute paths are preferred over relative paths.

%%%%%%%%%%%%%%%%%%%%%%%%%%%%%%%%%%%%%%%%%%%%%%%%%




\documentclass[]{BasiliskReportMemo}

\usepackage{cite}
\usepackage{AVS}
\usepackage{float} %use [H] to keep tables where you put them
\usepackage{array} %easy control of text in tables
\usepackage{graphicx}
\bibliographystyle{plain}


\newcommand{\submiterInstitute}{Autonomous Vehicle Simulation (AVS) Laboratory,\\ University of Colorado}


\newcommand{\ModuleName}{CSSWlsEst}
\newcommand{\subject}{Weight Least-Squares Minimum-Norm Coarse Sun Heading Estimator}
\newcommand{\status}{Released}
\newcommand{\preparer}{S. Piggott}
\newcommand{\summary}{This is a report documenting the results of the unit test performed for the 
   coarse sun sensor sun point vector estimation algorithm.}

\begin{document}

\makeCover

%
%	enter the revision documentation here
%	to add more lines, copy the table entry and the \hline, and paste after the current entry.
%
\pagestyle{empty}
{\renewcommand{\arraystretch}{2}
\noindent
\begin{longtable}{|p{0.5in}|p{3.5in}|p{1.07in}|p{0.9in}|}
\hline
{\bfseries Rev} & {\bfseries Change Description} & {\bfseries By}& {\bfseries Date} \\
\hline
1.0 & Initial Documentation & S. Piggott & 2016-10-01\\
\hline
1.1 & Modified Format to BSK documentation & H. Schaub & 2018-04-28\\
\hline

\end{longtable}
}



\newpage
\setcounter{page}{1}
\pagestyle{fancy}

\tableofcontents %Autogenerate the table of contents
~\\ \hrule ~\\ %Makes the line under table of contents




\section{Introduction}
When in safe mode, the spacecraft uses its coarse sun sensors in order to 
point the vehicle's solar arrays at the Sun.  This is done in order to ensure 
that the vehicle gets to a power-positive state with a minimum set of sensors 
in order to recover from whatever event triggered the transition to safe mode.  
The vehicle notionally has 8 coarse sun sensor (CSS) sensors available to it 
which allows it to resolve the exact sun direction in almost all body axes as 
long as all sensors are functional. 

Since there are so many CSSs available, the algorithm needs to be able to obtain 
the sun pointing vector that best fits the current outputs from all of the CSSs.  
This is done by a least squares estimation process that provides the sun vector 
that best fits from a weighted least squares perspective.  The weights are 
simply set based on the current output of each sensor which ensures that the 
sensors that have the best measurements are trusted the most.  The details of 
this algorithm are available in Steve O'Keefe's PhD dissertation. 
\footnote{\href{http://gradworks.umi.com/3704787.pdf}
   {O'Keefe Public Dissertation Link}}

The algorithm stores its internal variables in the CSSWLSConfig data structure 
with the input message coming from the "css\_data\_aggregate" message and the 
output sun pointing vector going to the "css\_wls\_est" message.  This 
algorithm does not use any information stored from previous frames so it is a 
fresh computation every time it is called.  It can therefore be called at any 
rate needed by the system.

\section{Test Design}
The unit test for the cssWlsEst module is located in:\\

\noindent
{\tt fswAlgorithms/attDetermination/CSSEst/\_UnitTest/CSSWlsEstUnitTest.py} \\
\\

Please see the python script for information on test setup and initial 
conditions.  \\

\noindent This unit test is designed to functionally test the algorithm 
outputs as well as get complete code path coverage.  The test design is broken 
up into four main parts:\\
\begin{enumerate}
\item{Main Body Axis Estimates: The principal body axes (b1, b2, b3) are tested 
   with both positive and negative values to ensure that all axes are correctly 
   estimated.}
\item{Double Coverage Test: There are small regions of pointing where only two 
   sun sensors provide "good" values, which results in a case where only the 
   minimum norm solution can be used instead of a full least squares solution.  
   One of these regions is tested here.}
\item{Single Coverage Test: One of the sensors used for the double coverage test 
   is zeroed in order to simulate a sensor failure and hit the single coverage 
   code.  The accuracy of this estimate is severely compromised.}
\item{Zero Coverage Case: The case where no sensors provide an above-threshold 
   value is tested here.}
\end{enumerate}


\section{Test Results}

The values obtained in the test over time are best visualized in 
Figure~\ref{fig:point_fig}.  That shows a comparison between the Sun pointing 
vector input to the test and the estimate provided by the algorithm.
\begin{figure}[htb]
        \centerline{
        \includegraphics[scale=0.5]{Figures/sunEstAccuracy}
        }
        \caption{Truth and Estimated Sun Pointing Vector}
        \label{fig:point_fig}
\end{figure}

As this plot shows, the algorithm is very accurate up until we hit 6.0 seconds, 
so both directions of the three primary body axes are estimated precisely.  
Then the double coverage case is reasonably accurate, but no longer precise as 
there isn't sufficient information available to get a good pointing direction.  
The single coverage test is not accurate at all (~45 degrees of error), but that 
is simply the best that the algorithm can do with that limited information.

Figure ~\ref{fig:num_fig} shows the number of CSSs used by the algorithm to 
estimate the sun pointing vector over the duration of the test.  It continues 
for longer than Figure~\ref{fig:point_fig} because the algorithm stops setting 
its output message once it gets to the zero valid sensors case as there is no 
good information to provide.
\begin{figure}[htb]
        \centerline{
        \includegraphics[scale=0.5]{Figures/numGoodCSS}
        }
        \caption{Number of CSSs Used in Estimate}
        \label{fig:num_fig}
\end{figure}


\begin{enumerate}
\item{Main Body Axis Estimates: The sun pointing estimation algorithm is not 
   required to provide a precise estimate of the Sun direction.  This algorithm 
   is only intended to be used in safe mode where the arrays only need to be 
   approximately pointed at the Sun.  For this reason, a pointing vector 
   was flagged as successful when it provided the Sun direction within 17.5 
   degrees which corresponds to a cosine loss of approximately 5\%.  All body 
   axes met this criteria with large margins.  A check was also performed that 
   verified that the predicted number of CSSs matched up with what the 
   algorithm used and this check was also 100\% successful.  The UseWeights flag 
   was initially set to False, and then was changed to True after two cases to 
   ensure that the algorithm works correctly in both cases. 
    \textcolor{ForestGreen}{Test successful.}}
\item{Double Coverage Test: The same accuracy criteria was used for this test.  
   This is mostly a function of CSS geometry and it is also the main driving 
   case for the success criteria used.  It was correct to within 14 degrees. 
   The predicted number of CSSs used (2) also matched the algorithm's selection. 
    \textcolor{ForestGreen}{Test successful.}}
\item{Single Coverage Test: The single coverage case did not have its accuracy 
   tested as there are no accuracy requirements for this case.  It simply must 
   provide an estimate and exit.  The predicted number of CSSs used (1) did  
   match the algorithm's selection.  \textcolor{ForestGreen}{Test successful.}}
\item{Zero Coverage Test: The zero coverage test is only provided here to 
   demonstrate that the algorithm passivates its outputs without hitting any 
   unacceptable event.  It does correctly flag that no valid CSSs were found 
   during the test. \textcolor{ForestGreen}{Test successful.}}
\end{enumerate}

\section{Test Coverage}
The method coverage for all of the methods included in the cssWlsEst 
module are tabulated in Table~\ref{tab:cov_met}

\begin{table}[htbp]
    \caption{ADCS Coarse Sun Sensor Estimation Coverage Results}
   \label{tab:cov_met}
        \centering \fontsize{10}{10}\selectfont
   \begin{tabular}{c | r | r | r} % Column formatting, 
      \hline
      Method Name    & Unit Test Coverage (\%) & Runtime Self (\%) & Runtime Children (\%) \\
      \hline
      SelfInit\_cssWlsEst& 100.0 & 0.0 & 0.0 \\
      CrossInit\_cssWlsEst & 100.0 & 0.0 & 0.0 \\
      computeWlsmn & 100.0 & 0.01 & 0.64 \\
      Update\_cssWlsEst & 100.0 & 0.04 & 0.88 \\
      \hline
   \end{tabular}
\end{table}

For all of the code this test was designed for, the coverage percentage is 
100\%.  For Safe Mode, we do expect this algorithm to be the highest usage 
element from an ADCS perspective, so the CPU usage is almost certainly fine as 
is.  The main penalty comes from the use of matrix multiply in the computeWlsmn 
function.  The only issue of note here is that the matrix multiply algorithm(s) 
use in the FSW should be optimized as much as possible as they are major sources 
of CPU spin.

\section{Conclusions}
The safe mode sun vector estimator described in this document is functionally 
ready from a PDR perspective.  It has no noted failure cases, all code is tested 
for statement coverage, and it successfully meets its test criteria for all 
cases.  The only area where there might be a question is the desired behavior 
for zero-coverage cases.  We may wish to change the outputs to something more 
obviously in-error instead of just having the algorithm go silent.










%	
%% !TEX root = ./Basilisk-msisAtmosphere-20190221.tex

\section{Model Description}

The purpose of this module is to wrap the Brodowski implementation of the NRLMSISE-00 atmospheric neutral density model\footnote{\url{https://ccmc.gsfc.nasa.gov/models/modelinfo.php?model=MSISE}} for use with other Basilisk modules. NRLMSISE-00 is a high-fidelity, empirically-derived model of the Earth's atmosphere that considers the effects of common space phenomena (magnetic/solar influences) on neutral atmospheric density. A portfolio of individual models is valid up to 1000km, and provides both the neutral temperature and species in addition to the total mass density.
 %This section includes mathematical models, code description, etc.
%
%% !TEX root = ./Basilisk-ephemNavConverter-20190326.tex


\section{Module Functions}
\begin{itemize}
	\item \textbf{Read in an ephemeris message}: A single ephemeris message is read in to convert it to another output type.
	\item \textbf{Translate navigation message}: The output message must be of type {\tt NavTransIntMsg}
\end{itemize}

\section{Module Assumptions and Limitations}

The main assumptions used in this module are :

\begin{itemize}
\item Light-time is not modeled
\item The incoming message is a circle
\item The uncertainty is mapped using a first variation method, higher order terms are not taken into account
\end{itemize} %This includes a concise list of what the module does. It also includes model assumptions and limitations
%
%% !TEX root = ./Basilisk-planetEphemeris-20190422.tex

\section{Test Description and Success Criteria}
The unit test configures the module to model general orbital motions of Earth and Venus.  In each case the translational motion is compute for 3 times steps from 0 to 1 second using a 0.5 second time step.  The planet orientation information is only set if the appropriate simulation parameter is set.  The following sub-sections discuss these flags.  If none of these flags are set, then the module default orientation behavior is expected.  If all the flags are set, then a constant rotation is set.  If only a partial set of orientation information is provided then the reset routine will force the module to throw an error message and only output a default constant zero orientation of the planet. 

\subsection{{\tt setRAN}}
This flag specifies if a set of right ascension angles are specified in the unit test.

\subsection{{\tt setDEC}}
This flag specifies if a set of declination angles are specified in the unit test.

\subsection{{\tt setLST}}
This flag specifies if a set of local sidereal time angles are specified in the unit test.

\subsection{{\tt setRate}}
This flag specifies if a set of planetary polar rotation rates are specified in the unit test.




\section{Test Parameters}
The unit test verify that the module output guidance message vectors match expected values.
\begin{table}[htbp]
	\caption{Error tolerance for each test.}
	\label{tab:errortol}
	\centering \fontsize{10}{10}\selectfont
	\begin{tabular}{ c | c } % Column formatting, 
		\hline\hline
		\textbf{Output Value Tested}  & \textbf{Tolerated Error}  \\ 
		\hline
		{\tt J2000Current}        & \input{AutoTeX/toleranceValue} s   \\ 
		{\tt PositionVector}        & \input{AutoTeX/toleranceValue} m   \\ 
		{\tt VelocityVector}        & \input{AutoTeX/toleranceValue} m/s   \\ 
		{\tt J20002Pfix}        & \input{AutoTeX/toleranceValue}    \\ 
		{\tt J20002Pfix\_dot}        & $10^{-10}$ rad/s   \\ 
		{\tt computeOrient}        & \input{AutoTeX/toleranceValue}    \\ 
		\hline\hline
	\end{tabular}
\end{table}




\section{Test Results}
All orientation flag permutations are tested and are expected to pass.

\begin{table}[H]
	\caption{Test results}
	\label{tab:results}
	\centering \fontsize{10}{10}\selectfont
	\begin{tabular}{c | c  | c | c | c } % Column formatting, 
		\hline\hline
		{\tt setRAN} & {\tt setDEC} & {\tt setLST} & {\tt setRate} &\textbf{Pass/Fail} \\ 
		\hline
	   True & True & True & True & \input{AutoTeX/passFailTrueTrueTrueTrue} \\ 
	   True & True & True & False & \input{AutoTeX/passFailTrueTrueTrueFalse} \\ 
	   True & True & False & True & \input{AutoTeX/passFailTrueTrueFalseTrue} \\ 
	   True & True & False & False & \input{AutoTeX/passFailTrueTrueFalseFalse} \\ 
	   True & False & True & True & \input{AutoTeX/passFailTrueFalseTrueTrue} \\ 
	   True & False & True & False & \input{AutoTeX/passFailTrueFalseTrueFalse} \\ 
	   True & False & False & True & \input{AutoTeX/passFailTrueFalseFalseTrue} \\ 
	   True & False & False & False & \input{AutoTeX/passFailTrueFalseFalseFalse} \\ 
	   False & True & True & True & \input{AutoTeX/passFailFalseTrueTrueTrue} \\ 
	   False & True & True & False & \input{AutoTeX/passFailFalseTrueTrueFalse} \\ 
	   False & True & False & True & \input{AutoTeX/passFailFalseTrueFalseTrue} \\ 
	   False & True & False & False & \input{AutoTeX/passFailFalseTrueFalseFalse} \\ 
	   False & False & True & True & \input{AutoTeX/passFailFalseFalseTrueTrue} \\ 
	   False & False & True & False & \input{AutoTeX/passFailFalseFalseTrueFalse} \\ 
	   False & False & False & True & \input{AutoTeX/passFailFalseFalseFalseTrue} \\ 
	   False & False & False & False & \input{AutoTeX/passFailFalseFalseFalseFalse} \\ 
	   \hline\hline
	\end{tabular}
\end{table}

 % This includes test description, test parameters, and test results
%
%\section{User Guide}
When using this model, the user should follow the setup procedure corresponding to his or her desired conversion described below. The procedures outline the required inputs and for each conversion and some recommended parameter values.
\subsection{Element to Cartesian}
	\begin{itemize}
		\item \textit{Elements2Cart} = \textit{True}
		\item \textit{useEphemFormat}
		\begin{itemize}
			\item \textit{True} for planet state data
			\item \textit{False} for spacecraft state data
		\end{itemize}
		\item \textit{inputsGood} = True
		\item $\mu$ is recommended to be $3.86\text{e+}14$ $\frac{\text{m}^3}{\text{s}^2}$.
		\item Keplerian orbital elements should abide by the cases layed out in the Model Functions section.
	\end{itemize}
\subsection{Cartesian to Element}
	\begin{itemize}
		\item \textit{Elements2Cart} = \textit{False}
		\item \textit{useEphemFormat}
		\begin{itemize}
			\item \textit{True} for planet state data
			\item \textit{False} for spacecraft state data
		\end{itemize}
		\item \textit{inputsGood} = \textit{True}
		\item $\mu$ is recommended to be $3.86\text{e+}14$ $\frac{\text{m}^3}{\text{s}^2}$.
		\item Recommended Cartesian vectors can be obtained from Figures \ref{fig:2} through \ref{fig:13}, which correspond to the allowed orbit types.
	\end{itemize}

\subsection{Variable Definition and Code Description}
The variables in Table \ref{tabular:vars} are available for user input. Variables used by the module but not available to the user are not mentioned here. Variables with default settings do not necessarily need to be changed by the user, but may be.
\begin{table}[H]
	\caption{Definition and Explanation of Variables Used.}
	\label{tab:errortol}
	\centering \fontsize{10}{10}\selectfont
	\begin{tabular}{  m{3cm}| m{3cm} | m{3cm} | m{6cm} } % Column formatting, 
		\hline
		\textbf{Variable}   							& \textbf{LaTeX Equivalent} 	&		\textbf{Variable Type} & \textbf{Notes}			  \\ \hline
		r$_N$	&$\bm{r}$ & double & [km]Default setting: 0.0. Position vector either used as an input to or obtained as an output from the conversions.\\ \hline
		v$_N$	& $\bm{\dot{r}}$ & double & [km/s]Default setting: 0.0. Velocity vector either used as an input to or obtained as an output from the conversions.\\ 
		\hline
		$\mu$	& $\mu$ & double & [m3/s2] Required Input. This is the gravitational parameter of the body. Replaces the product of Earth's gravitational force and mass for this test..\\
		\hline
		a & $a$ & double & [km] Required Input. The semimajor axis of the body's orbit.\\ 
		\hline
		e & $e$ & double & Required Input. The eccentricity of the body's orbit.\\ 
		\hline
		i & $i$ & double & [rad] Required Input. The inclination of the body's orbit\\ 
		\hline
		Omega & $\Omega$ & double & [rad] Required Input. The ascending node of the body's orbit. \\ 
		\hline
		omega & $\omega$ & double & [rad] Required Input. The argument of periapses of the body's orbit. \\ 
		\hline
		f & $f$ & double & [rad] Required Input. The true anomaly of the body's orbit \\
		\hline
		Elements2Cart & N/A & bool & Default Setting: False. Identifies the desired conversion. \\ 
		\hline
		useEphemFormat & N/A & bool & Default Setting: False. Identifies whether the body is a spacecraft or a planet.\\
		\hline
		inputsGood & N/A & bool & Default Setting: False. Indicates whether the code reads valid state data for conversion.\\
		\hline
	\end{tabular}
	\label{tabular:vars}
\end{table}
\begin{thebibliography}{1}
	\bibitem{bib:1}
	Vallado, D. A., and McClain, W. D., \textit{Fundamentals of Astrodynamics and Applications, 4th ed}. Hawthorne, CA: Published by Microcosm Press, 2013.
	\bibitem{bib:2}
	Schaub, H., and Junkins, J. L., \textit{Analytical Mechanics of Space Systems, 3rd ed.}. Reston, VA: American Institute of Aeronautics and Astronautics.
\end{thebibliography} % Contains a discussion of how to setup and configure  the BSK module
%





\bibliography{bibliography} %This includes references used and mentioned.

\end{document}
