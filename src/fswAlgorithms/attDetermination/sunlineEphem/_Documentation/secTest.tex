% !TEX root = ./Basilisk-MODULENAME-yyyymmdd.tex

\section{Test Description and Success Criteria}
The unit test confirms that the vectors produced from the module do reflect the unit vector in the body frame pointing to the sun. 


\section{Test Parameters}

Test and simulation parameters and inputs go here. Basically, describe your test in the section above, but put any specific numbers or inputs to the tests in this section.

The unit test verify that the module output guidance message vectors match expected values.
\begin{table}[htbp]
	\caption{Error tolerance for each test.}
	\label{tab:errortol}
	\centering \fontsize{10}{10}\selectfont
	\begin{tabular}{ c | c } % Column formatting, 
		\hline\hline
		\textbf{Output Value Tested}  & \textbf{Tolerated Error}  \\ 
		\hline
		{\tt estVector}        & 1e-05	   \\ 
		\hline\hline
	\end{tabular}
\end{table}




\section{Test Results}


All of the tests passed:
\begin{table}[H]
	\caption{Test results}
	\label{tab:results}
	\centering \fontsize{10}{10}\selectfont
	\begin{tabular}{c | c  } % Column formatting, 
		\hline\hline
		\textbf{Check} 						  		&\textbf{Pass/Fail} \\ 
		\hline
	   1	   			& \input{AutoTeX/passFail11} \\ 
	   2	   			& \input{AutoTeX/passFail13} \\ 
	   3	   			& \input{AutoTeX/passFail22} \\ 
	   \hline\hline
	\end{tabular}
\end{table}




