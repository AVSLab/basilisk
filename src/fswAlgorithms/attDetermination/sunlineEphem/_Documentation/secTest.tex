% !TEX root = ./Basilisk-SunlineEphem-20181204.tex

\section{Test Description and Success Criteria}
The unit test configures the sun to sit at the origin, and a spacecraft to be located along each of the coordinate axes, with an orientation 90 degree rotate about the z-axis. The body-frame sun-heading is then computed to confirm that the vectors produced from the module do reflect the unit vector in the body frame pointing to the sun. 


\section{Test Parameters}

The sun was placed at $[0, 0, 0]$ and the spacecraft is tested at each of the unit coordinate axes $[1, 0, 0]$, $[0, 1, 0]$, $[-1, 0, 0]$, etc. 

The unit test verify that the module output guidance message vectors match expected values.
\begin{table}[htbp]
	\caption{Error tolerance for each test.}
	\label{tab:errortol}
	\centering \fontsize{10}{10}\selectfont
	\begin{tabular}{ c | c } % Column formatting, 
		\hline\hline
		\textbf{Output Value Tested}  & \textbf{Tolerated Error}  \\ 
		\hline
		{\tt estVector}        & see script	   \\ 
		\hline\hline
	\end{tabular}
\end{table}




%\section{Test Results}
%
%
%All of the tests passed:
%\begin{table}[H]
%	\caption{Test results}
%	\label{tab:results}
%	\centering \fontsize{10}{10}\selectfont
%	\begin{tabular}{c | c  } % Column formatting, 
%		\hline\hline
%		\textbf{Check} 						  		&\textbf{Pass/Fail} \\ 
%		\hline
%%	   1	   			& \input{AutoTeX/passFail11} \\ 
%%	   2	   			& \input{AutoTeX/passFail13} \\ 
%%	   3	   			& \input{AutoTeX/passFail22} \\ 
%%	   4	   			& \input{AutoTeX/passFail11} \\ 
%%	   5	   			& \input{AutoTeX/passFail13} \\ 
%	   6	   			& \input{AutoTeX/passFail22} \\ 
%	   \hline\hline
%	\end{tabular}
%\end{table}




